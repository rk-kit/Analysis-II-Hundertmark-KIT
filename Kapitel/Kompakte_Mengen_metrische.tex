\section{[*] Kompakte Mengen und metrische Räume}
\imaginarysubsection{Kompakte Mengen}
\thispagestyle{pagenumberonly}

\begin{definition}[Kompaktheit]
    Sei $\pair{M, d}$ ein metrischer Raum. Eine Teilmenge $A\subseteq M$ ist kompakt, falls jede Folge in $A$ eine konvergente Teilfolge besitzt, deren Grenzwert wieder in $A$ liegt. Das heißt
    \begin{align*}
        \forall \text{Folge } (x_n)_n\subseteq A\ex\text{Teilfolge } (x_{n_k})_k\colon x\coloneqq \lim_{k\toinf} x_{n_k}\text{ existiert und }x\in A
    \end{align*}
\end{definition}

\begin{bemerkung}[Topologische Kompaktheit]
    Es gibt auch eine topologische Definition, die nicht über Folgen argumentiert und im metrischen Fall äquivalent zur Folgenkompaktheit ist: $A\subseteq M$ ist topologisch kompakt, falls jede offene Überdeckung eine endliche Teilüberdeckung hat.\\
    Das heißt für eine beliebige Indexmenge $I$ und $U_j$ offen $\forall j\in I$ mit $A\subseteq \bigcup_{j\in I} U_j$ muss ein $N\in\N$ und $j_1, j_2, \ldots, j_N \in I$ existieren, sodass $A \subseteq U_{j_1} \cup U_{j_2} \cup \dots \cup U_{j_N}$.
\end{bemerkung}

\begin{definition}
    Ein metrischer Raum $\pair{M, d}$ ist kompakt, falls $M$ kompakt ist.
\end{definition}

\begin{bemerkung}
    Jeder kompakte metrische Raum ist vollständig.
    \begin{proof}
        Wir haben gezeigt, dass eine Cauchy-Folge genau dann konvergiert, wenn sie eine konvergente Teilfolge hat. Damit konvergiert jede Cauchy-Folge in einem kompakten Raum.
    \end{proof}
\end{bemerkung}

\begin{satz}
    Wir betrachten $\pair{\R^d, \norm{\cdot}_{\infty}}$ mit Maximumsnorm. Dann gilt $A\subseteq \R^d$ ist genau dann kompakt, wenn $A$ abgeschlossen und beschränkt ist.
    \begin{proof}
        \anf{$\impl$} Sei $A\subseteq\R^d$ kompakt. Sei $(x_n)_n \subseteq A$ eine Folge in $A$, welche gegen $x\in\R^d$ konvergiert. Dann existiert eine Teilfolge $(x_{n_k})_k$ mit $y\coloneqq \lim_{\ntoinf} x_{n_k}\in A$. Da $(x_n)_n$ konvergiert, ist $x=y\in A$.\\
        Angenommen $A$ ist nicht beschränkt. Das heißt
        \begin{align*}
            \forall n\in\N\ex x_n\in A\colon \norm{x_n}_{\infty} \geq n
        \end{align*}
        Dann gilt für jede Teilfolge $(x_{n_k})_k$ auch $\norm{x_{n_k}} \geq n_k \geq k$. Das heißt jede Teilfoge von $(x_n)_n$ ist unbeschränkt und kann somit nicht konvergieren.\\[.5\baselineskip]
        \anf{$\Leftarrow$} Sei $A$ abgeschlossen und beschränkt. Dann wählen wir eine Folge $(x_n)_n\subseteq A$. Dann besteht $x_n$ aus mehreren Koordinaten. Das heißt $x_n = \pair{x_n^1~x_n^2~\dots~x_n^d}$ Wir definieren $(x_n^j)_n$ als Folge in $\R$. Da $A$ beschränkt ist, muss auch $(x_n^j)_n$ beschränkt sein. Das heißt für $j=1$ gibt es eine Teilfolge $(x_{n_k}^1)_k$ von $(x_n^1)_n$, die konvergiert. Es gibt also eine Ausdünnung $\sigma_1: \N\fromto\N$ mit $\sigma_1\of{n+1} > \sigma_1\of{n}~\forall n\in\N$, sodass $x^1\coloneqq \lim_{k\toinf} x_{\sigma_{1}\of{k}}^1$ existiert. Dann konvergiert die erste Koordinate von $\pair{x_{\sigma_1\of{n}}^1~x_{\sigma_1\of{n}}^2~\ldots~x_{\sigma_1\of{n}}^d}$.\\
        Wir führen das Prinzip iterativ fort. Das heißt $(x_{\sigma\of{n}}^2)_n$ ist beschränkt in $\R$. Damit hat es eine konvergente Teilfolge. Wir wählen eine neue Ausdünnung $\sigma_2: \N\fromto\N$ analog zu $\sigma_1$ und setzen $\kappa_2 \coloneqq \sigma_2 \circ \sigma_1$. Dann hat $x^2_{\kappa_2\of{n}}$ einen Grenzwert $x^2\coloneqq \biglim{\ntoinf} x^2_{\kappa_2\of{n}}$. Dabei konvergiert $x_{\kappa_2\of{n}}^1$ immer noch.\\
        Dieses Prinzip wenden wir für jede Koordinate an. Das heißt wir definieren $\sigma_j: \N\fromto\N$ und $\kappa_{j+1} \coloneqq \sigma_{j+1} \circ \kappa_j$. Damit hören wir bei $j=d$ auf. Dann haben wir $x_{\kappa_d\of{n}}^j$ konvergiert gegen ein $x^j$ für alle $j\in\set{1, \ldots, d}$. Das heißt wir haben einen Grenzwert $x=\pair{x^1~x^2~\dots~x^d}$ von $(x_{\kappa_d\of{n}})_n$. Da $A$ abgeschlossen ist und $x_{\kappa\of{n}} \fromto x$ ist auch $x\in A$. Damit ist $A$ kompakt.
    \end{proof}
\end{satz}

\begin{satz} % Satz 4
    \label{satz:weierstrass-alg}
    Sei $\pair{M, d}$ ein metrischer Raum und $A\subseteq M$ kompakt. Sei außerdem $f: A\fromto\R$ stetig. Dann nimmt $f$ sein Maximum und Minimum an. Das heißt
    \begin{align*}
        \exists\un{x}\in A\colon f\of{\un{x}} &= \inf\set{f\of{x}: x\in A} \text{ und } \\
        \exists\ov{x}\in A\colon f\of{\ov{x}} &= \sup\set{f\of{x}: x\in A}
    \end{align*}
    \begin{proof}
        \textsc{Schritt 1}: Sei $a\coloneqq\inf\set{f\of{x}: x\in A}$. Angenommen $a=-\infty$. Dann würde gelten
        \begin{align*}
            &\forall n\in\N\ex x_0\in A\colon f\of{x_n} \leq -n\\
            \intertext{Das heißt es existiert eine konvergente Teilfolge $(x_{n_k})_k$, da $A$ kompakt ist. Wir definieren $x \coloneqq \biglim{k\toinf} x_{n_k}\in A$. Wegen der Stetigkeit von $f$ gilt dann}
            \R\ni f\of{x} &= f\of{\lim_{k\toinf} x_{n_k}} = \lim_{k\toinf} f\of{x_{n_k}} = -\infty
        \end{align*}
        \textsc{Schritt 2}: Sei $a\coloneqq\inf\set{f\of{x}: x\in A}$. Das heißt
        \begin{align*}
            \forall n\in\N\ex x_n\in A\colon &f\of{x_n} < a + \frac{1}{n}\\
            \impl a \leq &f\of{x_n} < a + \frac{1}{n}
            \intertext{Da $A$ kompakt ist, existiert eine konvergente Teilfolge $(x_{n_k})_k$ von $(x_n)_n$. Wir definieren}
            \un{x}\coloneqq&\lim_{k\toinf} x_{n_k} \leq a + \frac{1}{n}\\
            \impl a \leq &f\of{\un{x}} = f\of{\lim_{k\toinf} x_{n_k}} = \lim_{k\toinf} x_{n_k} = a\\
            \impl &f\of{\un{x}} = a
        \end{align*}
        \textsc{Schritt 3}: Für das Maximum betrachte $-f$ und wende \textsc{Schritt 1} und \textsc{Schritt 2} an.
    \end{proof}
\end{satz}

\begin{satz} % Satz 5
    \marginnote{[28. Jun]}
    Auf endliche-dimensionalen Vektorräumen sind alle Normen äquivalent. Das heißt für einen endlich-dimensionalen Vektorraum $X$ mit zwei Normen $\norm{\cdot}_a, \norm{\cdot}_b$ gibt es $0 <c_1 \leq c_2 < \infty$, sodass
    \begin{align*}
        c_1\cdot\norm{h}_a &\leq \norm{h}_b \leq c_2\cdot\norm{h}_a\quad\forall h\in X
    \end{align*}
    \begin{proof}
        \textsc{Schritt 1}: Sei $d\coloneqq\dim\of{X} < \infty$. Wir wählen eine geordnete Basis $\pair{e_1, e_2, \ldots, e_d}$ von $X$. Dann können wir einen beliebigen Vektor $h\in X$ schreiben als
        \begin{align*}
            h &= \sum_{j=1}^{d} t_j\cdot e_j
            \intertext{Für eindeutige $t_j\in\R$. Damit erhalten wir eine Bijektion}
            h: \R^d &\fromto X\\
            \R^d \ni t&\mapsto h\of{t} = \sum_{j=1}^{d} t_j\cdot e_j \in X
            \intertext{Also reicht es, zu zeigen, dass für beliebige Normen $\norm{\cdot}$ auf $X$ und gewählte $0< c_1 \leq c_2 < \infty$ gilt}
            c_1\cdot\norm{t}_{\infty} &\leq \norm{h\of{t}} \leq c_2\cdot\norm{t}_{\infty}\quad\forall t\in\R^d\tag{1}
            \intertext{Setzen wir nämlich voraus, dass (1) gilt, dann gilt für zwei beliebige Normen $\norm{\cdot}_a, \norm{\cdot}_b$ auf $X$}
            c_1\cdot\norm{t}_{\infty} &\leq \norm{h\of{t}}_a \leq c_2\cdot\norm{t}_{\infty}\\
            d_1\cdot\norm{t}_{\infty} &\leq \norm{h\of{t}}_b \leq d_2\cdot\norm{t}_{\infty}
            \intertext{Sei $h\coloneqq h\of{t}$}
            \impl \norm{h}_a &\leq c_2\cdot \norm{t}_{\infty} \leq \frac{c_2}{d_1}\cdot\norm{h}_b\\
            &\leq \frac{c_2}{d_1}\cdot d_2 \cdot \norm{t}_{\infty}\\
            &\leq \frac{c_2\cdot d_2}{c_1\cdot d_1} \cdot\norm{h}_a\\
            \impl \frac{d_1}{c_2} \cdot\norm{h}_a &\leq \norm{h}_b \leq \frac{d_2}{c_1}\cdot\norm{h}_a
            \intertext{Damit wäre die Behauptung gezeigt. Das heißt wir müssen nur noch (1) zeigen. \textsc{Schritt 2}: Sei $\norm{\cdot}$ eine beliebige Norm auf $X$. Wir definieren}
            f: \R^d &\fromto \R\\
            h &\mapsto f\of{h\of{t}} = \norm{h\of{t}}
            \intertext{Dann existiert ein $0 < C < \infty$, sodass}
            \abs{f\of{t}} = f\of{t} &\leq C\cdot\norm{t}_{\infty}\quad\forall t\in\R^d\tag{2}
            \intertext{und}
            \abs{f\of{t_1} - f\of{t}} &\leq C\cdot\norm{t_1 - t_2}_{\infty}\quad\forall t_1, t_2\in\R^d\tag{3}
            \intertext{Dabei folgt aus (3) gerade, dass $f$ Lipschitz-stetig ist. Um (2) zu beweisen, betrachten wir}
            \abs{f\of{t}} &= \norm{h\of{t}} = \norm{ \sum_{j=1}^{d} t_j\cdot e_j}\\
            &\leq \sum_{j=1}^{d} \norm{t_j\cdot e_j}\\
            &\leq \max_{1\leq j \leq d} \abs{t_j} \cdot \underbrace{\sum_{j=1}^{d} \norm{e_j}}_{\eqqcolon C} = C\cdot\norm{t}_{\infty}
            \intertext{Außerdem zeigen wir (3):}
            \abs{f\of{t_1} - f\of{t_2}} &= \abs{\norm{h\of{t_1}} - \norm{h\of{t_2}}}\\
            &= \abs{\norm{h_1} - \norm{h_2}} \leq \norm{h_1 - h_2}\\
            \impl \abs{f\of{t_1} - f\of{t_2}} &\leq \norm{h\of{t_1} - h\of{t_2}}\\
            &= \norm{h\of{t_1 - t_2}} \leq C\cdot\norm{t_1 - t_2}_{\infty}
            \intertext{Damit haben wir (3) gezeigt. \textsc{Schritt 3}: Nach (3) gilt $\norm{h\of{t}} \leq C\cdot\norm{t}_{\infty}$. Also reicht es zu zeigen, dass}
            \exists c_1 > 0\colon c_1\cdot\norm{t}_{\infty} &\leq \norm{h\of{t}}
            \intertext{Sei $t\neq 0$. Dann ist auch $\norm{t}_{\infty} > 0$}
            f\of{\frac{t}{\norm{t}}_{\infty}} &= \norm{h\of{\frac{t}{\norm{t}_{\infty}}}}\\
            h\of{\frac{t}{\norm{t}_{\infty}}} &= \sum_{j=1}^{d} \frac{t_j}{\norm{t}_{\infty}}\cdot e_j\\
            &= \frac{1}{\norm{t}_{\infty}} \sum_{j=1}^{d} t_j\cdot e_j\\
            &= \frac{1}{\norm{t}_{\infty}}\cdot f\of{t}\\
            \impl \norm{t}_{\infty}\cdot f\of{\frac{t}{\norm{t}_{\infty}}} &= f\of{t}\quad\forall t\in \R^d\exclude\set{\textbf{0}}\tag{4}
            \intertext{Wir definieren}
            S &\coloneqq \set{t\in\R^d : \norm{t}_{\infty} = 1}
            \intertext{Damit ist $S$ beschränkt und abgeschlossen. Wir wählen eine Folge $(t_n)_n\subseteq S$ mit $t_n\fromto t$}
            \impl \norm{t}_{\infty} &= \norm{\lim_{\ntoinf} t_n}_{\infty} = \lim_{\ntoinf}\norm{t_n}_{\infty} = 1\\
            \impl t&\in S
            \intertext{Wir definieren außerdem}
            c_1 &\coloneqq \inf\set{f\of{t}: t\in S}\\
            \annot{\impl}{(4)} f\of{t} &= \norm{t}_{\infty} \cdot f\of{\frac{t}{\norm{t}_{\infty}}} \geq c_1\cdot\norm{t}_{\infty}\quad\forall t\in\R^d
            \intertext{Frage: Was ist ein guter Grund, dass $c_1 > 0$? \textsc{Schritt 4}: $S$ ist kompakt, da $\R^d$ endlich-dimensional ist und $S$ beschränkt und abgeschlossen ist. Außerdem ist $f: S\fromto\R$ stetig und $f\of{t} > 0~\forall t\in S$. Nach Satz~\ref{satz:weierstrass-alg} nimmt $f$ auf $S$ sein Minimum ein. Das heißt es existiert ein $\un{t}\in S$, sodass}
            c_1 &= \inf\set{f\of{t}: t\in S} = f\of{\un{t}} > 0\qedhere
        \end{align*}
    \end{proof}
\end{satz}

\begin{bemerkung}
    Sei $d = \dim\of{X}$ und $\pair{e_1, e_2,\ldots, e_d}$ eine Basis von $X$. Dann ist
    \begin{align*}
        F: \R^d &\fromto X\\
        t &\mapsto F\of{t} = \sum_{j=1}^{d} t_j\cdot e_j
        \intertext{eine lineare Bijektion. Wir haben im vorherigen Beweis gesehen, dass}
        \norm{F\of{t}} & \leq C\cdot\norm{t}_{\infty}\text{ für } C \coloneqq \sum_{j=1}^{d} \norm{e_j}
        \intertext{Damit ist $F$ sogar stetig. Frage: Ist $F^{-1}: X\fromto\R^d$ auch stetig?}
        c_1 \cdot\norm{t}_{\infty} &\leq \norm{F\of{t}}\\
        t &= F^{-1}\of{h}\tag{$h\coloneqq F\of{t}$}\\
        \impl c_1\cdot\norm{F^{-1}\of{h}} &\leq \norm{h}\\
        \impl \norm{F^{-1}\of{h}} &\leq \frac{1}{c_1}\cdot\norm{h}
    \end{align*}
    Das heißt $F^{-1}: X\fromto \R^d$ ist auch stetig.
\end{bemerkung}

\begin{satz} % Satz 6
    Seien $\pair{M, d_M}$ und $\pair{N, d_N}$ metrische Räume, $M$ kompakt und $f: M\fromto N$. Dann gilt $f$ ist genau dann stetig, wenn $f$ gleichmäßig stetig ist. Das heißt
    \begin{align*}
        \forall \varepsilon > 0\ex\delta > 0\fa x,y\in M\colon d_M\of{x,y} < \delta \impl d_N\of{f\of{x}, f\of{y}} < \varepsilon
    \end{align*}
    \begin{proof}
        \anf{$\Leftarrow$} Ist klar.\\
        \anf{$\impl$} Angenommen $f$ ist stetig, aber nicht gleichmäßig stetig. Das heißt
        \begin{align*}
            \exists\varepsilon > 0\fa\delta > 0\ex x,y\in M\colon d_M\of{x,y} < \delta \text{ aber } d_N\of{f\of{x}, f\of{y}} \geq \varepsilon
            \intertext{Wähle $\delta = \frac{1}{n}$. Das heißt es existieren zwei Folgen $(x_n)_n, (y_n)_n \subseteq M$, sodass}
            d_N\of{x_n, y_n} < \frac{1}{n} \text{ aber } d_N\of{f\of{x_n}, f\of{y_n}} \geq \varepsilon\quad\forall n\in\N
            \intertext{Da $M$ kompakt ist, existiert eine Teilfolge $(y_{n_k})_k$ von $(y_n)_n$, die konvergiert. Sei}
            y \coloneqq \lim_{k\toinf} y_{n_k} \in M
            \intertext{Behauptung: $x_{n_K} \fromto y$, da}
            d_M\of{x_{n_k}, y} \leq d_M\of{x_{n_k}, y_{n_k}} + d_M\of{y_{n_k}, y} \fromto 0
            \intertext{Da $f$ stetig ist folgt}
            \impl d_N\of{f\of{x_{n_k}}, f\of{y}}\fromto 0 \text{ für } k\toinf\\
            0 < \varepsilon \leq d_N\of{f\of{x_{n_k}}, f\of{y_{n_k}}} \leq d_N\of{f\of{x_{n_k}}, f\of{y}} + d_N\of{f\of{y}, f\of{y_{n_k}}}\fromto 0
        \end{align*}
        Damit ergibt sich ein Widerspruch, da $\varepsilon$ fest gewählt war.
    \end{proof}
\end{satz}

\newpage