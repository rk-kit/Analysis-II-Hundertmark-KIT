\section{Integrale und gleichmäßige Konvergenz}
\imaginarysubsection{Gleichmäßige Konvergenz}
\thispagestyle{pagenumberonly}

Sei $I=\interv{a,b}$ und $f: I\fromto\R$, $f_n: I\fromto \R$. Wenn die Funktionenfolge $(f_n)_n$ \anf{irgendwie} gegen $f$ konvergiert. Wann gilt dann
\begin{align*}
    \int_{a}^{b} f_n\of{x} \dif x \fromto \int_{a}^{b} f\of{x} \dif x \text{ für } \ntoinf \text{ ?}
\end{align*}
Wir werden in diesem Kapitel einsehen, dass punktweise Konvergenz dafür nicht ausreichend ist, sondern wir gleichmäßige Konvergenz fordern müssen.
\begin{beispiel}[Punktweise Konvergenz]
    Sei $f_n: \interv{0, 1}\fromto\R$ mit
    \begin{align*}
        f_n\of{x} &\coloneqq \begin{cases}
                                 n &0 < x < \frac{1}{n}\\
                                 0 &\text{sonst}
        \end{cases}
        \intertext{$(f_n)_n$ konvergiert punktweise gegen die Nullfunktion ($f_n\of{x}\fromto 0$ für $n\toinf~\forall x\in\interv{0,1}$). Außerdem gilt für ein $n\in\N$}
        \int_{0}^{1} f_n\of{x} \dif x &= \int_{0}^{\frac{1}{n}} n \dif x = \frac{n}{n} = 1
    \end{align*}
    Das Integral über die Nullfunktion ist aber 0. Das heißt punktweise Konvergenz ist kein ausreichendes Kriterium, damit die Integrale gleich sind.
\end{beispiel}

\begin{satz} % Satz 1
    \label{satz:gleichm-int}
    Seien $f, f_n: \interv{a,b}\fromto\R$ (oder $\C, \dots$) und $n\in\N$. Außerdem konvergiere $(f_n)_n$ gleichmäßig gegen $f$ auf $\interv{a,b}$ und $f_n\in\mR\of{\interv{a,b}}$. Dann gilt $f\in\mR\of{\interv{a,b}}$ und
    \begin{align*}
        \lim_{\ntoinf} \int_{a}^{b} f_n\of{x} \dif x &= \int_{a}^{b} f\of{x} \dif x = \int_{a}^{b} \lim_{\ntoinf} f_n\of{x} \dif x
    \end{align*}

    \begin{proof}
        Sei $\varepsilon > 0$ und $N\in\N$ groß genug. Dann gilt
        \begin{align*}
            \norm{f-f_N}_{\infty} &= \sup_{a \leq x \leq b} \abs{f\of{x} - f_N\of{x}} < \frac{\varepsilon}{4\cdot\pair{b-a}}\\
            \impl f_N\of{x} - \frac{\varepsilon}{4\cdot\pair{b-a}} &\leq f\of{x} \leq f_N\of{x} + \frac{\varepsilon}{4\cdot\pair{b-a}}\quad\forall n\geq N\tag{1}
            \intertext{Halte $N$ fest und nehme Zerlegung $Z$ von $I=\interv{a,b}$ mit $\ov{S}_Z\of{f_N} - \un{S}_Z\of{f_N} < \frac{\varepsilon}{2}$. Dann gilt jeweils nach (1)}
            \ov{S}_Z\of{f} \leq \ov{S}_Z\of{f_N + \frac{\varepsilon}{4\cdot\pair{b-a}}} &= \ov{S}_Z\of{f_N} + \ov{S}_Z\of{\frac{\varepsilon}{4\cdot\pair{b-a}}} = \ov{S}_Z\of{f_N} + \frac{\varepsilon}{4}\\
            \un{S}_Z\of{f} \geq \un{S}_Z\of{f_N - \frac{\varepsilon}{4\cdot\pair{b-a}}} &= \un{S}_Z\of{f_N} - \un{S}_Z\of{\frac{\varepsilon}{4\cdot\pair{b-a}}} = \un{S}_Z\of{f_N} - \frac{\varepsilon}{4}\\[.6\baselineskip]
            \impl \ov{S}_Z\of{f} - \un{S}_Z\of{f} &\leq \ov{S}_Z\of{f_N} + \frac{\varepsilon}{4} - \pair{\un{S}_Z\of{f_N} - \frac{\varepsilon}{4}}\\
            &= \ov{S}_Z\of{f_N} - \un{S}_Z\of{f_N} + \frac{\varepsilon}{2}\\
            &< \frac{\varepsilon}{2} + \frac{\varepsilon}{2} = \varepsilon
        \end{align*}
        Damit folgt $f\in\mR\of{I}$. Wir beweisen die Gleichheit der Integrale.
        \begin{align*}
            \int_{a}^{b} f_n\of{x} \dif x - \frac{\varepsilon}{4} &= \int_{a}^{b} \pair{f_n\of{x} - \frac{\varepsilon}{4\cdot\pair{b-a}}} \dif x\\
            &\leq \int_{a}^{b} f\of{x} \dif x \leq \int_{a}^{b} \pair{f_n\of{x} + \frac{\varepsilon}{4\cdot\pair{b-a}}} \dif x\\
            &= \int_{a}^{b} f_n\of{x} \dif x + \frac{\varepsilon}{4}\quad\forall n\geq N\\
            \impl \limsup_{\ntoinf} \int_{a}^{b} f_n\of{x} \dif x - \frac{\varepsilon}{4} &\leq \int_{a}^{b} f\of{x} \dif x \leq \liminf_{\ntoinf} \int_{a}^{b} f_n\of{x} \dif x + \frac{\varepsilon}{4}\quad\forall\varepsilon > 0\\
            \impl \limsup_{\ntoinf} \int_{a}^{b} f_n\of{x} \dif x &\leq \int_{a}^{b} f\of{x} \dif x \leq \liminf \int_{a}^{b} f_n\of{x} \dif x\qedhere
        \end{align*}
    \end{proof}
\end{satz}

\begin{beispiel}[Integral von Potenzreihen]
    \marginnote{[17. Mai]}
    Wir betrachten die Potenzreihe
    \begin{align*}
        f\of{x} &= \sum_{n=0}^{\infty} a_n\pair{x-x_0}^n
        \intertext{mit Konvergenzradius $R>0$ und}
        R &= \frac{1}{\displaystyle \limsup_{\ntoinf} \abs{a_n}^{\frac{1}{n}}}
        \intertext{Wir haben also eine Funktion $f: \pair{x_0 - R, x_0 + R}\fromto\R$ (oder $\C$). Die Stammfunktion zu $a_n\pair{x-x_0}^n$ ist $\frac{a_n}{n+1}\pair{x-x_0}^{n+1}$. Das heißt für endliche Summen können wir leicht ein Integral finden. Ob das auch für unendliche Summen funktioniert, ist nicht direkt klar, aber wir definieren trotzdem in diesem Sinne}
        F\of{x} \coloneqq \sum_{n=0}^{\infty} \frac{a_n}{n+1}&\pair{x-x_0}^{n+1} = \sum_{n=1}^{\infty} c_n\pair{x-x_0}^n\tag{$c_n\coloneqq \frac{a_{n-1}}{n}$}
        \intertext{Wir prüfen den Konvergenzradius}
        \limsup_{\ntoinf} \abs{c_n}^{\frac{1}{n}} &= \limsup_{\ntoinf} \abs{\frac{a_{n-1}}{n}}^{\frac{1}{n}}\\
        \intertext{Es gilt}
        \pair{\frac{\abs{a_{n-1}}}{n}}^{\frac{1}{n}} &= \frac{1}{n^{\frac{1}{n}}}\pair{\abs{a_{n-1}}^{\frac{1}{n-1}}}^{\frac{n-1}{n}}\\
        \impl \limsup_{\ntoinf} \abs{c_n}^{\frac{1}{n}} &= \limsup_{\ntoinf} \abs{a_n}^{\frac{1}{n}}
        \intertext{Das heißt $F$ hat denselben Konvergenzradius wie $f$. Unsere Hoffnung ist also, dass $F$ eine Stammfunktion von $f$ ist, bzw.}
        \int_{x_0}^{x} f\of{t} \dif t &= F\of{x}
        \intertext{Das gilt tatsächlich und lässt sich folgendermaßen zeigen. Wir definieren eine Funktionenfolge}
        f_n\of{x} &\coloneqq \sum_{k=0}^{n} a_k\pair{x-x_0}^k
    \end{align*}
    Wir wissen $\forall\delta > 0$ klein genug (konkret heißt konkret $\delta < R$) konvergiert $f_n$ gleichmäßig gegen $f$ auf dem Intervall $\interv{x_0-R+\delta, x_0+R-\delta}$. Dann gilt für ein festes $x\in\interv{x_0-R+\delta, x_0+R-\delta}$
    \begin{align*}
        \int_{x_0}^{x} f_n\of{t} \dif t &= \int_{x_0}^{x} \sum_{k=0}^{n} a_k\pair{t-x_0}^k \dif t = \sum_{k=0}^{n} a_k \int_{x_0}^{x} \pair{t-x_0}^k \dif t\\
        &= \sum_{k=0}^{n} a_k\cdot \interv{\frac{1}{k+1}\pair{t-x_0}^{k+1}}_{x_0}^{x} = \sum_{k=0}^{n} \frac{a_k}{k+1}\pair{x-x_0}^{k+1}
        \intertext{Das heißt nach Satz~\ref{satz:gleichm-int} wissen wir, dass}
        \int_{x_0}^{x} f\of{t} \dif t &= \lim_{\ntoinf} \int_{x_0}^{x} f_n\of{t} \dif t = \lim_{\ntoinf} \sum_{k=0}^{n} \frac{a_k}{k+1}\pair{x-x_0}^{k+1} \dif x = F(x)
    \end{align*}
\end{beispiel}

\begin{satz}
    \label{satz:ableitung-gleichm-konv}
    Sei $I=\interv{a,b}$ sowie $f_n: I\fromto\R$ (oder $\C$) und die folgenden Voraussetzungen gelten
    \begin{enumerate}[label=(\roman*)]
        \item $\exists x_0\in I\colon f_n\of{x_0}$ konvergiert gegen $f\of{x_0}$
        \item $\pair{f_n'}_n$ konvergiert gleichmäßig gegen eine Funktion $g$
        \item $f_n'$ ist stetig für alle $n\in\N$
    \end{enumerate}
    Dann gilt $f(x) \coloneqq \displaystyle\lim_{\ntoinf} f_n\of{x}~\forall x\in I$ und $f$ ist stetig differenzierbar mit Ableitung $f' = g$.
    \begin{proof}
        Sei $x\in I$. Da alle Ableitungen von $f_n$ stetig sind, können wir den Hauptsatz verwenden und es gilt
        \begin{align*}
            f_n\of{x} - f_n\of{x_0} &= \int_{x_0}^{x} f_n'\of{t} \dif t\\
            \impl f_n\of{x} &= f_n\of{x_0} + \int_{x_0}^{x} f_n'\of{t} \dif t \underset{\ntoinf}{\to} f\of{x_0} + \int_{x_0}^{x} g\of{t} \dif t\\
            \impl f\of{x} &\coloneqq \lim_{\ntoinf} f_n\of{x} \text{ existiert } \forall x\in I \text{ und}\\
            f\of{x} &= f\of{x_0} + \int_{x_0}^{x} g\of{t} \dif t
        \end{align*}
        Nach dem Hauptsatz gilt, dass $f$ stetig differenzierbar ist mit $f' = g$.
    \end{proof}
\end{satz}

\begin{anwendung}
    \label{anwendung:potenzreihe-diff}
    Wir betrachten eine Potenzreihe
    \begin{align*}
        f\of{x} &= \sum_{n=0}^{\infty} a_n\pair{x-x_0}^{n}
        \intertext{mit Konvergenzradius}
        R &= \frac{1}{\displaystyle \limsup_{\ntoinf} \abs{a_n}^{\frac{1}{n}}} > 0\\
        f_n\of{x} &\coloneqq \sum_{k=0}^{n} a_k\pair{x-x_0}^k\\
        \impl f\of{x} &= \lim_{\ntoinf} f_n\of{x}\\
        f_n'\of{x} &= \sum_{k=1}^{\infty} k\cdot a_k\pair{x-x_0}^{k-1}
        \intertext{Es gilt}
        \limsup_{\ntoinf} \abs{n\cdot a_{n+1}}^{\frac{1}{n-1}} &= \limsup_{\ntoinf} \sqrt[n-1]{n} \cdot \abs{a_{n+1}}^{\frac{1}{n-1}} = \limsup_{\ntoinf} \abs{a_{n}}^{\frac{1}{n}}
        \intertext{Das heißt die Konvergenzradien sind gleich und}
        f_n'\of{x} &= \sum_{k=1}^{n} k\cdot a_k\pair{x-x_0}^{k-1}
        \intertext{konvergiert auch auf $\pair{x_0-R, x_0+R}$ und gleichmäßig auf $\interv{x_0-R+\delta, x_0+R-\delta}$. Also konvergiert nach Satz~\ref{satz:ableitung-gleichm-konv} auch die Potenzreihe}
        f\of{x} &= \sum_{n=0}^{\infty} a_n\pair{x-x_0}^n
        \intertext{und ihre Ableitung ist gegeben durch}
        &\sum_{n=1}^{\infty} n\cdot a_n\pair{x-x_0}^{n-1}
    \end{align*}
    Also ist jede Potenzreihe auf ihrem Konvergenzintervall differenzierbar.
\end{anwendung}

\begin{korollar}
    \label{korollar:potenzreihe-diffb}
    Jede Potenzreihe $f\of{x} = \sum_{n=0}^{\infty} a_n\pair{x-x_0}^{n}$ ist unendlich oft differenzierbar auf ihrem Konvergenzintervall.
    \begin{proof}
        Nach Anwendung~\ref{anwendung:potenzreihe-diff} ist eine Potenzreihe einmal differenzierbar mit einer Potenzreihe als Ableitung. Damit folgt induktiv die Behauptung. Insbesondere gilt
        \begin{align*}
            f'\of{x} &= \sum_{n=1}^{\infty} n\cdot a_n\cdot\pair{x-x_0}^{n-1}\\
            f''\of{x} &= \sum_{n=2}^{\infty} n\cdot\pair{n-1}\cdot a_n\cdot\pair{x-x_0}^{n-2}\\
            f^{(k)}\of{x} &= \sum_{n=k}^{\infty} n\cdot\pair{n-1}\cdot\ldots\cdot\pair{n-k+1}\cdot a_n\cdot \pair{x-x_0}^{n-k}\\
            \impl f^{(k)}\of{x_0} &= k!\cdot a_k\\
            \equivalent a_k &= \frac{f^{(k)}\of{x_0}}{k!}
        \end{align*}
    \end{proof}
\end{korollar}

\begin{anwendung}
    Wir wollen
    \begin{align*}
        &\sum_{n=1}^{\infty} n\cdot x^n
        \intertext{für $\abs{x} < 1$ berechnen. Wir wissen, dass für $\abs{x} < 1$}
        \sum_{n=0}^{\infty} x^n &= \frac{1}{1-x}\\
        \impl \sum_{n=1}^{\infty} n\cdot x^n &= x\cdot \sum_{n=1}^{\infty} n\cdot x^{n-1}\\
        &= x\cdot \frac{\dif}{\dif x} \sum_{n=0}^{\infty} x^n = x\cdot \frac{\dif}{\dif x} \frac{1}{1-x}\\
        &= x\cdot \frac{-1}{\pair{1-x}^2}\cdot\pair{-1} = \frac{x}{\pair{1-x}^2}
    \end{align*}
\end{anwendung}

\newpage
