\section{Uneigentliche Integrale}
\thispagestyle{pagenumberonly}

Bisher haben wir immer nur Integrale auf kompakten Intervallen berechnet und dabei waren alle Funktionen $f\in\mR\of{I}$ insbesondere beschränkt. Frage: Was ist $\int_{0}^{1} \frac{1}{\sqrt{x}} \dif x$? Was ist $ \int_{0}^{\infty} e^{-t} \dif t$?
\begin{align*}
    \int_{0}^{b} e^{-t} \dif t &= \interv{-e^{-t}}_0^b = e^{-0} - e^{-b} = 1 - e^{-b} = 1 -\frac{1}{e^b}\fromto 1 \text{ für } b\fromto\infty
\end{align*}
Wir betrachten unterschiedliche Fälle von solchen sogenannten uneigentlichen Integralen.

\subsection{Uneigentliche Integrale - Fall I}

\begin{mdframed}
    \condition{cond:uneigentlich-fall1}: In diesem Teilkapitel sei $I=\linterv{a, \infty}$, $f: I\to\R$ und $f\in\mR\of{\interv{a,b}}~\forall a < b < \infty$. Wir definieren $F\of{b} \coloneqq \int_{a}^{b} f\of{x} \dif x$.
\end{mdframed}

\begin{definition}
    Es gelte~(\ref{cond:uneigentlich-fall1}). Wir definieren
    \begin{align*}
        \int_{a}^{\infty} f\of{x} \dif x &\coloneqq \lim_{b\toinf} F\of{b} = \lim_{b\toinf} \int_{a}^{b} f\of{x} \dif x
    \end{align*}
    sofern der Grenzwert existiert. Dann nennen wir dies das uneigentliche Integral von $f$ über $\linterv{a,\infty}$. Wenn der Grenzwert existiert, sagen wir das Integral konvergiert.\\
    Divergiert das Integral und gilt $F\of{b}\toinf$ für $b\toinf$ (oder $F\of{b}\fromto -\infty$ für $b\toinf$), so nennen wir das Integral bestimmt divergent und schreiben
    \begin{align*}
        \int_{a}^{\infty} f\of{x} \dif x = +\infty\quad\text{oder}\quad\int_{a}^{\infty} f\of{x} \dif x = -\infty
    \end{align*}
\end{definition}

\begin{satz} % Satz 2
    \label{satz:int-uneigentlich-epsilon}
    Es gelte~(\ref{cond:uneigentlich-fall1}). Das Integral $ \int_{a}^{\infty} f\of{x} \dif x$ existiert genau dann, wenn
    \begin{align*}
        \forall\varepsilon > 0\ex R\geq a\colon \abs{F\of{b_2} - F\of{b_1}} &= \abs{ \int_{b_1}^{b_2} f\of{x} \dif x} < \varepsilon\quad\forall b_1, b_2 \geq R
    \end{align*}
    \begin{proof}
        Wir wollen die Existenz von $\biglim{b\toinf} F\of{b}$ für $F\of{b} = \int_{a}^{b} f\of{x} \dif x$. Dann folgt der Satz aus dem Cauchy-Kriterium für Grenzwerte.
    \end{proof}
\end{satz}

\begin{definition}[Absolut konvergente uneigentliche Integrale]
    Das Integral $\int_{a}^{\infty} f\of{x} \dif x$ heißt absolut konvergent, falls $\int_{a}^{\infty} \abs{f\of{x}} \dif x$ konvergiert.
\end{definition}


\begin{satz}
    Es gelte~(\ref{cond:uneigentlich-fall1}). Ist das Integral $\int_{a}^{\infty} f\of{x} \dif x$ absolut konvergent, so ist es auch konvergent. Das heißt ist $ \int_{a}^{\infty} \abs{f\of{x}} \dif x < \infty$, so konvergiert auch $ \int_{a}^{\infty} f\of{x} \dif x$.
    \begin{proof}
        Wir setzen $G\of{b} = \int_{a}^{b} \abs{f\of{x}} \dif x$ und $F\of{b} = \int_{a}^{b} f\of{x} \dif x$. Wir nehmen an, dass $\biglim{b\toinf} G\of{b}$ existiert, das heißt
        \begin{align*}
            \forall\varepsilon > 0\ex R\geq a\colon \abs{G\of{b_2} - G\of{b_1}} &< \varepsilon\quad\forall b_1, b_2\geq R\\
            \impl \abs{F\of{b_2} - F\of{b_1}} = \abs{ \int_{b_1}^{b_2} f\of{x} \dif x} &\leq \int_{b_1}^{b_2} \abs{f\of{x}} \dif x = G\of{b_2} - G\of{b_1}
        \end{align*}
        Damit folgt die Behauptung aus Satz~\ref{satz:int-uneigentlich-epsilon}.
    \end{proof}
\end{satz}

\newpage

\begin{satz}[Majorantenkriterium] % Satz 5
    \label{satz:int-majorant}
    Es gelte~(\ref{cond:uneigentlich-fall1}). Sei $\varphi: \linterv{a,\infty}\fromto\linterv{0, \infty}$ mit $\int_{a}^{\infty} \varphi\of{x} \dif x < \infty$ und es existiert ein $R_0 \geq 0$, sodass
    \begin{align*}
        \abs{f\of{x}} &\leq \varphi\of{x}\quad\forall x \geq R_0
    \end{align*}
    Dann ist $\int_{a}^{\infty} f\of{x} \dif x$ absolut konvergent.
    \begin{proof}
        Für $b_2 \geq b_1\geq R_0$ gilt
        \begin{align*}
            \abs{F\of{b_2} - F\of{b_1}} &= \abs{ \int_{b_1}^{b_2} f\of{x} \dif x}\\
            &\leq \int_{b_1}^{b_2} \abs{f\of{x}} \dif x \leq \int_{b_1}^{b_2} \varphi\of{x} \dif x\\
            &\leq \int_{b_1}^{\infty} \varphi\of{x} \dif x\fromto 0 \text{ für } b_1\toinf
        \end{align*}
        Also folgt die Behauptung aus Satz~\ref{satz:int-uneigentlich-epsilon}.
    \end{proof}
\end{satz}

\begin{beispiel}[Konvergenz ohne absolute Konvergenz]
    Das Integral
    \begin{align*}
        \int_{\pi}^{\infty} \frac{\sin x}{x} \dif x&\numbereq{eq:int-frac-sin-x}
        \intertext{ist konvergent, aber nicht absolut konvergent. Um die Konvergenz nachzuweisen, verwenden wir partielle Integration}
        \int_{\pi}^{b} \frac{\sin x}{x} \dif x &= \interv{-\cos\of{x}\cdot\frac{1}{x}}_\pi^b - \int_{\pi}^{b} \frac{\cos x}{x^2} \dif x\\
        &= \frac{\cos\pi}{\pi} - \frac{\cos b}{b} - \int_{\pi}^{b} \frac{\cos x}{x^2} \dif x
        \intertext{Wir wollen Satz~\ref{satz:int-majorant} anwenden und definieren $\varphi\of{x} = \frac{1}{x^2}$ mit}
        \int_{\pi}^{b} \frac{1}{x^2} \dif x &= \interv{-\frac{1}{x}}_{\pi}^b = \frac{1}{\pi} - \frac{1}{b}\fromto \frac{1}{\pi}
        \intertext{Außerdem gilt}
        \abs{\frac{\cos x}{x^2}} &\leq \frac{1}{x^2}
        \intertext{Damit ist (\ref{eq:int-frac-sin-x}) nach dem Majorantenkriterium konvergent. Um einzusehen, dass es nicht absolut konvergent ist, betrachten wir für $N\in\N$}
        \int_{N\pi}^{\pair{N+1}\pi} \abs{\frac{\sin x}{x}} \dif x &= \int_{N\pi}^{\pair{N+1}\pi} \frac{\abs{\sin x}}{x} \dif x\\
        &\geq \frac{1}{\pi\pair{N+1}} \cdot \int_{N\pi}^{\pair{N+1}\pi} \abs{\sin x} \dif x = \frac{2}{\pi\pair{N+1}}\\
        \impl \int_{\pi}^{\pair{N+1}\pi} \abs{\frac{\sin x}{x}} \dif x &= \sum_{k=1}^{N} \int_{k\pi}^{\pair{k+1}\pi} \frac{\abs{\sin x}}{x} \dif x\\
        &\geq \sum_{k=1}^{N} \frac{2}{\pi\pair{k+1}} = \frac{2}{\pi} \sum_{n=0}^{N} \frac{1}{k+1}\fromto \infty \text{ für } N\toinf
    \end{align*}
\end{beispiel}

\begin{bemerkung}
    Es gelte~(\ref{cond:uneigentlich-fall1}). Dann wollen wir analog zu $\linterv{a, \infty}$ auch die Integrale in $\rinterv{-\infty, b}$ betrachten. Wir definieren
    \begin{align*}
        \int_{-\infty}^{b} f\of{x} \dif x &\coloneqq \lim_{a\fromto -\infty} \int_{a}^{b} f\of{x} \dif x
    \end{align*}
    sofern der Grenzwert existiert. Alle Aussagen für $\linterv{a, \infty}$ gelten analog auch für $\rinterv{-\infty, b}$.
\end{bemerkung}

\begin{definition} % Definition 6
    Sei $f: \pair{-\infty, \infty}\fromto \R$ und $f\in\mR\of{\interv{a,b}}~\forall a,b\in\R$. Dann nehmen wir $c\in\R$ beliebig und definieren, dass
    \begin{align*}
        \int_{-\infty}^{\infty} f\of{x} \dif x&
        \intertext{konvergiert, falls}
        \int_{-\infty}^{c} f\of{x} \dif x& \text{ und } \int_{c}^{\infty} f\of{x} \dif x
        \intertext{beide konvergieren. Und setzen}
        \int_{-\infty}^{\infty} f\of{x} \dif x &\coloneqq \int_{-\infty}^{c} f\of{x} \dif x + \int_{c}^{\infty} f\of{x} \dif x
    \end{align*}
\end{definition}

\begin{uebung}
    Weisen Sie nach, dass sowohl die Konvergenz, als auch der Wert des Integrals in der vorherigen Definition unabhängig von der Wahl von $c$ ist.
\end{uebung}

\begin{bemerkung}
    Es ist allerdings zu beachten, dass im Allgemeinen
    \begin{align*}
        \lim_{a\toinf} \int_{a}^{c} f\of{x} \dif x + \lim_{b\toinf} \int_{c}^{b}  \dif x &\neq \lim_{R\toinf} \int_{-R}^{R} f\of{x} \dif x
    \end{align*}
    Das heißt die Integrale müssen tatsächlich getrennt betrachtet werden. Zum Beispiel bei der Funktion $f\of{x} = x$ gilt $\displaystyle\lim_{R\toinf} \int_{-R}^{R} x \dif x = 0$, aber diese ist eigentlich nicht auf $\pair{-\infty, \infty}$ integrierbar, da sich bei der Trennung in zwei Integrale kein Grenzwert ergibt.
\end{bemerkung}

\subsection{Uneigentliche Integrale - Fall II}

\begin{mdframed}
    \condition{cond:uneigentlich-fall2}: In diesem Teilkapitel sei $I=\linterv{a, b}$ (oder $I=\rinterv{a, b}$) und $f:I\fromto\R$ hat Singularität bei $x=a$ (oder $x=b$). Außerdem $f\in\mR\of{\interv{a,c}}~\forall a<c < b$ (oder $f\in\mR\of{\interv{c, b}}~\forall a < c < b$).
\end{mdframed}

\begin{definition}
    Existiert
    \begin{align*}
        \lim_{c\fromto b-} \int_{a}^{c} f\of{x} \dif x\quad &\pair{\text{oder } \lim_{c\fromto a+} \int_{c}^{b} f\of{x} \dif x}
        \intertext{so setzen wir}
        \int_{a}^{b} f\of{x} \dif x = \lim_{c\fromto b-} \int_{a}^{c} f\of{x} \dif x\quad &\pair{\text{oder } \int_{a}^{b} f\of{x} \dif x = \lim_{c\fromto a+} \int_{c}^{b} f\of{x} \dif x}
    \end{align*}
    und sagen $\displaystyle\int_{a}^{b} f\of{x} \dif x$ konvergiert.
\end{definition}

\begin{satz}
    Ist $\abs{f\of{x}} \leq \varphi\of{x}~\forall x\in\linterv{a,b}$ (oder $\forall x\in\rinterv{a,b}$) und konvergiert $ \int_{a}^{b} \varphi\of{x} \dif x$, so konvergiert auch $\int_{a}^{b} f\of{x} \dif x$.
\end{satz}

\begin{beispiel}
    Sei $f: \rinterv{0, 1}\fromto\R,~x\mapsto \frac{1}{\sqrt{x}}$. Dann gilt $F\of{x}  = 2\sqrt{x}$
    \begin{align*}
        \int_{0}^{1} \frac{1}{\sqrt{x}} \dif x &= \lim_{c\to 0+}\int_{c}^{1} \frac{1}{\sqrt{x}} \dif x = \lim_{c\to 0+}\interv{2\sqrt{x}}_c^1 = \lim_{c\to 0+}\pair{2-2\sqrt{c}} = 2\numbereq{eq:int-0-1-frac-sqrt-x}
    \end{align*}
\end{beispiel}

\subsection{Uneigentliche Integrale - Fall III}
\marginnote{[14. Mai]}
Im vorherigen Kapitel haben wir den Fall betrachtet, dass $f$ eine Singularität auf einer der Intervallgrenzen hat. Jetzt wollen wir das ergänzen, indem wir eine Singularität an einer beliebigen Stelle $\xi$ im Inneren von $\interv{a,b}$ zulassen.

\begin{definition}
    Wir sagen, dass
    \begin{align*}
        \int_{a}^{b} f\of{x} \dif x
        \intertext{existiert/konvergiert, falls die uneigentlichen Integrale}
        \int_{\xi}^{b} f\of{x} \dif x &\text{ und } \int_{a}^{\xi} f\of{x} \dif x
        \intertext{konvergieren. Wir setzen}
        \int_{a}^{b} f\of{x} \dif x &\coloneqq \int_{a}^{\xi} f\of{x} \dif x + \int_{\xi}^{b} f\of{x} \dif x\numbereq{eq:un-iii}
    \end{align*}
\end{definition}

\begin{beispiel}
    Wir betrachten die Funktion $f\of{x} = \frac{1}{\sqrt{\abs{x}}}$ mit der Singularität bei $\xi = 0$. Dann gilt
    \begin{align*}
        \int_{-1}^{1} \frac{1}{\sqrt{\abs{x}}} \dif x &= \int_{0}^{1} \frac{1}{\sqrt{\abs{x}}} \dif x + \int_{-1}^{0} \frac{1}{\sqrt{\abs{x}}} \dif x = 2\int_{0}^{1} \frac{1}{\sqrt{x}} \dif x \annot{=}{(\ref{eq:int-0-1-frac-sqrt-x})} 4
    \end{align*}
\end{beispiel}

\begin{bemerkung}
    Die Existenz von (\ref{eq:un-iii}) ist echt stärker als die Existenz des \emph{Cauchyschen Hauptwerts}
    \begin{align*}
        \lim_{\varepsilon\searrow 0} \int_{I_{\varepsilon}}^{} f\of{x} \dif x
    \end{align*}
    mit $I=\interv{a,b}$ und $I_{\varepsilon}\coloneqq I\exclude\pair{\xi-\varepsilon, \xi+\varepsilon} = \interv{a, \xi-\varepsilon} \cup \interv{\xi+\varepsilon, b}$.
\end{bemerkung}

\begin{beispiel}[Cauchyscher Hauptwert bei Nichtexistenz des Integrals]
    Sei $f\of{x} = \frac{1}{x^2}$, $I=\interv{-1, 1}$. Dann existiert der Cauchysche Hauptwert, aber nicht (\ref{eq:un-iii}).
\end{beispiel}

\subsection{Uneigentliche Integrale - Fall IV}

\begin{definition}
    Sei $f$ eine Funktion mit Singularitäten in $\R$, deren Integral wir auf einem Intervall $\interv{a,b}$ (oder $\linterv{a, \infty}$, $\rinterv{-\infty, b}$, $\pair{-\infty, \infty}$) berechnen wollen. Dann zerlegen wir das Intervall in endlich viele Intervalle, wobei die Singularitäten (oder $-\infty, \infty$) die Randpunkte sind. Wir sagen, dass das gesamte Integral existiert, falls die endlich vielen uneigentlichen Integrale existieren und setzen den Wert auf die Summe all dieser uneigentlichen Integrale.
\end{definition}

\begin{satz}[Integralvergleichskriterium] % Satz 10
    \label{satz:integral-vergleich}
    Sei $f: \linterv{1, \infty} \fromto\linterv{0, \infty}$ monoton fallend. Dann gilt
    \begin{align*}
        \sum_{n=1}^{\infty}  f\of{n} \text{ konvergiert } \equivalent \int_{1}^{\infty} f\of{x} \dif x \text{ existiert }
    \end{align*}
    \begin{proof}
        Siehe Saalübung.
    \end{proof}
\end{satz}

\begin{beispiel}
    Es sei $f\of{x} = x^{-p}$ mit $p\neq 1$. Dann ist $F\of{x} = \frac{1}{1-p}x^{1-p}$ für $F'=f$. Es gilt
    \begin{align*}
        \int_{1}^{\infty} \frac{1}{x^p} \dif x &= \lim_{R\toinf} \interv{\frac{1}{1-p}x^{1-p}}_1^R
        \intertext{Dieses Integral existiert für $p > 1$. Das heißt nach Satz~\ref{satz:integral-vergleich} gilt}
        \sum_{n=1}^{\infty} \frac{1}{n^p} \text{ konvergiert } &\equivalent p > 1
    \end{align*}
\end{beispiel}

\subsection{Die Gamma-Funktion}

Als ein Beispiel für ein uneigentliches Integral betrachten wir in diesem Teilkapitel noch die Gamma-Funktion, die über ein uneigentliches Integral definiert ist. Wir werden zunächst nur ein paar Eigenschaften beweisen und später in Kapitel~\ref{sec:gamma-funktion} noch genauer auf diese Funktion eingehen.

\begin{definition}[Gamma-Funktion]
    Wir definieren
    \begin{align*}
        \Gamma\of{x} &\coloneqq \int_{0}^{\infty} t^{x-1} e^{-t} \dif t\tag{$x > 0$}
    \end{align*}
    Warum ist das wohldefiniert? Also warum existiert das Integral für alle $x > 0$? Wir teilen das Integral in zwei Teile und schätzen ab mit $t^{x-1}e^{-t} \leq t^{x-1}~\forall t \geq 0$
    \begin{align*}
        \int_{0}^{1} t^{x-1}e^{-t} \dif t &\leq \int_{0}^{1} t^{x-1} \dif t = \lim_{c\to 0+} \interv{\frac{1}{x}t^{x}}_c^1\\
        &= \lim_{c\to 0+} \frac{1}{x}\pair{1-c^x} = \frac{1}{x}
        \intertext{Damit existiert der erste Teil. Wir schätzen den Rest ab. Es gilt für $t\geq 1$}
        t^{x-1}e^{-t} &= t^{x-1}e^{-\frac{t}{2}}e^{-\frac{t}{2}} \leq C_x e^{-\frac{t}{2}}\tag{$C_x\coloneqq \sup_{t\geq 1} t^{x-1}e^{-\frac{t}{2}} < \infty$}\\
        \impl 0 \leq \int_{1}^{\infty} t^{x-1}e^{-t} \dif t &= \lim_{b\to\infty} \int_{1}^{b} t^{x-1}e^{-t} \dif t \leq \lim_{b\fromto\infty} C_x \int_{1}^{b} e^{-\frac{t}{2}} \dif t\\
        &= C_x\lim_{b\fromto \infty}\interv{-2e^{-\frac{t}{2}}}_1^b = C_x\cdot\pair{2e^{-\frac{1}{2}}} < \infty
    \end{align*}
\end{definition}

\begin{satz}[Funktionalgleichung der $\Gamma$-Funktion] % Satz 12
    \label{satz:gamma-funktion}
    Es gilt $\Gamma\of{n+1} = n!$ für alle $n\in\N$ und $x\Gamma\of{x} = \Gamma\of{x+1}$ für alle $x>0$.
    \begin{proof}
        \begin{align*}
            \Gamma\of{x+1} &= \int_{0}^{\infty} t^{(x+1)-1}e^{-t} \dif t = \int_{0}^{\infty} t^{x}e^{-t} \dif t\\
            \intertext{Wir integrieren partiell. Sei $0 < a < b < \infty$}
            \int_{a}^{b} t^{x}e^{-t} \dif t &= \interv{-t^x e^{-t}}_a^b + \int_{a}^{b} xt^{x-1}e^{-t} \dif t\\
            &= a^{x}e^{-a} - b^{x}e^{-b} + x \int_{a}^{b} t^{x-1}e^{-t} \dif t\\
            \impl \int_{a}^{\infty} t^x e^{-t} \dif t &= \lim_{b\toinf} \int_{a}^{b} t^x e^{-t} \dif t = a^x e^{-a} + x \int_{a}^{\infty} t^{x-1} e^{-t} \dif t\\
            \impl \Gamma\of{x+1} &= \int_{0}^{\infty} t^x e^{-t} \dif t = x\Gamma\of{x}
            \intertext{Damit folgt die zweite Behauptung. Wir betrachten außerdem}
            \Gamma\of{n+1} &= n\Gamma\of{n} = n\Gamma\of{n-1+1}\\
            &= n\pair{n-1}\Gamma\of{n-1} = n\cdot \pair{n-1}\cdot\ldots\cdot 2 \cdot 1 \cdot \Gamma\of{1}\\
            &= n!\qedhere
        \end{align*}
    \end{proof}
\end{satz}

\begin{anwendung}
    Wir können auch andere Integrale auf die Gamma-Funktion zurückführen. Nach Substitution mit $t^2 = x$ gilt
    \begin{align*}
        \int_{0}^{\infty} e^{-t^2} \dif t &= \int_{0}^{\infty} e^{-x}\cdot\frac{1}{2\sqrt{x}} \dif x = \frac{1}{2} \int_{0}^{\infty} x^{-\frac{1}{2}}\cdot e^{-x} \dif x\\
        &= \frac{1}{2} \int_{0}^{\infty} t^{\frac{1}{2}-1}\cdot e^{-t} \dif t = \frac{1}{2}\Gamma\of{\frac{1}{2}}\\
        \impl 2 \int_{0}^{\infty} e^{-t^2} \dif t &= \Gamma\of{\frac{1}{2}}
        \intertext{Wie wir später in Satz~\ref{satz:gamma-einhalb} sehen werden, gilt damit}
        \int_{0}^{\infty} e^{-t^2} \dif t &= \frac{\sqrt{\pi}}{2}
    \end{align*}
\end{anwendung}

\newpage
