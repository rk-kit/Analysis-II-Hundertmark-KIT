\section{Der Hauptsatz der Differential- und Integralrechnung}
\thispagestyle{pagenumberonly}

Wir haben unseren Integralbegriff bisher über Zerlegungen definiert. Allerdings ist das für das konkrete Berechnen von Integralen nicht wirklich praktikabel. Wenn wir also ein Intervall $I=\interv{a,b}$ und eine Funktion $f\in\mC\of{I}$ betrachten, wie rechnet man das Integral dann konkret aus?

\subsection{Hauptsatz der Integralrechnung}

\begin{definition}[Erinnerung: Stammfunktion]
    Wir nennen $F$ eine Stammfunkton von $f$, falls $F$ differenzierbar ist und $F'=f$.
\end{definition}

\begin{satz}[Hauptsatz der Differential- und Integralrechnung] % Satz 1
    \label{satz:hauptsatz-dif-int}
    Sei $f\in\mC\of{I}$. Dann ist für jedes $c\in\interv{a,b}$ die Funktion
    \begin{align*}
        F\of{x} \coloneqq \int_{c}^{x} f\of{t} \dif t\tag{$x\in I$}
    \end{align*}
    stetig differenzierbar und $F' = f$. Das heißt $F'\of{x} = f\of{x}~\fa x\in I$.

    \begin{proof}
        \marginnote{[3. Mai]}
        Sei $F(x) = \int_{c}^{x} f\of{t} \dif t$ und $h \neq 0$. Wir wollen über den Differenzenquotient zeigen, dass $F' = f$. Wir berechnen zuerst den Zähler
        \begin{align*}
            F(x+h) - F(x) &= \int_{c}^{x+h} f\of{t} \dif t - \int_{c}^{x} f\of{t} \dif t = \int_{x}^{x+h} f\of{t} \dif t
            \intertext{Das können wir in den Differenzenquotienten einsetzen}
            \frac{F\of{x+h} - F\of{x}}{h} &= \frac{1}{h} \int_{x}^{x+h} f\of{t} \dif t\\
            \impl \frac{F\of{x+h} - F\of{x}}{h} - f\of{x} &= \frac{1}{h} \int_{x}^{x+h} f\of{t} \dif t - f\of{x}\\
            &= \frac{1}{h} \int_{x}^{x+h} f\of{t} \dif t - \frac{1}{h} \int_{x}^{x+h} f\of{x} \dif t\\
            &= \frac{1}{h} \int_{x}^{x+h} \pair{f\of{t} - f\of{x}} \dif t
            \intertext{Wir definieren $I_h\of{x} = \interv{x, x+h}$, falls $h> 0$ und $I_h\of{x} = \interv{x+h, x}$, falls $h < 0$. Damit können wir das Integral gegen seinen betragsmäßig größten Wert abschätzen}
            \impl \abs{\frac{F\of{x+h} - F\of{x}}{h} - f\of{x}} &\leq \frac{1}{\abs{h}} \cdot \sup_{t\in I_h\of{x}} \abs{f\of{t} - f\of{x}} \cdot \abs{h}\\
            &\leq \sup_{t\in I_h\of{x}} \abs{f\of{t} - f\of{x}}
        \end{align*}
        Da $f$ stetig in $x$ ist, folgt
        \begin{align*}
            \sup_{t\in I_h\of{x}} \abs{f\of{t} - f\of{x}} &\fromto 0 \text{ für } h\fromto 0\\
            \impl \lim_{h\fromto 0} \abs{\frac{F\of{x+h} - F\of{x}}{h} - f\of{x}} &= 0\\
            \equivalent \lim_{h\fromto 0} \frac{F\of{x+h} - F\of{x}}{h} &= f\of{x}\qedhere
        \end{align*}
    \end{proof}
\end{satz}

\begin{korollar}
    Sei $G\in \mC^{1}\of{I}$ (die Klasse der stetig differenzierbaren Funktionen auf $I$) eine Stammfunktion von $f\in\mC\of{I}$. Dann gilt
    \begin{align*}
        \int_{a}^{b} f\of{x} \dif x = G\of{b} - G\of{a} \eqqcolon G\vert_a^b \eqqcolon \interv{G}_a^b \eqqcolon \interv{G\of{x}}_{x=a}^{x=b}
    \end{align*}
    \begin{proof}
        Wir setzen $c = a$ in Satz~\ref{satz:hauptsatz-dif-int} und $F: I\fromto\R~x\mapsto F\of{x} = \int_{a}^{x} f\of{t} \dif t$. Dann ist $F$ nach Satz~\ref{satz:hauptsatz-dif-int} eine Stammfunktion von $f$ auf $I$. Sei $G\in\mC^1\of{I}$ eine beliebige Stammfunktion von $f$. Wir definieren für $t\in I$
        \begin{align*}
            h\of{t} &\coloneqq F\of{t} - G\of{t}\\
            \impl h'\of{t} &= F'\of{t} - G'\of{t} = f\of{t}-f\of{t} = 0
            \intertext{Damit ist $h$ konstant, d.h. $h\of{x} = k$ für alle $x\in I$}
            \impl k &= F\of{x} - G\of{x}\quad\forall x\in I\\
            \impl k &= F\of{a} - G\of{a} = -G\of{a}\\
            \impl F\of{x} - G\of{x} &= -G\of{a}\\
            F\of{x} &= G\of{x} - G\of{a}\\
            \impl F\of{b} &= G\of{b} - G\of{a}\qedhere
        \end{align*}
    \end{proof}
\end{korollar}
\noindent Damit sind wir nun in der Lage Integrale für bestimmte Arten von Funktionen konkret anzugeben.

\begin{beispiel}
    Sei $p\in\N$ und $f\of{x} = x^{p}$, $x\in\R$. Dann hat $f$ die Stammfunktion $F\of{x} = \frac{1}{p+1}\cdot x^{p+1}$. Damit folgt
    \begin{align*}
        \int_{a}^{b} x^p \dif x &= \frac{1}{p+1} \cdot \interv{b^{p+1} - a^{p+1}}\quad\forall a, b\in\R
    \end{align*}
\end{beispiel}

\begin{beispiel}
    Sei $p\in\N$, $p\geq 2$ und $f\of{x} = x^{-p}$, $x\neq 0$. Dann ist die Stammfunktion $F\of{x} = \frac{1}{1-p}\cdot x^{1-p}$. Damit folgt
    \begin{align*}
        \int_{a}^{b} x^{-p} \dif x &= \frac{1}{1-p}\cdot\interv{b^{1-p} - a^{1-p}}\quad \fa a,b < 0 \text{ oder } a,b > 0
    \end{align*}
\end{beispiel}

\begin{beispiel}
    Sei $\alpha\in\R\exclude\set{-1}$, $f\of{x} = x^{\alpha} = e^{\alpha\cdot\ln\of{x}}$, $x > 0$. Dann ist die Stammfunktion $F\of{x} = \frac{1}{\alpha + 1}\cdot x^{\alpha + 1}$. Damit gilt
    \begin{align*}
        \int_{a}^{b} x^{\alpha} \dif x &= \frac{1}{\alpha +1}\cdot\interv{b^{\alpha+1} - a^{\alpha+1}}\quad\forall a,b > 0
    \end{align*}
\end{beispiel}

\begin{beispiel}
    Sei $f\of{x} = \frac{1}{x}$, $x\neq 0$. Dann ist die Stammfunktion $F\of{x} = \ln \abs{x}$.
    \begin{proof}
        Falls $x > 0$. Dann ist $F\of{x} = \ln x$ und $F'\of{x} = \frac{1}{x}$.\\
        Falls $x < 0$. Dann ist $F\of{x} = \ln -x$ und $F'\of{x} = \frac{1}{-x}\cdot\pair{-1} = \frac{1}{x}$.
    \end{proof}
    \noindent Damit gilt
    \begin{align*}
        \int_{a}^{b} \frac{1}{x} \dif x &= \ln\abs{b} - \ln\abs{a} = \ln\abs{\frac{b}{a}}\quad\forall a,b < 0 \text{ oder } a,b > 0
    \end{align*}
\end{beispiel}

\begin{beispiel}
    Es gilt $\pair{\sin x}' = \cos x$ und $\pair{\cos x}' = -\sin x$. Damit gilt
    \begin{align*}
        \int_{a}^{b} \cos x \dif x = \interv{\sin x}_{a}^b &= \sin b - \sin a\\
        \int_{a}^{b} \sin x \dif x = \interv{-\cos x}_a^b &= -\cos b + \cos a\\
    \end{align*}
\end{beispiel}

\begin{beispiel}
    Es gilt $\tan x = \frac{\sin x}{\cos x}$ für $\abs{x} < \frac{\pi}{2}$. Damit folgt $\pair{\tan x}' = \frac{1}{\cos^2 x}$. Das heißt
    \begin{align*}
        \int_{0}^{\varphi} \frac{1}{\cos^2 x} \dif x &= \interv{\tan x}_0^{\varphi} = \tan\of{\varphi}\quad\forall\abs{\varphi} < \frac{\pi}{2}
    \end{align*}
\end{beispiel}

\begin{beispiel}[Fläche des Einheitskreises]
    Die Funktion $\sqrt{1-x^2}$ beschreibt die obere Hälfte des Einheitskreises. Das heißt wir wollen das Integral $ \int_{a}^{b} \sqrt{1-x^2} \dif x$ berechnen. Wie wir später in Beispiel~\ref{beispiel:stammfunktion-sqrt-1-xsqrt} sehen werden, hat $\sqrt{1-x^2}$ die Stammfunktion
    \begin{align*}
        \phi\of{x} &= \frac{1}{2}\pair{\arcsin x + x\cdot\sqrt{1-x^2}}
        \intertext{Das lässt sich durch Ableiten mit $\pair{\arcsin\of{x}}' = \frac{1}{\sqrt{1-x^2}}$ prüfen}
        \phi'\of{x} &= \frac{1}{2}\pair{\frac{1}{\sqrt{1-x^2}} + \sqrt{1-x^2} + x\cdot \frac{1}{2\sqrt{1-x^2}}\cdot \pair{-2x}}\\
        &= \frac{1}{2}\pair{\frac{1}{\sqrt{1-x^2}} + \sqrt{1-x^2} - \frac{x^2}{\sqrt{1-x^2}}}\\
        &= \frac{1}{2}\pair{\frac{1-x^2}{\sqrt{1-x^2}} + \sqrt{1-x^2}}\\
        &= \frac{1}{2}\pair{\sqrt{1-x^2} + \sqrt{1-x^2}} = \sqrt{1-x^2}\\
        \impl \int_{a}^{b} \sqrt{1-x^2} \dif x &= \interv{\frac{1}{2}\pair{\arcsin x + x\cdot\sqrt{1-x^2}}}_a^b\tag{$-1\leq a,b\leq 1$}
        \intertext{Geometrisch gesehen können wir damit nun die Fläche der oberen Hälfte des Einheitskreises berechnen}
        \int_{-1}^{1} \sqrt{1-x^2} \dif x &= \frac{1}{2} \cdot\pair{\arcsin\of{1} + 0 - \arcsin\of{-1} -0} = \arcsin\of{1} = \frac{\pi}{2}
    \end{align*}
\end{beispiel}

\begin{bemerkung}
    Satz~\ref{satz:hauptsatz-dif-int} gilt auch für Funktionen in $\C$ oder $\R^d$ bwz. $\C^d$.
\end{bemerkung}

\begin{notation}[Stammfunktion]
    Wir schreiben
    \begin{align*}
        &\int_{}^{} f\of{x} \dif x
        \intertext{für die Gesamtheit aller Stammfunktionen zu $f$ oder das unbestimmte Integral. Genauer gilt, wenn $\Phi$ eine Stammfunktion von $f$ ist}
        \int_{}^{} f\of{x} \dif x &= \set{\Phi + k: k \text{ Konstante} }
    \end{align*}
    Wir schreiben manchmal statt einer Menge auch nur eine konkrete Stammfunktion und die mögliche Addition einer Konstante ist dann implizit enthalten.
\end{notation}

\newpage

\subsection{Integrationstechniken - Partielle Integration}
\begin{satz}[Partielle Integration] % Satz 3
    \label{satz:partielle-integration}
    Seien $f, g\in \mC^1\of{I, \K}$ ($\K\in\set{\R, \C}$) für ein Intervall $I$. Dann gilt für $a,b\in I$
    \begin{align*}
        \int_{a}^{b} f'\of{x}g\of{x} \dif x &= \interv{f\of{x}g\of{x}}_a^b - \int_{a}^{b} f\of{x}g'\of{x} \dif x\\
        &= f\of{b}g\of{b} - f\of{a}g\of{a} - \int_{a}^{b} f\of{x}g'\of{x} \dif x\numberthis\label{eq:partial-int}
    \end{align*}
    \begin{proof}
        Wir wenden die Produktregel der Ableitung an. Es gilt
        \begin{align*}
            \pair{f\of{x}g\of{x}}' &= f'\of{x}g\of{x} + f\of{x}g'\of{x}\\
            \impl \int_{a}^{b} \pair{f\of{x}g\of{x}}' \dif x &= \int_{a}^{b} f'\of{x}g\of{x} \dif x + \int_{a}^{b} f\of{x}g'\of{x} \dif x\tag{i}
            \intertext{Außerdem gilt nach Satz~\ref{satz:hauptsatz-dif-int}}
            \int_{a}^{b} \pair{f\of{x}g\of{x}}' \dif x &= \interv{f\of{x}g\of{x}}_a^b = \interv{f\of{x}g\of{x}}_{a}^b\\
            &= f\of{b}g\of{b} - f\of{a}g\of{a}\tag{ii}
            \intertext{Wir setzen (i) und (ii) gleich}
            \impl \int_{a}^{b} f'\of{x}g\of{x} \dif x + &\int_{a}^{b} f\of{x}g'\of{x} \dif x = f\of{b}g\of{b} - f\of{a}g\of{a}\qedhere
        \end{align*}
    \end{proof}
\end{satz}

\begin{bemerkung}[Stammfunktion mittels partieller Integration]
    Wenn wir kein konkretes Integral ausrechnen wollen, sondern lediglich eine beliebige Stammfunktion für eine Funktion ermitteln wollen, lässt sich Satz~\ref{satz:partielle-integration} auch umformen zu
    \begin{align*}
        \int_{}^{} f'\of{x}g\of{x} \dif x &= f\of{x}g\of{x} - \int_{}^{} f\of{x}g'\of{x} \dif x
    \end{align*}
    weil wir konstante Terme für die Stammfunktion vernachlässigen können und so der zweite Term aus (\ref{eq:partial-int}) wegfällt.
\end{bemerkung}

\begin{beispiel}[Anwendung von partieller Integration]
    Wir bestimmen eine Stammfunktion von $\ln x$ mittels partieller Integration
    \begin{align*}
        \int_{}^{} \ln x \dif x &= \int_{}^{} 1\cdot\ln x \dif x = x\cdot\ln x - \int_{}^{} x\cdot \frac{1}{x} \dif x = x\cdot\ln x - x
    \end{align*}
\end{beispiel}

\begin{beispiel}
    Sei $-1\neq\alpha\in\R$. Dann gilt
    \begin{align*}
        \int_{}^{} x^{\alpha}\cdot \ln\of{x} \dif x &= \frac{1}{\alpha+1}x^{\alpha+1}\cdot \ln\of{x} - \int_{}^{} \frac{1}{\alpha+1}x^{\alpha+1}\cdot\frac{1}{x} \dif x\\
        &= \frac{1}{\alpha+1}x^{\alpha}\cdot\ln\of{x} - \frac{1}{\pair{\alpha+1}^2}x^{\alpha+1}
    \end{align*}
\end{beispiel}

\begin{beispiel}[Integrationstrick bei partieller Integration]
    \label{beispiel:stammfunktion-sqrt-1-xsqrt}
    Bei partieller Integration bleibt immer noch ein Integralteil übrig. Wenn dieser wieder dem ursprünglichen Integral entspricht, ist es möglich, die Gleichung danach aufzulösen und so das Integral zu bestimmen. Wir bestimmen auf diese Weise eine Stammfunktion für $\sqrt{1-x^2}$. \newpage
    \begin{align*}
        \int_{}^{} \sqrt{1-x^2} \dif x &= \int_{}^{} 1\cdot\sqrt{1-x^2} \dif x\\
        &= x\cdot\sqrt{1-x^2} - \int_{}^{} x\cdot\frac{1}{\sqrt{1-x^2}}\cdot\pair{-2x} \dif x\\
        &= x\cdot\sqrt{1-x^2} + \int_{}^{} \frac{x^2}{\sqrt{1-x^2}} \dif x\\
        &= x\cdot\sqrt{1-x^2} + \int_{}^{} \frac{1}{\sqrt{1-x^2}} \dif x - \int_{}^{} \frac{1-x^2}{\sqrt{1-x^2}} \dif x\\
        &= x\cdot\sqrt{1-x^2} + \arcsin\of{x} - \int_{}^{} \sqrt{1-x^2} \dif x\\
        \impl 2 \int_{}^{} \sqrt{1-x^2} \dif x &= x\cdot\sqrt{1-x^2} + \arcsin x\\
        \impl \int_{}^{} \sqrt{1-x^2} \dif x &= \frac{1}{2}\pair{x\cdot\sqrt{1-x^2} + \arcsin x}
    \end{align*}
\end{beispiel}

\begin{uebung}
    Beweisen Sie analog zum vorherigen Beispiel mittels partieller Integration, dass $ \int_{}^{} \sqrt{1+x^2} \dif x = \frac{1}{2}\pair{x\cdot\sqrt{1+x^2} + \arcsinh x}$ und $ \int_{}^{} \sqrt{x^2-1} \dif x = \frac{1}{2}\pair{x\cdot\sqrt{x^2-1} - \arccosh x}$.
\end{uebung}

\begin{beispiel}
    Seien $a,b\in\R$
    \begin{align*}
        \int_{}^{} \underset{\scriptscriptstyle g\of{x}}{\vphantom{\frac{i}{i}}e^{ax}}\cdot \underset{\scriptscriptstyle f\of{x}}{\vphantom{\frac{i}{i}}\sin\of{bx}} \dif x &= e^{ax}\cdot\pair{-\frac{1}{b}\cos\of{bx}} + \int_{}^{} \frac{a}{b} e^{ax} \cos\of{bx} \dif x\\
        &= -\frac{1}{b}e^{ax} \cos\of{bx} + \frac{a}{b}\int_{}^{} e^{ax} \cos\of{bx} \dif x\\
        \intertext{Wir wenden nochmal partielle Integration an und erhalten}
        &= -\frac{1}{b}e^{ax} \cos\of{bx} + \frac{a}{b}\cdot\pair{\frac{1}{b} e^{ax} \sin\of{bx} - \frac{a}{b}\int_{}^{} e^{ax}\sin\of{bx} \dif x}\\
        \impl \pair{1+\frac{a^2}{b^2}} \int_{}^{} e^{ax}\sin\of{bx} \dif x &= -\frac{1}{b} e^{ax}\cos\of{bx} + \frac{a}{b^2}e^{ax}\sin\of{bx}\\
        \impl \int_{}^{} e^{ax}\sin\of{bx} \dif x &= \frac{1}{a^2+b^2}\pair{e^{ax}\pair{a\sin\of{bx} - b\cos\of{bx}}}
    \end{align*}
\end{beispiel}

\begin{beispiel}
    \marginnote{[07. Mai]}
    Es gilt
    \begin{align*}
        \int_{0}^{\frac{\pi}{2}} \sin^2\of{x} \dif x &= \int_{0}^{\frac{\pi}{2}} \cos^2\of{x} \dif x = \frac{\pi}{4}\\
    \end{align*}
    \begin{proof}
        \begin{align*}
            \int_{0}^{\frac{\pi}{2}} \sin^2\of{x} \dif x &= \int_{0}^{\frac{\pi}{2}} \sin\of{x}\sin\of{x} \dif x\\
            &= \interv{-\cos\of{x}\sin\of{x}}_{0}^{\frac{\pi}{2}} + \int_{0}^{\frac{\pi}{2}} \cos\of{x}\cos\of{x} \dif x\\
            &= 0 - 0 + \int_{0}^{\frac{\pi}{2}} \cos^2\of{x} \dif x
            \intertext{Mit dem trigonometrischen Pythagoras wissen wir außerdem, dass}
            \frac{\pi}{2} &= \int_{0}^{\frac{\pi}{2}} 1 \dif x = \int_{0}^{\frac{\pi}{2}} \pair{\cos^2\of{x} + \sin^2\of{x}} \dif x\\
            &= \int_{0}^{\frac{\pi}{2}} \cos^2\of{x} \dif x + \int_{0}^{\frac{\pi}{2}} \sin^2\of{x} \dif x = 2 \int_{0}^{\frac{\pi}{2}} \cos^2\of{x} \dif x\\
            \impl\frac{\pi}{4} &= \int_{0}^{\frac{\pi}{2}} \cos^2\of{x} \dif x\qedhere
            \intertext{Analog lässt sich zeigen, dass}
            \pi &= \int_{0}^{2\pi} \cos^2\of{x} \dif x = \int_{0}^{2\pi} \sin^2\of{x} \dif x
        \end{align*}
    \end{proof}
\end{beispiel}

\begin{beispiel}
    Sei $n\in\N$ mit $n\geq 2$
    \begin{align*}
        \int_{}^{} \cos^{n}\of{x} \dif x &= \int_{}^{} \cos\of{x}\cos^{n-1}\of{x} \dif x\\
        &= \sin\of{x}\cos^{n-1}\of{x} + \int_{}^{} \sin\of{x}\cdot\pair{n-1}\cos^{n-2}\of{x}\sin\of{x} \dif x\\
        &= \sin\of{x}\cos^{n-1}\of{x} + \pair{n-1} \int_{}^{} \underbrace{\sin^2\of{x}}_{= 1- \cos^2\of{x}}\cos^{n-2}\of{x} \dif x\\
        &= \sin\of{x}\cos^{n-1}\of{x} + \pair{n-1} \int_{}^{} \cos^{n-2}\of{x} \dif x - \pair{n-1} \int_{}^{} \cos^{n}\of{x}\dif x\\
        \impl \int_{}^{} \cos^{n}\of{x} \dif x &= \frac{1}{n} \sin\of{x}\cos^{n-1}\of{x} + \frac{n-1}{n} \int_{}^{} \cos^{n-2}\of{x}\dif x\tag{Rekursionsformel}
        \intertext{Analog lässt sich zeigen, dass}
        \int_{}^{} \sin^{n}\of{x} \dif x &= -\frac{1}{n}\cos\of{x}\sin^{n-1}\of{x} + \frac{n-1}{n} \int_{}^{} \sin^{n-2}\of{x} \dif x
        \intertext{Wir nutzen nun die Rekursionsformel, um einen Wert für das Intervall $\interv{0, \frac{\pi}{2}}$ zu berechnen}
        c_n &\coloneqq \int_{0}^{\frac{\pi}{2}} \cos^{n}\of{x} \dif x\numberthis\label{eq:int-cos-potenz}\\
        &= \interv{\frac{1}{n}\sin\of{x}\cos^{n-1}\of{x}}_0^{\frac{\pi}{2}} + \frac{n-1}{n} \int_{0}^{\frac{\pi}{2}} \cos^{n-2}\of{x} \dif x\\
        &= \frac{n-1}{n} \underbrace{\int_{0}^{\frac{\pi}{2}} \cos^{n-2}\of{x} \dif x}_{= c_{n-2}}\\
        \impl c_n &= \frac{n-1}{n}\cdot c_{n-2}\qquad\forall n\geq 2\\
        c_0 &= \frac{\pi}{2}\quad \text{ und } \quad c_1 = \int_{0}^{\frac{\pi}{2}} \cos\of{x} \dif x = \interv{\sin\of{x}}_0^{\frac{\pi}{2}} = 1-0 = 1\\
        c_n &= \frac{n-1}{n}\cdot c_{n-2} = \frac{n-1}{n}\cdot\frac{n-3}{n-2}\cdot c_{n-4}\\
        &= \frac{n-1}{n}\cdot\ldots\cdot\frac{n-2j-1}{n-2j}\cdot c_{n-2j-2}\qquad\forall j\in\N: n-2j-2 \geq 1
        \intertext{Damit folgt für $k\in\N$}
        c_{2k} &= \frac{2k-1}{2k} \cdot\frac{2k-3}{2k-2}\cdot \ldots\cdot\frac{3}{4}\cdot \frac{1}{2}\cdot c_0 = \frac{2k-1}{2k} \cdot\frac{2k-3}{2k-2}\cdot \ldots\cdot\frac{3}{4}\cdot\frac{1}{2}\cdot\frac{\pi}{2}\\
        c_{2k+1} &= \frac{2k}{2k+1}\cdot\frac{2k-2}{2k-1}\cdot\ldots\cdot\frac{4}{5}\cdot \frac{2}{3}\cdot c_1 = \frac{2k}{2k+1}\cdot\frac{2k-2}{2k-1}\cdot\ldots\cdot\frac{4}{5}\cdot \frac{2}{3}
    \end{align*}
\end{beispiel}

\noindent Aus diesem Beispiel können wir nun auch eine Grenzwertdarstellung für $\pi$ ableiten.

\begin{satz}[Wallisches Produkt] % Satz 3
    Sei $n\in\N$ und
    \begin{align*}
        W_n \coloneqq \frac{2\cdot 2}{1\cdot 3}\cdot \frac{4\cdot 4}{3\cdot 5}\cdot \ldots\cdot&\frac{2n\cdot 2n}{\pair{2n-1}\cdot\pair{2n+1}} = \prod_{j=1}^{n} \frac{2j\cdot 2j}{\pair{2j-1}\cdot\pair{2j+1}}
        \intertext{Dann gilt}
        &\lim_{\ntoinf} W_n = \frac{\pi}{2}
    \end{align*}
    \begin{proof}
        Mit der Definition von $c_n$ aus (\ref{eq:int-cos-potenz}) ergibt sich
        \begin{align*}
            W_n &= \frac{\pi}{2} \cdot \frac{c_{2n+1}}{c_{2n}}
            \intertext{Für $x\in\interv{0, \frac{\pi}{2}}$ ist $0\leq\cos\of{x}\leq 1$. Damit folgt $\cos^{2n}\of{x} \leq \cos^{2n-1}\of{x}\leq \cos^{2n-2}\of{x}$. Also gilt}
            \int_{0}^{\frac{\pi}{2}} \cos^{2n}\of{x} \dif x &\leq \int_{0}^{\frac{\pi}{2}} \cos^{2n-1}\of{x} \dif x \leq \int_{0}^{\frac{\pi}{2}} \cos^{2n-2}\of{x} \dif x\\
            \impl c_{2n} &\leq c_{2n-1} \leq c_{2n-2}\qquad\forall n\in\N
            \intertext{Nach Definition gilt}
            c_{2n} &= \frac{\pi}{2}\cdot\prod_{j=1}^{n} \frac{2j-1}{2j}\\[5pt]
            \impl \frac{c_{2n+2}}{c_{2n}} &= \frac{\displaystyle\frac{\pi}{2}\cdot\prod_{j=1}^{n+1} \frac{2j-1}{2j}}{\displaystyle\frac{\pi}{2}\cdot \prod_{j=1}^{n} \frac{2j-1}{2j}} = \frac{2\pair{n+1}-1}{2\pair{n+1}} = \frac{2n+1}{2n+2}\fromto 1 \text{ für } \ntoinf
            \intertext{Außerdem gilt}
            1 &= \frac{c_{2n}}{c_{2n}} \geq \pair{\frac{c_{2n+1}}{c_{2n}}} \geq \frac{c_{2n+2}}{c_{2n}} = \frac{2n+1}{2n+2} \fromto 1 \text{ für } \ntoinf\\
            &\impl \lim_{\ntoinf} \frac{c_{2n+1}}{c_n} = 1\qedhere
            \intertext{Außerdem erhalten wir auch eine Reihendarstellung von $\sqrt{\pi}$}
            W_n &= \frac{2^2\cdot 4^2\cdot 6^2\cdot\ldots\cdot \pair{2n-2}^2}{3^2\cdot 5^2\cdot 7^2\cdot\ldots\cdot\pair{2n-1}^2} \cdot 2n \cdot \frac{2n}{2n+1}\\
            \impl \sqrt{W_n} &= \frac{2\cdot 4 \cdot \ldots \pair{2n-2}}{3\cdot 5 \cdot\ldots\cdot \pair{2n-1}}\cdot\sqrt{2n}\cdot\sqrt{\frac{2n}{2n+1}}\\
            \impl \sqrt{\frac{\pi}{2}} &= \lim_{\ntoinf} \frac{2\cdot 4\cdot\ldots\cdot\pair{2n-2}}{3\cdot 5 \cdot\ldots \cdot\pair{2n-1}}\cdot\sqrt{2n}\\
            &= \lim_{\ntoinf} \frac{2^2 \cdot 4^2\cdot \ldots \cdot \pair{2n-2}^2}{2\cdot 3 \cdot \ldots \cdot \pair{2n-2}\cdot\pair{2n-1}}\cdot\sqrt{2n}\\
            &= \frac{2^2\cdot 4^2\cdot\ldots \cdot \pair{2n-2}^2\cdot\pair{2n}^2}{\pair{2n-1}!\cdot 2n\cdot\sqrt{2n}}\\
            &= \frac{2^{2n}\cdot\pair{n!}^2}{\pair{2n}!\cdot\sqrt{2n}} = \frac{2^{2n}}{\binom{2n}{n}\cdot\sqrt{n}}\cdot \frac{1}{\sqrt{2}}\\
            \impl \sqrt{\pi} &= \lim_{\ntoinf} \frac{2^{2n}}{\binom{2n}{n}\cdot\sqrt{n}}
        \end{align*}
    \end{proof}
    Wir werden später in Bemerkung~\ref{bemerkung:reihendarstellung-pi} sehen, dass z.B. mit Taylor-Polynomen noch andere Reihendarstellungen von $\pi$ herleitbar sind, die im Vergleich schneller konvergieren und dadurch für die tatsächliche Anwendung geigneter sind.
\end{satz}

\subsection{Integrationstechniken - Substitution}

\begin{satz}[Substitutionsregel] % Satz 4
    \label{satz:substitution}
    Seien $I=\interv{a,b}$ und $I^{*}$ kompakte Intervalle und $f\in\mC\of{I, \C}$, $\varphi\in\mC^{1}\of{I^{*}, \R}$ sowie $\varphi\of{I^{*}}\subseteq I$. Dann gilt für $\alpha, \beta \in I^{*}$
    \begin{align*}
        \int_{\alpha}^{\beta} f\of{\varphi\of{t}}\cdot\varphi'\of{t} \dif t &= \int_{\varphi\of{\alpha}}^{\varphi\of{\beta}} f\of{x} \dif x\numberthis\label{eq:substitution}
    \end{align*}
    \begin{proof}
        Sei $F$ eine Stammfunktion von $f$. Wir definieren $h\of{t}\coloneqq F\of{\varphi\of{t}}$. Dann ist $h\in\mC^{1}\of{I^{*}, \C}$ wegen der Kettenregel
        \begin{align*}
            h'\of{t} &= \frac{\dif}{\dif t} h\of{t} = F'\of{\varphi\of{t}}\cdot\varphi'\of{t} = f\of{\varphi\of{t}}\cdot\varphi'\of{t}\\
            \int_{\alpha}^{\beta} h'\of{t} \dif t &= \interv{h\of{t}}_{\alpha}^{\beta} = h\of{\beta}-h\of{\alpha} = F\of{\varphi\of{\beta}} - F\of{\varphi\of{\alpha}}\\
            &= \int_{\varphi\of{\alpha}}^{\varphi\of{\beta}} F'\of{x} \dif x = \int_{\varphi\of{\alpha}}^{\varphi\of{\beta}} f\of{x} \dif x\qedhere
        \end{align*}
    \end{proof}
\end{satz}

\begin{bemerkung}[Erste Lesart von Satz~\ref{satz:substitution}]
    Wir betrachten folgendes Szenario: Wir wollen $\int_{\alpha}^{\beta} g\of{t} \dif t$ ausrechnen und es existiert eine Substitution $x=\varphi\of{t}$ für eine Funktion $f\of{x}$, sodass
    \begin{align*}
        g\of{t} &= f\of{\varphi\of{t}}\cdot\varphi'\of{t}\\
        \intertext{Dann verwenden wir den Satz, setzen $b=\varphi\of{\beta}$, $a = \varphi\of{\alpha}$ und erhalten}
        \impl \int_{\alpha}^{\beta} g\of{t} \dif t &= \int_{a}^{b} f\of{x} \dif x
    \end{align*}
\end{bemerkung}

\begin{beispiel}
    Wir betrachten das Integral
    \begin{align*}
        \int_{\alpha}^{\beta} &g\of{t+c} \dif t
        \intertext{Wir definieren $\varphi\of{t} \coloneqq t+c$ und $f\of{x} = g\of{x}$. Dann gilt $\varphi'\of{t} = 1$}
        \impl \int_{\alpha}^{\beta} g\of{t+c} \dif t = \int_{\alpha}^{\beta} g\of{\varphi\of{t}}\cdot\varphi'\of{t} \dif t &= \int_{\varphi\of{\alpha}}^{\varphi\of{\beta}} g\of{x} \dif x = \int_{\alpha+c}^{\beta+c} g\of{x} \dif x\tag{Translation}
    \end{align*}
\end{beispiel}

\begin{beispiel}
    Sei $a,b >0$. Wir betrachten
    \begin{align*}
        &\int_{a}^{b} \frac{g\of{t}}{t} \dif t
        \intertext{und definieren $\varphi\of{t} \coloneqq \ln\of{t}$, $\varphi'\of{t} = \frac{1}{t}$, $t=e^{\varphi\of{t}}$. Dann gilt}
        g\of{t}\cdot \frac{1}{t} &= g\of{t}\cdot\varphi'\of{t} = g\of{e^{\varphi\of{t}}}\cdot\varphi'\of{t}\\
        f\of{x} &\coloneqq g\of{e^x}
        \intertext{Wir wenden Substitution an und erhalten}
        \impl \int_{a}^{b} g\of{t} \frac{\dif t}{t} &= \int_{a}^{b} f\of{\varphi\of{t}}\cdot\varphi'\of{t} \dif t = \int_{\varphi\of{a}}^{\varphi\of{b}} f\of{x} \dif x\\
        &= \int_{\ln a}^{\ln b} f\of{x} \dif x = \int_{\ln a}^{\ln b} g\of{e^x} \dif x
    \end{align*}
\end{beispiel}

\begin{beispiel}
    Wir betrachten
    \begin{align*}
        \int_{0}^{1} &\pair{1+t^2}^{n}\cdot t \dif t
        \intertext{und definieren $\varphi\of{t} \coloneqq 1 + t^2$, $f\of{x} \coloneqq \frac{1}{2}x^n$}
        \impl \varphi'\of{t} &= 2t\\
        \impl \pair{1+t^2}^n\cdot t &= f\of{\varphi\of{t}}\cdot\varphi'\of{t}\\
        \impl \int_{0}^{1} \pair{1+t^2}^{n}\cdot t \dif t &= \int_{\varphi\of{0}}^{\varphi\of{1}} \frac{1}{2}\cdot x^n \dif t = \int_{1}^{2} \frac{1}{2}\cdot x^n \dif t\\
        &= \interv{\frac{1}{2\pair{n+1}}\cdot x^{n+1}}_1^2\\
        &= \frac{1}{2\pair{n+1}}\cdot\pair{2^{n+1} - 1}
    \end{align*}
\end{beispiel}

\begin{bemerkung}[Zweite Lesart: Transformationssatz]
    \marginnote{[10. Mai]}
    Es ist auch möglich, Satz~\ref{satz:substitution} in die andere Richtung anzuwenden. Das heißt wir wollen das Integral auf der rechten Seite von (\ref{eq:substitution}) zum Integral auf der linken Seite umformen. Wir führen eine Transformation mit $\varphi\of{t} = x$ durch und erhalten ein Integral der Form
    \begin{align*}
        \int_{a}^{b} f\of{x} \dif x &= \int_{\varphi^{-1}\of{a}}^{\varphi^{-1}\of{b}} f\of{\varphi\of{t}}\cdot\varphi'\of{t} \dif t
    \end{align*}
    Dazu benötigt man, dass $\varphi: \interv{\alpha, \beta}\fromto\interv{a,b}$ invertierbar ist. (Also zum Beispiel $\varphi' > 0$ oder $\varphi' < 0$ auf ganz $\interv{\alpha, \beta}$)
\end{bemerkung}

\begin{notation}[Leibnitz'sche Schreibweise]
    Schreiben $x=\varphi\of{t}$ oder auch informell $\frac{\dif x}{\dif t} = \varphi'\of{t}$, $\dif x\phantom{|}\text{\anf{=}}\phantom{|}\varphi'\of{t}\dif t$. Eine Anwendung der 2. Lesart wäre das folgende Integral mit $x = \sin\of{t}$
    \begin{align*}
        \int_{0}^{1} \sqrt{1-x^2} \dif x &= \int_{0}^{\frac{\pi}{2}} \sqrt{1-\sin^2\of{t}}\cdot\cos\of{t} \dif t = \int_{0}^{\frac{\pi}{2}} \cos^2\of{t} \dif t = \frac{\pi}{4}\tag{$\frac{\dif x}{\dif t} = \cos t$}\\
    \end{align*}
\end{notation}

\newpage
