\documentclass[11pt, twoside, a4paper]{article}

% Setup
\usepackage[margin=2.4cm, top=3.5cm]{geometry}
\usepackage[utf8]{inputenc}
\usepackage[ngerman]{babel}

% Package imports
\usepackage{amsfonts}
\usepackage{amsmath}
\usepackage{amssymb}
\usepackage{amsthm}
\usepackage{mathtools}
\usepackage{setspace}
\usepackage{float}
\usepackage{enumitem}
\usepackage{hyperref}
\usepackage[pagestyles]{titlesec}
\usepackage{fancyhdr}
\usepackage{colonequals}
\usepackage{caption}
\usepackage{tikz}
\usepackage{marginnote}
\usepackage{etoolbox}
\usepackage{mdframed}
\usepackage{aligned-overset}

% Font-Encoding
\usepackage[T1]{fontenc}
\usepackage{lmodern}
\usepackage{watermark}

% Theorems
\newtheorem{blockelement}{Blockelement}[subsection]
\newtheoremstyle{plain}{}{}{}{}{\bfseries}{.}{ }{}
\theoremstyle{plain}
\newtheorem{bemerkung}[blockelement]{Bemerkung}
\newtheorem{definition}[blockelement]{Definition}
\newtheorem{lemma}[blockelement]{Lemma}
\newtheorem{satz}[blockelement]{Satz}
\newtheorem{notation}[blockelement]{Notation}
\newtheorem{korollar}[blockelement]{Korollar}
\newtheorem{uebung}[blockelement]{Übung}
\newtheorem{beispiel}[blockelement]{Beispiel}
\newtheorem{folgerung}[blockelement]{Folgerung}
\newtheorem{axiom}[blockelement]{Axiom}
\newtheorem{beobachtung}[blockelement]{Beobachtung}
\newtheorem{konzept}[blockelement]{Konzept}
\newtheorem{visualisierung}[blockelement]{Visualisierung}
\newtheorem{anwendung}[blockelement]{Anwendung}
\newtheorem{skizze}[blockelement]{Skizze}
\newtheorem{genv}[blockelement]{}

% Marginnotes left
\makeatletter
\patchcmd{\@mn@@@marginnote}{\begingroup}{\begingroup\@twosidefalse}{}{\fail}
\reversemarginpar
\makeatother

% Long equations
\allowdisplaybreaks

% \left \right
\newcommand{\set}[1]{\left\{#1\right\}}
\newcommand{\pair}[1]{\left(#1\right)}
\newcommand{\of}[1]{\mathopen{}\mathclose{}\bgroup\left(#1\aftergroup\egroup\right)}
\newcommand{\abs}[1]{\left\lvert#1\right\rvert}
\newcommand{\norm}[1]{\left\lVert#1\right\rVert}
\newcommand{\linterv}[1]{\left[#1\right)}
\newcommand{\rinterv}[1]{\left(#1\right]}
\newcommand{\interv}[1]{\left[#1\right]}
\newcommand{\sprod}[1]{\left<#1\right>}

% Shorten commands
\newcommand{\equivalent}[0]{\Leftrightarrow{}}
\newcommand{\impl}[0]{\Rightarrow{}}
\newcommand{\fromto}{\rightarrow{}}
\newcommand{\definedas}[0]{\coloneqq}
\newcommand{\definedasbackwards}[0]{\eqqcolon}
\newcommand{\definedasequiv}[0]{\ratio\Leftrightarrow{}}
\newcommand{\exclude}[0]{\setminus}
\renewcommand{\emptyset}{\varnothing}
\newcommand{\sbset}{\subseteq}

\newcommand{\ntoinf}[0]{n\fromto\infty}
\newcommand{\toinf}{\fromto\infty}
\newcommand{\fa}{\;\forall\,}
\newcommand{\ex}{\;\exists\,}
\newcommand{\conj}[1]{\overline{#1}}

\newcommand{\annot}[3][]{\overset{\text{#3}}#1{#2}}
\newcommand{\biglim}[1]{{\displaystyle \lim_{#1}}}
\newcommand{\nn}[0]{\\[2\baselineskip]}
\newcommand{\anf}[1]{\glqq{}#1\grqq}
\newcommand{\OBDA}{o.B.d.A. }
\newcommand{\theoremescape}{\leavevmode}
\newcommand{\aligntoright}[2]{\hfill#1\hspace{#2\textwidth}~}
\newcommand{\horizontalline}[0]{\par\noindent\rule{0.05\textwidth}{0.1pt}\\}
\newcommand{\rgbcolor}[3]{rgb,255:red,#1;green,#2;blue,#3}
\newcommand{\fixedspace}[2]{\makebox[#1][l]{#2}}

\let\Re\relax
\let\Im\relax

% MathOperators
\DeclareMathOperator{\grad}{Grad}
\DeclareMathOperator{\bild}{Bild}
\DeclareMathOperator{\Re}{Re}
\DeclareMathOperator{\Im}{Im}

% Mengenbezeichner
\newcommand{\R}{\mathbb{R}}
\newcommand{\N}{\mathbb{N}}
\newcommand{\C}{\mathbb{C}}
\newcommand{\Z}{\mathbb{Z}}
\newcommand{\Q}{\mathbb{Q}}
\newcommand{\K}{\mathbb{K}}

\newcommand\imaginarysubsection[1]{
    \refstepcounter{subsection}
    \subsectionmark{#1}
}

% Unfassbar hässlich, aber effektiv für temporäre schnelle Lösungen
\def\:={\coloneqq}
\def\->{\fromto}
\def\=>{\impl}
\def\<={\leq}
\def\>={\geq}
\def\!={\neq}

% Envs
\newenvironment{induktionsanfang}{
    \rule{0pt}{3ex}\noindent
    \begin{minipage}[t]{0.11\textwidth}
    {I-Anfang}
    \end{minipage}
    \hfill
    \begin{minipage}[t]{0.89\textwidth}
    }
    {
    \end{minipage}
}
\newenvironment{induktionsvoraussetzung}{
    \rule{0pt}{3ex}\noindent
    \begin{minipage}[t]{0.11\textwidth}
    {I-Vor.}
    \end{minipage}
    \hfill
    \begin{minipage}[t]{0.89\textwidth}
    }
    {
    \end{minipage}
}
\newenvironment{induktionsschritt}{
    \rule{0pt}{3ex}\noindent
    \begin{minipage}[t]{0.11\textwidth}
    {I-Schritt}
    \end{minipage}
    \hfill
    \begin{minipage}[t]{0.89\textwidth}
    }
    {
    \end{minipage}
}

% Section style
\titleformat*{\section}{\LARGE\bfseries}
\titleformat*{\subsection}{\large\bfseries}

% Page styles
\newpagestyle{pagenumberonly}{
    \sethead{}{}{}
    \setfoot[][][\thepage]{\thepage}{}{}
}
\newpagestyle{headfootdefault}{
    \sethead[][][\thesubsection~\textit{\subsectiontitle}]{\thesection~\textit{\sectiontitle}}{}{}
    \setfoot[][][\thepage]{\thepage}{}{}
}
\pagestyle{headfootdefault}

\begin{document}
    \title{\vspace{3cm} Skript zur Vorlesung\\Analysis II\\bei Prof. Dr. Dirk Hundertmark}
    \author{Karlsruher Institut für Technologie}
    \date{Sommersemester 2024}
    \maketitle
    \begin{center}
        Dieses Skript ist inoffiziell. Es besteht kein\\Anspruch auf Vollständigkeit oder Korrektheit.
    \end{center}
    \thispagestyle{empty}
    \newpage

    \tableofcontents
    ~\\
    Alle mit [*] markierten Kapitel sind noch nicht Korrektur gelesen und bedürfen eventuell noch Änderungen.

    \newpage


    \section{[*] Das eindimensionale Riemann-Integral}
    \imaginarysubsection{Das eindimensionale Riemann-Integral}
    \thispagestyle{pagenumberonly}
    \marginnote{[16. Apr]}
    Frage: Was ist die Fläche unter einem Graphen?
    % TODO: Vis
    % TODO: Konstant.
    \begin{definition}[Zerlegung eines Intervalls]
        Eine Zerlegung $Z$ eines kompakten Intervalls $I=\interv{a,b}$ in Teilintervalle $I_j$ ($j=1,\dots, k$) der Längen $\abs{I_j}$ ist eine Menge von Punkten $x_0, x_1,\dots, x_k\in I$ (Teilpunkte von $Z$) mit $a=x_0 < x_1 < x_2 < \dots < x_k = b$ und $I_j = \interv{x_{j-1}, x_j}$.\\
        Wir setzen $\varDelta x_j \definedas x_j - x_{j-1} \definedasbackwards\abs{I_j}$
    \end{definition}
    \begin{definition}[Feinheit einer Zerlegung]
        Die Feinheit der Zerlegung $Z$ ist definiert als die Länge des längsten Teilintervalls von $Z$:
        \begin{align*}
            \varDelta\of{z}\definedas \max\of{\abs{I_1}, \abs{I_2}, \dots, \abs{I_k}} = \max\of{\varDelta x_1, \varDelta x_2, \dots, \varDelta x_k}
        \end{align*}
    \end{definition}
    \begin{notation}[Riemannsche Zwischensumme]
        Wir setzen
        \begin{align*}
            B\of{I} = \set{f: I\fromto \R: \sup_{x\in I} \abs{f(x)} < \infty}
        \end{align*}
        als die Menge aller beschränkten reellwertigen Funktionen auf $I$. In jedem $I_j$ wählen wir ein $\xi_j\in I_j$ und setzen $\xi=\pair{\xi_1, \xi_2, \dots, \xi_k}$. Für $f\in B\of{I}$ setzen wir die Riemannsche Zwischensumme
        \begin{align*}
            S_Z\of{f} &= S_Z\of{f, \xi} \definedas \sum_{j=1}^{k} f\of{\xi_j}\cdot\varDelta x_j = \sum_{j=1}^{k} f\of{\xi_j}\cdot\abs{I_j}
        \end{align*}
    \end{notation}
    \begin{notation}[Ober- und Untersumme]
        Für $f\in B\of{I}$ setzen wir
        \begin{align*}
            \underline{m}_j &\definedas \inf_{I_j} f = \inf\set{f(x): x\in I_j}\\
            \overline{m}_j &\definedas \sup_{I_j} f = \sup\set{f(x): x\in I_j}\\
            \overline{S}_Z\of{f}&\definedas \sum_{j=1}^{k} \overline{m}_j\cdot \varDelta x_j\tag{Obersumme}\\
            \underline{S}_Z\of{f}&\definedas \sum_{j=1}^{k} \underline{m}_j\cdot \varDelta x_j\tag{Untersumme}
            \intertext{Damit gilt für $x \in I_j$}
            \underline{m}_j &\leq f\of{x} \leq \overline{m}_j\\
            \impl \underline{m}_j &\leq f\of{\xi_j} \leq \overline{m}_j\\
            \impl \underline{S}_Z\of{f} &\leq S_Z\of{f, \xi} \leq \overline{S}_Z\of{f}
        \end{align*}
    \end{notation}
    \horizontalline \noindent Wir wollen die Zerlegung $Z$ systematisch verfeinern.
    \begin{definition}[Verfeinerung einer Zerlegung]
        Eine Zerlegung $Z^{*}$ von $I$ ist eine Verfeinerung der Zerlegung $Z$ von $I$, falls alle Teilpunkte von $Z$ auch Teilpunkte von $Z^{*}$ sind.
    \end{definition}
    % Vis
    \begin{definition}[Gemeinsame Verfeinerung]
        Die gemeinsame Verfeinerung $Z_1 \lor Z_2$ zweier Zerlegungen $Z_1, Z_2$ von $I$ ist die Zerlegung von $I$, deren Teilpunkte gerade die Teilpunkte von $Z_1$ und $Z_2$ sind.
    \end{definition}
    % Vis

    \begin{lemma} % Lemma 3
        \label{lemma:verfeinerung-ober-unter-summe}
        Ist $Z^{*}$ eine Verfeinerung der Zerlegung $Z$ von $I$ und $f\in B\of{I}$. Dann gilt
        \begin{align*}
            \underline{S}_Z\of{f} \leq \underline{S}_{Z^{*}}\of{f} \leq \overline{S}_{Z^{*}}\of{f} \leq \overline{S}_Z\of{f}
        \end{align*}
        \begin{proof}
            $Z^{*}$ enthält alle Teilpunkte von $Z$, nur mehr.\\
            \textsc{Schritt 1:} Angenommen $Z^{*}$ enthält genau einen Teilpunkt mehr als $Z$.
            \begin{align*}
                y_j &= x_j\quad 0\leq j \leq l\\
                x_l &< y_{l+1} < x_{l+1}\\
                y_{l+1} &= x_j\quad l+ 1 \leq j \leq k\\
                \underline{S}_Z\of{f} &= \sum_{j=1}^{k} \underline{m}_j\varDelta x_j\\
                &= \sum_{j=1}^{l} \underline{m}_j \varDelta x_j + \sum_{j=l+2}^{k} \underline{m}_j \varDelta x_j\\
                \underline{m}_j &= \inf_{I_j} f = \inf_{I_j^{*}} f = \underline{m}_j^{*} \text{ für } 1 \leq j \leq l
                \intertext{$j\geq l+2$}
                \underline{m}_j &= \inf_{I_j} f = \inf_{I_{j+1}^{*}}f = \underline{m}_{j+1}^{*}\\
                I_j &= \interv{x_j, x_{j-1}} = \interv{y_{j+1}, y_j} = I_{j+1}^{*} \text{ für } j \geq l+2\\
                \impl \sum_{j=l+2}^{k} \underline{m}_j \varDelta x_j &= \sum_{j=l+2}^{k} \underline{m}_{j+1}^{*} \varDelta y_{j+1} = \sum_{j=l+3}^{k+1} \underline{m}_j^{*} \varDelta y_{j}\\
                \underline{m}_{l+1} \varDelta x_{l+1} &= \underline{m}_{l+1}\of{x_{l+1} - x_l} = \underline{m}_{l+1}\of{y_{l+2}-y_l}\\
                &= \underline{m}_{l+1}\of{y_{l+2} - y_{l+1} + y_{l+1} - y_l}\\
                &= \underline{m}_{l+1} \varDelta y_{l+1} + \underline{m}_{l+1}\varDelta y_{l+1} \leq m_{l+2}\varDelta y_{l+2} + m_{l+1}\varDelta y_{l+1}\\
                \underline{m}_{l+1} \inf_{I_{l+1}} f &\leq \inf_{I_{l+1}} f = \underline{m}^{*}_{l+1} \text{ und } \underline{m}_{l+1} \leq \inf_{I_{l+2}} f = \underline{m}^{*}_{l+2}
                \intertext{zusammen}
                \underline{S}_{Z}\of{f} &\leq \sum_{j=1}^{l} \underline{m}_j^{*} \varDelta y_j + \underline{m}_{l+1} \varDelta y_{l+1} + \underline{m}_{l+1}\varDelta y_{l+2} + \sum_{j=l+3}^{k+1} \underline{m}_j^{*}\varDelta y_j = \underline{S}_{z}\of{f}
            \end{align*}
            ähnlich zeigt man $\overline{S}_Z\of{f} \geq \overline{S}_{Z^{*}}\of{f}$.\\
            \textsc{Schritt 2:} Sei $Z^{*}$ eine beliebige Verfeinerung von $Z$. Nehmen eine endliche Folge von Einpunkt-Verfeinerungen $Z=Z_0, Z_1, Z_2, \dots, Z_r = Z^{*}$ und $Z_{s+1}$ hat genau einen Punkt mehr als $Z_{s}$. Dann gilt nach \textsc{Schritt 1}, dass
            \begin{align*}
                \underline{S}_{Z}\of{f} \leq \underline{S}_{Z_1}\of{f} \leq \dots \leq \underline{S}_{Z^{*}}\of{f}\\
                \overline{S}_{Z}\of{f} \geq \overline{S}_{Z_1}\of{f} \geq \overline{S}_{Z_2} \geq \dots \geq \overline{S}_{Z^{*}}\of{f}
            \end{align*}
            \textsc{Schritt 3:} Sei $\xi^{*} = \pair{\xi_1^{*}, \xi_2^{*}, \dots, \xi_l^{*}}$ Zwischenpunkt zur Zerlegung $Z^{*}$. Dann gilt
            \begin{align*}
                S_{Z^{*}}\of{f} \leq S_{Z^*}\of{f, \xi^{*}} &\leq \overline{S}_{Z^*}\of{f}\qedhere
            \end{align*}
        \end{proof}
    \end{lemma}

    \begin{lemma} % Lemma 4
        \label{lemma:temp-4}
        Seien $Z_1$, $Z_2$ Zerlegungen von $I$. Dann gilt $\underline{S}_{Z_1}\of{f} \leq \overline{S}_{Z_2}\of{f}~\forall f\in B\of{I}$.
        \begin{proof}
            Es gilt nach Lemma~\ref{lemma:verfeinerung-ober-unter-summe}, dass
            \begin{align*}
                \underline{S}_{Z_1}\of{f} &\leq \underline{S}_{Z_1\lor Z_2} \leq \overline{S}_{Z_1\lor Z_2} \leq \overline{S}_{Z_2}\of{f}\qedhere
            \end{align*}
        \end{proof}
    \end{lemma}

    \begin{bemerkung}
        Für $I=\interv{a,b}$ und $f\in B\of{I}$ gilt immer $\underbrace{\abs{I}}_{=b-a} \inf_{I} f \leq \underline{S}_Z\of{f} \leq \overline{S}_Z\of{f} \leq \abs{I} \sup_{I} f$ für alle Zerlegungen $Z$ von $I$.\\
        $\set{\overline{S}_Z\of{f} : Z \text{ ist eine Zerlegung von } I}$ und $\set{\underline{S}_Z\of{f} : Z \text{ ist eine Zerlegung von } I}$ sind bereits beschränkte, nicht-leere Teilmengen von $\R$.
    \end{bemerkung}

    \begin{definition}
        Für $I=\interv{a,b}$ und $f\in B\of{I}$ definieren wir
        \begin{align*}
            \underline{J}\of{f} \definedas \sup\set{\underline{S}_Z\of{f} : \text{ $Z$ ist Zerlegung von $I$ } }\tag{Unterintegral}\\
            \overline{J}\of{f} \definedas \sup\set{\overline{S}_Z\of{f} : \text{ $Z$ ist Zerlegung von $I$ } }\tag{Oberintegral}\\
        \end{align*}
    \end{definition}

    \begin{lemma} % Lemma 6
        Es sei $Z$ eine Zerlegung von $I$. Dann gilt
        \begin{align*}
            \underline{S}_{Z}\of{f} \leq \underline{J}\of{f} \leq \overline{J}\of{f} \leq \overline{S}_{Z}\of{f}
        \end{align*}
        \begin{proof}
            Nach Lemma~\ref{lemma:temp-4} gilt
            \begin{align*}
                \underline{S}_{Z_1}\of{f} \leq \overline{S}_{Z_2}\of{f}\\
                \impl \sup\set{\underline{S}_{Z_1}\of{f}: Z_1 \text{ ist eine Zerlegung von } I} \leq \overline{S}_{Z_1}\of{f}\\
                \impl \underline{J}\of{f} \leq \overline{S}_{Z_1}\of{f}\\
                \impl \underline{J}\of{f} \leq \inf\set{\overline{S}_{Z_2}\of{f}: "} = \overline{J}\of{f}\\
                \underline{S}_{Z}\of{f} \leq \underline{J}\of{f}\leq \overline{J}\of{f} \leq \overline{S}_Z\of{f}
            \end{align*}
        \end{proof}
    \end{lemma}

    \begin{definition}
        $I=\interv{a,b}$, $f\in B\of{I}$ heißt (Riemann-)integrierbar, falls
        \begin{align*}
            \underline{J}\of{f} = \overline{J}\of{f}
        \end{align*}
        In diese Fall nennen wir $J(f) \definedas \underline{J}\of{f} = \overline{J}\of{f}$ das bestimmte Integral über $\interv{a,b}$ und schreiben
        \begin{align*}
            \int_{a}^{b} f(x) dx = \int_{a}^{b} fdx = \int_{I} f(x)dx = \int_{I} fdx = J(f)
        \end{align*}
        Die Klasse der Riemann-integrierbaren Funktionen nennen wir $R(I)$.
    \end{definition}


\end{document}
