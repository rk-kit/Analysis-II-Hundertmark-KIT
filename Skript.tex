\documentclass[11pt, twoside, a4paper]{article}

% Setup
\usepackage[margin=2.4cm, top=3.5cm]{geometry}
\usepackage[utf8]{inputenc}
\usepackage[ngerman]{babel}

% Package imports
\usepackage{amsfonts}
\usepackage{amsmath}
\usepackage{amssymb}
\usepackage{amsthm}
\usepackage{mathtools}
\usepackage{setspace}
\usepackage{float}
\usepackage{enumitem}
\usepackage{hyperref}
\usepackage[pagestyles]{titlesec}
\usepackage{fancyhdr}
\usepackage{colonequals}
\usepackage{caption}
\usepackage{tikz}
\usepackage{marginnote}
\usepackage{etoolbox}
\usepackage{mdframed}
\usepackage{aligned-overset}
\usepackage{esint}
\usepackage{scalerel}

% Font-Encoding
\usepackage[T1]{fontenc}
\usepackage{lmodern}
\usepackage{stmaryrd}

% Theorems
\newtheoremstyle{plain}{}{}{}{}{\bfseries}{.}{ }{}
\theoremstyle{plain}
\newtheorem{blockelement}{Blockelement}[subsection]
\newtheorem{bemerkung}[blockelement]{Bemerkung}
\newtheorem{definition}[blockelement]{Definition}
\newtheorem{lemma}[blockelement]{Lemma}
\newtheorem{satz}[blockelement]{Satz}
\newtheorem{notation}[blockelement]{Notation}
\newtheorem{korollar}[blockelement]{Korollar}
\newtheorem{uebung}[blockelement]{Übung}
\newtheorem{beispiel}[blockelement]{Beispiel}
\newtheorem{folgerung}[blockelement]{Folgerung}
\newtheorem{axiom}[blockelement]{Axiom}
\newtheorem{beobachtung}[blockelement]{Beobachtung}
\newtheorem{konzept}[blockelement]{Konzept}
\newtheorem{visualisierung}[blockelement]{Visualisierung}
\newtheorem{anwendung}[blockelement]{Anwendung}
\newtheorem{skizze}[blockelement]{Skizze}
\newtheorem{genv}[blockelement]{}

% Equation numbering
\numberwithin{equation}{subsection}
\newcommand{\numberthis}[0]{\addtocounter{equation}{1}\tag{\theequation}}

% Marginnotes left
\makeatletter
\patchcmd{\@mn@@@marginnote}{\begingroup}{\begingroup\@twosidefalse}{}{\fail}
\reversemarginpar
\makeatother

% Long equations
\allowdisplaybreaks

% \left \right
\newcommand{\set}[1]{\left\{#1\right\}}
\newcommand{\pair}[1]{\left(#1\right)}
\newcommand{\of}[1]{\mathopen{}\mathclose{}\bgroup\left(#1\aftergroup\egroup\right)}
\newcommand{\abs}[1]{\left\lvert#1\right\rvert}
\newcommand{\norm}[1]{\left\lVert#1\right\rVert}
\newcommand{\linterv}[1]{\left[#1\right)}
\newcommand{\rinterv}[1]{\left(#1\right]}
\newcommand{\interv}[1]{\left[#1\right]}
\newcommand{\sprod}[1]{\left<#1\right>}

% Shorten commands
\newcommand{\equivalent}[0]{\Leftrightarrow{}}
\newcommand{\impl}[0]{\Rightarrow{}}
\newcommand{\fromto}{\rightarrow{}}
\newcommand{\definedas}[0]{\coloneqq}
\newcommand{\definedasbackwards}[0]{\eqqcolon}
\newcommand{\definedasequiv}[0]{\ratio\Leftrightarrow{}}
\newcommand{\exclude}[0]{\setminus}
\renewcommand{\emptyset}{\varnothing}
\newcommand{\sbset}{\subseteq}
\newcommand{\dif}{\mathop{}\!\mathrm{d}}

\newcommand{\ntoinf}[0]{n\fromto\infty}
\newcommand{\toinf}{\fromto\infty}
\newcommand{\fa}{\;\forall}
\newcommand{\ex}{\;\exists}
\newcommand{\conj}[1]{\overline{#1}}

\newcommand{\annot}[3][]{\overset{\text{#3}}#1{#2}}
\newcommand{\biglim}[1]{{\displaystyle \lim_{#1}}}
\newcommand{\nn}[0]{\\[2\baselineskip]}
\newcommand{\anf}[1]{\glqq{}#1\grqq}
\newcommand{\OBDA}{o.B.d.A. }
\newcommand{\theoremescape}{\leavevmode}
\newcommand{\aligntoright}[2]{\hfill#1\hspace{#2\textwidth}~}
\newcommand{\horizontalline}[0]{\par\noindent\rule{0.05\textwidth}{0.1pt}\\}
\newcommand{\rgbcolor}[3]{rgb,255:red,#1;green,#2;blue,#3}
\newcommand{\fixedspace}[2]{\makebox[#1][l]{#2}}
\newcommand{\ov}[1]{\overline{#1}}
\newcommand{\un}[1]{\underline{#1}}
\newcommand{\verteq}{\rotatebox{-90}{$~=$}}
\newcommand{\equalto}[2]{\underset{\scriptstyle\overset{\mkern4mu\verteq}}{#1}}
\newcommand{\eqbelow}[1]{\underset{\verteq}{#1}}

\let\Re\relax
\let\Im\relax

% MathOperators
\DeclareMathOperator{\grad}{Grad}
\DeclareMathOperator{\bild}{Bild}
\DeclareMathOperator{\Re}{Re}
\DeclareMathOperator{\Im}{Im}
\DeclareMathOperator{\arcsinh}{arcsinh}
\DeclareMathOperator{\arccosh}{arccosh}
\DeclareMathOperator{\diam}{diam}
\DeclareMathOperator{\fehler}{Fehler}
\DeclareMathOperator{\D}{D\!}

% Mengenbezeichner
\newcommand{\R}{\mathbb{R}}
\newcommand{\N}{\mathbb{N}}
\newcommand{\C}{\mathbb{C}}
\newcommand{\Z}{\mathbb{Z}}
\newcommand{\Q}{\mathbb{Q}}
\newcommand{\K}{\mathbb{K}}

\newcommand{\mR}{\mathcal{R}}
\newcommand{\mB}{\mathcal{B}}
\newcommand{\mC}{\mathcal{C}}
\newcommand{\mL}{\mathcal{L}}
\newcommand{\mJ}{\mathcal{J}}
\newcommand{\mPC}{\mathcal{PC}}

% Spezielle Symbole
\NewDocumentCommand{\Tau}{e{^_}}{
    \scalerel*{\tau}{X}
    \IfValueT{#1}{^{#1}}
    \IfValueT{#2}{_{\!\!#2}}
}

% Spezielle Commands
\newcommand\imaginarysubsection[1]{
    \refstepcounter{subsection}
    \subsectionmark{#1}
}

% Unfassbar hässlich, aber effektiv für temporäre schnelle Lösungen
\def\:={\coloneqq}
\def\->{\fromto}
\def\=>{\impl}
\def\<={\leq}
\def\>={\geq}

% Envs
\newenvironment{induktionsanfang}{
    \rule{0pt}{3ex}\noindent
    \begin{minipage}[t]{0.11\textwidth}
    {I-Anfang}
    \end{minipage}
    \hfill
    \begin{minipage}[t]{0.89\textwidth}
    }
    {
    \end{minipage}
}
\newenvironment{induktionsvoraussetzung}{
    \rule{0pt}{3ex}\noindent
    \begin{minipage}[t]{0.11\textwidth}
    {I-Vor.}
    \end{minipage}
    \hfill
    \begin{minipage}[t]{0.89\textwidth}
    }
    {
    \end{minipage}
}
\newenvironment{induktionsschritt}{
    \rule{0pt}{3ex}\noindent
    \begin{minipage}[t]{0.11\textwidth}
    {I-Schritt}
    \end{minipage}
    \hfill
    \begin{minipage}[t]{0.89\textwidth}
    }
    {
    \end{minipage}
}

% Section style
\titleformat*{\section}{\LARGE\bfseries}
\titleformat*{\subsection}{\large\bfseries}

% Page styles
\newpagestyle{pagenumberonly}{
    \sethead{}{}{}
    \setfoot[][][\thepage]{\thepage}{}{}
}
\newpagestyle{headfootdefault}{
    \sethead[][][\thesubsection~\textit{\subsectiontitle}]{\thesection~\textit{\sectiontitle}}{}{}
    \setfoot[][][\thepage]{\thepage}{}{}
}
\pagestyle{headfootdefault}

\begin{document}
    \title{\vspace{3cm} Skript zur Vorlesung\\Analysis II\\bei Prof. Dr. Dirk Hundertmark}
    \author{Karlsruher Institut für Technologie}
    \date{Sommersemester 2024}
    \maketitle
    \begin{center}
        Dieses Skript ist inoffiziell. Es besteht kein\\Anspruch auf Vollständigkeit oder Korrektheit.
    \end{center}
    \thispagestyle{empty}
    \newpage

    \tableofcontents
    ~\\
    Alle mit [*] markierten Kapitel sind noch nicht Korrektur gelesen und bedürfen eventuell noch Änderungen.

    \newpage

    \section{[*] Das eindimensionale Riemann-Integral}
\thispagestyle{pagenumberonly}

\marginnote{[16. Apr]}
Frage: Was ist die Fläche unter einem Graphen?

% Vis

\subsection{Der Integralbegriff nach Riemann}

\begin{definition}[Zerlegung]
    Eine Zerlegung $Z$ eines kompakten Intervalls $I=\interv{a,b}$ in Teilintervalle $I_j$ ($j=1,\dots, k$) der Längen $\abs{I_j}$ ist eine Menge von Punkten $x_0$, $x_1$,$\dots$, $x_k\in I$ (Teilpunkte von $Z$) mit
    \begin{align*}
        a=x_0 < x_1 < x_2 < \dots < x_k = b
    \end{align*}
    und $I_j = \interv{x_{j-1}, x_j}$. Wir setzen $\varDelta x_j \coloneqq x_j - x_{j-1} \eqqcolon\abs{I_j}$.
\end{definition}

\begin{definition}[Feinheit einer Zerlegung]
    Die Feinheit der Zerlegung $Z$ ist definiert als die Länge des längsten Teilintervalls von $Z$:
    \begin{align*}
        \varDelta\of{Z}\coloneqq \max\of{\abs{I_1}, \abs{I_2}, \dots, \abs{I_k}} = \max\of{\varDelta x_1, \varDelta x_2, \dots, \varDelta x_k}
    \end{align*}
\end{definition}

\begin{notation}
    Wir setzen
    \begin{align*}
        \mB\of{I} = \set{f: I\fromto \R~\middle\vert~\sup_{x\in I} \abs{f(x)} < \infty}
    \end{align*}
    als die Menge aller beschränkten reellwertigen Funktionen auf $I$.
\end{notation}

\begin{definition}[Riemannsche Zwischensumme]
    In jedem $I_j$ wählen wir ein $\xi_j\in I_j$ als Stützstelle und setzen $\xi=\pair{\xi_1, \xi_2, \dots, \xi_k}$. Für eine Funktion $f\in \mB\of{I}$ setzen wir die Riemannsche Zwischensumme
    \begin{align*}
        S_Z\of{f} &= S_Z\of{f, \xi} \coloneqq \sum_{j=1}^{k} f\of{\xi_j}\cdot\varDelta x_j = \sum_{j=1}^{k} f\of{\xi_j}\cdot\abs{I_j}
    \end{align*}
\end{definition}

\begin{definition}[Ober- und Untersumme]
    Für $f\in \mB\of{I}$ setzen wir außerdem
    \begin{align*}
        \underline{m}_j &\coloneqq \inf_{I_j} f = \inf\set{f(x): x\in I_j}\\
        \overline{m}_j &\coloneqq \sup_{I_j} f = \sup\set{f(x): x\in I_j}\\
        \overline{S}_Z\of{f}&\coloneqq \sum_{j=1}^{k} \overline{m}_j\cdot \varDelta x_j\tag{Obersumme}\\
        \underline{S}_Z\of{f}&\coloneqq \sum_{j=1}^{k} \underline{m}_j\cdot \varDelta x_j\tag{Untersumme}
        \intertext{Damit gilt für $x \in I_j$}
        \underline{m}_j &\leq f\of{x} \leq \overline{m}_j\\
        \impl \underline{m}_j &\leq f\of{\xi_j} \leq \overline{m}_j\\
        \impl \underline{S}_Z\of{f} &\leq S_Z\of{f, \xi} \leq \overline{S}_Z\of{f}\numbereq{eq:7}
    \end{align*}
    Wir wollen die Zerlegung $Z$ nun systematisch verfeinern.
\end{definition}

\begin{definition}[Verfeinerung einer Zerlegung]
    \theoremescape
    \begin{enumerate}[label=(\alph*)]
        \item Eine Zerlegung $Z^{*}$ von $I$ ist eine Verfeinerung der Zerlegung $Z$ von $I$, falls alle Teilpunkte von $Z$ auch Teilpunkte von $Z^{*}$ sind.
        \item Die gemeinsame Verfeinerung $Z_1 \lor Z_2$ zweier Zerlegungen $Z_1, Z_2$ von $I$ ist die Zerlegung von $I$, deren Teilpunkte gerade die Teilpunkte von $Z_1$ und $Z_2$ sind.
    \end{enumerate}
\end{definition}

% Vis

% Vis

\begin{lemma} % Lemma 3
    \label{lemma:temp-3}
    Ist $Z^{*}$ eine Verfeinerung der Zerlegung $Z$ von $I$ und $f\in \mB\of{I}$. Dann gilt
    \begin{align*}
        \underline{S}_Z\of{f} \leq \underline{S}_{Z^{*}}\of{f} \leq \overline{S}_{Z^{*}}\of{f} \leq \overline{S}_Z\of{f}
    \end{align*}
    \begin{proof}
        $Z^{*}$ enthält alle Teilpunkte von $Z$, nur mehr.\\[10pt]
        \textsc{Schritt 1:} Wir nehmen an $Z^{*}$ enthielte genau einen Teilpunkt ($y_{l+1}$) mehr als $Z$. Das heißt
        \begin{alignat*}{3}
            y_j &= x_j\quad &&\fa 0\leq j \leq l\\
            x_l < y_{l+1} &< x_{l+1}\quad &&\\
            y_{j+1} &= x_j\quad &&\fa l+ 1 \leq j \leq k
        \end{alignat*}
        Dann gilt
        \begin{align*}
            \underline{S}_Z\of{f} &= \sum_{j=1}^{k} \underline{m}_j\varDelta x_j = \sum_{j=1}^{l} \underline{m}_j \varDelta x_j + \underline{m}_{l+1}\varDelta x_{l+1} + \sum_{j=l+2}^{k} \underline{m}_j \varDelta x_j\\
            \underline{m}_j &= \inf_{I_j} f = \inf_{I_j^{*}} f = \underline{m}_j^{*}\quad \fa 1 \leq j \leq l\\
            \underline{m}_j &= \inf_{I_j} f = \inf_{I_{j+1}^{*}}f = \underline{m}_{j+1}^{*}\quad \fa j\geq l+2\\
            I_j &= \interv{x_j, x_{j-1}} = \interv{y_{j+1}, y_j} = I_{j+1}^{*}\quad\fa j \geq l+2\\
            \impl \sum_{j=l+2}^{k} \underline{m}_j \varDelta x_j &= \sum_{j=l+2}^{k} \underline{m}_{j+1}^{*} \varDelta y_{j+1} = \sum_{j=l+3}^{k+1} \underline{m}_j^{*} \varDelta y_{j}\\
            \underline{m}_{l+1} \varDelta x_{l+1} &= \underline{m}_{l+1}\of{x_{l+1} - x_l} = \underline{m}_{l+1}\of{y_{l+2}-y_l}\\
            &= \underline{m}_{l+1}\of{y_{l+2} - y_{l+1} + y_{l+1} - y_l}\\
            &= \underline{m}_{l+1} \varDelta y_{l+2} + \underline{m}_{l+1}\varDelta y_{l+1}\\
            &\leq \underline{m}_{l+2}^{*}\varDelta y_{l+2} + \underline{m}_{l+1}^{*}\varDelta y_{l+1}
            \intertext{Insgesamt ergibt sich}
            \underline{S}_{Z}\of{f} &\leq \sum_{j=1}^{l} \underline{m}_j^{*} \varDelta y_j + \underline{m}_{l+1}^{*} \varDelta y_{l+1} + \underline{m}_{l+2}^{*}\varDelta y_{l+2} + \sum_{j=l+3}^{k+1} \underline{m}_j^{*}\varDelta y_j = \underline{S}_{Z^{*}}\of{f}
        \end{align*}
        ähnlich zeigt man $\overline{S}_Z\of{f} \geq \overline{S}_{Z^{*}}\of{f}$.\\[10pt]
        \textsc{Schritt 2:} Sei $Z^{*}$ eine beliebige Verfeinerung von $Z$. Wir nehmen eine endliche Folge von Einpunkt-Verfeinerungen $Z=Z_0, Z_1, Z_2, \dots, Z_r = Z^{*}$. Dabei hat $Z_{s+1}$ genau einen Punkt mehr als $Z_{s}$. Dann gilt nach \textsc{Schritt 1}, dass $\underline{S}_{Z}\of{f} \leq \underline{S}_{Z_1}\of{f} \leq \dots \leq \underline{S}_{Z^{*}}\of{f}$ und $\overline{S}_{Z}\of{f} \geq \overline{S}_{Z_1}\of{f} \geq \dots \geq \overline{S}_{Z^{*}}\of{f}$.\\[10pt]
        \textsc{Schritt 3:} Sei $\xi^{*} = \pair{\xi_1^{*}, \xi_2^{*}, \dots, \xi_l^{*}}$ der Zwischenpunkt zur Zerlegung $Z^{*}$. Dann gilt nach (\ref{eq:7})
        \begin{align*}
            \un{S}_{Z^{*}}\of{f} \leq S_{Z^*}\of{f, \xi^{*}} &\leq \overline{S}_{Z^*}\of{f}\qedhere
        \end{align*}
    \end{proof}
\end{lemma}

\begin{lemma} % Lemma 4
    \label{lemma:temp-4}
    Seien $Z_1$, $Z_2$ Zerlegungen von $I$. Dann gilt
    \begin{align*}
        \underline{S}_{Z_1}\of{f} \leq \overline{S}_{Z_2}\of{f}\qquad\forall f\in \mB\of{I}
    \end{align*}
    \begin{proof}
        Es gilt nach Lemma~\ref{lemma:temp-3}, dass
        \begin{align*}
            \underline{S}_{Z_1}\of{f} &\leq \underline{S}_{Z_1\lor Z_2}\of{f}\leq \overline{S}_{Z_1\lor Z_2}\of{f} \leq \overline{S}_{Z_2}\of{f}\qedhere
        \end{align*}
    \end{proof}
\end{lemma}

\begin{bemerkung}
    Für $I=\interv{a,b}$ und $f\in \mB\of{I}$ gilt immer
    \begin{align*}
        \abs{I}\cdot \inf_{I} f \leq \underline{S}_Z\of{f} \leq \overline{S}_Z\of{f} \leq \abs{I}\cdot \sup_{I} f
    \end{align*}
    für alle Zerlegungen $Z$ von $I$. Somit sind
    \begin{align*}
        \set{\overline{S}_Z\of{f} : Z \text{ ist eine Zerlegung von } I}
        \intertext{und}
        \set{\underline{S}_Z\of{f} : Z \text{ ist eine Zerlegung von } I}
    \end{align*}
    beschränkte, nicht-leere Teilmengen von $\R$. Das erlaubt uns die folgende Definition, mit der wir nun mithilfe der bereits definierten Summen einem tatsächlichen Integralbegriff nähern wollen.
\end{bemerkung}

\begin{definition}[Ober- und Unterintegral]
    Es sei $I=\interv{a,b}$ und $f\in \mB\of{I}$. Wir definieren
    \begin{align*}
        \overline{J}\of{f} \coloneqq \inf\set{\overline{S}_Z\of{f} : Z\text{ ist Zerlegung von $I$ } }\tag{Oberintegral}\\
        \underline{J}\of{f} \coloneqq \sup\set{\underline{S}_Z\of{f} : Z\text{ ist Zerlegung von $I$ } }\tag{Unterintegral}
    \end{align*}
\end{definition}

\begin{lemma} % Lemma 6
    \label{lemma:temp-6}
    Es sei $Z$ eine Zerlegung von $I$. Dann gilt
    \begin{align*}
        \underline{S}_{Z}\of{f} \leq \underline{J}\of{f} \leq \overline{J}\of{f} \leq \overline{S}_{Z}\of{f}
    \end{align*}
    \begin{proof}
        Nach Lemma~\ref{lemma:temp-4} gilt für zwei beliebige Zerlegungen $Z_1$, $Z_2$
        \begin{align*}
            \underline{S}_{Z_1}\of{f} &\leq \overline{S}_{Z_2}\of{f}
            \intertext{Wir fixieren $Z_2$ und erhalten}
            \impl \sup\set{\underline{S}_{Z_1}\of{f}: Z_1 \text{ Zerlegung von } I} &\leq \overline{S}_{Z_2}\of{f}\\
            \impl \underline{J}\of{f} &\leq \overline{S}_{Z_2}\of{f}\\
            \impl \underline{J}\of{f} &\leq \inf\set{\overline{S}_{Z_2}\of{f}: Z_2 \text{ Zerlegung von } I}\\
            \impl \underline{J}\of{f} &\leq \overline{J}\of{f}\\
            \impl\underline{S}_{Z}\of{f} \leq \underline{J}\of{f}&\leq \overline{J}\of{f} \leq \overline{S}_Z\of{f}\qedhere
        \end{align*}
    \end{proof}
\end{lemma}

\begin{definition}[Integral]
    Es sei $I=\interv{a,b}$. $f\in \mB\of{I}$ heißt (Riemann-)integrierbar, falls
    \begin{align*}
        \underline{J}\of{f} = \overline{J}\of{f}
    \end{align*}
    In diese Fall nennen wir $J(f) \coloneqq \underline{J}\of{f} = \overline{J}\of{f}$ das (bestimmte) Integral von $f$ über $\interv{a,b}$ und schreiben
    \begin{align*}
        \int_{a}^{b} f(x) \dif x = \int_{a}^{b} f\dif x = \int_{I} f(x)\dif x = \int_{I} f\dif x \eqqcolon J(f)
    \end{align*}
    Die Klasse der Riemann-integrierbaren Funktionen $f\in \mB\of{I}$ nennen wir $\mR\of{I}$.
\end{definition}

\begin{beispiel}[Konstante Funktion]
    \marginnote{[18. Apr]}
    \label{beispiel:int-konstant}
    $f(x) \coloneqq c$ auf $\interv{a,b}$ für eine Konstante $c\in\R$. Dann gilt
    \begin{align*}
        \int_{a}^{b} f\of{x} \dif x &= c\cdot\pair{b-a}
    \end{align*}
\end{beispiel}

\begin{beispiel}[Dirichlet-Funktion]
    \label{beispiel:int-dirichlet}
    Die Funktion $f: \interv{0,1}\fromto\R$
    \begin{align*}
        f(x) \coloneqq \begin{cases}
                           1 &x\in \Q\\
                           0 &\text{sonst}
        \end{cases}
    \end{align*}
    ist nicht Riemann-integrierbar, weil $\overline{J}\of{f} = 1$ und $\underline{J}\of{f} = 0$.
\end{beispiel}

\begin{uebung}
    Beweisen Sie die Aussagen aus Beispiel~\ref{beispiel:int-konstant} und \ref{beispiel:int-dirichlet} mittels der formalen Definition von $\un{J}\of{f}$ und $\ov{J}\of{f}$.
\end{uebung}

\subsection{[*] Integrabilitätskriterien}

\begin{satz}[1. Kriterium] % Satz 8
    \label{satz:integr-kriterium-1}
    Es sei $f\in \mB\of{I}$. Dann gilt $f\in \mR\of{I}$ genau dann, wenn
    \begin{align*}
        \fa\varepsilon > 0\ex\text{Zerlegung } Z\text{ von } I\text{ mit } \overline{S}_{Z}\of{f}-\underline{S}_{Z}\of{f} < \varepsilon
    \end{align*}
    \begin{proof}
        \anf{$\Leftarrow$} Nach Lemma~\ref{lemma:temp-6} gilt
        \begin{align*}
            \underline{S}_Z\of{f}\leq \underline{J}\of{f} &\leq \overline{J}\of{f} \leq \overline{S}_Z\of{f}
            \intertext{Sei $\varepsilon > 0$, dann gilt}
            0\leq\overline{J}\of{f}-\underline{J}\of{f}&\leq\overline{S}_Z\of{f} - \underline{S}_Z\of{f} < \varepsilon\\
            \impl 0\leq \overline{J}\of{f} - \underline{J}\of{f} &\leq 0\\
            \impl f\in \mR\of{I}&
        \end{align*}
        \anf{$\impl$} Angenommen $f\in \mR\of{I}$, das heißt
        \begin{align*}
            \overline{J}\of{f} &= \underline{J}\of{f}\\
            \overline{J}\of{f} &= \inf\set{\overline{S}_Z\of{f}: Z \text{ Zerlegung von } I}\\
            \underline{J}\of{f} &=\sup\set{\underline{S}_Z\of{f}: Z \text{ Zerlegung von } I}
            \intertext{Das heißt zu $\varepsilon > 0$ existieren Zerlegungen $Z_1$, $Z_2$ von $I$ mit}
            \overline{J}\of{f} + \frac{\varepsilon}{2} &> \overline{S}_{Z_1}\of{f}\\
            \underline{J}\of{f} - \frac{\varepsilon}{2} &< \un{S}_{Z_2}\of{f}
            \intertext{Da $f\in \mR\of{I}$ gilt $\underline{J}\of{f} = \overline{J}\of{f}$. Wir definieren die gemeinsame Verfeinerung $Z\coloneqq Z_1 \lor Z_2$. Dann gilt nach Lemma~\ref{lemma:temp-3}}
            \overline{S}_Z\of{f} - \underline{S}_Z\of{f} &< \overline{J}\of{f} + \frac{\varepsilon}{2} - \pair{\underline{J}\of{f} - \frac{\varepsilon}{2}}\\
            &= \underbrace{\overline{J}\of{f} - \underline{J}\of{f}}_{=0} + \frac{\varepsilon}{2} + \frac{\varepsilon}{2} = \varepsilon\qedhere
        \end{align*}
    \end{proof}
\end{satz}

\begin{satz}[2. Kriterium] % Satz 9
    \label{satz:temp-9}
    Sei $f\in \mB\of{I}$. Dann gilt $f\in \mR\of{I}$ genau dann, wenn
    \begin{align*}
        \fa\varepsilon > 0\ex\delta > 0\fa \text{Zerlegungen }Z \text{ von } I \text{ mit Feinheit } \Delta\of{Z} < \delta\colon \overline{S}_Z\of{f} - \underline{S}_Z\of{f} < \varepsilon
    \end{align*}
    \begin{proof}
        \anf{$\Leftarrow$} wird von Satz~\ref{satz:integr-kriterium-1} bereits impliziert.\\[10pt]
        \anf{$\impl$} Sei $f\in \mR\of{I}$ und $\varepsilon > 0$. Dann gilt nach Satz~\ref{satz:integr-kriterium-1}, dass eine Zerlegung $Z'=\pair{x_0', x_1', \dots, x_l' = b}$ von $I$ mit
        \begin{align*}
            \overline{S}_Z\of{f} - \underline{S}_Z\of{f} &< \frac{\varepsilon}{2}
            \intertext{existiert. Wähle eine andere Zerlegung $Z$ von $I$ mit $\Delta\of{Z} < \delta$, wobei $\delta > 0$ noch später gewählt wird. Setze $Z^{*} = Z'\lor Z$. Nach Lemma~\ref{lemma:temp-3} und Satz~\ref{satz:integr-kriterium-1} gilt}
            \overline{S}_{Z^{*}}\of{f} - \underline{S}_{Z^{*}}\of{f} &< \frac{\varepsilon}{2}
            \intertext{Wir wollen die Ober- und Untersumme von $Z^{*}$ mit denen in $Z$ vergleichen.}
            \overline{S}_Z\of{f} - \underline{S}_{Z^{*}}\of{f} &= \sum_{j} \overline{m}_j\cdot\abs{I_j} - \sum_{t}^{} \overline{m}_t\cdot\abs{I_t}
            \intertext{wobei $I_j = \interv{x_{j-1}, x_j}$. Da $Z^*$ eine Verfeinerung von $Z$ ist, sind alle Teilpunkte von $Z$ auch Teilpunkte von $Z^*$. Das heißt die Intervalle $I_j$ (zu $Z$) unterscheiden sich von den Intervallen $I_j^*$ (zu $Z^*$) sofern Punkte $x_{\nu}'$ (Teilpunkte von $Z^*$) im Inneren von $I_j$ liegen. Also gilt}
            I_Z^{*} \cap I_j &\neq \emptyset \impl I_Z^{*} \subseteq I_j
            \intertext{Frage: Wie viele Intervalle $I_j$ existieren maximal, für die $I_j$ eine Verfeinerung von $Z$ oder ? hinter reellen $I_j^*$ ist? Dann muss mindestens ein Punkt von der Zerlegung $Z'$ unterhalb von $I_j$ liegen. Wir haben $l$ Punkte in Zerlegung $Z'$. Das heißt die Anzahl solcher Intervalle $I_j$ ist maximal $l$.}
            \overline{S}_Z\of{f} - \overline{S}_{Z^{*}}\of{f} &= \sum_{j}^{} \overline{m}_j\cdot\abs{I_j} - \sum_{t}^{} \overline{m}_t^{*} \cdot\abs{I_j^{*}}\\
            &= \sum_{j}^{} \pair{\overline{m}_j \cdot\abs{I_j} - \sum_{t: I_{Z}^{*} \subseteq I_j}^{} \overline{m}_t^{*} \cdot\abs{I_t^{*}}}\\
            &= \sum_{j}^{} \sum_{t: I_t^{*}}^{} \pair{\overline{m}_j - \overline{m}_t^{*}}\cdot\abs{I_t^{*}}\\
            \overline{S}_Z\of{f} - \overline{S}_Z\of{f} &= \sum_{j}^{} \sum_{t: I_t^{*}}^{} \pair{\underbrace{\overline{m}_j - \overline{m}_t^{*}}_{= 0 \text{ falls } I_t^{*} = I_j}}\cdot\abs{I_t^{*}}\\
            &= \sum_{j}^{} \sum_{t: I_t^{*}}^{} \pair{\overline{m}_j - \overline{m}_t^{*}}\cdot\abs{I_Z^{*}}\\
            f(x) &= f(y) + f(x) - f(y)\\
            &\leq f(y) + \sup_{s_1, s_2\in I}\set{f(s_1) - f(s_2)}\\
            f(x) &\leq f(y) + 2\norm{f}_{\infty}
            \intertext{genauso}
            f(x) &= f(y) + f(x) - f(y)\\
            &\geq f(y) + \inf_{s_1, s_2\in I}\set{f(s_1) - f(s_2)}\\
            &\geq f(y) - 2\norm{f}_{\infty}\\[10pt]
            \impl \overline{m}_j &= \sup_{s\in I_j}f(x) \leq 2\norm{f}_{\infty} + f(y)\quad\forall y\in I_t^{*}\\
            \impl \overline{m}_j &\leq 2\norm{f}_{\infty} + \sup_{?} f = 2\norm{f}_{\infty} + \overline{m}_z^{*}\\
            \vdots \quad &???
            \intertext{Genauso zeigt man}
            \underline{S}_Z\of{f} - \underline{S}_{Z^{*}}\of{f} &\geq -2\norm{f}_{\infty} l\cdot\delta\\
            \impl \overline{S}_Z\of{f} &\leq \overline{S}_{Z^{*}} + 2\norm{f}_{\infty} l \cdot \delta\\
            \underline{S}_Z\of{f} &\geq \underline{S}_{Z^{*}} - 2 \norm{f}_{\infty} l \cdot \delta\\
            \impl \overline{S}_Z\of{f} - \underline{S}_Z\of{f} &\leq \overline{S}_{Z^{*}}\of{f} + 2\norm{f}_{\infty} l\delta - \pair{\underline{S}_{Z^{*}}\of{f} - 2 \norm{f}_{\infty} l \cdot\delta}\\
            &= ?\\
            &< \frac{\varepsilon}{2} + 4\norm{f}_{\infty} l \cdot\delta
            \intertext{Jetzt wähle $\delta = \frac{\varepsilon}{\delta\pair{\norm{f}_{\infty} + 1}\cdot l}$}
            \impl &\leq \frac{\varepsilon}{2} + 4\norm{f}_{\infty} \cdot \frac{\varepsilon}{\delta\pair{\norm{f}+1}\cdot l} < \frac{\varepsilon}{2} + \frac{\varepsilon}{2} = \varepsilon
        \end{align*}
        sofern um $\Delta\of{z} < \delta$ ist.
    \end{proof}
\end{satz}

\begin{anwendung}
    Es sei $(Z_n)_n$ eine Folge von Zerlegungen von $I$ mit Feinheit $\Delta\of{Z_n}\fromto 0$ für $\ntoinf$. $\xi_n$ seien die Zwischenpunkt von Zerlegung $Z_n = \pair{x_0^n, x_1^n, \dots, x_{k_n}^n}$. Die Riemannnsumme
    \begin{align*}
        S_{Z_n}\of{f, \xi_n} &= \sum_{j=1}^{k_n} f\of{\xi_j^n}\cdot\abs{I_j^n}
    \end{align*}
    konvergiert nach Satz~\ref{satz:temp-9} gegen $J(f)$ falls $f\in \mR\of{I}$.
\end{anwendung}

\begin{bemerkung}[Linearität der Riemannschen Zwischensumme]
    \marginnote{[19. Apr]}
    Seien $Z=\pair{x_0, x_1, \dots, x_k}$ Zerlegung von $I=\interv{a,b}$ und $\xi = \pair{\xi_1, \xi_2, \dots, \xi_k}$ Zwischenpunkt zur Zerlegung $Z$, sodass
    \begin{align*}
        x_{j-1}&\leq \xi_j \leq x_j\quad\fa j=1,\dots, k
        \intertext{Dann ist die Riemannsche Zwischensumme}
        S_Z\of{f}= S_Z\of{f,\xi} &\coloneqq \sum_{j=1}^{k} f\of{\xi_j}\cdot\abs{I_j}\tag{$I_j = \interv{x_j-1, x_j}$}
    \end{align*}
    linear in Bezug zu $f$. Wir werden diese Aussage und weitere interessante Vektorraumeigenschaften des $\mR\of{I}$ später in Satz~\ref{satz:temp-11} noch beweisen.
\end{bemerkung}

\newpage

\begin{korollar} % Korollar 10
    \label{korollar:temp-10}
    Sei $f\in \mB\of{I}$. Dann gilt $f\in \mR\of{I}$ genau dann, wenn für jede Folge $(Z_n)_n$ von Zerlegungen $Z_n$ von $I$ mit Feinheit $\Delta\of{Z_n}\fromto 0$ für $\ntoinf$ und jede Folge $(\xi_n)_n$ von Zwischenpunkten $\xi_n$ zugehörig zu $Z_n$ der Grenzwert $\biglim{\ntoinf} S_{Z_n}\of{f, \xi_n}$ existiert.\\[4pt]
    Darüber hinaus ist in diesem Fall obiger Grenzwert unabhängig von der Wahl der Zerlegung $Z_n$ und der Zwischenpunkten $\xi_n$ und es gilt
    \begin{align*}
        \int_{a}^{b} f\of{x} \dif x &= \lim_{\ntoinf} S_{Z_n}\of{f, \xi_n}\tag{$I=\interv{a,b}$}
    \end{align*}
    \begin{proof}
        \anf{$\impl$} Sei $f\in \mR\of{I}$. Dann gilt nach Satz~\ref{satz:integr-kriterium-1}
        \begin{align*}
            \fa\varepsilon > 0\ex\delta > 0\colon \overline{S}_Z\of{f} - \underline{S}_Z\of{f} &< \varepsilon\quad\fa\text{Zerlegungen } Z \text{ mit } \Delta\of{Z} < \delta
            \intertext{Da $\Delta\of{Z_n}\fromto 0$ für $\ntoinf$ gilt außerdem}
            \impl \ex N\in\N\colon \Delta\of{Z_n} &< \delta\quad\forall n\geq N
            \intertext{und für alle $n\in\N$ gilt}
            \underline{S}_{Z_n}\of{f} \leq \underline{J}\of{f} &= \ov{J}\of{f} \leq \overline{S}_{Z_n}\of{f}\\
            \underline{S}_{Z_n}\of{f} \leq S_{Z_n}\of{f,\xi_n} &\leq \overline{S}_{Z_n}\of{f}\\
            \impl \abs{J\of{f} - S_{Z_n}\of{f, \xi_n}} &< \varepsilon\quad\forall n\geq N
            \intertext{das heißt}
            \lim_{\ntoinf} S_{Z_n}\of{f, \xi_n} &= J\of{f} = \int_{a}^{b} f \dif x
        \end{align*}
        \anf{$\Leftarrow$} \textsc{Schritt 1:} Angenommen $\biglim{\ntoinf} S_{Z_n}\of{f, \xi_n}$ existiert für jede Folge $\pair{Z_n}_n$ von Zerlegungen von $I$ mit $\Delta\of{Z_n}\fromto 0$ und jede Wahl von Zwischenpunkten $\pair{\xi_n}_n$ zu $Z_n$.\\
        Seien $\pair{Z_n^{1}}_n$, $\pair{Z_n^{2}}_n$ zwei solche Folgen von Zerlegungen mit $\pair{\xi_n^1}_n$, $\pair{\xi_n^2}_n$ zugehörigen Folgen von Zwischenpunkten. Sei $\pair{Z_n}_n$ eine neue Folge von Zerlegungen von $I$, wobei $Z_{2k} = Z_k^2$ und $Z_{2k-1} = Z^1_k$, außerdem sei $\xi_{2k} = \xi^2_k$ und $\xi_{2k-1}=\xi^1_k$. Dann wissen wir, dass
        \begin{align*}
            \lim_{\ntoinf} &S_{Z_n}\of{f, \xi_n}
            \intertext{existiert und gilt}
            \lim_{\ntoinf} S_{Z_n}\of{f, \xi_n} &= \lim_{\ntoinf} S_{Z_{2n}}\of{f, \xi_{2n}}\\
            &= \lim_{\ntoinf} S_{Z_{2n-1}}\of{f, \xi_{2n-1}}\\
            &= \lim_{\ntoinf} S_{Z_n^2}\of{f, \xi_n^2}\\
            &= \lim_{\ntoinf} S_{Z_n^1}\of{f, \xi_n^1}
        \end{align*}
        \textsc{Schritt 2:} (Später)
    \end{proof}
\end{korollar}

\newpage

\begin{satz}[$\mR\of{I}$ als Vektorraum] % Satz 11
    \label{satz:temp-11}
    Der Raum $\mR\of{I}$ auf einem kompakten Intervall $I=\interv{a,b}$ ist ein Vektorraum und $J: \mR\of{I}\fromto\R~~f\mapsto J\of{f} = \int_{a}^{b} f \dif x$ ist eine lineare Abbildung. Für $f,g\in \mR\of{I}$ und $\alpha,\beta\in\R$ folgt also $\alpha f + \beta g \in \mR\of{I}$ und $J\of{\alpha f + \beta g} = \alpha J\of{f} + \beta J\of{g}$.
    \begin{proof}
        \textsc{Teil 1:} Sei $h: I\fromto\R$ eine zusätzliche Funktion auf dem Intervall und $Z$ eine Zerlegung von $I$ mit zugehörigen Intervallen $Ij$. Dann gilt
        \begin{align*}
            \overline{m}_j = \sup_{x\in I_j}h(x)&\quad \underline{m}_j = \inf_{y\in I_j} h(y)\\
            \impl \overline{m}_j - \underline{m}_j &= \sup_{x\in I_j}h(x) - \inf_{y\in I_j} h(y)\\
            &= \sup_{x\in I_j} h(x) + \sup_{y\in I_j}\pair{-h\of{y}}\\
            &= \sup_{x,y\in I_j}\pair{h\of{x}-h\of{y}}\\
            &= \sup_{x,y\in I_j}\pair{h(y) - h(x)}\tag{Vertauschen von $x,y$}\\
            &= \sup_{x,y\in I_j}\pair{\abs{h(x)-h(y)}}\\
            \impl \overline{m}_j\of{h} - \underline{m}_j\of{h} &= \sup_{x,y\in I_j}\pair{\abs{h(x)-h(y)}}\tag{1}
            \intertext{Wir wählen $h=\alpha f + \beta g$, wobei $f,g\in \mR\of{I}$ und $\alpha, \beta\in\R$}
            h(x) - h(y) &= \alpha\cdot\pair{f(x)-f(y)} + \beta\cdot\pair{g(x)-g(y)}\\
            \impl \abs{h(x)-h(y)} &\leq \abs{\alpha}\cdot\abs{f(x)-f(y)}+ \abs{\beta}\cdot\abs{g(x)-g(y)}\tag{2}\\
            \overline{m}_j\of{h} - \underline{m}_j\of{h} &= \sup_{x\in I_j} h(x) - \inf_{y\in I_j} h(y)\\
            \annot[{&}]{=}{(1)} \sup_{x,y\in I_j} \pair{\abs{h(x)-h(y)}}\\
            \annot[{&}]{\leq}{(2)} \abs{\alpha}\cdot \sup_{x,y\in I_j}\abs{f(x)-f(y)} + \abs{\beta}\cdot\sup_{x,y\in I_j} \abs{g(x)-g(y)}\\
            &= \abs{\alpha}\cdot\pair{\ov{m}_j\of{f} - \un{m}_j\of{f}} + \abs{\beta}\cdot\pair{\ov{m}_j\of{g} - \un{m}_j\of{g}}\\
            \impl \overline{S}_Z\of{h} - \underline{S}_{Z}\of{h} &= \sum_{j=1}^{k} \pair{\overline{m}_j\of{h} - \underline{m}_j\of{h}]}\cdot\abs{I_j}\\
            &\leq \abs{\alpha}\cdot\sum_{j=1}^{k} \pair{\overline{m}_j\of{f} - \underline{m}_j\of{f}}\cdot\abs{I_j} + \abs{\beta}\cdot\sum_{j=1}^{k} \pair{\overline{m}_j\of{g} - \underline{m}_j\of{g}}\cdot\abs{I_j}\\
            \impl \overline{S}_Z\of{h} - \underline{S}_Z\of{h} &\leq \abs{\alpha}\cdot\pair{\overline{S}_Z\of{f} - \underline{S}_Z\of{f}} + \abs{\beta}\cdot\pair{\overline{S}_Z\of{g} - \underline{S}_Z\of{g}}\tag{3}
            \intertext{Nach Satz~\ref{satz:integr-kriterium-1} und der Riemann-Integrierbarkeit von $f$ und $g$ gilt}
            \impl \fa\varepsilon > 0\ex Z_1\colon \overline{S}_{Z_1}\of{f} - \underline{S}_{Z_1}\of{f} &< \frac{\varepsilon}{2\cdot\pair{1+\abs{\alpha} + \abs{\beta}}}\\
            \fa\varepsilon > 0\ex Z_2\colon \overline{S}_{Z_2}\of{g} - \underline{S}_{Z_2}\of{g} &< \frac{\varepsilon}{2\cdot\pair{1+\abs{\alpha} + \abs{\beta}}}\\
            \intertext{Wähle $Z=Z_1\lor Z_2$ und verwende (3)}
            \impl \overline{S}_Z\of{h} - \underline{S}_Z\of{h} &< \abs{\alpha}\frac{\varepsilon}{2\cdot\pair{1+\abs{\alpha}+\abs{\beta}}} + \abs{\beta}\frac{\varepsilon}{2\cdot\pair{1+\abs{\alpha}+\abs{\beta}}}\\
            &\leq \frac{\varepsilon}{2} +\frac{\varepsilon}{2} = \varepsilon
        \end{align*}
        Nach Satz~\ref{satz:integr-kriterium-1} ist $h=\alpha f + \beta g$ damit Riemann-integrierbar.\\[5pt]
        \textsc{Teil 2:} Für Zwischensummen
        \begin{align*}
            S_Z\of{h, \xi} &= \sum_{j=1}^{k} h\of{\xi_j}\cdot\abs{I_j} =\alpha\cdot S_Z\of{f, \xi} + \beta\cdot S_Z\of{g, \xi}
        \end{align*}
        haben wir bereits Linearität. Für $h,f,g\in \mR\of{I}$ gilt nach Korollar~\ref{korollar:temp-10}
        \begin{align*}
            J\of{h} &= \lim_{n\toinf} S_{Z_n}\of{h, \xi_n}\tag{$\Delta\of{Z_n}\fromto 0$}\\
            &= \lim_{n\toinf}\pair{\alpha\cdot S_{Z_n}\of{f, \xi_n} + \beta\cdot S_{Z_n}\of{g, \xi_n}}\\
            &= \alpha\cdot \lim_{\ntoinf} S_{Z_n}\of{f, \xi_n} + \beta\cdot \lim_{\ntoinf} S_{Z_n}\of{g, \xi_n}\\
            &= \alpha\cdot J\of{f} + \beta\cdot J\of{g}\qedhere
        \end{align*}
    \end{proof}
\end{satz}

\begin{satz}[Kompositionen von integrierbaren Funktionen] % Satz 12
    Seien $f, g\in \mR\of{I}$. Dann gilt
    \begin{enumerate}[label=(\roman*)]
        \item $f\cdot g\in \mR\of{I}$
        \item $\abs{f} \in \mR\of{I}$
        \item Ist außerdem $\abs{g} \geq c > 0$ auf $I$ für ein konstantes $c>0$, so ist auch $\frac{f}{g}\in \mR\of{I}$.
    \end{enumerate}

    \begin{proof}
        \theoremescape
        \begin{enumerate}[label=(\roman*)]
            \item Es sei $h(x) = f(x)\cdot g(x)$ für $x\in I$. Dann gilt
            \begin{align*}
                \abs{h(x) - h(y)} &= \abs{f(x)\cdot g(x) - f(y)\cdot g(y)}\\
                &= \abs{g(x)\cdot \pair{f(x)-f(y)} + f(y)\cdot \pair{g(x)-g(y)}}\\
                &\leq \norm{g}_{\infty}\cdot\abs{f(x)-f(y)} + \norm{f}_{\infty} \cdot \abs{g(x)-g(y)}\tag{1}
                \intertext{Sei $Z$ Zerlegung von $I$ und $I_j$ die entsprechenden Teilintervalle. Dann gilt}
                \overline{S}_Z\of{h} - \underline{S}_Z\of{h} &= \sum_{j=1}^{k} \pair{\overline{m}_j\of{h} - \underline{m}_j\of{h}}\cdot\abs{I_j}\\
                \overline{m}_j\of{h} - \underline{m}_j\of{h} &= \sup_{I_j} h - \inf_{I_j} h = \sup_{x,y\in I_j} \abs{h(x)-h(y)}\\
                \annot[{&}]{\leq}{(1)} \norm{g}_{\infty}\cdot \pair{\overline{m}_j\of{f}- \underline{m}_j\of{f}} + \norm{f}_{\infty}\cdot \pair{\overline{m}_j\of{g} - \underline{m}_j\of{g}}\\
                \impl \overline{S}_Z\of{h} - \underline{S}_Z\of{h} &\leq \norm{g}_{\infty}\cdot\pair{\overline{S}_Z\of{f} - \underline{S}_Z\of{f}} + \norm{f}_{\infty}\cdot\pair{\overline{S}_Z\of{g} - \underline{S}_Z\of{g}}
                \intertext{Für ein $\varepsilon > 0$ gilt nach Satz~\ref{satz:integr-kriterium-1}}
                \exists Z_1\colon \overline{S}_{Z_1}\of{f} - \underline{S}_{Z_1}\of{f} &< \frac{\varepsilon}{2\cdot\pair{1+\norm{g}_{\infty}}}\\
                \exists Z_2\colon \overline{S}_{Z_2}\of{g} - \underline{S}_{Z_2}\of{g} &< \frac{\varepsilon}{2\cdot\pair{1+\norm{f}_{\infty}}}
                \intertext{Es sei $Z\:= Z_1 \lor Z_2$}
                \impl \overline{S}_Z\of{h} - \underline{S}_Z\of{h} &\leq \norm{g}_{\infty} \cdot \frac{\varepsilon}{2\cdot\pair{1+\norm{g}_{\infty}}} +\norm{f}_{\infty}\cdot\frac{\varepsilon}{2\cdot\pair{1+\norm{f}_{\infty}}}\\
                &\leq \frac{\varepsilon}{2} + \frac{\varepsilon}{2} = \varepsilon
                \intertext{Damit gilt $h=f\cdot g\in \mR\of{I}$ nach Satz~\ref{satz:integr-kriterium-1}.\item Für $\abs{f}$ verwenden wir $\abs{\abs{f(x)}- \abs{f(y)}} \leq \abs{f(x) - f(y)}$}
                \impl \overline{m}_j\of{\abs{f}} - \underline{m}_j\of{\abs{f}} &= \sup_{x,y\in I_j}\pair{\abs{\abs{f(x)} - \abs{f(y)}}}\\
                &\leq \sup_{x,y\in I_j} \pair{\abs{f(x)-f(y)}}\\
                &= \overline{m}_j\of{f} - \underline{m}_j\of{f}
                \intertext{wie vorher folgt also $\abs{f} \in \mR\of{I}$. \item Für $\frac{f}{g}$ muss nur $\frac{1}{g}$ betrachtet und die Multiplikationsregel angewendet werden. Es gilt}
                \abs{\frac{1}{g(x)}- \frac{1}{g(y)}} &= \frac{\abs{g(x)-g(y)}}{\abs{g(x)}\cdot\abs{g(y)}} \leq \frac{1}{c^2}\cdot\abs{g(x) - g(y)}\\
                \impl \overline{m}_j\of{\frac{1}{y}}  - \underline{m}_j\of{\frac{1}{y}} &\leq \frac{1}{c^2}\cdot \pair{\overline{m}_j\of{y} - \underline{m}_j\of{y}}
            \end{align*}
            Damit gilt analog zu (ii) die Behauptung.\qedhere
        \end{enumerate}
    \end{proof}
\end{satz}

\begin{beispiel}[Exponentialfunktion]
    \marginnote{[23. Apr]}
    Sei $f: \R\fromto\R~x\mapsto e^{\alpha x}$, $n\in\N$, $I=\interv{a,b}$ und $\alpha\in\R$ mit $\alpha > 0$. Wir betrachten eine äquidistante Zerlegung $Z_n = \pair{x_0^n, x_1^n, \dots, x_k^n}$ mit $x_j^n = a + j\cdot h_n$, wobei $h_n = \frac{b-a}{n} = h = \abs{I_j}$. Da $f$ streng monoton wachsend ist gilt
    \begin{align*}
        \overline{m}_j &= \sup_{I_j} f = f\of{x_j} = f\of{x_j^n} = e^{\alpha x_j}\\
        \underline{m}_j &= \inf_{I_j} f = f\of{x_{j-1}} = f\of{x_{j-1}^n} = e^{\alpha x_{j-1}}\\
        \impl \overline{S}_Z\of{f} &= \overline{S}_{Z_n}\of{f} = \sum_{j=1}^{n} \overline{m}_j \cdot\abs{I_j} = \sum_{j=1}^{n} e^{\alpha x_j} \cdot h\\
        &= h\cdot \sum_{j=1}^{n} e^{\alpha\pair{a+jh}} = h\cdot \sum_{j=1}^{n} e^{\alpha a}\cdot e^{\alpha jh}\\
        &= h\cdot e^{\alpha a}\cdot e^{\alpha h}\cdot \sum_{j=1}^{n} \pair{e^{\alpha h}}^{j-1}\\
        &= h\cdot e^{\alpha a}\cdot e^{\alpha h}\cdot\frac{\pair{e^{\alpha h}}^n - 1}{e^{\alpha h} - 1}\tag{Geometr. Summe}\\
        &= \frac{h}{e^{\alpha h} - 1}\cdot e^{\alpha h}\cdot e^{\alpha a} \cdot \pair{e^{\alpha h\cdot n} - 1}\\
        &= \frac{h_n}{e^{\alpha h_n } - 1}\cdot e^{\alpha h_n}\cdot \pair{e^{\alpha b} - e^{\alpha a}}
        \intertext{Es gilt $\biglim{n\fromto\infty} \frac{e^{\alpha h_n} - 1}{h_n} = \biglim{h\fromto 0} \frac{e^{\alpha h} - 1}{h} = \alpha$ sowie $\biglim{\ntoinf} e^{\alpha h_n} = 1$. Damit folgt}
        \lim_{n\fromto\infty} \overline{S}_{Z_n}\of{f} &= \frac{1}{\alpha}\cdot\pair{e^{\alpha b} - e^{\alpha a}}
        \intertext{Wir betrachten die Untersumme}
        \underline{S}_Z &= \underline{S}_{Z_n} = \sum_{j=1}^{n} \underline{m}_j \cdot \abs{I_j} = h\cdot \sum_{j=1}^{n} \pair{e^{\alpha x_{j-1}}}\\
        &= h\cdot e^{\alpha a}\cdot \sum_{j=1}^{n} \pair{e^{\alpha h}}^{j-1} = h\cdot e^{\alpha a} \sum_{j=0}^{n-1} \pair{e^{\alpha h}}^j\\
        &= h\cdot e^{\alpha a} \frac{\pair{e^{\alpha h}}^n - 1}{e^{\alpha h} - 1}\\
        &= \frac{h}{e^{\alpha h} - 1}\cdot e^{\alpha a}\cdot\pair{e^{\alpha\pair{b-a}} - 1} \fromto \frac{1}{\alpha}\cdot \pair{e^{\alpha b} - e^{\alpha a}}
        \intertext{Also gilt $f\in \mR\of{I}$ sowie}
        \int_{a}^{b} e^{\alpha x} \dif x &= \frac{1}{\alpha}\cdot \pair{e^{\alpha b} - e^{\alpha a}}
    \end{align*}
\end{beispiel}

\begin{beispiel}[Polynome]
    \label{beispiel:int-polynom}
    Es sei $f: \linterv{0, \infty}\fromto \linterv{0, \infty}$, $x\mapsto x^{\alpha}$ $\pair{\alpha\neq -1}$. Dann $f\in \mR\of{I}$ und
    \begin{align*}
        \int_{a}^{b} x^{\alpha} \dif x &= \frac{1}{\alpha+1}\pair{b^{\alpha +1} - a^{\alpha +1}}
    \end{align*}
    \begin{proof}[Beweisansatz]
        Wir wählen eine geometrische Zerlegung. Sei $q = q_n = \sqrt[n]{\frac{b}{a}}$, $Z= Z_n = \pair{x_0^n, x_1^n, \dots, x_n^n}$, $I_j = \interv{x_{j-1}, x_j}$, $x_j = x_j^n = a\cdot q^j$
        \begin{align*}
            \abs{I_j} &= \Delta x_j = x_j - x_{j-1} = a\cdot q^{j} - a \cdot q^{j-1}\\
            &= a\cdot q^{j-1}\cdot\pair{q-1} \leq b\cdot\pair{q_n - 1}\fromto 0 \text{ für } \ntoinf
            \intertext{Beobachtung: Ober- und Untersumme lassen sich \anf{leicht} mittels geometrischer Summen ausrechnen}
            \overline{m}_j &= \sup_{I_j} f = \pair{x_j}^{\alpha} = \pair{a\cdot q^j}^{\alpha}\tag{Nach Monotonie}\\
            \underline{m}_j &= \inf_{I_j} f = \pair{x_{j-1}}^{\alpha} = \pair{a\cdot q^{j-1}}^{\alpha}\\
            \un{S}_Z\of{f} &= \un{S}_{Z_n}\of{f} = \sum_{j=1}^{n} \un{m}_j\cdot\abs{I_j} = \sum_{j=1}^{n} \pair{a\cdot q^{j-1}}^{\alpha}\cdot a\cdot q^{j-1}\cdot\pair{q-1}\\
            &= \pair{q-1}\cdot a^{\alpha + 1} \cdot \sum_{j=1}^{n} q^{\pair{\alpha +1}\cdot\pair{j-1}}
        \end{align*}
        Damit erhalten wir eine geometrische Summe, dessen Grenzwert sich gut ermitteln lässt.
    \end{proof}
\end{beispiel}

\begin{uebung}
    Bestimmen Sie den Grenzwert der Ober- und Untersummen aus Beispiel~\ref{beispiel:int-polynom}, um die Riemann-Integrierbarkeit der Polynome nachzuweisen.
\end{uebung}

\begin{satz}[Monotonie des Integrals] % Satz 13
    \label{satz:temp-13}
    Seien $f, g\in \mR\of{I}$, $I=\interv{a,b}$. Dann erfüllt das Integral Monotonieeigenschaften. Das heißt konkret
    \begin{enumerate}[label=(\roman*)]
        \item Wenn $\fa x\in\R\colon f\of{x}\leq g\of{x}$, dann folgt
        \begin{align*}
            \int_{a}^{b} f\of{x} \dif x &\leq \int_{a}^{b} g\of{x} \dif x\numbereq{eq:int-monton}
            \intertext{\item Insbesondere gilt für $f\in\mR\of{I}$ beliebig}
            \abs{\int_{a}^{b} f\of{x} \dif x} &\leq \int_{a}^{b} \abs{f\of{x}} \dif x\numbereq{eq:int-abs-mon}
            \intertext{\item Sowie}
            \abs{\int_{a}^{b} f\cdot g \dif x} &\leq \sup_{I} \abs{f} \cdot \int_{a}^{b} \abs{g} \dif x
        \end{align*}
    \end{enumerate}

    \begin{proof}
        \theoremescape
        \begin{enumerate}[label=(\roman*)]
            \item Sei $h=g-f\geq 0$. Dann gilt nach Satz~\ref{satz:temp-11} $h\in \mR\of{I}$ und $\int_{a}^{b} h \dif x \geq 0$
            \begin{align*}
                \impl 0\leq \int_{a}^{b} h \dif x &= \int_{a}^{b} g \dif x + \int_{a}^{b} \pair{-f} \dif x = \int_{a}^{b} g \dif x - \int_{a}^{b} f \dif x\\
                \impl \int_{a}^{b} f \dif x &\leq \int_{a}^{b} g \dif x
                \intertext{\item Es gilt $\pm f \leq \abs{f}$. Damit folgt aus (\ref{eq:int-monton})}
                \int_{a}^{b} \pair{\pm f} \dif x &\leq \int_{a}^{b} \abs{f} \dif x\\
                \impl \abs{\int_{a}^{b} f \dif x} &= \max\pair{\int_{a}^{b} f \dif x, - \int_{a}^{b} f \dif x}\leq \int_{a}^{b} \abs{f} \dif x
                \intertext{\item Nach (\ref{eq:int-abs-mon}) gilt}
                \abs{\int_{a}^{b} fg \dif x} \leq \int_{a}^{b} \abs{fg} \dif x &\leq \int_{a}^{b} \pair{\sup_I \abs{f}}\abs{g} \dif x = \sup_I\of{\abs{f}}\cdot \int_{a}^{b} \abs{g} \dif x\qedhere
            \end{align*}
        \end{enumerate}
    \end{proof}
\end{satz}

\newpage

\begin{satz}[Cauchy-Schwarz] % Satz 14
    Seien $f, g\in \mR\of{I}$ und $I=\interv{a,b}$. Dann gilt
    \begin{align*}
        \abs{\int_{a}^{b} fg \dif x}^2 &\leq \pair{\int_{a}^{b} \abs{fg} \dif x}^2\\
        &\leq \int_{a}^{b} \abs{f}^2 \dif x \cdot \int_{a}^{b} \abs{g}^2 \dif x
        \intertext{mit}
        \norm{f} &= \sqrt{\int_{a}^{b} \abs{f}^2 \dif x}\\
        \impl \abs{\int_{}^{} fg \dif x} &\leq \norm{f}\cdot {g}
    \end{align*}
    \begin{proof}
        \begin{align*}
            0 &\leq \pair{a\pm b}^2 = a^2\pm 2ab + b^2\\
            \impl \mp ab \leq \frac{a^2+b^2}{2}\\
            \impl \abs{ab} &\leq \frac{1}{2}\pair{a^2+b^2}
            \intertext{$t > 0$}
            \abs{\alpha\beta} &= \abs{t\alpha - \frac{\beta}{t}} \leq \frac{1}{2}\pair{t\alpha^2 + \frac{1}{t}\beta^2}\\
            \abs{\int_{a}^{b} fg \dif x} &\leq \int_{a}^{b} \abs{f\of{x}}\abs{g\of{x}} \dif x\\
            &\leq \frac{1}{2}\pair{t\cdot \underbrace{\int_{a}^{b} \abs{f\of{x}}^2 \dif x}_{A} + \frac{1}{t} \underbrace{\int_{a}^{b} \abs{g}^2 \dif x}_{B}}\\
            &\leq \frac{1}{2}\pair{t\cdot\abs{f\of{x}}^2+ \frac{1}{t}\abs{g\of{x}}^2} = \frac{1}{2}\pair{tA + \frac{1}{t}B}
            \intertext{Frage: Welches $t> 0$ maximiert $h$?}
            A = 0 \impl h\of{t} &= \frac{1}{2t}B \fromto 0 \text{ für } \ntoinf\\
            B = 0\impl h\of{t} &= \frac{1}{2}A \fromto 0 \text{ für } \ntoinf\\
            \impl \lim_{t\fromto} h\of{t} &= \infty, \lim_{t\searrow 0} h\of{t} = \infty
            \intertext{Minimum existiert für ein $t_0 > 0$ und es gilt $0=h'\of{t_0}$}
            \impl 0 &= \frac{1}{2}\pair{A - \frac{1}{t_0}B}\\
            \impl \pair{t_0}^2 &= \frac{B}{A}\quad t_0 = \sqrt {\frac{b}{A}}\\
            \impl \inf_{\pair{0, \infty}} h\of{t} &= \frac{1}{2}t_0\pair{A + \frac{1}{t_0^2}B}\\
            &= \frac{1}{2}\sqrt {\frac{b}{A}} \pair{A + \frac{A}{B}B} = \sqrt{AB}\qedhere
        \end{align*}
    \end{proof}
\end{satz}

\begin{bemerkung}
    \begin{align*}
        \sprod{f,g} &= \int_{a}^{b} f\of{x}g\of{x} \dif x\\
        \norm{f} &\coloneqq\sqrt{\int_{a}^{b} \abs{f}^2 \dif x} \text{ ist eine Norm}\\
        \impl \abs{\sprod{f,g}} &\leq \norm{f}\norm{g}
    \end{align*}
\end{bemerkung}

\begin{satz} % Satz 15
    \label{satz:temp-15}
    Sei $\mC\of{I} = \mC\of{\interv{a,b}}$ der Raum der stetigen reellen Funktionen auf einem $I=\interv{a,b}$. Es gilt $\mC\of{I} \subseteq \mR\of{I}$.
    \begin{proof}
        $I=\interv{a,b}$ ist kompakt und $f: \interv{a,b}\fromto\R$ ist stetig und damit auch gleichmäßig stetig. Das heißt
        \begin{align*}
            \fa\varepsilon > 0\ex\delta > 0\colon \abs{f\of{x} - f\of{y}} &<\delta\quad\fa x,y\in I \text{ mit } \abs{x-y} < \delta
        \end{align*}
        Sei $Z$ eine Zerlegung von $I$ mit $\Delta\of{Z} < \delta$. $I_j = \interv{x_{j-1}, x_j}$ und $Z=\pair{x_0, x_1, \dots, x_k}$. Dann gilt
        \begin{align*}
            \overline{m}_j - \underline{m}_j &= \sup_{x\in I_j} f\of{x} - \inf_{y\in I_j} f\of{y}\\
            &= \sup_{x,y\in I_j} \abs{f\of{x} - f\of{y}} = \sup_{x,y\in I_j}\pair{f\of{x} - f\of{y}}
            \intertext{Da $\abs{x-y} \leq \abs{I_j} < \delta$ gilt}
            \overline{m}_j - \underline{m}_j &\leq \varepsilon\\
            \impl \overline{S}_Z\of{f} - \underline{S}_{Z}\of{f} &= \sum_{j=1}^{n} \pair{\overline{m}_j - \underline{m}_j}\cdot\abs{I_j}\\
            &\leq \varepsilon \sum_{j=1}^{n} \abs{I_j} = \varepsilon \cdot \abs{I} = \varepsilon\cdot\pair{b-a}\\
            \impl 0 &\leq \overline{J}\of{f} - \underline{J}\of{f}\leq \overline{S}_Z\of{f} - \underline{S}_Z\of{f}\\
            &\leq \varepsilon\pair{b-a}\quad\forall\varepsilon > 0\\
            \impl \overline{J}\of{f} &= \underline{J}\of{f} \impl f\in \mR\of{I}
        \end{align*}
    \end{proof}
\end{satz}

\begin{definition}
    Eine Funktion $f: I\fromto\R$ auf $I=\interv{a,b}$ heißt stückweise stetig, falls es eine Zerlegung $Z = \pair{x_0, x_1, \dots, x_k}$ von $I$ gibt so, dass $f$ auf jedem der offenen Intervalle $\pair{x_{j-1}, x_j}$ stetig ist und die einseitigen Grenzwerte
    \begin{align*}
        f\of{a+} &= \lim_{x\fromto a+} f\of{x}, f\of{b-} = \lim_{x\fromto b-} f\of{x}\\
        f\of{x_j-} &= \lim_{x\fromto x_j-} f\of{x}, f\of{x_j+} = \lim_{x\fromto x_j+} f\of{x}
    \end{align*}
    für $j=1, \dots, k-1$ existieren.\\
    $f\of{\pair{x_{j-1}, x_j}}$ können zu stetigen Funktionen auf $I_j=\interv{x_{j-1}, x_j}$ fortgesetzt werden. Wir nennen diese Klasse von Funktionen $\mPC\of{I}$\footnote{Piecewise continuos function in $I$}.
\end{definition}

\begin{satz} % Satz 17
    \label{satz:temp-17}
    Es gilt $P\mC\of{I} \subseteq \mR\of{I}$. $I=\interv{a,b}$. Ist $Z=\pair{x_0, \dots, x_k}$ eine Zerlegung von $f\in \mPC\of{I}$ und $f$ stetig auf $\pair{x_{j-1}, x_j}~\fa j$ und $f_j$ eine stetige Fortsetzung von $f\vert_{\pair{x_{j-1}, x_j}}$ auf $I_j = \interv{x_{j-1}, x_j}$. So gilt
    \begin{align*}
        \int_{a}^{b} f\of{x} \dif x &= \sum_{l=1}^{k} \int_{x_{l-1}}^{x_l} f_l\of{x}\dif x
    \end{align*}
    \begin{proof}
        Arbeite auf $I_l = \interv{x_{l-1}, x_{l}}$ dann ist $f_l$ stetig nach Satz~\ref{satz:temp-15} und summiere zusammen. (Details selber machen).
    \end{proof}
\end{satz}

\begin{bemerkung}[Treppenfunktion]
    Ist $f$ stückweise konstant auf $I$. Das heißt es existiert eine Zerlegung $Z=\pair{x_0, \dots, x_{\nu}}$ von $I$ mit $f$ ist konstant auf $\pair{x_{k-1}, x_k}~\fa k=1, \dots, \nu$. So heißt $f$ Treppenfunktion. Schreiben $\mJ\of{I}$ für die Klasse der Treppenfunktionen.
\end{bemerkung}

\begin{satz} % Satz 18
    \label{satz:temp-18}
    \marginnote{[26. Apr]}
    Sei $I=\interv{a,b}$, $f: I\fromto\R$ mit den folgenden Eigenschaften
    \begin{enumerate}[label=(\alph*)]
        \item In jedem Punkt $x\in\pair{a,b}$ existieren die rechts- und linksseitigen Grenzwerte.
        \item In $a$ existiert der rechtsseitige und in $b$ der linksseitige Grenzwert.
    \end{enumerate}
    Dann gilt $f\in \mR\of{I}$.\\
    Zum Beweis dieses Satzes benötigen wir zunächst das folgende Approximationslemma~\ref{lemma:temp-19}.
\end{satz}

\begin{bemerkung}
    Insbesondere erfüllt $P\mC\of{I}$ die Bedingungen a) und b) aus Satz~\ref{satz:temp-18}.
\end{bemerkung}

\begin{lemma} % Lemma 19
    \label{lemma:temp-19}
    Sei $f: I\fromto\R$ eine Funktion, die die Bedingungen aus Satz~\ref{satz:temp-18} erfüllt. Dann gibt es eine Folge $(\varphi_n)_n$ von Treppenfunktionen $\varphi_n: I\fromto\R$, die gleichmäßig gegen $f$ konvergiert. Das heißt
    \begin{align*}
        \lim_{\ntoinf} \norm{f-\varphi_n}_{\infty} &= \lim_{\ntoinf} \sup_{x\in\interv{a,b}} \abs{f(x)-\varphi_n(x)} = 0
        \intertext{Also}
        \fa\varepsilon > 0\ex\text{Treppenfunktion }&\varphi: I\fromto\R \text{ mit } \norm{f-\varphi}_{\infty} = \sup_{x\in I}\abs{f(x) - \varphi(x)} < \varepsilon
    \end{align*}
    \begin{proof}
    (Später)
    \end{proof}
\end{lemma}
\horizontalline
Mithilfe dieses Lemmas können wir nun Satz~\ref{satz:temp-18} beweisen.
\begin{proof}
    Sei $f: \interv{a,b}\fromto\R$ wie in Satz~\ref{satz:temp-18} verlangt und $\varepsilon > 0$, sowie $\varphi: I\fromto\R$ Treppenfunktion mit $\norm{f-\varphi}_{\infty} < \frac{\varepsilon}{2}$. Wir definieren $\Psi_1\coloneqq \varphi - \frac{\varepsilon}{2}$, $\Psi_2 = \varphi + \frac{\varepsilon}{2}$ auch als Treppenfunktionen.
    Dann gilt $\Psi_1 = \varphi - \frac{\varepsilon}{2} \leq f$ und $\Psi_2 \geq f$. Für alle Zerlegungen $Z$ von $I$ mit
    \begin{align*}
        \un{S}_Z\of{\Psi_1} &\leq \un{S}_Z\of{f}\\
        \impl \un{S}_Z\of{f} \geq \un{S}_Z\of{\varphi - \frac{\varepsilon}{2}} &= \un{S}_Z\of{\varphi} - \frac{\varepsilon}{2}\cdot\abs{I} = \un{S}_Z\of{\varphi} - \frac{\varepsilon}{2}\pair{b-a}
        \intertext{Analog gilt}
        \ov{S}_Z\of{\varphi} + \frac{\varepsilon}{2}\pair{b-a} &\geq \ov{S}_Z\of{f}
        \intertext{Damit folgt insgesamt}
        \underline{S}_Z\of{\varphi} - \frac{\varepsilon}{2}\pair{b-a} &\leq \underline{S}_Z\of{f} \leq \un{J}\of{f}\\
        \ov{S}_Z\of{\varphi} + \frac{\varepsilon}{2}\pair{b-a} &\geq \ov{S}_Z\of{f} \leq \ov{J}\of{f}\\
        \intertext{Da $\varphi$ eine Treppenfunktion ist, ist $\varphi\in P\mC\of{I}\subseteq \mR\of{I}$. Also existiert eine Folge $(z_n)_n$ von Zerlegungen von $I$ mit}
        \lim_{\ntoinf} \ov{S}_{Z_n}\of{\varphi} &= \lim_{\ntoinf}\un{S}_{Z_n}\of{\varphi} = \int_{a}^{b} \varphi\of{x} \dif x
        \intertext{(sofern $\Delta\of{Z_n}\fromto 0$ für $\ntoinf$)}
        \impl \ov{J}\of{f} - \un{J}\of{f} &\leq \ov{S}_{Z_n}\of{\varphi} + \frac{\varepsilon}{2}\pair{b-a} - \pair{\un{S}_{Z_n}\of{\varphi} - \frac{\varepsilon}{2}\pair{b-a}}\\
        &= \ov{S}_{Z_n}\of{\varphi} - \un{S}_{Z_n}\of{\varphi} + \varepsilon\pair{b-a}\\
        &\fromto_{\ntoinf} \int_{a}^{b} \varphi(x) \dif x - \int_{a}^{b} \varphi\of{x} \dif x + \varepsilon\pair{b-a}\\
        &= \varepsilon\pair{b-a}\\
        \impl \ov{J}\of{f} - \un{J}\of{f} &\leq \varepsilon\pair{b-a}\quad\forall\varepsilon > 0\\
        \impl \ov{J}\of{f} - \un{J}\of{f} &\leq 0\\
        \impl \ov{J}\of{f} &= \un{J}\of{f}\\
        \impl f&\in \mR\of{I}\qedhere
    \end{align*}
\end{proof}

\begin{bemerkung}
    Welche $f\in \mB\of{I}$ sind genau Riemann-integrierbar?
\end{bemerkung}

\begin{definition}[Nullmenge]
    Eine Menge $N\subseteq\R$ heißt Nullmenge, falls zu jedem $\varepsilon > 0$ höchstens abzählbar viele Intervalle $I_1, I_2, \dots$ existieren mit
    \begin{align*}
        N \subseteq \bigcup_{j} I_j\tag{$I_j$ überdecken $N$}
        \intertext{und}
        \sum_{j}^{} \abs{I_j} < \varepsilon
    \end{align*}
\end{definition}

\begin{beispiel}
    $\Q$ ist eine Nullmenge.
    \begin{align*}
        \Q&\subseteq \bigcup_{j\in \N} I_j
        \intertext{Nehme $\varepsilon > 0$}
        \mathcal{Q} &= \set{q_j \middle | j\in \N}
        \intertext{Zu $q_j$ nehme  $I_J= \interv{q_j - \frac{\varepsilon}{2}, q_j + \frac{\varepsilon}{2}}$}
        q_j \in I_j\quad\abs{I_j} &= \varepsilon 2^{-j}\\
        \sum_{j\in\N}^{} \abs{I_j} &= \varepsilon \sum_{j=1}^{\infty} 2^{-j}\\
        &= \varepsilon \cdot \frac{1}{2-1} = \varepsilon
    \end{align*}
\end{beispiel}

% Alle abzählbaren Mengen sind Nullmenge

\begin{definition}
    Eine Funktion $f: I\fromto\R$ heißt fast überall stetig auf $I$, falls die Menge der Unstetigkeitsstellen von $f$ eine Nullmenge ist.
\end{definition}

\begin{genv}[Lebesgue'sches Integrabilitätskriterium]
    $\mR\of{I} = \set{f\in \mB\of{I}: f \text{ ist fast überall stetig auf } I}$
\end{genv}

\begin{bemerkung}
    Sei $f$ wie in Satz~\ref{satz:temp-18}. Dann ist die Menge der Unstetigkeitsstelle von $f$ höchstens abzählbar, also eine Nullmenge.\\
    Ist $f\in P\mC\of{I}$ so ist die Menge der Unstetigkeitsstellen endlich.
\end{bemerkung}

\begin{proof}[Beweis von Lemma~\ref{lemma:temp-19}]
    Wir führen einen Widerspruchsbeweis. Angenommen die Aussage stimmt nicht, dann existiert ein $\varepsilon_0 > 0$ sowie ein $f: I\fromto\R$ wie in Satz~\ref{satz:temp-18}, sodass
    \begin{align*}
        \fa\text{Treppenfunktionen }\varphi: I\fromto\R\colon \norm{f-\varphi}_{\infty} = \sup_{x\in\interv{a,b}} \abs{f(x) - \varphi(x)} \geq \varepsilon_0 > 0
    \end{align*}
    \textsc{Schritt 1:} $I_1=\interv{a,b}$, $a_1=a$, $b_1=b$. Dann weiter mit Divide \& Conquer:
    \begin{align*}
        \sup_{I_1} \abs{f-\varphi} &\geq\varepsilon_0
        \intertext{Behauptung: Es existiert eine Folge $(I_n)_n$ von Intervallschachtelungen $I_{n+1}\subseteq I_n$ mit $\abs{I_n} = b-a\fromto 0$ für $\ntoinf$ mit}
        \sup_{x\in I_n}\abs{f(x) - \varphi(x)} &\geq \varepsilon_0\quad \forall n\in\N \text{ und alle Treppenfunktionen } \varphi \text{ (auf $I_n$)}\tag{*}
        \intertext{Beweis: Angenommen $I_n = \interv{a_n, b_n}$ ist gegeben und erfüllt die obige Bedingung}
        M_N &= \frac{b_n+a_n}{2}\\
        \impl \sup_{x\in\interv{a_n, M_n}} \abs{f(x) - \varphi(x)} &\geq \varepsilon_0 \text{ oder } \sup_{x\in\interv{M_n, b_n}} \abs{f(x) - \varphi(x)} \geq \varepsilon_0\tag{Für alle Treppenfunktionen $\varphi$}
        \intertext{Im ersten Fall wählen wir die linke Hälfte des Intervalls, also $a_{n+1} = a_n$, $b_{n+1} = M_n$. Im zweiten Fall die rechte Hälfte, also $a_{n+1} = M_n$, $b_{n+1} = b_n$. Damit gilt im Sinne der Intervallhalbierung}
        \impl I_{n+1} &\subseteq I_{n}
        \intertext{sowie}
        b_n - a_n &= \frac{1}{2}\pair{b_{n-1} - a_{n-1}} \leq \frac{1}{2^n}\pair{b-a} \fromto 0
        \intertext{Nehme $c_n\subseteq I_n$}
        a = a_1 \leq a_2 \leq \dots \leq a_n &\leq b_n \leq b_{n-1} \leq\dots\leq b_1 = b\\
        \lim_{\ntoinf} a_n \text{ existiert und } &\lim_{\ntoinf} b_n \text{ texistiert aufgrund der monotonen Konvergenz}
        \intertext{und}
        \lim_{\ntoinf} a_n&= \lim_{\ntoinf} b_n \eqqcolon \xi \tag{da $b_n-a_n\fromto 0$}\\
        \impl \forall n\in\N\colon a_n &\leq \xi \leq b_n\\
        \impl a_n &\leq \xi\quad\forall n\in\N
        \intertext{Analog ergibt sich}
        b_n &\geq\xi\quad\forall n\in\N\\
        \impl \xi \in I_n &= \interv{a_n, b_n}\quad\forall n\in\N\\
        \impl \bigcap_{n\in\N} I_n &= \set{\xi}
    \end{align*}
    \textsc{Schritt 2:} Angenommen $a < \xi < b$. Dann ist
    \begin{align*}
        c_l &= f\of{\xi-} = \lim_{x\fromto\xi-} f\of{x}\\
        c_r &= f\of{\xi+} = \lim_{x\fromto\xi+} f\of{x}
        \intertext{Nehmen $\delta > 0$}
        \abs{f\of{x} - c_l} &< \varepsilon_0\quad \xi-\delta \leq x \leq \xi\\
        \abs{f\of{x} - c_r} &< \varepsilon_0\quad \xi < x \leq \xi+\delta
        \intertext{Wir definieren $\varphi: \interv{\xi-\delta, \xi + \delta}$ durch}
        \varphi\of{x} &\coloneqq \begin{cases}
                                     c_r &\xi < x < \xi + \delta\\
                                     f\of{x} &x = \xi\\
                                     c_l &\xi - \delta < x < \xi + \delta
        \end{cases}
        \intertext{und}
        \sup_{\xi-\delta < x \leq \xi + \delta} \abs{f(x) - \varphi\of{x}} &< \varepsilon_0 \tag{**}
    \end{align*}
    Aber $I_n\subseteq\interv{\xi-\delta, \xi+\delta}$ für fast alle $n\in\N$. Für $n$ groß genug ist (**) im Widerspruch zu (*). Damit folgt die Aussage des Lemmas.
\end{proof}

\begin{satz} % Satz 20
    \label{satz:temp-20}
    Seien $f, g\in \mR\of{I}$ ???.
\end{satz}

\begin{lemma} % Lemma 21
    \label{lemma:temp-21}
    Seien $f,g\in \mR\of{I}$ und gebe es eine Menge $G\subseteq I$ welche in $I$ dicht liegt und für die $f(x) = g(x)~\fa x\in G$ gilt. Dann folgt $\int_{a}^{b} f(x)\dif x = \int_{a}^{b} g(x)\dif x$
\end{lemma}

\subsection{[*] Mittelwertsätze der Integralrechnung}

\begin{definition}
    Sei $f\in \mR\of{I}$, $I=\interv{a,b}$. Dann ist
    \begin{align*}
        \fint_I f\of{x} \dif x = \fint_a^{b} f(x) \dif x &\coloneqq \frac{1}{b-a} \int_{a}^{b} f(x) \dif x
        \intertext{definiert als der Mittelwert von $f$ über $I$. Wir schreiben auch}
        \overline{f}_I &= \fint_{a}^{b} f(x) \dif x
    \end{align*}
\end{definition}

\begin{satz} % Satz 22
    \label{satz:temp-22}
    Es sei $I=\interv{a,b}$, $f\in \mC\of{I}$. Dann gilt
    \begin{align*}
        \ex\xi\colon a < \xi < b \text{ mit } f\of{\xi} = \fint_{a}^{b} f(x) \dif x
    \end{align*}
    \begin{proof}
        \begin{align*}
            \ov{m} &= \sup_{I} f = \max_{I} f\\
            \un{m} &= \inf_{I} f = \min_{I} f
            \intertext{Nach Satz~\ref{satz:temp-13} gilt}
            \un{m} &\leq f(x) \leq \ov{m}\quad\forall x \in I\\
            \impl \un{m}\pair{b-a} &= \int_{a}^{b} \underline{m} \dif x \leq \int_{a}^{b} f(x) \dif x\leq \int_{a}^{b} \overline{m} \dif x = \ov{m}\pair{b-a}\\
            \impl \un{m} &\leq \fint_{a}^{b} f(x) \dif x \leq \ov{m}
            \intertext{Ist $\un{m} = \ov{m} \impl f$ ist konstant auf $\interv{a,b}$}
            \impl \un{m} = \ov{m} = \fint_{a}^{b} f(x) \dif x
            \intertext{und $\forall a < \xi < b$ ist $f\of{x} = \un{m}$. Damit gilt die Behauptung. Sei also $\un{m} < \ov{m}$. Dann folgt aus der Stetigkeit von $f$, dass $x_1$ und $x_2$ in $I$ existieren, sodass $f(x_1) = \un{m}$ und $f(x_2) = \ov{m}$ mit $x_1\neq x_2$. Außerdem folgt aus $\un{m} < \ov{m}$, $f\in \mC\of{I}$ auch}
            \un{m} &\leq \fint_{a}^{b} f\of{x} \dif x < \ov{m}
            \intertext{Nach dem Zwischenwertsatz für stetige Funktionen folgt}
            \impl \ex\xi \text{ zwischen } x_1, x_2 \text{ mit } f\of{x} &= \fint_{a}^{b} f(x) \dif x\qedhere
        \end{align*}
    \end{proof}
\end{satz}

\begin{satz}[Verallgemeinerung des vorherigen Satzes] % Satz 23
    \label{satz:temp-23}
    \marginnote{[30. Apr]}
    Es sei $I=\interv{a,b}$, $f\in\mC\of{I}$, $p\in\mR\of{I}$. Falls $p\geq 0$ folgt $\exists\xi$ mit $a<\xi<b$ und
    \begin{align*}
        \int_{a}^{b} f\of{x}p\of{x} \dif x &= f\of{\xi}\cdot \int_{a}^{b} p\of{x} \dif x\numbereq{eq:mittelwertsatz-2}
    \end{align*}
    \begin{proof}
        Angenommen $ \int_{a}^{b} p\of{x} \dif x = 0$
        \begin{align*}
            \impl \abs{\int_{a}^{b} f\of{x}p\of{x} \dif x} &\leq \sup_{x\in I} \int_{a}^{b} \abs{p\of{x}} \dif x = 0
        \end{align*}
        Damit gilt (\ref{eq:mittelwertsatz-2}) für alle $a < \xi< b$.\\
        Ist $ \int_{a}^{b} p\of{x} \dif x > 0$, dann definieren wir ein neues Mittel:
        \begin{align*}
            \text{Mittel}\of{f} &\coloneqq \frac{1}{\int_{a}^{b} p\of{x}\dif x} \cdot \int_{a}^{b} f\of{x}p\of{x} \dif x
        \end{align*}
        Durch scharfes Hinschauen folgt dann die Aussage aus dem Beweis des vorherigen Satzes.
    \end{proof}
\end{satz}

\newpage

    \section{Orientiertes und vektorwertiges Riemann-Integral}
\thispagestyle{pagenumberonly}

\subsection{Das orientierte Riemann-Integral}

Sei $I=\interv{a,b}$; $a',b'\in I$ mit $a'< b'$ und $I'=\interv{a', b'}$. Wenn $f\in\mR\of{I}$, ist dann auch $f\in\mR\of{I'}$?\\
Ist also die Einschränkung $\varphi\coloneqq f\vert_{I'}: I'\fromto\R,~x\mapsto f\of{x}$ Riemann-integrierbar?

\begin{satz} % Satz 1
    \label{satz:einschr-integrierbar}
    Ist $f\in\mR\of{I}$ und $I'=\interv{a', b'}\subseteq I = \interv{a,b}$, so ist $f\vert_{I'} \in\mR\of{I}$.
    \begin{proof}
        \textsc{Schritt 1}: Angenommen $I' = \interv{a, b'}$ (also $a' = a$). Dann folgt aus der Riemann-Integrierbarkeit von $f$ und Satz~\ref{satz:integr-kriterium-1}, dass
        \begin{align*}
            \forall\varepsilon > 0\ex \text{Zerlegung } &Z \text{ von } I\colon \ov{S}_Z\of{f} - \un{S}_Z\of{f} < \varepsilon
            \intertext{Sei $Z_0\coloneqq\pair{a, b', b}$ eine Zerlegung und $Z_1 = Z_0\lor Z$ die gem. Verfeinerung mit $Z_1 = \pair{x_0, x_1, \dots, x_k}$. Dann gilt $x_0 = a$, $x_k = b$ und $\exists l\in\set{1, \dots, k-1}\colon x_l = b'$. Dann ist $Z' = \pair{x_0, x_1, \dots, x_l}$ eine Zerlegung von $I'$ mit zugehörigen Intervallen $I_j = \interv{x_{j-1}, x_j}$ für ($j=1,\dots, l$). Wir definieren $\varphi = f\vert_{I'}$. Dann folgt}
            \ov{m}_j\of{f} &= \sup_{I_J} f = \sup_{I_j} \varphi\quad\forall 1 \leq j \leq l\\
            \un{m}_j\of{f} &= \inf_{I_j} f = \inf_{I_j} \varphi\quad\forall 1 \leq j \leq l\\
            \ov{S}_Z\of{\varphi} - \un{S}_Z\of{\varphi} &= \sum_{j=1}^{l} \pair{\ov{m}_j\of{\varphi} - \un{m}_j\of{\varphi}}\cdot\abs{I_j}\\
            &\leq \sum_{j=1}^{k} \pair{\ov{m}_j\of{f} - \un{m}_j\of{f}}\cdot\abs{I_j}\\
            &= \ov{S}_Z\of{f} - \un{S}_Z\of{f} < \varepsilon
        \end{align*}
        Damit gilt die Aussage für $I' = \interv{a, b'}$ nach Satz~\ref{satz:integr-kriterium-1}.\\[5pt]
        \textsc{Schritt 2:} Sei $b' = b$, $a < a' < b$. Dann funktioniert der Beweis analog zu \textsc{Schritt 1}.\\[5pt]
        \textsc{Schritt 3:} Sei $a < a' < b' < b$ mit $ f\in\mR\of{\interv{a,b}}$. Dann folgt aus \textsc{Schritt 1}, dass
        \begin{align*}
            \varphi_1\coloneqq f\vert_{\interv{a,b'}} &\in\mR\of{\interv{a, b'}}
            \intertext{Außerdem gilt nach \textsc{Schritt 2}}
            \varphi_2\coloneqq \varphi_1\vert_{\interv{a', b'}} &\in\mR\of{\interv{a', b'}}\\
            \impl f\vert_{I'}&\in \mR\of{I}\qedhere
        \end{align*}
    \end{proof}
\end{satz}

\begin{definition}[Integral auf Teilintervall]
    Sei $f\in\mR\of{I}$ mit $I=\interv{a,b}$ und $I'=\interv{a', b'}\subseteq I$. Dann folgt $f\vert_{I'}\in\mR\of{I}$. Und wir definieren
    \begin{align*}
        \int_{a'}^{b'} f\of{x} \dif x \coloneqq \int_{a'}^{b'} \varphi\of{x} \dif x
    \end{align*}
    mit $\varphi\coloneqq f\vert_{\interv{a', b'}}$.
\end{definition}

\begin{satz} % Satz 2
    \label{satz:integral-zerlegt}
    Sei $I=\interv{a, b}$ zerlegt in endlich viele Intervalle $I_j$~($j=1,\dots, m$), die höchstens die Randpunkt gemeinsam haben. Das heißt
    \begin{align*}
        I &= \bigcup_{j=1}^{m} I_j\quad I_j = \interv{a_j, b_j}
    \end{align*}
    und der Schnitt des Inneren der Intervalle leer ($\pair{a_j, b_j}\cap\pair{a_k, b_k} = \emptyset$ für $j\neq k$). Dann gilt
    \begin{align*}
        \int_{I}^{} f\of{x} \dif x &= \sum_{j=1}^{m} \int_{I_j}^{} f\of{x} \dif x
    \end{align*}
    \begin{proof}
        Sei $\pair{Z'_n}_n$ eine Folge von Zerlegungen von $I$ mit $\Delta\of{Z'_n}\fromto 0$ für $\ntoinf$ sowie $Z_0$ die ursprüngliche Zerlegung aus der Voraussetzung des Satzes (das heißt die Zerlegung in die Intervalle $I_j$). Wir betrachten die Verfeinerung $Z_n\coloneqq Z_n' \lor Z_0$, für die auch gilt $\Delta\of{Z_n}\fromto 0$. Wir haben Zwischenpunkte $\xi_n$ zu $Z_n$.\\
        Dadurch, dass $Z_n$ verfeinert ist, lässt es sich in Zerlegungen $Z_n^j$ von den Intervallen $I_j$ aufteilen. Dann gilt auch, dass $\Delta\of{Z_n^j}\fromto 0$ für $j\in\set{1,\ldots, m}$. Die Zwischenpunkte $\xi_n$ lassen sich aufteilen in $\xi_n^j$ von $Z_n^j$. Dann gilt
        \begin{align*}
            S_{Z_n}\of{f, \xi_n} &= \sum_{j=1}^{m} S_{Z_n^j}\of{f, \xi_n^j}\\
            \impl \int_{I}^{} f\of{x} \dif x &= \sum_{j=1}^{m} \int_{I_j}^{} f\of{x} \dif x\qedhere
        \end{align*}
    \end{proof}
\end{satz}

\begin{definition}[Orientiertes Riemann-Integral]
    Sei $\alpha, \beta\in I = \interv{a,b}$, $f\in\mR\of{I}$. Dann definieren wir
    \begin{alignat*}{2}
        \int_{\alpha}^{\beta} f\of{x} \dif x&\coloneqq \int_{\alpha}^{\beta} \varphi\of{x}\dif x  &\quad \varphi \coloneqq f\vert_{\interv{\alpha, \beta}} \\
        \int_{\alpha}^{\beta} f\of{x} \dif x&\coloneqq -\int_{\beta}^{\alpha} f\of{x} \dif x &\quad\text{ falls } \alpha \neq \beta\\
        \int_{\alpha}^{\beta} f\of{x} \dif x&\coloneqq 0 &\quad\text{ falls } \alpha = \beta
    \end{alignat*}
\end{definition}

\begin{satz} % Satz 4
    Sei $f\in\mR\of{I}$ und $\alpha, \beta, \gamma\in I=\interv{a,b}$. Dann gilt
    \begin{align*}
        \int_{\alpha}^{\beta} f\of{x} \dif x + \int_{\beta}^{\gamma} f\of{x} \dif x &= \int_{\alpha}^{\gamma} f\of{x} \dif x\numbereq{eq:int-add}
    \end{align*}
    \begin{proof}
        Sind mindestens 2 Punkte von $\alpha, \beta, \gamma$ gleich, so stimmt die Aussage. Also seien \OBDA $\alpha, \beta, \gamma$ paarweise verschieden. Dann ist (\ref{eq:int-add}) äquivalent zu
        \begin{align*}
            \int_{\alpha}^{\beta} f\of{x} \dif x + \int_{\beta}^{\gamma} f\of{x} \dif x + \int_{\gamma}^{\alpha} f\of{x} \dif x = 0
        \end{align*}
        Diese Gleichung ist invariant unter zyklischem Vertauschen von $\alpha, \beta, \gamma$. (Z.B. $\gamma, \alpha, \beta$ oder $\beta, \gamma, \alpha$).\\
        \textsc{Fall 1}: Sei $\alpha < \beta < \gamma$. Dann folgt die Aussage aus Satz~\ref{satz:integral-zerlegt}.\\
        \textsc{Fall 2}: Sei $\beta < \alpha < \gamma$. Dann folgt aus \textsc{Fall 1}, dass
        \begin{align*}
            \int_{\beta}^{\alpha} f\of{x} \dif x + \int_{\alpha}^{\gamma} f\of{x} \dif x &= \int_{\beta}^{\gamma} f\of{x} \dif x\\
            \impl -\int_{\alpha}^{\beta} f\of{x} \dif x + \int_{\alpha}^{\gamma} f\of{x} \dif x &= \int_{\beta}^{\gamma} f\of{x} \dif x\\
            \impl \int_{\alpha}^{\beta} f\of{x} \dif x + \int_{\beta}^{\gamma} f\of{x} \dif x &= \int_{\alpha}^{\gamma} f\of{x} \dif x
        \end{align*}
        Die restlichen Fälle ergeben sich durch zyklisches Vertauschen von \textsc{Fall 1} oder zyklischem Vertauschen von \textsc{Fall 2}. Damit gilt die Gleichung für alle Fälle.
    \end{proof}
\end{satz}

\subsection{Riemann-Integral für vektorwertige Funktionen}
Sei $I=\interv{a,b}$. Angenommen wir haben eine Funktion, die von $I$ nach $\R^d$ abbildet
\begin{alignat*}{3}
    f: I&\fromto\R^d\\
    x&\mapsto f\of{x} &&= \pair{f_1\of{x}, f_2\of{x}, \dots, f_d\of{x}}\\
    &&&= \begin{pmatrix}
             \tag{Komponentenfunktionen}
             f_1\of{x} \\
             \vdots    \\
             f_d\of{x}
    \end{pmatrix}
\end{alignat*}
Wie können wir dann ein Integral für diese Funktion definieren?

\begin{definition}
    Wir definieren für komplexe oder vektorwertige Funktionen einen Integralbegriff, indem wir jede Komponente einzeln integrieren.
    \begin{enumerate}[label=(\roman*)]
        \item Sei $f: I\fromto\C~x\mapsto f\of{x} = \Re\of{f\of{x}} + i\Im\of{f\of{x}}$. Dann definieren wir
        \begin{align*}
            \mB\of{I, \C} &\coloneqq\set{f: I\fromto\C~\middle\vert~\Re\of{f}, \Im\of{f} \in \mB\of{I}}\\
            \mR\of{I, \C} &\coloneqq\set{f\in\mB\of{I, \C}~\middle\vert~\Re\of{f}, \Im\of{f}\in \mR\of{I}}\\
            \intertext{und setzen}
            \int_{a}^{b} f\of{x} \dif x &\coloneqq \int_{a}^{b} \Re\of{f\of{x}} \dif x + i\int_{a}^{b} \Im\of{f\of{x}} \dif x
        \end{align*}
        \item Sei $\K\in\set{\R, \C}$. Dann ist eine Funktion $f\in\mB\of{I, \K^d}$ Riemann-integrierbar, falls alle Komponentenfunktionen $f_1, f_2, \dots, f_d$ Riemann-integrierbar auf $I$ sind. Wir setzen
        \begin{align*}
            \int_{a}^{b} f\of{x} \dif x\coloneqq \begin{pmatrix}
                                                     \displaystyle\int_{a}^{b} f_1\of{x} \dif x \\
                                                     \displaystyle\int_{a}^{b} f_2\of{x} \dif x \\
                                                     \vdots                                     \\
                                                     \displaystyle\int_{a}^{b} f_d\of{x} \dif x
            \end{pmatrix}
        \end{align*}
    \end{enumerate}
\end{definition}

\begin{bemerkung}
    Das lässt sich auch auf Matrizen übertragen. Eine Funktion $f: I\fromto \K^{n\times m}$ ist Riemann-integrierbar, falls jede Komponentenfunktion integrierbar ist. Das Integral wird analog zu Vektoren definiert.
\end{bemerkung}

\begin{bemerkung}
    Entsprechend ist auch $\mR\of{I, \R^d}$ ein reeller und $\mR\of{I, \C^d}$ ein komplexer Vektorraum und die Linearität des Integrals gilt. Das heißt $\forall \alpha,\beta\in\K, f,g\in\mR\of{I, \K^d}$ ist
    \begin{align*}
        \int_{a}^{b} \alpha f\of{x} + \beta g\of{x} \dif x &= \alpha \int_{a}^{b} f\of{x} \dif x + \beta \int_{a}^{b} g\of{x} \dif x
    \end{align*}
    Alle Rechenregeln und Sätze gelten entsprechend!
\end{bemerkung}

\newpage

    \section{Der Hauptsatz der Differential- und Integralrechnung}
\thispagestyle{pagenumberonly}

Wir haben unseren Integralbegriff bisher über Zerlegungen definiert. Allerdings ist das für das konkrete Berechnen von Integralen nicht wirklich praktikabel. Wenn wir also ein Intervall $I=\interv{a,b}$ und eine Funktion $f\in\mC\of{I}$ betrachten, wie rechnet man das Integral dann konkret aus?

\subsection{Hauptsatz der Integralrechnung}

\begin{definition}[Erinnerung: Stammfunktion]
    Wir nennen $F$ eine Stammfunkton von $f$, falls $F$ differenzierbar ist und $F'=f$.
\end{definition}

\begin{satz}[Hauptsatz der Differential- und Integralrechnung] % Satz 1
    \label{satz:hauptsatz-dif-int}
    Sei $f\in\mC\of{I}$. Dann ist für jedes $c\in\interv{a,b}$ die Funktion
    \begin{align*}
        F\of{x} \coloneqq \int_{c}^{x} f\of{t} \dif t\tag{$x\in I$}
    \end{align*}
    stetig differenzierbar und $F' = f$. Das heißt $F'\of{x} = f\of{x}~\fa x\in I$.

    \begin{proof}
        \marginnote{[3. Mai]}
        Sei $F(x) = \int_{c}^{x} f\of{t} \dif t$ und $h \neq 0$. Wir wollen über den Differenzenquotient zeigen, dass $F' = f$. Wir berechnen zuerst den Zähler
        \begin{align*}
            F(x+h) - F(x) &= \int_{c}^{x+h} f\of{t} \dif t - \int_{c}^{x} f\of{t} \dif t = \int_{x}^{x+h} f\of{t} \dif t
            \intertext{Das können wir in den Differenzenquotienten einsetzen}
            \frac{F\of{x+h} - F\of{x}}{h} &= \frac{1}{h} \int_{x}^{x+h} f\of{t} \dif t\\
            \impl \frac{F\of{x+h} - F\of{x}}{h} - f\of{x} &= \frac{1}{h} \int_{x}^{x+h} f\of{t} \dif t - f\of{x}\\
            &= \frac{1}{h} \int_{x}^{x+h} f\of{t} \dif t - \frac{1}{h} \int_{x}^{x+h} f\of{x} \dif t\\
            &= \frac{1}{h} \int_{x}^{x+h} \pair{f\of{t} - f\of{x}} \dif t
            \intertext{Wir definieren $I_h\of{x} = \interv{x, x+h}$, falls $h> 0$ und $I_h\of{x} = \interv{x+h, x}$, falls $h < 0$. Damit können wir das Integral gegen seinen betragsmäßig größten Wert abschätzen}
            \impl \abs{\frac{F\of{x+h} - F\of{x}}{h} - f\of{x}} &\leq \frac{1}{\abs{h}} \cdot \sup_{t\in I_h\of{x}} \abs{f\of{t} - f\of{x}} \cdot \abs{h}\\
            &\leq \sup_{t\in I_h\of{x}} \abs{f\of{t} - f\of{x}}
        \end{align*}
        Da $f$ stetig in $x$ ist, folgt
        \begin{align*}
            \sup_{t\in I_h\of{x}} \abs{f\of{t} - f\of{x}} &\fromto 0 \text{ für } h\fromto 0\\
            \impl \lim_{h\fromto 0} \abs{\frac{F\of{x+h} - F\of{x}}{h} - f\of{x}} &= 0\\
            \equivalent \lim_{h\fromto 0} \frac{F\of{x+h} - F\of{x}}{h} &= f\of{x}\qedhere
        \end{align*}
    \end{proof}
\end{satz}

\begin{korollar}
    Sei $G\in \mC^{1}\of{I}$ (die Klasse der stetig differenzierbaren Funktionen auf $I$) eine Stammfunktion von $f\in\mC\of{I}$. Dann gilt
    \begin{align*}
        \int_{a}^{b} f\of{x} \dif x = G\of{b} - G\of{a} \eqqcolon G\vert_a^b \eqqcolon \interv{G}_a^b \eqqcolon \interv{G\of{x}}_{x=a}^{x=b}
    \end{align*}
    \begin{proof}
        Wir setzen $c = a$ in Satz~\ref{satz:hauptsatz-dif-int} und $F: I\fromto\R~x\mapsto F\of{x} = \int_{a}^{x} f\of{t} \dif t$. Dann ist $F$ nach Satz~\ref{satz:hauptsatz-dif-int} eine Stammfunktion von $f$ auf $I$. Sei $G\in\mC^1\of{I}$ eine beliebige Stammfunktion von $f$. Wir definieren für $t\in I$
        \begin{align*}
            h\of{t} &\coloneqq F\of{t} - G\of{t}\\
            \impl h'\of{t} &= F'\of{t} - G'\of{t} = f\of{t}-f\of{t} = 0
            \intertext{Damit ist $h$ konstant, d.h. $h\of{x} = k$ für alle $x\in I$}
            \impl k &= F\of{x} - G\of{x}\quad\forall x\in I\\
            \impl k &= F\of{a} - G\of{a} = -G\of{a}\\
            \impl F\of{x} - G\of{x} &= -G\of{a}\\
            F\of{x} &= G\of{x} - G\of{a}\\
            \impl F\of{b} &= G\of{b} - G\of{a}\qedhere
        \end{align*}
    \end{proof}
\end{korollar}
\noindent Damit sind wir nun in der Lage Integrale für bestimmte Arten von Funktionen konkret anzugeben.

\begin{beispiel}
    Sei $p\in\N$ und $f\of{x} = x^{p}$, $x\in\R$. Dann hat $f$ die Stammfunktion $F\of{x} = \frac{1}{p+1}\cdot x^{p+1}$. Damit folgt
    \begin{align*}
        \int_{a}^{b} x^p \dif x &= \frac{1}{p+1} \cdot \interv{b^{p+1} - a^{p+1}}\quad\forall a, b\in\R
    \end{align*}
\end{beispiel}

\begin{beispiel}
    Sei $p\in\N$, $p\geq 2$ und $f\of{x} = x^{-p}$, $x\neq 0$. Dann ist die Stammfunktion $F\of{x} = \frac{1}{1-p}\cdot x^{1-p}$. Damit folgt
    \begin{align*}
        \int_{a}^{b} x^{-p} \dif x &= \frac{1}{1-p}\cdot\interv{b^{1-p} - a^{1-p}}\quad \fa a,b < 0 \text{ oder } a,b > 0
    \end{align*}
\end{beispiel}

\begin{beispiel}
    Sei $\alpha\in\R\exclude\set{-1}$, $f\of{x} = x^{\alpha} = e^{\alpha\cdot\ln\of{x}}$, $x > 0$. Dann ist die Stammfunktion $F\of{x} = \frac{1}{\alpha + 1}\cdot x^{\alpha + 1}$. Damit gilt
    \begin{align*}
        \int_{a}^{b} x^{\alpha} \dif x &= \frac{1}{\alpha +1}\cdot\interv{b^{\alpha+1} - a^{\alpha+1}}\quad\forall a,b > 0
    \end{align*}
\end{beispiel}

\begin{beispiel}
    Sei $f\of{x} = \frac{1}{x}$, $x\neq 0$. Dann ist die Stammfunktion $F\of{x} = \ln \abs{x}$.
    \begin{proof}
        Falls $x > 0$. Dann ist $F\of{x} = \ln x$ und $F'\of{x} = \frac{1}{x}$.\\
        Falls $x < 0$. Dann ist $F\of{x} = \ln -x$ und $F'\of{x} = \frac{1}{-x}\cdot\pair{-1} = \frac{1}{x}$.
    \end{proof}
    \noindent Damit gilt
    \begin{align*}
        \int_{a}^{b} \frac{1}{x} \dif x &= \ln\abs{b} - \ln\abs{a} = \ln\abs{\frac{b}{a}}\quad\forall a,b < 0 \text{ oder } a,b > 0
    \end{align*}
\end{beispiel}

\begin{beispiel}
    Es gilt $\pair{\sin x}' = \cos x$ und $\pair{\cos x}' = -\sin x$. Damit gilt
    \begin{align*}
        \int_{a}^{b} \cos x \dif x = \interv{\sin x}_{a}^b &= \sin b - \sin a\\
        \int_{a}^{b} \sin x \dif x = \interv{-\cos x}_a^b &= -\cos b + \cos a\\
    \end{align*}
\end{beispiel}

\begin{beispiel}
    Es gilt $\tan x = \frac{\sin x}{\cos x}$ für $\abs{x} < \frac{\pi}{2}$. Damit folgt $\pair{\tan x}' = \frac{1}{\cos^2 x}$. Das heißt
    \begin{align*}
        \int_{0}^{\varphi} \frac{1}{\cos^2 x} \dif x &= \interv{\tan x}_0^{\varphi} = \tan\of{\varphi}\quad\forall\abs{\varphi} < \frac{\pi}{2}
    \end{align*}
\end{beispiel}

\begin{beispiel}[Fläche des Einheitskreises]
    Die Funktion $\sqrt{1-x^2}$ beschreibt die obere Hälfte des Einheitskreises. Das heißt wir wollen das Integral $ \int_{a}^{b} \sqrt{1-x^2} \dif x$ berechnen. Wie wir später in Beispiel~\ref{beispiel:stammfunktion-sqrt-1-xsqrt} sehen werden, hat $\sqrt{1-x^2}$ die Stammfunktion
    \begin{align*}
        \phi\of{x} &= \frac{1}{2}\pair{\arcsin x + x\cdot\sqrt{1-x^2}}
        \intertext{Das lässt sich durch Ableiten mit $\pair{\arcsin\of{x}}' = \frac{1}{\sqrt{1-x^2}}$ prüfen}
        \phi'\of{x} &= \frac{1}{2}\pair{\frac{1}{\sqrt{1-x^2}} + \sqrt{1-x^2} + x\cdot \frac{1}{2\sqrt{1-x^2}}\cdot \pair{-2x}}\\
        &= \frac{1}{2}\pair{\frac{1}{\sqrt{1-x^2}} + \sqrt{1-x^2} - \frac{x^2}{\sqrt{1-x^2}}}\\
        &= \frac{1}{2}\pair{\frac{1-x^2}{\sqrt{1-x^2}} + \sqrt{1-x^2}}\\
        &= \frac{1}{2}\pair{\sqrt{1-x^2} + \sqrt{1-x^2}} = \sqrt{1-x^2}\\
        \impl \int_{a}^{b} \sqrt{1-x^2} \dif x &= \interv{\frac{1}{2}\pair{\arcsin x + x\cdot\sqrt{1-x^2}}}_a^b\tag{$-1\leq a,b\leq 1$}
        \intertext{Geometrisch gesehen können wir damit nun die Fläche der oberen Hälfte des Einheitskreises berechnen}
        \int_{-1}^{1} \sqrt{1-x^2} \dif x &= \frac{1}{2} \cdot\pair{\arcsin\of{1} + 0 - \arcsin\of{-1} -0} = \arcsin\of{1} = \frac{\pi}{2}
    \end{align*}
\end{beispiel}

\begin{bemerkung}
    Satz~\ref{satz:hauptsatz-dif-int} gilt auch für Funktionen in $\C$ oder $\R^d$ bwz. $\C^d$.
\end{bemerkung}

\begin{notation}[Stammfunktion]
    Wir schreiben
    \begin{align*}
        &\int_{}^{} f\of{x} \dif x
        \intertext{für die Gesamtheit aller Stammfunktionen zu $f$ oder das unbestimmte Integral. Genauer gilt, wenn $\Phi$ eine Stammfunktion von $f$ ist}
        \int_{}^{} f\of{x} \dif x &= \set{\Phi + k: k \text{ Konstante} }
    \end{align*}
    Wir schreiben manchmal statt einer Menge auch nur eine konkrete Stammfunktion und die mögliche Addition einer Konstante ist dann implizit enthalten.
\end{notation}

\newpage

\subsection{Integrationstechniken - Partielle Integration}
\begin{satz}[Partielle Integration] % Satz 3
    \label{satz:partielle-integration}
    Seien $f, g\in \mC^1\of{I, \K}$ ($\K\in\set{\R, \C}$) für ein Intervall $I$. Dann gilt für $a,b\in I$
    \begin{align*}
        \int_{a}^{b} f'\of{x}g\of{x} \dif x &= \interv{f\of{x}g\of{x}}_a^b - \int_{a}^{b} f\of{x}g'\of{x} \dif x\\
        &= f\of{b}g\of{b} - f\of{a}g\of{a} - \int_{a}^{b} f\of{x}g'\of{x} \dif x\numberthis\label{eq:partial-int}
    \end{align*}
    \begin{proof}
        Wir wenden die Produktregel der Ableitung an. Es gilt
        \begin{align*}
            \pair{f\of{x}g\of{x}}' &= f'\of{x}g\of{x} + f\of{x}g'\of{x}\\
            \impl \int_{a}^{b} \pair{f\of{x}g\of{x}}' \dif x &= \int_{a}^{b} f'\of{x}g\of{x} \dif x + \int_{a}^{b} f\of{x}g'\of{x} \dif x\tag{i}
            \intertext{Außerdem gilt nach Satz~\ref{satz:hauptsatz-dif-int}}
            \int_{a}^{b} \pair{f\of{x}g\of{x}}' \dif x &= \interv{f\of{x}g\of{x}}_a^b = \interv{f\of{x}g\of{x}}_{a}^b\\
            &= f\of{b}g\of{b} - f\of{a}g\of{a}\tag{ii}
            \intertext{Wir setzen (i) und (ii) gleich}
            \impl \int_{a}^{b} f'\of{x}g\of{x} \dif x + &\int_{a}^{b} f\of{x}g'\of{x} \dif x = f\of{b}g\of{b} - f\of{a}g\of{a}\qedhere
        \end{align*}
    \end{proof}
\end{satz}

\begin{bemerkung}[Stammfunktion mittels partieller Integration]
    Wenn wir kein konkretes Integral ausrechnen wollen, sondern lediglich eine beliebige Stammfunktion für eine Funktion ermitteln wollen, lässt sich Satz~\ref{satz:partielle-integration} auch umformen zu
    \begin{align*}
        \int_{}^{} f'\of{x}g\of{x} \dif x &= f\of{x}g\of{x} - \int_{}^{} f\of{x}g'\of{x} \dif x
    \end{align*}
    weil wir konstante Terme für die Stammfunktion vernachlässigen können und so der zweite Term aus (\ref{eq:partial-int}) wegfällt.
\end{bemerkung}

\begin{beispiel}[Anwendung von partieller Integration]
    Wir bestimmen eine Stammfunktion von $\ln x$ mittels partieller Integration
    \begin{align*}
        \int_{}^{} \ln x \dif x &= \int_{}^{} 1\cdot\ln x \dif x = x\cdot\ln x - \int_{}^{} x\cdot \frac{1}{x} \dif x = x\cdot\ln x - x
    \end{align*}
\end{beispiel}

\begin{beispiel}
    Sei $-1\neq\alpha\in\R$. Dann gilt
    \begin{align*}
        \int_{}^{} x^{\alpha}\cdot \ln\of{x} \dif x &= \frac{1}{\alpha+1}x^{\alpha+1}\cdot \ln\of{x} - \int_{}^{} \frac{1}{\alpha+1}x^{\alpha+1}\cdot\frac{1}{x} \dif x\\
        &= \frac{1}{\alpha+1}x^{\alpha}\cdot\ln\of{x} - \frac{1}{\pair{\alpha+1}^2}x^{\alpha+1}
    \end{align*}
\end{beispiel}

\begin{beispiel}[Integrationstrick bei partieller Integration]
    \label{beispiel:stammfunktion-sqrt-1-xsqrt}
    Bei partieller Integration bleibt immer noch ein Integralteil übrig. Wenn dieser wieder dem ursprünglichen Integral entspricht, ist es möglich, die Gleichung danach aufzulösen und so das Integral zu bestimmen. Wir bestimmen auf diese Weise eine Stammfunktion für $\sqrt{1-x^2}$. \newpage
    \begin{align*}
        \int_{}^{} \sqrt{1-x^2} \dif x &= \int_{}^{} 1\cdot\sqrt{1-x^2} \dif x\\
        &= x\cdot\sqrt{1-x^2} - \int_{}^{} x\cdot\frac{1}{\sqrt{1-x^2}}\cdot\pair{-2x} \dif x\\
        &= x\cdot\sqrt{1-x^2} + \int_{}^{} \frac{x^2}{\sqrt{1-x^2}} \dif x\\
        &= x\cdot\sqrt{1-x^2} + \int_{}^{} \frac{1}{\sqrt{1-x^2}} \dif x - \int_{}^{} \frac{1-x^2}{\sqrt{1-x^2}} \dif x\\
        &= x\cdot\sqrt{1-x^2} + \arcsin\of{x} - \int_{}^{} \sqrt{1-x^2} \dif x\\
        \impl 2 \int_{}^{} \sqrt{1-x^2} \dif x &= x\cdot\sqrt{1-x^2} + \arcsin x\\
        \impl \int_{}^{} \sqrt{1-x^2} \dif x &= \frac{1}{2}\pair{x\cdot\sqrt{1-x^2} + \arcsin x}
    \end{align*}
\end{beispiel}

\begin{uebung}
    Beweisen Sie analog zum vorherigen Beispiel mittels partieller Integration, dass $ \int_{}^{} \sqrt{1+x^2} \dif x = \frac{1}{2}\pair{x\cdot\sqrt{1+x^2} + \arcsinh x}$ und $ \int_{}^{} \sqrt{x^2-1} \dif x = \frac{1}{2}\pair{x\cdot\sqrt{x^2-1} - \arccosh x}$.
\end{uebung}

\begin{beispiel}
    Seien $a,b\in\R$
    \begin{align*}
        \int_{}^{} \underset{\scriptscriptstyle g\of{x}}{\vphantom{\frac{i}{i}}e^{ax}}\cdot \underset{\scriptscriptstyle f\of{x}}{\vphantom{\frac{i}{i}}\sin\of{bx}} \dif x &= e^{ax}\cdot\pair{-\frac{1}{b}\cos\of{bx}} + \int_{}^{} \frac{a}{b} e^{ax} \cos\of{bx} \dif x\\
        &= -\frac{1}{b}e^{ax} \cos\of{bx} + \frac{a}{b}\int_{}^{} e^{ax} \cos\of{bx} \dif x\\
        \intertext{Wir wenden nochmal partielle Integration an und erhalten}
        &= -\frac{1}{b}e^{ax} \cos\of{bx} + \frac{a}{b}\cdot\pair{\frac{1}{b} e^{ax} \sin\of{bx} - \frac{a}{b}\int_{}^{} e^{ax}\sin\of{bx} \dif x}\\
        \impl \pair{1+\frac{a^2}{b^2}} \int_{}^{} e^{ax}\sin\of{bx} \dif x &= -\frac{1}{b} e^{ax}\cos\of{bx} + \frac{a}{b^2}e^{ax}\sin\of{bx}\\
        \impl \int_{}^{} e^{ax}\sin\of{bx} \dif x &= \frac{1}{a^2+b^2}\pair{e^{ax}\pair{a\sin\of{bx} - b\cos\of{bx}}}
    \end{align*}
\end{beispiel}

\begin{beispiel}
    \marginnote{[07. Mai]}
    Es gilt
    \begin{align*}
        \int_{0}^{\frac{\pi}{2}} \sin^2\of{x} \dif x &= \int_{0}^{\frac{\pi}{2}} \cos^2\of{x} \dif x = \frac{\pi}{4}\\
    \end{align*}
    \begin{proof}
        \begin{align*}
            \int_{0}^{\frac{\pi}{2}} \sin^2\of{x} \dif x &= \int_{0}^{\frac{\pi}{2}} \sin\of{x}\sin\of{x} \dif x\\
            &= \interv{-\cos\of{x}\sin\of{x}}_{0}^{\frac{\pi}{2}} + \int_{0}^{\frac{\pi}{2}} \cos\of{x}\cos\of{x} \dif x\\
            &= 0 - 0 + \int_{0}^{\frac{\pi}{2}} \cos^2\of{x} \dif x
            \intertext{Mit dem trigonometrischen Pythagoras wissen wir außerdem, dass}
            \frac{\pi}{2} &= \int_{0}^{\frac{\pi}{2}} 1 \dif x = \int_{0}^{\frac{\pi}{2}} \pair{\cos^2\of{x} + \sin^2\of{x}} \dif x\\
            &= \int_{0}^{\frac{\pi}{2}} \cos^2\of{x} \dif x + \int_{0}^{\frac{\pi}{2}} \sin^2\of{x} \dif x = 2 \int_{0}^{\frac{\pi}{2}} \cos^2\of{x} \dif x\\
            \impl\frac{\pi}{4} &= \int_{0}^{\frac{\pi}{2}} \cos^2\of{x} \dif x\qedhere
            \intertext{Analog lässt sich zeigen, dass}
            \pi &= \int_{0}^{2\pi} \cos^2\of{x} \dif x = \int_{0}^{2\pi} \sin^2\of{x} \dif x
        \end{align*}
    \end{proof}
\end{beispiel}

\begin{beispiel}
    Sei $n\in\N$ mit $n\geq 2$
    \begin{align*}
        \int_{}^{} \cos^{n}\of{x} \dif x &= \int_{}^{} \cos\of{x}\cos^{n-1}\of{x} \dif x\\
        &= \sin\of{x}\cos^{n-1}\of{x} + \int_{}^{} \sin\of{x}\cdot\pair{n-1}\cos^{n-2}\of{x}\sin\of{x} \dif x\\
        &= \sin\of{x}\cos^{n-1}\of{x} + \pair{n-1} \int_{}^{} \underbrace{\sin^2\of{x}}_{= 1- \cos^2\of{x}}\cos^{n-2}\of{x} \dif x\\
        &= \sin\of{x}\cos^{n-1}\of{x} + \pair{n-1} \int_{}^{} \cos^{n-2}\of{x} \dif x - \pair{n-1} \int_{}^{} \cos^{n}\of{x}\dif x\\
        \impl \int_{}^{} \cos^{n}\of{x} \dif x &= \frac{1}{n} \sin\of{x}\cos^{n-1}\of{x} + \frac{n-1}{n} \int_{}^{} \cos^{n-2}\of{x}\dif x\tag{Rekursionsformel}
        \intertext{Analog lässt sich zeigen, dass}
        \int_{}^{} \sin^{n}\of{x} \dif x &= -\frac{1}{n}\cos\of{x}\sin^{n-1}\of{x} + \frac{n-1}{n} \int_{}^{} \sin^{n-2}\of{x} \dif x
        \intertext{Wir nutzen nun die Rekursionsformel, um einen Wert für das Intervall $\interv{0, \frac{\pi}{2}}$ zu berechnen}
        c_n &\coloneqq \int_{0}^{\frac{\pi}{2}} \cos^{n}\of{x} \dif x\numberthis\label{eq:int-cos-potenz}\\
        &= \interv{\frac{1}{n}\sin\of{x}\cos^{n-1}\of{x}}_0^{\frac{\pi}{2}} + \frac{n-1}{n} \int_{0}^{\frac{\pi}{2}} \cos^{n-2}\of{x} \dif x\\
        &= \frac{n-1}{n} \underbrace{\int_{0}^{\frac{\pi}{2}} \cos^{n-2}\of{x} \dif x}_{= c_{n-2}}\\
        \impl c_n &= \frac{n-1}{n}\cdot c_{n-2}\qquad\forall n\geq 2\\
        c_0 &= \frac{\pi}{2}\quad \text{ und } \quad c_1 = \int_{0}^{\frac{\pi}{2}} \cos\of{x} \dif x = \interv{\sin\of{x}}_0^{\frac{\pi}{2}} = 1-0 = 1\\
        c_n &= \frac{n-1}{n}\cdot c_{n-2} = \frac{n-1}{n}\cdot\frac{n-3}{n-2}\cdot c_{n-4}\\
        &= \frac{n-1}{n}\cdot\ldots\cdot\frac{n-2j-1}{n-2j}\cdot c_{n-2j-2}\qquad\forall j\in\N: n-2j-2 \geq 1
        \intertext{Damit folgt für $k\in\N$}
        c_{2k} &= \frac{2k-1}{2k} \cdot\frac{2k-3}{2k-2}\cdot \ldots\cdot\frac{3}{4}\cdot \frac{1}{2}\cdot c_0 = \frac{2k-1}{2k} \cdot\frac{2k-3}{2k-2}\cdot \ldots\cdot\frac{3}{4}\cdot\frac{1}{2}\cdot\frac{\pi}{2}\\
        c_{2k+1} &= \frac{2k}{2k+1}\cdot\frac{2k-2}{2k-1}\cdot\ldots\cdot\frac{4}{5}\cdot \frac{2}{3}\cdot c_1 = \frac{2k}{2k+1}\cdot\frac{2k-2}{2k-1}\cdot\ldots\cdot\frac{4}{5}\cdot \frac{2}{3}
    \end{align*}
\end{beispiel}

\noindent Aus diesem Beispiel können wir nun auch eine Grenzwertdarstellung für $\pi$ ableiten.

\begin{satz}[Wallisches Produkt] % Satz 3
    Sei $n\in\N$ und
    \begin{align*}
        W_n \coloneqq \frac{2\cdot 2}{1\cdot 3}\cdot \frac{4\cdot 4}{3\cdot 5}\cdot \ldots\cdot&\frac{2n\cdot 2n}{\pair{2n-1}\cdot\pair{2n+1}} = \prod_{j=1}^{n} \frac{2j\cdot 2j}{\pair{2j-1}\cdot\pair{2j+1}}
        \intertext{Dann gilt}
        &\lim_{\ntoinf} W_n = \frac{\pi}{2}
    \end{align*}
    \begin{proof}
        Mit der Definition von $c_n$ aus (\ref{eq:int-cos-potenz}) ergibt sich
        \begin{align*}
            W_n &= \frac{\pi}{2} \cdot \frac{c_{2n+1}}{c_{2n}}
            \intertext{Für $x\in\interv{0, \frac{\pi}{2}}$ ist $0\leq\cos\of{x}\leq 1$. Damit folgt $\cos^{2n}\of{x} \leq \cos^{2n-1}\of{x}\leq \cos^{2n-2}\of{x}$. Also gilt}
            \int_{0}^{\frac{\pi}{2}} \cos^{2n}\of{x} \dif x &\leq \int_{0}^{\frac{\pi}{2}} \cos^{2n-1}\of{x} \dif x \leq \int_{0}^{\frac{\pi}{2}} \cos^{2n-2}\of{x} \dif x\\
            \impl c_{2n} &\leq c_{2n-1} \leq c_{2n-2}\qquad\forall n\in\N
            \intertext{Nach Definition gilt}
            c_{2n} &= \frac{\pi}{2}\cdot\prod_{j=1}^{n} \frac{2j-1}{2j}\\[5pt]
            \impl \frac{c_{2n+2}}{c_{2n}} &= \frac{\displaystyle\frac{\pi}{2}\cdot\prod_{j=1}^{n+1} \frac{2j-1}{2j}}{\displaystyle\frac{\pi}{2}\cdot \prod_{j=1}^{n} \frac{2j-1}{2j}} = \frac{2\pair{n+1}-1}{2\pair{n+1}} = \frac{2n+1}{2n+2}\fromto 1 \text{ für } \ntoinf
            \intertext{Außerdem gilt}
            1 &= \frac{c_{2n}}{c_{2n}} \geq \pair{\frac{c_{2n+1}}{c_{2n}}} \geq \frac{c_{2n+2}}{c_{2n}} = \frac{2n+1}{2n+2} \fromto 1 \text{ für } \ntoinf\\
            &\impl \lim_{\ntoinf} \frac{c_{2n+1}}{c_n} = 1\qedhere
            \intertext{Außerdem erhalten wir auch eine Reihendarstellung von $\sqrt{\pi}$}
            W_n &= \frac{2^2\cdot 4^2\cdot 6^2\cdot\ldots\cdot \pair{2n-2}^2}{3^2\cdot 5^2\cdot 7^2\cdot\ldots\cdot\pair{2n-1}^2} \cdot 2n \cdot \frac{2n}{2n+1}\\
            \impl \sqrt{W_n} &= \frac{2\cdot 4 \cdot \ldots \pair{2n-2}}{3\cdot 5 \cdot\ldots\cdot \pair{2n-1}}\cdot\sqrt{2n}\cdot\sqrt{\frac{2n}{2n+1}}\\
            \impl \sqrt{\frac{\pi}{2}} &= \lim_{\ntoinf} \frac{2\cdot 4\cdot\ldots\cdot\pair{2n-2}}{3\cdot 5 \cdot\ldots \cdot\pair{2n-1}}\cdot\sqrt{2n}\\
            &= \lim_{\ntoinf} \frac{2^2 \cdot 4^2\cdot \ldots \cdot \pair{2n-2}^2}{2\cdot 3 \cdot \ldots \cdot \pair{2n-2}\cdot\pair{2n-1}}\cdot\sqrt{2n}\\
            &= \frac{2^2\cdot 4^2\cdot\ldots \cdot \pair{2n-2}^2\cdot\pair{2n}^2}{\pair{2n-1}!\cdot 2n\cdot\sqrt{2n}}\\
            &= \frac{2^{2n}\cdot\pair{n!}^2}{\pair{2n}!\cdot\sqrt{2n}} = \frac{2^{2n}}{\binom{2n}{n}\cdot\sqrt{n}}\cdot \frac{1}{\sqrt{2}}\\
            \impl \sqrt{\pi} &= \lim_{\ntoinf} \frac{2^{2n}}{\binom{2n}{n}\cdot\sqrt{n}}
        \end{align*}
    \end{proof}
    Wir werden später in Bemerkung~\ref{bemerkung:reihendarstellung-pi} sehen, dass z.B. mit Taylor-Polynomen noch andere Reihendarstellungen von $\pi$ herleitbar sind, die im Vergleich schneller konvergieren und dadurch für die tatsächliche Anwendung geigneter sind.
\end{satz}

\subsection{Integrationstechniken - Substitution}

\begin{satz}[Substitutionsregel] % Satz 4
    \label{satz:substitution}
    Seien $I=\interv{a,b}$ und $I^{*}$ kompakte Intervalle und $f\in\mC\of{I, \C}$, $\varphi\in\mC^{1}\of{I^{*}, \R}$ sowie $\varphi\of{I^{*}}\subseteq I$. Dann gilt für $\alpha, \beta \in I^{*}$
    \begin{align*}
        \int_{\alpha}^{\beta} f\of{\varphi\of{t}}\cdot\varphi'\of{t} \dif t &= \int_{\varphi\of{\alpha}}^{\varphi\of{\beta}} f\of{x} \dif x\numberthis\label{eq:substitution}
    \end{align*}
    \begin{proof}
        Sei $F$ eine Stammfunktion von $f$. Wir definieren $h\of{t}\coloneqq F\of{\varphi\of{t}}$. Dann ist $h\in\mC^{1}\of{I^{*}, \C}$ wegen der Kettenregel
        \begin{align*}
            h'\of{t} &= \frac{\dif}{\dif t} h\of{t} = F'\of{\varphi\of{t}}\cdot\varphi'\of{t} = f\of{\varphi\of{t}}\cdot\varphi'\of{t}\\
            \int_{\alpha}^{\beta} h'\of{t} \dif t &= \interv{h\of{t}}_{\alpha}^{\beta} = h\of{\beta}-h\of{\alpha} = F\of{\varphi\of{\beta}} - F\of{\varphi\of{\alpha}}\\
            &= \int_{\varphi\of{\alpha}}^{\varphi\of{\beta}} F'\of{x} \dif x = \int_{\varphi\of{\alpha}}^{\varphi\of{\beta}} f\of{x} \dif x\qedhere
        \end{align*}
    \end{proof}
\end{satz}

\begin{bemerkung}[Erste Lesart von Satz~\ref{satz:substitution}]
    Wir betrachten folgendes Szenario: Wir wollen $\int_{\alpha}^{\beta} g\of{t} \dif t$ ausrechnen und es existiert eine Substitution $x=\varphi\of{t}$ für eine Funktion $f\of{x}$, sodass
    \begin{align*}
        g\of{t} &= f\of{\varphi\of{t}}\cdot\varphi'\of{t}\\
        \intertext{Dann verwenden wir den Satz, setzen $b=\varphi\of{\beta}$, $a = \varphi\of{\alpha}$ und erhalten}
        \impl \int_{\alpha}^{\beta} g\of{t} \dif t &= \int_{a}^{b} f\of{x} \dif x
    \end{align*}
\end{bemerkung}

\begin{beispiel}
    Wir betrachten das Integral
    \begin{align*}
        \int_{\alpha}^{\beta} &g\of{t+c} \dif t
        \intertext{Wir definieren $\varphi\of{t} \coloneqq t+c$ und $f\of{x} = g\of{x}$. Dann gilt $\varphi'\of{t} = 1$}
        \impl \int_{\alpha}^{\beta} g\of{t+c} \dif t = \int_{\alpha}^{\beta} g\of{\varphi\of{t}}\cdot\varphi'\of{t} \dif t &= \int_{\varphi\of{\alpha}}^{\varphi\of{\beta}} g\of{x} \dif x = \int_{\alpha+c}^{\beta+c} g\of{x} \dif x\tag{Translation}
    \end{align*}
\end{beispiel}

\begin{beispiel}
    Sei $a,b >0$. Wir betrachten
    \begin{align*}
        &\int_{a}^{b} \frac{g\of{t}}{t} \dif t
        \intertext{und definieren $\varphi\of{t} \coloneqq \ln\of{t}$, $\varphi'\of{t} = \frac{1}{t}$, $t=e^{\varphi\of{t}}$. Dann gilt}
        g\of{t}\cdot \frac{1}{t} &= g\of{t}\cdot\varphi'\of{t} = g\of{e^{\varphi\of{t}}}\cdot\varphi'\of{t}\\
        f\of{x} &\coloneqq g\of{e^x}
        \intertext{Wir wenden Substitution an und erhalten}
        \impl \int_{a}^{b} g\of{t} \frac{\dif t}{t} &= \int_{a}^{b} f\of{\varphi\of{t}}\cdot\varphi'\of{t} \dif t = \int_{\varphi\of{a}}^{\varphi\of{b}} f\of{x} \dif x\\
        &= \int_{\ln a}^{\ln b} f\of{x} \dif x = \int_{\ln a}^{\ln b} g\of{e^x} \dif x
    \end{align*}
\end{beispiel}

\begin{beispiel}
    Wir betrachten
    \begin{align*}
        \int_{0}^{1} &\pair{1+t^2}^{n}\cdot t \dif t
        \intertext{und definieren $\varphi\of{t} \coloneqq 1 + t^2$, $f\of{x} \coloneqq \frac{1}{2}x^n$}
        \impl \varphi'\of{t} &= 2t\\
        \impl \pair{1+t^2}^n\cdot t &= f\of{\varphi\of{t}}\cdot\varphi'\of{t}\\
        \impl \int_{0}^{1} \pair{1+t^2}^{n}\cdot t \dif t &= \int_{\varphi\of{0}}^{\varphi\of{1}} \frac{1}{2}\cdot x^n \dif t = \int_{1}^{2} \frac{1}{2}\cdot x^n \dif t\\
        &= \interv{\frac{1}{2\pair{n+1}}\cdot x^{n+1}}_1^2\\
        &= \frac{1}{2\pair{n+1}}\cdot\pair{2^{n+1} - 1}
    \end{align*}
\end{beispiel}

\begin{bemerkung}[Zweite Lesart: Transformationssatz]
    \marginnote{[10. Mai]}
    Es ist auch möglich, Satz~\ref{satz:substitution} in die andere Richtung anzuwenden. Das heißt wir wollen das Integral auf der rechten Seite von (\ref{eq:substitution}) zum Integral auf der linken Seite umformen. Wir führen eine Transformation mit $\varphi\of{t} = x$ durch und erhalten ein Integral der Form
    \begin{align*}
        \int_{a}^{b} f\of{x} \dif x &= \int_{\varphi^{-1}\of{a}}^{\varphi^{-1}\of{b}} f\of{\varphi\of{t}}\cdot\varphi'\of{t} \dif t
    \end{align*}
    Dazu benötigt man, dass $\varphi: \interv{\alpha, \beta}\fromto\interv{a,b}$ invertierbar ist. (Also zum Beispiel $\varphi' > 0$ oder $\varphi' < 0$ auf ganz $\interv{\alpha, \beta}$)
\end{bemerkung}

\begin{notation}[Leibnitz'sche Schreibweise]
    Schreiben $x=\varphi\of{t}$ oder auch informell $\frac{\dif x}{\dif t} = \varphi'\of{t}$, $\dif x\phantom{|}\text{\anf{=}}\phantom{|}\varphi'\of{t}\dif t$. Eine Anwendung der 2. Lesart wäre das folgende Integral mit $x = \sin\of{t}$
    \begin{align*}
        \int_{0}^{1} \sqrt{1-x^2} \dif x &= \int_{0}^{\frac{\pi}{2}} \sqrt{1-\sin^2\of{t}}\cdot\cos\of{t} \dif t = \int_{0}^{\frac{\pi}{2}} \cos^2\of{t} \dif t = \frac{\pi}{4}\tag{$\frac{\dif x}{\dif t} = \cos t$}\\
    \end{align*}
\end{notation}

\newpage

    \section{[*] Uneigentliche Integrale}
\thispagestyle{pagenumberonly}
Bisher haben wir immer nur Integrale auf kompakten Intervalle $I$ berechnet und dabei waren alle Funktionen $f\in\mR\of{I}$ insbesondere beschränkt.\\
Frage: Was ist $ \int_{0}^{1} \frac{1}{\sqrt{x}} \dif x$? Was ist $ \int_{0}^{\infty} e^{-t} \dif t$?
\begin{align*}
    \int_{a}^{b} e^{-t} \dif t &= \interv{-e^{-t}}_a^b = e^{-0} - e^{-b} = 1 - e^{-b} = 1 -\frac{1}{e^b}\fromto 1 \text{ für } b\fromto\infty
\end{align*}

\subsection{Uneigentliche Integrale: Fall I}
Es sei $I=\linterv{a, \infty}$, $f: I\fromto\R$ und $f\in\mR\of{\interv{a,b}}~\forall a<b<\infty$ sowie $F\of{b} = \int_{a}^{b} f\of{x} \dif x$.
\begin{definition}[Fall]
    Wir definieren
    \begin{align*}
        \int_{a}^{\infty} f\of{x} \dif x &\coloneqq \lim_{b\toinf} F\of{b} = \lim_{b\toinf} \int_{a}^{b} f\of{x} \dif x
    \end{align*}
    sofern der Grenzwert existiert nennen wir das das uneigentliche Integral von $f$ über $\linterv{a,\infty}$. Wenn der Grenzwert existiert, sagen wir das Integral konvergiert.\\
    Divergiert das Integral und gilt $F\of{b}\toinf$ für $b\toinf$ (oder $F\of{b}\fromto -\infty$ für $b\toinf$), so nennen wir das Integral bestimmt divergent und schreiben
    \begin{align*}
        \int_{a}^{\infty} f\of{x} \dif x &= +\infty
        \intertext{oder}
        \int_{a}^{\infty} f\of{x} \dif x &= -\infty
    \end{align*}
\end{definition}

\begin{satz} % Satz 2
    \label{satz:int-uneigentlich-epsilon}
    Das Integral $ \int_{a}^{\infty} f\of{x} \dif x$ existiert genau dann, wenn
    \begin{align*}
        \forall\varepsilon > 0\ex R\geq a\colon \abs{F\of{b_2} - F\of{b_1}} &= \abs{ \int_{b_1}^{b_2} f\of{x} \dif x} < \varepsilon\quad\forall b_1, b_2 \geq R
    \end{align*}
    \begin{proof}
        Wir wollen die Existenz von $\biglim{b\toinf} F\of{b}$ für $F\of{b} = \int_{a}^{b} f\of{x} \dif x$. Dann folgt der Satz aus dem Cauchy-Kriterium für Grenzwerte.
    \end{proof}
\end{satz}

\begin{definition}[Absolut konvergente uneigentliche Integrale]
    Das Integral
    \begin{align*}
        \int_{a}^{\infty} f\of{x} \dif x
        \intertext{heißt absolut konvergent, falls}
        \int_{a}^{\infty} \abs{f\of{x}} \dif x
    \end{align*}
    konvergiert.
\end{definition}

\begin{satz}
    Ist das Integral $\int_{a}^{\infty} f\of{x} \dif x$ absolut konvergent, so ist es auch konvergent. Das heißt ist $ \int_{a}^{\infty} \abs{f\of{x}} \dif x < \infty$, so konvergiert auch $ \int_{a}^{\infty} f\of{x} \dif x$.
    \begin{proof}
        Wir setzen $G\of{b} = \int_{a}^{b} \abs{f\of{x}} \dif x$ und $F\of{b} = \int_{a}^{b} f\of{x} \dif x$. Wir nehmen an, dass $\biglim{b\toinf} G\of{b}$ existiert, das heißt
        \begin{align*}
            \forall\varepsilon > 0\ex R\geq a\colon \abs{G\of{b_2} - G\of{b_1}} &< \varepsilon\quad\forall b_1, b_2\geq R\\
            \impl \abs{F\of{b_2} - F\of{b_1}} &= \abs{ \int_{b_1}^{b_2} f\of{x} \dif x}\\
            &\leq \int_{b_1}^{b_2} \abs{f\of{x}} \dif x = G\of{b_2} - G\of{b_1}
        \end{align*}
        Damit folgt die Behauptung aus Satz~\ref{satz:int-uneigentlich-epsilon}.
    \end{proof}
\end{satz}

\begin{satz} % Satz 5
    \label{satz:int-majorant}
    Sei $\varphi: \linterv{a,\infty}\fromto\linterv{0, \infty}$ mit
    \begin{align*}
        \int_{a}^{\infty} \varphi\of{x} \dif x &< \infty
        \intertext{une es existiert ein $R_0 \geq 0$, sodass}
        \abs{f\of{x}} &\leq \varphi\of{x}\quad\forall x \geq R_0
        \intertext{Dann ist}
        \int_{a}^{\infty} f\of{x} \dif x
    \end{align*}
    absolut konvergent.
    \begin{proof}
        Für $b_2 \geq b_1\geq R_0$ gilt
        \begin{align*}
            \abs{F\of{b_2} - F\of{b_1}} &= \abs{ \int_{b_1}^{b_2} f\of{x} \dif x}\\
            &\leq \int_{b_1}^{b_2} \abs{f\of{x}} \dif x < \int_{b_1}^{b_2} \varphi\of{x} \dif x\\
            &\leq \int_{b_1}^{b_2} \varphi\of{x} \dif x\fromto 0 \text{ für } b_1\toinf
        \end{align*}
    \end{proof}
\end{satz}

\begin{beispiel}
    Das Integral
    \begin{align*}
        \int_{a}^{\infty} \frac{\sin x}{x} \dif x&
        \intertext{ist konvergent, aber nicht absolut konvergent. Wir definieren}
        f\of{x} &= \begin{cases}
                       \frac{\sin x}{x} &x\neq 0\\
                       1 &x = 0
        \end{cases}
        \intertext{Damit ist $f$ stetig auf $\pair{-\infty, \infty}$ und damit folgt $f\in\mR\of{\interv{a,b}}~\forall a,b\in\R$. Insbesondere existiert}
        \int_{0}^{1} \frac{\sin x}{x} \dif x&\\
        \int_{a}^{b} \frac{\sin x}{x} \dif x &= \int_{a}^{1} \frac{\sin x}{x} \dif x + \int_{1}^{b} \frac{\sin x}{x} \dif x\\
        \int_{1}^{b} \frac{\sin x}{x} \dif x &= \interv{-\cos + \frac{1}{x}}_1^b - \int_{1}^{b} \frac{\cos x}{x^2} \dif x\\
        &= \cos 1 - \frac{\cos b}{b} - \int_{1}^{b} \frac{\cos x}{x^2} \dif x
        \intertext{Wir definieren $\varphi\of{x} = \frac{1}{x^2}$ mit}
        \int_{1}^{b} \frac{1}{x^2} \dif x &= \interv{-\frac{1}{x}}_1^b = 1 - \frac{1}{b}\fromto 1
        \intertext{Außerdem gilt}
        \abs{\frac{\cos x}{x^2}} &\leq \frac{1}{x^2}
        \intertext{Damit ist das Integral nach dem Majorantenkriterium konvergent. Um einzusehen, dass es nicht absolut konvergent ist, betrachten wir für $N\in\N$}
        \int_{N\pi}^{\pair{N+1}\pi} \abs{\frac{\sin x}{\pi}} \dif x &= \int_{N\pi}^{\abs{N+1}\pi} \frac{\abs{\sin x}}{x} \dif x\\
        &\geq \frac{1}{\pi\pair{N+1}} \cdot \int_{N\pi}^{\pair{N+1}\pi} \abs{\sin x} \dif x\\
        \impl \int_{0}^{\pair{k+1}\pi} \abs{\frac{\sin x}{x}} \dif x &= \sum_{n=0}^{k} \int_{n\pi}^{\pair{n+1}\pi} \frac{\abs{\sin x}}{x} \dif x\\
        &\geq \sum_{n=0}^{k} \frac{2}{\pi\pair{n+1}} = \frac{2}{\pi} \sum_{n=0}^{k} \frac{1}{n+1}\fromto \infty
    \end{align*}
\end{beispiel}

\begin{bemerkung}
    Analog zu $\linterv{a, \infty}$ wollen wir auch die Integrale in $\rinterv{-\infty, b}$ betrachten. Wir setzen
    \begin{align*}
        F\of{a} &= \int_{a}^{b} f\of{x} \dif x\\
        \int_{-\infty}^{b} f\of{x} \dif x &\coloneqq \lim_{a\fromto -\infty} \int_{a}^{b} f\of{x} \dif x
    \end{align*}
    sofern der Grenzwert existiert. Alle Aussagen für $\linterv{a, \infty}$ gelten analog auch für $\rinterv{-\infty, b}$.
\end{bemerkung}

\begin{definition} % Definition 6
    Sei $f: \pair{-\infty, \infty}\fromto \R$ und $f\in\mR\of{\interv{a,b}}~\forall a,b\in\R$. Dann nehmen wir $c\in\R$ beliebig und definieren, dass
    \begin{align*}
        \int_{-\infty}^{\infty} f\of{x} \dif x&
        \intertext{konvergiert, falls}
        \int_{-\infty}^{c} f\of{x} \dif& \text{ und } \int_{c}^{\infty} f\of{x} \dif x
        \intertext{beide konvergieren. Und setzen}
        \int_{-\infty}^{\infty} f\of{x} \dif x &\coloneqq \int_{-\infty}^{c} f\of{x} \dif x + \int_{c}^{\infty} f\of{x} \dif x
    \end{align*}
\end{definition}

\begin{uebung}
    Weisen Sie nach, dass sowohl die Konvergenz, als auch der Wert des Integrals in der vorherigen Definition unabhängig von der Wahl von $c$ ist.
\end{uebung}

\begin{bemerkung}
    Es ist allerdings zu beachten, dass
    \begin{align*}
        \lim_{a\toinf} \int_{a}^{c} f\of{x} \dif x + \lim_{b\toinf} \int_{c}^{b}  \dif x &\neq \lim_{R\toinf} \int_{-R}^{R} f\of{x} \dif x
    \end{align*}
    Das heißt die Integrale müssen tatsächlich getrennt betrachtet werden. Zum Beispiel bei der Funktion $f\of{x} = x$ geht $ \int_{-R}^{R} x \dif x \fromto 0$, aber ist eigentlich nicht auf $\pair{-\infty, \infty}$ integrierbar, da sich bei der Trennung in zwei Integrale kein Grenzwert ergibt.
\end{bemerkung}

\subsection{Uneigentliche Integrale: Fall II}
Es sei $I=\linterv{a, b}$ (oder $I=\rinterv{a, b}$) und $f:I\fromto\R$ unbeschränkt bei $x=a$ (oder $x=b$). Außerdem $f\in\mR\of{\interv{a,c}}~\forall a<c < b$ (oder $f\in\mR\of{\interv{c, b}}~\forall a < c < b$)

\begin{definition}
    Existiert
    \begin{align*}
        \lim_{c\fromto b-} \int_{a}^{c} f\of{x} \dif x\quad &\pair{\text{oder } \lim_{c\fromto a+} \int_{c}^{b} f\of{x} \dif x}
        \intertext{so setzen wir}
        \int_{a}^{b} f\of{x} \dif x = \lim_{c\fromto b-} \int_{a}^{c} f\of{x} \dif x\quad &\pair{\text{oder } \int_{a}^{b} f\of{x} \dif x = \lim_{c\fromto a+} \int_{c}^{b} f\of{x} \dif x}
        \intertext{und sagen}
        \int_{a}^{b} f\of{x} \dif x
    \end{align*}
    konvergiert.
\end{definition}

\begin{satz}
    Ist $\abs{f\of{x}} \leq \varphi\of{x}~\forall x\in\linterv{a,b}$ (oder $\forall x\in\rinterv{a,b}$) und konvergiert $ \int_{a}^{b} \varphi\of{x} \dif x$, so konvergiert auch $ \int_{a}^{b} f\of{x} \dif x$
\end{satz}

\begin{beispiel}
    Sei $f: \rinterv{0, 1}\fromto\R,~x\mapsto \frac{1}{\sqrt{x}}$. Dann gilt $F\of{x}  = 2\sqrt{x}$
    \begin{align*}
        \int_{0}^{1} \frac{1}{\sqrt{x}} \dif x &= \interv{2\sqrt{x}}_c^1 = 2-2\sqrt{c}\fromto 2
    \end{align*}
\end{beispiel}

\subsection{Uneigentliche Integrale Fall III}
\marginnote{[14. Mai]}
$f$ hat eine Singularität in $\xi$ im Inneren von $\interv{a,b}$.
\begin{beispiel}
    $f\of{x} = \frac{1}{\abs{\sqrt{x}}}$ auf $\linterv{-1, 0} \cup \rinterv{0, 1}$.
\end{beispiel}

\begin{definition}
    Wir sagen, dass
    \begin{align*}
        \int_{a}^{b} f\of{x} \dif x
        \intertext{existiert/konvergiert, falls die uneigentlichen Integrale}
        \int_{\xi}^{b} f\of{x} \dif x &\text{ und } \int_{a}^{\xi} f\of{x} \dif x
        \intertext{konvergieren. Wir setzen}
        \int_{a}^{b} f\of{x} \dif x &\coloneqq \int_{a}^{\xi} f\of{x} \dif x + \int_{\xi}^{b} f\of{x} \dif x\numberthis\label{eq:un-iii}
    \end{align*}
\end{definition}

\begin{bemerkung}
(\ref{eq:un-iii})
    ist stärker als die Existenz von
    \begin{align*}
        \lim_{\varepsilon\searrow} \int_{I_{\varepsilon}}^{} f\of{x} \dif x
    \end{align*}
    mit $I=\interv{a,b}$ und $I_{\varepsilon}\coloneqq I\exclude\pair{\xi-\varepsilon, \xi+\varepsilon} = \interv{a, \xi-\varepsilon} \cup \interv{\xi+\varepsilon, b}$. (Cauchyscher Hauptwert).
\end{bemerkung}

\begin{beispiel}
    Sei $f\of{x} = \frac{1}{x^2}$, $I=\interv{-1, 1}$. Dann existiert der Cauchysche Hauptwert, aber nicht (\ref{eq:un-iii}).
\end{beispiel}

\subsection{Uneigentliche Integrale Fall IV}

\begin{definition}
    Man hat Singularitäten in $\R$ für $f$ oder/und $b=+\infty$, $a=-\infty$. Dann zerlege $\linterv{a, \infty}$ oder $\rinterv{-\infty, b}$ oder $\pair{-\infty, \infty}$ in endlich viele Intervalle, wobei die Singularitäten die Randpunkte sind (oder $-\infty$, $\infty$). Dann existiert das Integral, falls die endlich vielen uneigentlichen Integrale existieren. Dann nehme Summe aller dieser uneigentlichen Integrale
\end{definition}

\begin{satz}[Integralvergleichskriterium] % Satz 10
    \label{satz:integral-vergleich}
    Sei $f: \linterv{1, \infty} \fromto\R$ monoton fallend. Dann gilt
    \begin{align*}
        \sum_{n=1}^{\infty}  f\of{n} \text{ konvergiert } \equivalent \int_{1}^{\infty} f\of{x} \dif x \text{ existiert }
    \end{align*}
    \begin{proof}
        Siehe Saalübung.
    \end{proof}
\end{satz}

\begin{beispiel}
    Es sei $f\of{x} = x^{-p}$ mit $p\neq 1$. Dann ist $F\of{x} = \frac{1}{1-p}x^{1-p}$ für $F'=f$.
    \begin{align*}
        \int_{1}^{\infty} \frac{1}{x^p} \dif x &= \lim_{R\toinf} \interv{\frac{1}{1-p}x^{1-p}}_1^R
    \end{align*}
    existiert nach Satz~\ref{satz:integral-vergleich} für $p>1$.
\end{beispiel}

\begin{beispiel}
    $f\of{x} = \log_2\of{x} = \log\of{\log\of{x}}$, $x > 1$
    \begin{align*}
        \frac{\dif}{\dif x} \log_2\of{x} &= \frac{1}{\log\of{x}}\cdot\frac{1}{x}\\
        \frac{\dif}{\dif x}\pair{\log_2\of{x}}^{1-s} &= \frac{1-s}{\pair{\log x}^s}\cdot \frac{1}{x}\\
        \impl \sum_{n=2}^{\infty} \frac{1}{n\pair{\log^s n}^s} \text{ konvergiert } &\equivalent s > 1
    \end{align*}
\end{beispiel}

\begin{beispiel}[Gamma-Funktion]
    \begin{align*}
        \Gamma\of{x} &\coloneqq \int_{0}^{\infty} t^{x-1} e^{-t} \dif t\tag{$x > 0$}
    \end{align*}
    \begin{enumerate}[label=(\alph*)]
        \item
        \begin{align*}
            t^{x-1}e^{-t} &\leq t^{x-1}\quad\forall t > 0
            \intertext{\item~}
            t^{x-1}e^{-t} &= t^{x-1}e^{-\frac{t}{2}}e^{-\frac{t}{2}}\\
            &\leq c_x e^{-\frac{t}{2}}\quad\forall t\geq 1\tag{$c_x\coloneqq \sup_{t\geq 1} t^{x-1}e^{-\frac{t}{2}}$}
            \intertext{$t^{x-1}e^{-\frac{t}{2}}$ ist beschränkt auf $\linterv{1, \infty}$}
            \int_{0}^{1} t^{x-1}e^{-t} \dif t &\leq \int_{0}^{1} t^{x-1} \dif t\\
            &= \lim_{c\toinf} \interv{\frac{1}{x}t^{x}}_c^1\\
            &= \lim_{c\fromto 0^{+}} \frac{1}{x}\pair{1-e^{x}}\\
            0 &\leq \int_{1}^{\infty} t^{x-1}e^{-t} \dif t\\
            &= \lim_{b\toinf} \int_{a}^{b} t^{x-1}e^{-t} \dif t\\
            &\leq c_x e^{-\frac{t}{2}}\\
            &\leq \lim_{b\toinf} c_x \int_{a}^{b} e^{-\frac{t}{2}} \dif t < \infty\\
            \int_{a}^{b} e^{-\frac{t}{2}} &= \interv{-2e^{-\frac{t}{2}}}_1^b = 2\pair{e^{-\frac{1}{2}} - e^{-\frac{b}{2}}}\fromto 2e^{-\frac{1}{2}}
        \end{align*}
    \end{enumerate}
\end{beispiel}

\begin{satz}[Funktionalgleichung der $\Gamma$-Funktion] % Satz 12
    \label{satz:gamma-funktion}
    Es gilt $\Gamma\of{n+1} = n!$ und $x\Gamma\of{x} = \Gamma\of{x+1}$ für alle $x>0$.
    \begin{proof}
        \begin{align*}
            \Gamma\of{x+1} &= \int_{0}^{\infty} t^{(x+1)-1}e^{-t} \dif t\\
            &= \int_{0}^{\infty} t^{x}e^{-t} \dif t\\
            \intertext{Wir integrieren partiell. Sei $0 < a < b < \infty$}
            \int_{a}^{b} t^{x}e^{-t} \dif t &= \interv{-t^x e^{-t}}_a^b + \int_{a}^{b} xt^{x-1}e^{-t} \dif t\\
            &= a^{x}e^{-b} - b^{x}e^{-b} + x \int_{a}^{b} t^{x-1}e^{-t} \dif t\\
            \impl \int_{a}^{\infty} t^x e^{-t} \dif t &= \lim_{b\toinf} \int_{a}^{b} t^x e^{-t} \dif t\\
            &= a^x e^{-a} + x \int_{a}^{\infty} t^{x-1} e^{-t} \dif t\\
            \impl \Gamma\of{x+1} &= \int_{0}^{\infty} t^x e^{-t} \dif x = x\Gamma\of{x}
            \intertext{Damit folgt die zweite Behauptung. Wir betrachten außerdem}
            \Gamma\of{n+1} &= n\Gamma\of{n} = n\Gamma\of{n-1+1}\\
            &= n\pair{n-1}\Gamma\of{n-1} = n\cdot \pair{n-1}\cdot\ldots\cdot 2 \cdot 1 \cdot \Gamma\of{1}\\
            &= n!\qedhere
        \end{align*}
    \end{proof}
\end{satz}

\begin{anwendung}
    Nach Substitution mit $t^2 = x$ gilt $\frac{\dif t}{\dif x} = \frac{1}{2\sqrt{x}}$
    \begin{align*}
        \int_{a}^{\xi} e^{-t^2} \dif t &= \int_{}^{} e^{-x}\frac{1}{2}\sqrt{x} \dif x\\
        &= \frac{1}{2} \int_{0}^{\infty} \frac{1}{\sqrt{x}}e^{-x} \dif x\\
        &= \frac{1}{2} \int_{?}^{b} s^{-\frac{1}{2}} e^{-s} \dif s\\
        \intertext{für $b\toinf$ und $a\searrow 0$}
        \impl 2 \int_{0}^{\infty} e^{-t^2} \dif x &= \int_{0}^{\infty} s^{-\frac{1}{2}} e^{-s} \dif s\\
        &= \Gamma\of{\frac{1}{2}}
    \end{align*}
    Berechnung von $\Gamma\of{\frac{1}{2}}$ später.
\end{anwendung}

\newpage



    \section{[*] Integrale und gleichmäßige Konvergenz}
    \section{Integrale und gleichmäßige Konvergenz}
\imaginarysubsection{Gleichmäßige Konvergenz}
\thispagestyle{pagenumberonly}

Sei $I=\interv{a,b}$ und $f: I\fromto\R$, $f_n: I\fromto \R$. Wenn die Funktionenfolge $(f_n)_n$ \anf{irgendwie} gegen $f$ konvergiert. Wann gilt dann
\begin{align*}
    \int_{a}^{b} f_n\of{x} \dif x \fromto \int_{a}^{b} f\of{x} \dif x \text{ für } \ntoinf \text{ ?}
\end{align*}
Wir werden in diesem Kapitel einsehen, dass punktweise Konvergenz dafür nicht ausreichend ist, sondern wir gleichmäßige Konvergenz fordern müssen.
\begin{beispiel}[Punktweise Konvergenz]
    Sei $f_n: \interv{0, 1}\fromto\R$ mit
    \begin{align*}
        f_n\of{x} &\coloneqq \begin{cases}
                                 n &0 < x < \frac{1}{n}\\
                                 0 &\text{sonst}
        \end{cases}
        \intertext{$(f_n)_n$ konvergiert punktweise gegen die Nullfunktion ($f_n\of{x}\fromto 0$ für $n\toinf~\forall x\in\interv{0,1}$). Außerdem gilt für ein $n\in\N$}
        \int_{0}^{1} f_n\of{x} \dif x &= \int_{0}^{\frac{1}{n}} n \dif x = \frac{n}{n} = 1
    \end{align*}
    Das Integral über die Nullfunktion ist aber 0. Das heißt punktweise Konvergenz ist kein ausreichendes Kriterium, damit die Integrale gleich sind.
\end{beispiel}

\begin{satz} % Satz 1
    \label{satz:gleichm-int}
    Seien $f, f_n: \interv{a,b}\fromto\R$ (oder $\C, \dots$) und $n\in\N$. Außerdem konvergiere $(f_n)_n$ gleichmäßig gegen $f$ auf $\interv{a,b}$ und $f_n\in\mR\of{\interv{a,b}}$. Dann gilt $f\in\mR\of{\interv{a,b}}$ und
    \begin{align*}
        \lim_{\ntoinf} \int_{a}^{b} f_n\of{x} \dif x &= \int_{a}^{b} f\of{x} \dif x = \int_{a}^{b} \lim_{\ntoinf} f_n\of{x} \dif x
    \end{align*}

    \begin{proof}
        Sei $\varepsilon > 0$ und $N\in\N$ groß genug. Dann gilt
        \begin{align*}
            \norm{f-f_N}_{\infty} &= \sup_{a \leq x \leq b} \abs{f\of{x} - f_N\of{x}} < \frac{\varepsilon}{4\cdot\pair{b-a}}\\
            \impl f_N\of{x} - \frac{\varepsilon}{4\cdot\pair{b-a}} &\leq f\of{x} \leq f_N\of{x} + \frac{\varepsilon}{4\cdot\pair{b-a}}\quad\forall n\geq N\tag{1}
            \intertext{Halte $N$ fest und nehme Zerlegung $Z$ von $I=\interv{a,b}$ mit $\ov{S}_Z\of{f_N} - \un{S}_Z\of{f_N} < \frac{\varepsilon}{2}$. Dann gilt jeweils nach (1)}
            \ov{S}_Z\of{f} \leq \ov{S}_Z\of{f_N + \frac{\varepsilon}{4\cdot\pair{b-a}}} &= \ov{S}_Z\of{f_N} + \ov{S}_Z\of{\frac{\varepsilon}{4\cdot\pair{b-a}}} = \ov{S}_Z\of{f_N} + \frac{\varepsilon}{4}\\
            \un{S}_Z\of{f} \geq \un{S}_Z\of{f_N - \frac{\varepsilon}{4\cdot\pair{b-a}}} &= \un{S}_Z\of{f_N} - \un{S}_Z\of{\frac{\varepsilon}{4\cdot\pair{b-a}}} = \un{S}_Z\of{f_N} - \frac{\varepsilon}{4}\\[.6\baselineskip]
            \impl \ov{S}_Z\of{f} - \un{S}_Z\of{f} &\leq \ov{S}_Z\of{f_N} + \frac{\varepsilon}{4} - \pair{\un{S}_Z\of{f_N} - \frac{\varepsilon}{4}}\\
            &= \ov{S}_Z\of{f_N} - \un{S}_Z\of{f_N} + \frac{\varepsilon}{2}\\
            &< \frac{\varepsilon}{2} + \frac{\varepsilon}{2} = \varepsilon
        \end{align*}
        Damit folgt $f\in\mR\of{I}$. Wir beweisen die Gleichheit der Integrale.
        \begin{align*}
            \int_{a}^{b} f_n\of{x} \dif x - \frac{\varepsilon}{4} &= \int_{a}^{b} \pair{f_n\of{x} - \frac{\varepsilon}{4\cdot\pair{b-a}}} \dif x\\
            &\leq \int_{a}^{b} f\of{x} \dif x \leq \int_{a}^{b} \pair{f_n\of{x} + \frac{\varepsilon}{4\cdot\pair{b-a}}} \dif x\\
            &= \int_{a}^{b} f_n\of{x} \dif x + \frac{\varepsilon}{4}\quad\forall n\geq N\\
            \impl \limsup_{\ntoinf} \int_{a}^{b} f_n\of{x} \dif x - \frac{\varepsilon}{4} &\leq \int_{a}^{b} f\of{x} \dif x \leq \liminf_{\ntoinf} \int_{a}^{b} f_n\of{x} \dif x + \frac{\varepsilon}{4}\quad\forall\varepsilon > 0\\
            \impl \limsup_{\ntoinf} \int_{a}^{b} f_n\of{x} \dif x &\leq \int_{a}^{b} f\of{x} \dif x \leq \liminf \int_{a}^{b} f_n\of{x} \dif x\qedhere
        \end{align*}
    \end{proof}
\end{satz}

\begin{beispiel}[Integral von Potenzreihen]
    \marginnote{[17. Mai]}
    Wir betrachten die Potenzreihe
    \begin{align*}
        f\of{x} &= \sum_{n=0}^{\infty} a_n\pair{x-x_0}^n
        \intertext{mit Konvergenzradius $R>0$ und}
        R &= \frac{1}{\displaystyle \limsup_{\ntoinf} \abs{a_n}^{\frac{1}{n}}}
        \intertext{Wir haben also eine Funktion $f: \pair{x_0 - R, x_0 + R}\fromto\R$ (oder $\C$). Die Stammfunktion zu $a_n\pair{x-x_0}^n$ ist $\frac{a_n}{n+1}\pair{x-x_0}^{n+1}$. Das heißt für endliche Summen können wir leicht ein Integral finden. Ob das auch für unendliche Summen funktioniert, ist nicht direkt klar, aber wir definieren trotzdem in diesem Sinne}
        F\of{x} \coloneqq \sum_{n=0}^{\infty} \frac{a_n}{n+1}&\pair{x-x_0}^{n+1} = \sum_{n=1}^{\infty} c_n\pair{x-x_0}^n\tag{$c_n\coloneqq \frac{a_{n-1}}{n}$}
        \intertext{Wir prüfen den Konvergenzradius}
        \limsup_{\ntoinf} \abs{c_n}^{\frac{1}{n}} &= \limsup_{\ntoinf} \abs{\frac{a_{n-1}}{n}}^{\frac{1}{n}}\\
        \intertext{Es gilt}
        \pair{\frac{\abs{a_{n-1}}}{n}}^{\frac{1}{n}} &= \frac{1}{n^{\frac{1}{n}}}\pair{\abs{a_{n-1}}^{\frac{1}{n-1}}}^{\frac{n-1}{n}}\\
        \impl \limsup_{\ntoinf} \abs{c_n}^{\frac{1}{n}} &= \limsup_{\ntoinf} \abs{a_n}^{\frac{1}{n}}
        \intertext{Das heißt $F$ hat denselben Konvergenzradius wie $f$. Unsere Hoffnung ist also, dass $F$ eine Stammfunktion von $f$ ist, bzw.}
        \int_{x_0}^{x} f\of{t} \dif t &= F\of{x}
        \intertext{Das gilt tatsächlich und lässt sich folgendermaßen zeigen. Wir definieren eine Funktionenfolge}
        f_n\of{x} &\coloneqq \sum_{k=0}^{n} a_k\pair{x-x_0}^k
    \end{align*}
    Wir wissen $\forall\delta > 0$ klein genug (konkret heißt konkret $\delta < R$) konvergiert $f_n$ gleichmäßig gegen $f$ auf dem Intervall $\interv{x_0-R+\delta, x_0+R-\delta}$. Dann gilt für ein festes $x\in\interv{x_0-R+\delta, x_0+R-\delta}$
    \begin{align*}
        \int_{x_0}^{x} f_n\of{t} \dif t &= \int_{x_0}^{x} \sum_{k=0}^{n} a_k\pair{t-x_0}^k \dif t = \sum_{k=0}^{n} a_k \int_{x_0}^{x} \pair{t-x_0}^k \dif t\\
        &= \sum_{k=0}^{n} a_k\cdot \interv{\frac{1}{k+1}\pair{t-x_0}^{k+1}}_{x_0}^{x} = \sum_{k=0}^{n} \frac{a_k}{k+1}\pair{x-x_0}^{k+1}
        \intertext{Das heißt nach Satz~\ref{satz:gleichm-int} wissen wir, dass}
        \int_{x_0}^{x} f\of{t} \dif t &= \lim_{\ntoinf} \int_{x_0}^{x} f_n\of{t} \dif t = \lim_{\ntoinf} \sum_{k=0}^{n} \frac{a_k}{k+1}\pair{x-x_0}^{k+1} \dif x = F(x)
    \end{align*}
\end{beispiel}

\begin{satz}
    \label{satz:ableitung-gleichm-konv}
    Sei $I=\interv{a,b}$ sowie $f_n: I\fromto\R$ (oder $\C$) und die folgenden Voraussetzungen gelten
    \begin{enumerate}[label=(\roman*)]
        \item $\exists x_0\in I\colon f_n\of{x_0}$ konvergiert gegen $f\of{x_0}$
        \item $\pair{f_n'}_n$ konvergiert gleichmäßig gegen eine Funktion $g$
        \item $f_n'$ ist stetig für alle $n\in\N$
    \end{enumerate}
    Dann gilt $f(x) \coloneqq \displaystyle\lim_{\ntoinf} f_n\of{x}~\forall x\in I$ und $f$ ist stetig differenzierbar mit Ableitung $f' = g$.
    \begin{proof}
        Sei $x\in I$. Da alle Ableitungen von $f_n$ stetig sind, können wir den Hauptsatz verwenden und es gilt
        \begin{align*}
            f_n\of{x} - f_n\of{x_0} &= \int_{x_0}^{x} f_n'\of{t} \dif t\\
            \impl f_n\of{x} &= f_n\of{x_0} + \int_{x_0}^{x} f_n'\of{t} \dif t \underset{\ntoinf}{\to} f\of{x_0} + \int_{x_0}^{x} g\of{t} \dif t\\
            \impl f\of{x} &\coloneqq \lim_{\ntoinf} f_n\of{x} \text{ existiert } \forall x\in I \text{ und}\\
            f\of{x} &= f\of{x_0} + \int_{x_0}^{x} g\of{t} \dif t
        \end{align*}
        Nach dem Hauptsatz gilt, dass $f$ stetig differenzierbar ist mit $f' = g$.
    \end{proof}
\end{satz}

\begin{anwendung}
    \label{anwendung:potenzreihe-diff}
    Wir betrachten eine Potenzreihe
    \begin{align*}
        f\of{x} &= \sum_{n=0}^{\infty} a_n\pair{x-x_0}^{n}
        \intertext{mit Konvergenzradius}
        R &= \frac{1}{\displaystyle \limsup_{\ntoinf} \abs{a_n}^{\frac{1}{n}}} > 0\\
        f_n\of{x} &\coloneqq \sum_{k=0}^{n} a_k\pair{x-x_0}^k\\
        \impl f\of{x} &= \lim_{\ntoinf} f_n\of{x}\\
        f_n'\of{x} &= \sum_{k=1}^{\infty} k\cdot a_k\pair{x-x_0}^{k-1}
        \intertext{Es gilt}
        \limsup_{\ntoinf} \abs{n\cdot a_{n+1}}^{\frac{1}{n-1}} &= \limsup_{\ntoinf} \sqrt[n-1]{n} \cdot \abs{a_{n+1}}^{\frac{1}{n-1}} = \limsup_{\ntoinf} \abs{a_{n}}^{\frac{1}{n}}
        \intertext{Das heißt die Konvergenzradien sind gleich und}
        f_n'\of{x} &= \sum_{k=1}^{n} k\cdot a_k\pair{x-x_0}^{k-1}
        \intertext{konvergiert auch auf $\pair{x_0-R, x_0+R}$ und gleichmäßig auf $\interv{x_0-R+\delta, x_0+R-\delta}$. Also konvergiert nach Satz~\ref{satz:ableitung-gleichm-konv} auch die Potenzreihe}
        f\of{x} &= \sum_{n=0}^{\infty} a_n\pair{x-x_0}^n
        \intertext{und ihre Ableitung ist gegeben durch}
        &\sum_{n=1}^{\infty} n\cdot a_n\pair{x-x_0}^{n-1}
    \end{align*}
    Also ist jede Potenzreihe auf ihrem Konvergenzintervall differenzierbar.
\end{anwendung}

\begin{korollar}
    \label{korollar:potenzreihe-diffb}
    Jede Potenzreihe $f\of{x} = \sum_{n=0}^{\infty} a_n\pair{x-x_0}^{n}$ ist unendlich oft differenzierbar auf ihrem Konvergenzintervall.
    \begin{proof}
        Nach Anwendung~\ref{anwendung:potenzreihe-diff} ist eine Potenzreihe einmal differenzierbar mit einer Potenzreihe als Ableitung. Damit folgt induktiv die Behauptung. Insbesondere gilt
        \begin{align*}
            f'\of{x} &= \sum_{n=1}^{\infty} n\cdot a_n\cdot\pair{x-x_0}^{n-1}\\
            f''\of{x} &= \sum_{n=2}^{\infty} n\cdot\pair{n-1}\cdot a_n\cdot\pair{x-x_0}^{n-2}\\
            f^{(k)}\of{x} &= \sum_{n=k}^{\infty} n\cdot\pair{n-1}\cdot\ldots\cdot\pair{n-k+1}\cdot a_n\cdot \pair{x-x_0}^{n-k}\\
            \impl f^{(k)}\of{x_0} &= k!\cdot a_k\\
            \equivalent a_k &= \frac{f^{(k)}\of{x_0}}{k!}
        \end{align*}
    \end{proof}
\end{korollar}

\begin{anwendung}
    Wir wollen
    \begin{align*}
        &\sum_{n=1}^{\infty} n\cdot x^n
        \intertext{für $\abs{x} < 1$ berechnen. Wir wissen, dass für $\abs{x} < 1$}
        \sum_{n=0}^{\infty} x^n &= \frac{1}{1-x}\\
        \impl \sum_{n=1}^{\infty} n\cdot x^n &= x\cdot \sum_{n=1}^{\infty} n\cdot x^{n-1}\\
        &= x\cdot \frac{\dif}{\dif x} \sum_{n=0}^{\infty} x^n = x\cdot \frac{\dif}{\dif x} \frac{1}{1-x}\\
        &= x\cdot \frac{-1}{\pair{1-x}^2}\cdot\pair{-1} = \frac{x}{\pair{1-x}^2}
    \end{align*}
\end{anwendung}

\newpage



    \section{[*] Taylors Theorem}
    \imaginarysubsection{Taylors Theorem}
\thispagestyle{pagenumberonly}

\begin{align*}
    f\of{x} &= f\of{x_0} + \int_{x_0}^{x} f'\of{t} \dif t\numberthis\label{eq:taylor}
\end{align*}

\begin{satz} % Satz 1
    \marginnote{[28. Mai]}
    \label{satz:taylor}
    Sei $f\in\mC^{(n+1)}\of{\pair{a,b}}$ ($n+1$ mal stetig differenzierbar auf $\pair{a,b}$). Dann gilt für alle $x, x_0\in\pair{a,b}$
    \begin{align*}
        f\of{x} &= f\of{x_0} + f'\of{x_0}\pair{x-x_0} + \frac{f''\of{x_0}}{2}\pair{x-x_0}^2\\
        &\quad+ \dots + \frac{f^{(n)}\of{x_0}}{n!}\pair{x-x_0}^n + R_n\of{f, x_0, x}
        \intertext{mit}
        R_n\of{f, x_0, x} &= \frac{1}{n!} \int_{x_0}^{x} \pair{x-t}^{n}f^{(n+1)}\of{t} \dif t
    \end{align*}
    \begin{proof}
        Wir verwenden Induktion. Der Induktionsanfang für $n=1$ ist gerade der Hauptsatz.\\[.5\baselineskip]
        Induktionsschritt: Angenommen $f\in\mC^{(n+2)}$. Dann gilt nach Induktionsannahme
        \begin{align*}
            f\of{x} &= \sum_{k=0}^{n} \frac{f^{(k)}\of{x_0}}{k!}\pair{x-x_0}^{k} + R_n\of{f, x_0, x}\tag{1}\\
            R_n\of{f, x_0, x} &= \frac{1}{n!} \int_{x_0}^{x} \pair{x-t}^{n}f^{(n+1)}\of{t} \dif t
            \intertext{Wir integrieren partiell}
            &= \frac{1}{n!}\pair{\interv{-\frac{1}{n+1}\pair{x-t}^{n+1}f^{(n+1)}\of{t}}_{x_0}^x - \int_{x_0}^{x} \frac{1}{n+1}\pair{x-t}^{n+1}f^{(n+2)}\of{t} \dif t}\\
            &= -\frac{1}{n+1}\cdot\frac{\dif}{\dif t}\pair{x-t}^{n+1}
            \intertext{Nach (1) folgt}
            f\of{x} &= \sum_{k=0}^{n+1} \frac{f^{(k)}\of{x_0}}{k!}\pair{x-x_0}^k + \underbrace{\frac{1}{\pair{n+1}!} \int_{x_0}^{x} \pair{x-t}^{n+1}\cdot f^{(n+2)}\of{t} \dif t}_{=R_{n+1}\of{f, x_0, x}}\qedhere
        \end{align*}
    \end{proof}
\end{satz}

\begin{korollar} % Korollar 1
    Sei $f\in\mC^n\of{\pair{a,b}}$. Dann gilt $\forall x, x_0\in\pair{a,b}$:
    \begin{align*}
        f\of{x} &= \sum_{k=0}^{n} \frac{f^{(k)}\of{x_0}}{k!}\pair{x-x_0}^k + \overline{R}_n\of{f, x_0, x}
        \intertext{mit}
        \ov{R}_n\of{f, x_0, x} &= \frac{1}{\pair{n-1}!} \int_{x_0}^{x} \pair{x-t}^{n-1}\cdot\interv{f^{(n)}\of{t} - f^{(n)}\of{x}} \dif t
    \end{align*}
    \begin{proof}
        Nach Satz~\ref{satz:taylor} gilt
        \begin{align*}
            f(x) &= \sum_{k=0}^{n-1} \frac{f^{(n)}\of{x_0}}{k!}\cdot\pair{x-x_0}^k + \ov{R}_{n-1}\of{f, x_0, x}\\
            &= \sum_{k=0}^{n} \frac{f^{(k)}\of{x_0}}{k!}\pair{x-x_0}^k + R_{n-1}\of{f, x_0, x} - \frac{f^{(n)}\of{x_0}}{n!}\pair{x-x_0}^{n}\\
            \pair{n-1}!\cdot R_{n-1}\of{f, x_0, x} &= \int_{x_0}^{x} \pair{x-t}^{n-1}f{(n)}\of{t} \dif t - \frac{1}{n}f^{(n)}\of{x_0}\pair{x-x_0}^n\\
            &= \int_{x_0}^{x} \pair{x-t}^{n-1}\cdot\interv{f^{(n)}\of{t} - f^{(n)}\of{x_0}} \dif t\qedhere
        \end{align*}
    \end{proof}
\end{korollar}

\begin{bemerkung}
    \begin{align*}
        n!\cdot\abs{\frac{R_n\of{f, x_0, x}}{\pair{x-x_0}^n}} &= \abs{\int_{x_0}^{x} \frac{\pair{x-t}^n}{\pair{x-x_0}^n}f^{(n+1)}\of{t} \dif t}\\
        &\leq \int_{x_0}^{x} \abs{\frac{x-t}{x-x_0}}^n\cdot\abs{f^{(n+1)}\of{t}} \dif t\\
        &\leq \int_{x_0}^{x} \abs{f^{(n+1)}\of{t}} \dif t\fromto 0
    \end{align*}
\end{bemerkung}

\begin{definition}
    Sei $f\in\mC\of{\pair{a,b}}$ und $x_0\in\pair{a,b}$. Wir definieren
    \begin{align*}
        T_n\of{f, x_0}\of{x} &\coloneqq \sum_{k=0}^{n} \frac{f^{(k)}\of{x_0}}{k!}\pair{x-x_0}^k \tag{$n$-tes Taylorpolynom}
        \intertext{Ist $f$ unendlich oft differenzierbar, so nennen wir}
        T\of{f, x_0, x} &= \sum_{n=0}^{\infty} \frac{f^{(n)}\of{x_0}}{n!}\pair{x-x_0}^n
    \end{align*}
    Taylorreihe (von $f$ im Entwicklungspunkt $x_0$).
\end{definition}

\begin{bemerkung}
    \theoremescape
    \begin{enumerate}[label=(\roman*)]
        \item Die Taylorreihe kann Konvergenzradius $R > 0$ haben
        \item Ist eine Taylorreihe konvergent, so muss sie nicht unbedingt gegen $f$ konvergieren
    \end{enumerate}
\end{bemerkung}

\begin{beispiel}
    Wir betrachten $f: \R\fromto\R$
    \begin{align*}
        x\mapsto f\of{x}&\coloneqq \begin{cases}
                                       e^{-\frac{1}{x^2}} & x\neq 0\\
                                       0 &x=0
        \end{cases}
    \end{align*}
    Dann ist $f$ unendlich oft differenzierbar und es gilt $f^{(n)}\of{0} = 0~\forall n\in\N_0$.
    \begin{proof}
        \textsc{Schritt 1}: Sei $x\neq 0$. Dann existiert $\forall n\in\N_0$ ein Polynom $p_n$, sodass
        \begin{align*}
            f^{(n)}\of{x} &= p_n\of{\frac{1}{n}}\cdot e^{-\frac{1}{x^2}}
        \end{align*}
        Wir beweisen diese Behauptung mittels Induktion.~\\
        \begin{induktionsanfang}
            Es ist $n=0$. Wir wählen $p_0\of{x} = 1$.
        \end{induktionsanfang}
        \begin{induktionsschritt}
            \begin{align*}
                f^{(n+1)}\of{x} &= \frac{\dif}{\dif x}\pair{f^{(n)}\of{x}}\\
                &= \frac{\dif}{\dif x}\pair{p_n\of{\frac{1}{x}}\cdot e^{-\frac{1}{x^2}}}\\
                &= p_n'\of{\frac{1}{x}}\cdot\pair{-\frac{1}{x^2}}\cdot e^{-\frac{1}{x^2}} + p_n\of{\frac{1}{x}}\cdot e^{-\frac{1}{x^2}}\cdot\frac{2}{x^3}\\
                &= \underbrace{\pair{-p_n'\of{\frac{1}{x}}\cdot\frac{1}{x^2} + 2p_n\of{\frac{1}{x}}\cdot\frac{1}{x^3}}}_{\eqqcolon p_{n+1}\of{\frac{1}{x}}}\cdot e^{-\frac{1}{x^2}}\\
                p_{n+1}\of{t} &\coloneqq -p'_n\of{t}\cdot t^2 + 2t^3\cdot p_n\of{t}
            \end{align*}
        \end{induktionsschritt}~\\
        \textsc{Schritt 2}: $f^{(n)}\of{0} = 0~\forall n\in\N_0$. Wir nutzen wieder Induktion. Der Induktionsanfang ist klar.
        \begin{induktionsschritt}
            Angenommen $f^{(n)}\of{0} = 0$. Dann gilt
            \begin{align*}
                f^{(n+1)}\of{0} &= \lim_{x\fromto 0} \frac{f^{(n)}\of{x} - f^{(n)}\of{0}}{x}\\
                &= \lim_{x\fromto 0} \frac{f^{(n)}\of{x}}{x}\\
                &= \lim_{x\fromto 0}\pair{\frac{1}{x}\cdot p_n\of{\frac{1}{x}}\cdot e^{-\frac{1}{x^2}}}\\
                &= \lim_{\abs{R}\fromto \infty} \pair{R\cdot p_n\of{R}\cdot e^{-R^2}} = 0\qedhere
            \end{align*}
        \end{induktionsschritt}
    \end{proof}
\end{beispiel}

\begin{satz} % Satz 4
    Ist $f\of{x} = \sum_{n=0}^{\infty} a_n\pair{x-x_0}^{n}$ eine Potenzreihe mit Konvergenzradius $r>0$, so ist die Taylorreihe von $f$ gleich dieser Potenzreihe.
    \begin{proof}
        Folgt aus Korollar~\ref{korollar:potenzreihe-diffb} und Gleichung ??.
    \end{proof}
\end{satz}

\begin{beispiel}
    \begin{align*}
        \sum_{n==}^{\infty} \frac{\pair{cx}^n}{n!} &= \sum_{n=0}^{\infty} \frac{c^n}{n!}\cdot x^n\\
        \exp\of{cx} &= \exp\of{cx_0 + c\pair{x-x_0}}\\
        &= \exp\of{cx_0}\cdot\exp\of{c\pair{x-x_0}}\\
        &= \exp\of{cx_0}\cdot \sum_{n=0}^{\infty} \frac{c^n}{n!}\pair{x-x_0}^n\\
        &= \sum_{n=0}^{\infty} \frac{\exp\of{cx_0}c^n}{n!}\pair{x-x_0}^n
    \end{align*}
\end{beispiel}

\begin{satz}[Restglieddarstellung von Schlömilch] % Satz 5
    \label{satz:restglied-schloemilch}
    Sei $f\in\mC^{n+1}\of{\pair{a,b}}$ und $x_0\in\pair{a,b}$. Dann gilt
    \begin{align*}
        f\of{x} &= T_n\of{f, x_0, x} + R_n\of{f, x_0, x}
        \intertext{mit}
        R_n\of{f, x_0, x} &= \frac{1}{p \cdot n!}\cdot f^{(n+1)}\of{\xi}\cdot\pair{x-\xi}^{n+1-p}\pair{x-x_0}^p \tag{8}\\
        \forall 1 \leq p \leq n+1 \text{ und } \xi \text{ zwischen } x_0 \text{ und } x
    \end{align*}
\end{satz}

\begin{bemerkung}
    Ist $p=n+1$, dann haben wir die Lagrangsche Darstellung
    \begin{align*}
        R_n\of{f, x_0, x} &= \frac{1}{\pair{n+1}!}\cdot f^{(n+1)}\of{\xi}\cdot\pair{x-x_0}^{n+1}\of{\xi}
        \intertext{und wenn $p=1$, dann haben wir die Cauchysche Darstellung}
        R_n\of{f, x_0, x} &= \frac{1}{n!}\cdot f^{(n+1)}\of{\xi}\cdot\pair{x-\xi}^n\cdot\pair{x-x_0}
    \end{align*}
    für das Restglied.
\end{bemerkung}

\begin{satz}[Logarithmus] % Satz 6
    Für die Logarithmusreihe $f_n: -1\leq x \leq 1$ gilt
    \begin{align*}
        \log\of{1+x} &= x-\frac{x^2}{2} + \frac{x^2}{3} \pm \dots = \sum_{n=1}^{\infty} \pair{-1}^{n+1}\cdot \frac{x^n}{n}
    \end{align*}

    \begin{proof}
        \begin{align*}
            f\of{x} &= \log\of{1+x}\\
            f'\of{x} &= \pair{1+x}^{-1}\\
            f''\of{x} &= -1\cdot\pair{1+x}^{-2}\\
            \vdots\\
            f^{(n)}\of{x} &= \pair{-1}^{n+1}\cdot\pair{n-1}!\cdot \pair{1+x}^{-n}\\
            T_n\of{f, 0}\of{x} &= \sum_{k=0}^{n} \frac{f^{(k)}\of{0}}{k} \cdot x^k = \sum_{k=0}^{n} \pair{-1}^{n+1}\cdot \frac{\pair{k-1}}{k!}x^k\\
            &= \sum_{k=0}^{n} \pair{-1}^{k+1}\cdot \frac{x^k}{k}
            \intertext{\textsc{Schritt 1}: Aus Satz~\ref{satz:restglied-schloemilch} folgt}
            f\of{x} &= \sum_{k=1}^{n} \pair{-1}^{k+1}\cdot \frac{x^k}{k} + R_n\of{f, 0, x}\\
            R_n\of{f, 0, x} &= \frac{1}{pn!}\cdot f^{(n+1)}\of{x}\cdot\pair{x-\xi}^{n+1-p}\cdot\pair{x-x_0}^p\\
            &= n!\cdot\pair{-1}^{n+1}\cdot\pair{1+\xi}^{-\pair{n+1}}\\
            \impl \abs{R_n\of{f, 0, x}} &= \frac{1}{pn!}\cdot n! \cdot\pair{1+\xi}^{-n-1}\cdot\abs{x-\xi}^{n+1-p}\cdot\abs{x}^p
            \intertext{Angenommen $0\leq x \leq 1$. $0 < \xi < x$, wir wählen $p=n+1$}
            \impl \abs{R_n\of{f, 0, x}}&\leq \frac{1}{p} = \frac{1}{n+1}\fromto 0\marginnote{[31. Mai]}
            \intertext{Angenommen $-1\leq x \leq 0$. Dann gibt es ein $\xi$ zwischen 0 und $x$, das heißt $\xi = \Theta x$ mit $0 < \Theta < 1$. Dann gilt}
            R_n\of{f, 0, x} &= \frac{1}{p}\cdot\pair{-1}^n\cdot\pair{1+\Theta x}^{-(n+1)}\cdot\pair{x-\Theta x}^{n+1-p}\cdot x^p\\
            \impl \abs{R_n\of{f, 0, x}} &= \frac{1}{p} \cdot \pair{1+\Theta x}^{-(n+1)}\cdot\abs{x}^{n+1-p}\cdot\pair{1-\Theta}^{n+1-p}\cdot\abs{x}^p\\
            &= \frac{1}{p}\cdot \pair{1+\Theta x}^{-(n+1)}\cdot \pair{1-\Theta}^{n+1-p}\cdot \abs{x}^{n+1}
            \intertext{Da $-1\leq x \leq 0$}
            \impl 1 + \Theta x &= 1 - \Theta\cdot\abs{x} \geq 1 - \Theta > 0\\
            \impl \pair{1+\Theta x}^{-n} &\leq \pair{1-\Theta}^{-n}\\
            \impl \abs{R_n\of{f, 0, x}} &\leq \frac{1}{p}\cdot\pair{1-\Theta}^{-n}\cdot\pair{1-\abs{x}}^{-1}\cdot\pair{1-\Theta}^{n+1-p}\cdot\abs{x}^{n+1}
            \intertext{Wähle $p=1$}
            \impl \abs{R_n\of{f, 0, x}} &\leq \pair{1-\Theta}^{-n}\cdot\pair{1-\Theta}^{n} \cdot \frac{\abs{x}^{n+1}}{1-\abs{x}}= \frac{\abs{x}^{n+1}}{1-\abs{x}}\fromto 0
            \intertext{\textsc{Schritt 2}: Wir wollen zeigen, dass die Taylorreihe $ \sum_{n=1}^{\infty} \frac{(-1)^{n+1}}{n}\cdot x^n$ für alle $-1\leq x \leq 1$ konvergiert. Für $-1\leq x \leq 0$ gilt}
            \abs{\frac{(-1)^{n+1}}{n}\cdot x^n} &\leq \frac{1}{n}\cdot\abs{x}^n\leq \abs{x}^n
        \end{align*}
        Damit folgt die Konvergenz aus dem Vergleich mit der geometrischen Reihe. Das gleiche Prinzip lässt sich für $0\leq x< 1$ anwenden. Für $x=1$ ist $ \sum_{n=1}^{\infty} \frac{(-1)^{n+1}}{n}$ eine alternierende monotone Reihe, die damit nach Leibniz konvergiert.\\[.2\baselineskip]
        Aus \textsc{Schritt 1} und \textsc{Schritt 2} folgt damit die Behauptung.
    \end{proof}
\end{satz}

\begin{korollar}
    Für $a > 0$ und $0< x \leq 2a$ folgt
    \begin{align*}
        \log x &= \log a + \sum_{n=1}^{\infty} \frac{\pair{-1}^{n+1}}{n\cdot a^n}\pair{x-a}^n
    \end{align*}
    \begin{proof}
        \begin{align*}
            \log x &= \log\of{a+\pair{x-a}} = \log\of{a\cdot\pair{1+\frac{x}{a}}} \\
            &= \log a + \log\of{1+\frac{x}{a}}\qedhere
        \end{align*}
    \end{proof}
\end{korollar}

\begin{bemerkung}
    Es gilt
    \begin{align*}
        \log 2 &= \log\of{1+1} = \sum_{n=1}^{\infty} \frac{(-1)^{n+1}}{n}\tag{konvergiert langsam}\\
        \log\of{1+x} &= \sum_{n=1}^{\infty} \frac{(-1)^{n+1}}{n}\cdot x^n\\
        \log\of{1-x} &= \sum_{n=1}^{\infty} \frac{(-1)^{n+1}}{n}\cdot (-1)^n\cdot x^n = - \sum_{n=1}^{\infty} \frac{x^n}{n}\\
        \impl \log\of{1+x} - \log\of{1-x} &= \sum_{n \text{ ungerade}}^{} \pair{\frac{x^n}{n}+\frac{x^n}{n}} = 2\cdot \sum_{k=0}^{\infty} \frac{x^{2k+1}}{2k+1}= \log\of{\frac{1+x}{1-x}}
        \intertext{Für ein $y>1$ mit $y=\frac{1+x}{1-x}$ gilt}
        \pair{1-x}\cdot y &= 1+x\\
        \equivalent y-1 &= x\cdot\pair{y+1}\\
        \equivalent x &= \frac{y-1}{y+1}
        \intertext{Für $y=2$ gilt also $x=\frac{1}{3}$. Das heißt}
        \log y &= 2\cdot \sum_{k=0}^{\infty} \frac{1}{2k+1}\cdot\pair{\frac{y-1}{y+1}}^{2k+1}\\
        \impl \log 2 &= 2\cdot \sum_{k=0}^{\infty} \frac{1}{2k+1}\cdot\pair{\frac{1}{3}}^{2k+1}\tag{konvergiert schneller}
    \end{align*}
\end{bemerkung}

\begin{satz}[Abelscher Grenzwertsatz] % Satz 8
    \label{satz:abel-grenzwert}
    Angenommen $\displaystyle \sum_{n=0}^{\infty} a_n$ konvergiert. Dann ist die Potenzreihe $\displaystyle f(x) \coloneqq\sum_{n=0}^{\infty} a_n\cdot x^n$
    \begin{enumerate}[label=(\roman*)]
        \item konvergent für alle $-1 < x \leq 1$
        \item stetig in $x=1$ und
        \item Die Potenzreihe konvergiert gleichmäßig auf allen Intervallen $\interv{a, 1}$ mit $-1 < a < 1$. (Das heißt sie konvergiert lokal gleichmäßig auf $\interv{-1, 1}$). Insbesondere in jeder $\varepsilon$-Umgebung um $x=1$.
    \end{enumerate}

    \begin{proof}
        \textsc{Schritt 1}: Wir zeigen zunächst (ii) und setzen dafür
        \begin{align*}
            A_n &\coloneqq \sum_{k=n+1}^{\infty} a_{k}\fromto 0 \text{ für } \ntoinf
            \intertext{Insbesondere ist}
            \sup_{n\geq 0} \abs{A_n} &< \infty\\
            \impl \sup_{n\geq k+1} \abs{A_n} &\fromto 0 \text{ für } k\toinf\\
            a_n &= A_{n-1} - A_n\tag{Wir setzen $A_{-1} = \sum_{n=0}^{\infty} a_n$}
            \intertext{Für ein $L\in\N$ gilt}
            \sum_{n=0}^{L} a_n \cdot x^n &= \sum_{n=0}^{L} \pair{A_{n-1}-A_n}\cdot x^n\\
            &= \sum_{n=0}^{L} A_{n-1}\cdot x^n - \sum_{n=0}^{L} A_n\cdot x^n\\
            &= \sum_{j=-1}^{L-1} A_j\cdot x^{j+1} - \sum_{j=0}^{L} A_j \cdot x^j\\
            &= A_{-1}\cdot x^0 - A_{L}\cdot x^L + \sum_{n=0}^{L} A_n\cdot\pair{x^{n+1}-x^n}\\
            &= f\of{1} - A_{L}\cdot x^L + \pair{x-1}\cdot \sum_{n=0}^{L-1} A_n \cdot x^n
            \intertext{Es gilt $\abs{A_L\cdot x^L} \leq \abs{A_L}$ und $\abs{A_n}\leq C$ für eine Konstante $C$. Das heißt für $\abs{x} < 1$}
            \impl \sum_{n=0}^{\infty} A_n\cdot x^n &\text{ hat Limes für }L\toinf\\
            \impl f\of{x} &= \lim_{L\toinf} \sum_{n=0}^{L} a_n\cdot x^n = f\of{1}+ \pair{x-1}\cdot \sum_{n=0}^{\infty} A_n\cdot x^n\\
            \impl \abs{f\of{1} - f\of{x}} &= \pair{1-x} \cdot \abs{\sum_{n=0}^{\infty} A_n \cdot x^n} \leq \pair{1-x} \cdot \sum_{n=0}^{\infty} \abs{A_n}\cdot x^n
            \intertext{Sei $K\in\N$. Dann gilt}
            \impl \abs{f\of{1} - f\of{x}} &\leq \pair{1-x}\cdot \sum_{n=0}^{K} \abs{A_n}\cdot x^n + \pair{1-x} \cdot \sum_{n=K+1}^{\infty} \abs{A_n}\cdot x^n\\
            &\leq \underbrace{\pair{1-x} \cdot \sup_{n\geq 0}\of{\abs{A_n} }\cdot \sum_{n=0}^{K} x^n}_{\eqqcolon I_{K}\of{x}} + \underbrace{\pair{1-x}\cdot \sup_{n\geq K+1}\of{\abs{A_n}}\cdot \sum_{n=K+1}^{\infty} x^n}_{\eqqcolon J_{K}\of{x}}
            \intertext{Für ein festes $K\in\N$ geht $I_{K}\fromto 0$ für $x\fromto 1-$ und es gilt}
            J_{K}\of{x} &= \sup_{n\geq K + 1}\of{\abs{A_n}}\cdot\pair{1-x}\cdot \sum_{n=K+1}^{\infty} x^n
            \intertext{Nach der geometrischen Summenformel gilt}
            &= \sup_{n\geq K + 1}\of{\abs{A_n}}\cdot \pair{1-x}\cdot\frac{x^{K+1}}{1-x}\\
            &\leq \sup_{n\geq K + 1}\of{\abs{A_n}}\fromto 0 \text{ für } L\toinf \tag{gleichmäßig in $0\leq x <  1$}\\
            \impl \limsup_{x\fromto 1-} \abs{f\of{1} - f\of{x}} &\leq 0 + \limsup_{x\fromto 1-} J_K\of{x}\\
            &\leq \sup_{n\geq K+1}\of{\abs{A_n}}\fromto 0 \text{ für } K\toinf\quad\forall K\in\N\\
            \impl \limsup_{x\fromto 1-} \abs{f\of{1}-f\of{x}} &= 0\\
            \impl \lim_{x\fromto 1-} f\of{x} &= f\of{1}
            \intertext{\textsc{Schritt 2}: $f_n\of{x} = \sum_{k=0}^{n} a_k\cdot x^k$}
            \impl f\of{x} - f_n\of{x} &= \pair{x-1}\cdot \sum_{k=n+1}^{\infty} A_k \cdot x^k - A_n\cdot x^n\\
            \impl \abs{f\of{x} - f_n\of{x}} &\leq \pair{1-x} \cdot \sum_{k=n+1}^{\infty} \abs{A_k}\cdot x^k + \abs{A_n}\cdot x^n\tag{$0\leq x < 1$}\\
            &\leq \pair{1-x} \cdot \sup_{k\geq n+1}\of{\abs{A_k}}\cdot \sum_{k=n+1}^{\infty} x^k + \abs{A_n}\\
            &\leq \sup_{k\geq n+1}\of{\abs{A_k}} \cdot\pair{1-x}\cdot x^{n+1}\cdot \sum_{k=0}^{\infty} x^k + \abs{A_n}\\
            &\leq 2\cdot \sup_{k\geq n}\of{\abs{A_k}}
            \intertext{Mit (ii) folgt $\fa 0\leq x \leq 1$}
            \abs{f\of{x} - f_n\of{x}} &\leq 2\cdot \sup_{k\geq n}\of{\abs{A_k}}\\
            \impl \sup_{0 \leq x \leq 1}\of{\abs{f\of{x}-f_n\of{x}}} &\leq 2\cdot \sup_{k\geq n}\of{\abs{A_k}}
        \end{align*}
        Das heißt $(A_n)_n$ ist eine Nullfolge. Damit gilt gleichmäßige Konvergenz auf $\interv{0,1}$.\\
        $f\of{x}$ konvergiert gleichmäßig auf kompakten Teilintervallen innerhalb des Konvergenzradius und $ \sum_{}^{} a_n$ konvergiert mit Konvergenzradius $R\geq 1$. Das heißt $f\of{x}$ konvergiert gleichmäßig auf allen $\interv{-\delta, \delta}$ für $0<\delta < 1$.
    \end{proof}
\end{satz}

\begin{satz}[Arctan Reihe] % Satz 8?
    Für $\abs{x} \leq 1$ gilt
    \begin{align*}
        \arctan x &= x- \frac{x^3}{3} + \frac{x^5}{5} \pm \dots\\
        &= \sum_{n=0}^{\infty} \pair{-1}^n \cdot \frac{x^{2n+1}}{2n+1}
    \end{align*}
    \begin{proof}
        Es sei $f\of{x} = \arctan x$. Dann gilt
        \begin{align*}
            f'\of{x} &= \frac{1}{1+x^2} = \frac{1}{1-\pair{-x}^2}\\
            &= \sum_{n=0}^{\infty} \pair{-x^2}^n = \sum_{n=0}^{\infty} \pair{-1}^n \cdot x^{2n}
            \intertext{Nach dem Hauptsatz gilt}
            f\of{x} &= f\of{0} + \int_{0}^{x} f'\of{t} \dif t\\
            &= 0 + \int_{0}^{x} \frac{1}{1+t^2} \dif t\\
            &= \int_{0}^{x} \sum_{n=0}^{\infty} \pair{-1}^n\cdot t^{2n}  \dif t\\
            &= \sum_{n=0}^{\infty} \pair{-1}^{2n}\cdot \int_{0}^{x} t^{2n} \dif t\\
            &= \sum_{n=0}^{\infty} \pair{-1}^n\cdot \frac{x^{2n+1}}{2n+1} \text{ falls } \abs{x} < 1
        \end{align*}
        Für $x=1$ gilt
        \begin{align*}
            f\of{x} &= \sum_{n=0}^{\infty} \pair{-1}^n \cdot \frac{x^{2n+1}}{2n+1}\\
            f\of{1} &= \sum_{n=0}^{\infty} \pair{-1}^n \cdot \frac{1^{2n+1}}{2n+1}\\
            &= \sum_{n=0}^{\infty} \frac{(-1)^n}{2n+1} \text{ konvergiert nach Leibniz }
        \end{align*}
        Das heißt aus Satz~\ref{satz:abel-grenzwert} folgt die gleichmäßige Konvergenz von dieser Reihe für alle $\abs{x} \leq 1$.\\
        Das heißt aus der Stetigkeit von $\arctan$ bei $\pm 1$ und dem Satz folgt
        \begin{align*}
            \arctan x &= \sum_{n=0}^{\infty} \pair{-1}^n\cdot \frac{x^{2n+1}}{2n+1}\quad\forall \abs{x}\leq 1\qedhere
        \end{align*}
    \end{proof}
\end{satz}

\begin{bemerkung}[Reihendarstellung von $\pi$]
    \marginnote{[04. Jun]}
    Es gilt $\tan x = \frac{\sin x}{\cos x}$ und damit $1=\tan \frac{\pi}{4}$, $\arctan 1 = \frac{\pi}{4}$. So ergibt sich mit dem Arctan eine Reihendarstellung von $\pi$
    \begin{align*}
        \frac{\pi}{4} &= \sum_{n=0}^{\infty} \frac{\pair{-1}^n}{2n+1} = 1 - \frac{1}{3} + \frac{1}{5} - \frac{1}{7} + \dots
        \intertext{Diese Reihe konvergiert für die tatsächliche Anwendung allerdings zu langsam. Viel schneller ist die Berechnung über die \emph{Machinsche Formel}}
        \frac{\pi}{4} &= 4\cdot\arctan \frac{1}{5} - \arctan \frac{1}{239}
    \end{align*}
\end{bemerkung}

\begin{satz}[Binomische Reihe]
    Sei $\alpha\in\R$. Dann gilt für $\abs{x} < 1$
    \begin{align*}
        \pair{1+x}^{\alpha} &= \sum_{n=0}^{\alpha} \binom{\alpha}{n} x^n\tag{\footnotemark}
        \intertext{wobei wir den verallgemeinerten Binomialkoeffizient verwenden}
        \binom{\alpha}{n} &\coloneqq \prod_{k=1}^{n} \frac{\alpha - k +1}{k} = \frac{\alpha\cdot\pair{\alpha-1}\cdot\ldots\cdot\pair{\alpha-k+1}}{k!}\\
        \binom{\alpha}{n} &\coloneqq 0 \text{ für } n\geq \alpha + 1
        \intertext{Daraus folgt der speziellere Binomische Lehrsatz für $m\in\N$}
        \impl \pair{1+x}^m &= \sum_{n=0}^{m} \binom{m}{n}\cdot x^n
    \end{align*}
    \footnotetext{Gefunden von Newton 1665}
    \begin{proof}
        \textsc{Schritt 1}: Sei $f\of{x} = \pair{1+x}^{\alpha}$ für $x > -1$. Dann gilt
        \begin{align*}
            f'\of{x} &= \alpha\cdot\pair{1+x}^{\alpha-1}\\
            f''\of{x} &= \alpha\cdot\pair{\alpha-1}\cdot\pair{1+x}^{\alpha-2}\\
            \vdots\\
            f^{(n)}\of{x} &= \alpha\cdot\pair{\alpha-1}\cdot \ldots\cdot \pair{\alpha-n+1}\cdot\pair{1-x}^{\alpha-n}
            \intertext{Das heißt die Taylorreihe für $f$ in $0$ ist}
            T\of{f, 0}\of{x} &= \sum_{n=0}^{\infty} \frac{f^{(n)}\of{0}}{n!}\cdot x^n\\
            &= \sum_{n=0}^{\infty} \frac{\alpha\cdot\pair{\alpha-1}\cdot\ldots\cdot\pair{\alpha-n+1}}{n!}\cdot x^n = \sum_{n=0}^{\infty} \binom{\alpha}{n} \cdot x^n
            \intertext{\textsc{Schritt 2}: Wir wollen zeigen, dass die obige Taylorreihe konvergiert}
            a_n &\coloneqq \binom{a}{n}x^n\quad \abs{x} < 1\\
            \abs{\frac{a_{n+1}}{a_n}} &= \abs{\frac{\binom{\alpha}{n+1}\cdot x^{n+1}}{\binom{\alpha}{n}\cdot x^n}}\\
            &= \abs{x}\cdot \abs{\frac{\alpha-n}{n+1}} \underset{\ntoinf}{\fromto} \abs{x} < 1\\
            \impl \exists x < 1, N_0\in\N\colon \abs{\frac{a_{n+1}}{a_n}} &\leq x < 1 \quad\forall n\geq N_0\\
            \impl \sum_{n=0}^{\infty} a_n &= \sum_{n=0}^{\infty} \binom{\alpha}{n}x^n \text{ ist absolut konvergent }
            \intertext{\textsc{Schritt 3}: Der Restterm soll verschwinden. Sei $0 < \Theta < 1$ und $\xi=\Theta x$, sowie $1 \leq p \leq n+1$}
            R_n\of{f, 0, x} &= \frac{1}{p\cdot n!} \cdot f^{(n+1)}\cdot \pair{\Theta x}\cdot\pair{x-\Theta x}^{n+1-p}\cdot x^p\tag{Schlömilch}
            \intertext{Für $p=1$ ergibt sich die Restglieddarstellung von Cauchy}
            R_n\of{f, 0, x} &= \frac{1}{n!}\cdot \alpha\cdot\pair{\alpha-1}\cdot\ldots\cdot \pair{\alpha - n+1}\cdot\pair{\alpha-n}\cdot \pair{1+\Theta x}^{-n-1}\cdot\pair{x-\Theta x}^n\cdot x\\
            &= \binom{\alpha}{n+1}\cdot x^{n+1}\cdot\pair{1-\Theta}^n\cdot\pair{1+\Theta x}^{-(n+1)}\\
            \impl \abs{R_n\of{f, 0, x}} &= \underbrace{\abs{\binom{\alpha}{n+1}\cdot x^{n+1}}}_{\fromto 0\text{ nach \textsc{Schritt 2}}} \cdot\pair{1-\Theta}^{n}\cdot \pair{1+\Theta x}^{-n-1}\\
            \pair{1+\Theta x}^{-(n+1)} &= \frac{1}{\pair{1+\Theta x}^{-n+1}}\\
            &= \frac{1}{1+\Theta x}\cdot \frac{1}{\pair{1+\Theta x}^{n}}\\
            &\leq \frac{1}{1-\abs{x}}\cdot \frac{1}{\pair{1-\Theta}^n}\\[5pt]
            \impl \abs{R_n\of{f, 0, x}} &\leq \abs{\binom{\alpha}{n+1}\cdot x^{n+1}}\cdot \frac{1-\Theta^n}{1-\Theta}\cdot \frac{1}{1-\abs{x}}\\
            &= \abs{\binom{\alpha}{n+1}\cdot x^{n+1}} \cdot \frac{1}{1-\abs{x}} \underset{\ntoinf}{\fromto}
            \intertext{Das heißt nach dem Satz von Taylor gilt}
            f\of{x} &= \sum_{n=0}^{\infty} \binom{\alpha}{n}\cdot x^n\qedhere
        \end{align*}
    \end{proof}
\end{satz}

\newpage


    \section{[*] Die Gamma-Funktion}
    \section{Die Gamma-Funktion}
\imaginarysubsection{Die $\Gamma$-Funktion}
\thispagestyle{pagenumberonly}

Erinnerung: Die $\Gamma$-Funktion ist für $x>0$ definiert als
\begin{align*}
    \Gamma\of{x} &\coloneqq \int_{0}^{\infty} t^{x-1}\cdot e^{-t} \dif t
    \intertext{Das funktioniert bei $0$, da $x - 1 > -1$ und es funktioniert bei $\infty$, da}
    C_x &\coloneqq \sup_{t \geq 1} t^{x-1}\cdot e^{-\frac{t}{2}} < \infty
    \intertext{und}
    t^{x-1}\cdot e^{-t} &= t^{x-1}\cdot e^{-\frac{t}{2}}\cdot e^{-\frac{t}{2}}\leq C_x\cdot e^{-\frac{t}{2}}\\
    \impl \Gamma\of{x} &= \lim_{a\searrow 0}\lim_{b\toinf} \int_{a}^{b} t^{x-1}\cdot e^{-t} \dif t \text{ existiert } \forall x>0
    \intertext{Wir hatten außerdem in Satz~\ref{satz:gamma-funktion}, dass}
    \Gamma\of{x+1} &= x\cdot\Gamma\of{x}\quad\forall x > 0
\end{align*}

\begin{definition}[Konvexität\footnote{Siehe auch Skript \textsc{Analysis I}, Kapitel 19}]
    Eine Funktion $F: I\fromto \R$ -- wobei $I$ ein Intervall ist ($I=\linterv{0, \infty}$ ist dabei erlaubt) -- heißt konvex, falls $\forall x,y\in I$ und für alle $0\leq \Theta \leq 1$ gilt
    \begin{align*}
        F\of{\Theta x + \pair{1-\Theta}y} &\leq \Theta F\of{x} + \pair{1-\Theta}F\of{y}
    \end{align*}
\end{definition}

\begin{skizze}[Konvexe Funktion]
    Wähle ein $\Theta\in\pair{0,1}$ und formuliere die Interpolation $\Theta x + \pair{1-\Theta}\cdot y$.
    \begin{figure}[H]
        \centering
        \begin{tikzpicture}
            \draw[->] (-1, 0) -- (3, 0);
            \draw[->] (0, -1) -- (0, 4);
            \draw (-0.95*0.5, 0.1) -- (-0.95*0.5, -0.1) node[below] {$x$};
            \draw (5.2*0.5, 0.1) -- (5.2*0.5, -0.1) node[below] {$y$};

            \fill (-0.95*.5,0.7875*.5) circle[radius=1.5pt] node[left] {$F(x)$};
            \fill (5.2*.5,5.4*.5) circle[radius=1.5pt] node[right] {$F(y)$};
            \draw[domain=-2:6, smooth, variable=\x] plot ({0.5*\x}, {(0.2*(\x-0.25)^2+0.5)*0.5}) node[anchor=east] {$F$};
            \draw[domain=-0.95:5.2, smooth, variable=\x, dashed] plot ({0.5*\x}, {(0.75*\x+1.5)*0.5});
        \end{tikzpicture}
        \caption{Konvexe Funktion mit eingezeichneter Sekante}
    \end{figure}
\end{skizze}

\begin{beispiel}
    Die Funktionen $F\of{t} = e^{t}$ und $F\of{t} = e^{-t}$ sind konvex auf $\R$. (Übung)
\end{beispiel}

\begin{definition}
    Eine Funktion $F$ heißt konkav, falls $-F$ konvex ist. Das heißt
    \begin{align*}
        F\of{\Theta x + \pair{1-\Theta}y} &\geq \Theta F\of{x} + \pair{1-\Theta}F\of{y}\quad\fa 0\leq \Theta \leq 1\fa x,y\in I
    \end{align*}
\end{definition}

\begin{definition}[Logarithmische Konvexität]
    Eine Funktion $F$ heißt logarithmisch konvex, falls $\log \circ F = \log\of{F}$ konvex ist. Das heißt
    \begin{align*}
        \log F\of{\Theta x + \pair{1-\Theta} y} &\leq \Theta\log F\of{x} + \pair{1-\Theta}\cdot \log F\of{y}\\
        &= \log\of{F\of{x}^{\Theta}} + \log\of{F\of{y}^{1-\Theta}}\\
        &= \log\of{F\of{x}^{\Theta}\cdot F\of{y}^{1-\Theta}}\\
        \intertext{Das heißt die Bedingung für logarithmische Konvexität ist äquivalent zu}
        \equivalent F\of{\Theta x + \pair{1-\Theta}y} &\leq F\of{x}^{\Theta} \cdot F\of{y}^{1-\Theta}\quad\fa x,y\in I\fa 0\leq \Theta \leq 1\numbereq{eq:logarithm-konvex}
    \end{align*}
\end{definition}

\begin{satz} % Satz 2
    Die Gamma-Funktion $\Gamma: \pair{0, \infty}\fromto \pair{0, \infty},~x\mapsto \Gamma\of{x}$ ist logarithmisch konvex.
    \begin{proof}
    (Übung)
    \end{proof}
\end{satz}

\begin{satz}[Bohr, Mollerup] % Satz 3
    Ist $F: \pair{0, \infty}\fromto\pair{0, \infty}$ eine Funktion mit
    \begin{enumerate}[label=(\alph*)]
        \item $F\of{1} = 1$
        \item $F\of{x+1} = x\cdot F\of{x}$ und
        \item $F$ ist logarithmisch konvex
    \end{enumerate}
    Dann gilt $F = \Gamma$, das heißt $F\of{x} = \Gamma\of{x}~\forall x > 0$.
    \begin{proof}
        Es reicht zu zeigen, dass die obigen Eigenschaften die Funktion $F$ eindeutig bestimmen, da wir bereits wissen, dass $\Gamma$ die Eigenschaften erfüllt.\\[.5\baselineskip]
        \textsc{Schritt 1}:
        \begin{align*}
            F\of{x+n} \annot[{&}]{=}{(b)} \pair{x+n-1}\cdot F\of{x+n-1}\\
            &= \pair{x+n-1}\cdot\pair{x-n-2}\cdot \ldots \cdot \pair{x-1}\cdot x\cdot F\of{x}\quad\forall x > 0
            \intertext{Für $n\in\N$ gilt}
            F\of{n+1} &= n!\cdot F\of{1} \annot{=}{(a)} n!\\
            \impl F\of{n} &= \Gamma\of{n}\quad\forall n\in\N
            \intertext{Das heißt es reicht zu zeigen, dass $F\of{x}$ bei $0< x < 1$ eindeutig bestimmt ist.\endgraf\vspace{.5\baselineskip}\noindent\textsc{Schritt 2}: Sei $0 < x < 1$}
            x + n &= \pair{1-x}\cdot n + x\cdot\pair{n+1}\\
            \impl F\of{x+n} &= F\of{\pair{1-x}\cdot n + x\cdot\pair{n+1}}
            \intertext{Nach (c) können wir (\ref{eq:logarithm-konvex}) mit $\Theta \coloneqq 1-x$ anwenden}
            &\leq F\of{n}^{1-x} \cdot F\of{n-1}^{x} = F\of{n}^{1-x} \cdot \pair{n\cdot F\of{n}}^x\\
            &= F\of{n}\cdot n^x = \pair{n-1}! \cdot n^x\quad\fa n\in\N\fa 0 < x < 1\tag{1}\\[10pt]
            n+1 &= x\cdot\pair{n+x} + \pair{1-x}\cdot\pair{n+1+x}\\
            \impl F\of{n+x} &\leq F\of{n+x}^x\cdot F\of{n+1+x}^{1-x}\\
            &= F\of{n+x}\cdot \pair{n+x}^{1-x}\tag{2}
            \intertext{Durch die Kombination von (1) und (2) folgt}
            \impl n!\cdot \pair{n+x}^{x-1} &\leq F\of{n+x}\leq \pair{n-1}!\cdot n^x
            \intertext{Es gilt $F\of{n+x} = x\cdot\pair{x+1}\cdot\ldots\cdot \pair{x+n-1}\cdot F\of{x}$}
            \impl \frac{n!\cdot\pair{n+x}^{x-1}}{x\cdot \pair{x+1}\cdot\ldots\cdot\pair{x+n-1}} &\leq F\of{x} \leq \frac{\pair{n-1}!\cdot n^x}{x\cdot\pair{x+1}\cdot\ldots\cdot \pair{x+n-1}}\\
            \intertext{Wir definieren}
            a_n\of{x} &\coloneqq \frac{n!\cdot\pair{x+n}^{x-1}}{x\cdot \pair{x+1}\cdot\ldots\cdot\pair{x+n-1}}\\
            b_n\of{x} &\coloneqq \frac{\pair{n-1}!\cdot n^x}{x\cdot\pair{x+1}\cdot\ldots\cdot \pair{x-n-1}}\\[5pt]
            \impl a_n\of{x} &\leq F\of{x} \leq b_n\of{x}\quad\forall n\in\N\fa 0 < x < 1\\
            \impl \frac{a_n\of{x}}{b_n\of{x}} &\leq \frac{F\of{x}}{b_n\of{x}} \leq 1\\
            \frac{b_n\of{x}}{a_n\of{x}} &= \frac{n^x}{n\cdot\pair{n+x}^{x-1}} = \frac{\pair{n+x}\cdot n^x}{n\cdot\pair{n+x}^x}\\
            &= \frac{n+x}{n}\cdot\pair{\frac{n}{n+x}}^x\underset{\ntoinf}{\fromto}  1\\[10pt]
            \impl F\of{x} &= \lim_{\ntoinf} b_n\of{x}\\
            &= \lim_{\ntoinf} \frac{\pair{n-1}! \cdot n^x}{x\cdot\pair{x+1}\cdot \ldots\cdot \pair{x+n-1}}\numbereq{eq:gamma-alt}
        \end{align*}
        also ist $F\of{x}$ eindeutig bestimmt.
    \end{proof}
\end{satz}

\begin{korollar}[Gaußsche Darstellung von $\Gamma$] % Korollar 4
    \begin{align*}
        \Gamma\of{x} &= \lim_{\ntoinf} \frac{n!\cdot n^x}{x\cdot\pair{x+1}\cdot\ldots\cdot \pair{x+n}}\numbereq{eq:gamma-gauss}
    \end{align*}
    \begin{proof}
        Da $\frac{n}{n+1}\fromto 1$ für $\ntoinf$ folgt die Behauptung for $0 < x < 1$ direkt aus (\ref{eq:gamma-alt}). Für $x=1$ rechnet sich die Formel leicht nach.
        Also ist noch zu zeigen: Gilt (\ref{eq:gamma-gauss}) für ein $x$, so gilt die Aussage auch für $y\coloneqq x+1$.
        \begin{align*}
            \Gamma\of{y} &=\Gamma\of{x+1} = x\cdot\Gamma\of{x}\\
            &= x \cdot \lim_{\ntoinf} \frac{n!\cdot n^x}{x\cdot\pair{x+1}\cdot\ldots\cdot\pair{x+n}}\\
            &= \lim_{\ntoinf} \frac{n!\cdot n^{y-1}}{y\cdot\pair{y+1}\cdot\ldots\cdot \pair{y+n-1}}\\
            \intertext{Multiplikation im Zähler mit $n$ und im Nenner mit $y+n$ (was sich für $\ntoinf$ entspricht) liefert}
            &= \lim_{\ntoinf} \frac{n!\cdot n^{y}}{y\cdot\pair{y+1}\cdot \ldots\cdot \pair{y+n-1}\cdot\pair{y+n}}\qedhere
        \end{align*}
    \end{proof}
\end{korollar}

\begin{satz} % Satz 5
    \begin{align*}
        \Gamma\of{\frac{1}{2}} &= \sqrt {\pi}
    \end{align*}

    \begin{proof}
        \begin{align*}
            \Gamma\of{\frac{1}{2}} &= \lim_{\ntoinf} \frac{n!\cdot n^{\frac{1}{2}}}{\frac{1}{2}\cdot \pair{1+\frac{1}{2}}\cdot\ldots \cdot \pair{n+\frac{1}{2}}}\\
            &= \lim_{\ntoinf} \frac{n!\cdot n^{\frac{1}{2}}}{\pair{1-\frac{1}{2}}\cdot\pair{2-\frac{1}{2}}\cdot\ldots\cdot \pair{n+1-\frac{1}{2}}}\\
            \impl \Gamma\of{\frac{1}{2}}^{2} &= \lim_{\ntoinf} \frac{2\cdot \pair{n!}^2}{\pair{1+\frac{1}{2}}\cdot \pair{1-\frac{1}{2}}\cdot\pair{2+\frac{1}{2}}\cdot\pair{2-\frac{1}{2}}\cdot\ldots\cdot \pair{n+\frac{1}{2}}\cdot \pair{n-\frac{1}{2}}}\\
            &= \lim_{\ntoinf} \frac{2\cdot \pair{n!}^2}{\pair{1-\frac{1}{4}}\cdot\pair{4-\frac{1}{4}}\cdot\ldots\cdot\pair{n^2-\frac{1}{4}}}\\
            &= 2\lim_{\ntoinf} \prod_{k=1}^{n} \frac{k^2}{k^2-\frac{1}{4}}
            \intertext{Wir haben ein Wallisches Produkt, was nach Satz~\ref{satz:wallisches-produkt} gegen $\frac{\pi}{2}$ konvergiert}
            \impl \Gamma\of{\frac{1}{2}}^{2} &= \pi\qedhere
        \end{align*}
    \end{proof}
\end{satz}

\newpage


    \section{[*] Metrische Räume, topologische Räume und normierte Vektorräume}
    \section{[*] Metrische Räume, topologische Räume und normierte Vektorräume}

\subsection{Metrische Räume}
\thispagestyle{pagenumberonly}

\marginnote{[07. Jun]}
Frage: Wie definiert man Abstand auf allgemeinen Mengen?

\begin{beispiel}[Abstände zweier reeller Zahlen]
    Seien $x,y\in\R$. Wir definieren den Abstand über den Betrag. Das heißt $d\of{x,y}\coloneqq \abs{x-y}$.
    Der Abstand hat in diesem Fall die Eigenschaften
    \begin{enumerate}[label=(\roman*)]
        \item $\abs{x-y} \geq 0$ und $\abs{x-y} = 0 \equivalent x=y$
        \item $\abs{x-y} = \abs{y-x}$
        \item $\abs{x-y} \leq \abs{x-z} + \abs{z-y}$ für ein beliebiges $z\in\R$
    \end{enumerate}
\end{beispiel}

\begin{definition}[Metrik]
    Sei $X$ eine Menge und $d: X \times X\fromto \R$ eine Abbildung mit den Eigenschaften
    \begin{enumerate}[label=(\roman*)]
        \item $d\of{x,y} \geq 0~\forall x,y\in X$ und $d\of{x,y} = 0\equivalent x = y$
        \item $d\of{x,y} = d\of{y,x}~\forall x,y\in X$
        \item $d\of{x,y} \leq d\of{x,z} + d\of{z,y}~\forall x,y,z\in X$
    \end{enumerate}
    In diesem Fall nennen wir $d$ eine Metrik auf $X$ und das Paar $\pair{X, d}$ einen metrischen Raum.
\end{definition}

\begin{beispiel}
    \theoremescape
    \begin{enumerate}
        \item Auf $\R$ oder $\C$ definieren wir $d\of{x,y} \coloneqq \abs{x-y}$
        \item Für $x,y\in\R^2$ definieren wir zum Beispiel die euklidische Metrik $d\of{x,y} \coloneqq \sqrt{\pair{x_1 - y_1}^2 + \pair{x_2 - y_2}^2}$
        \item Sei $\pair{X, d}$ ein metrischer Raum und sei $A\subseteq X$. Dann definieren wir die \emph{induzierte Metrik} $d_A: A \times A \fromto \R,~\pair{x,y}\mapsto d\of{x,y}$
        \item Die diskrete Metrik ist definiert durch $d\of{x,y} \coloneqq \begin{cases}
                                                                               0 & x = y\\
                                                                               1 & x\neq y
        \end{cases}$
        \item Die französische Eisenbahnmetrik im $\R^2$ ist definiert durch $d\of{x,y} \coloneqq \sqrt{x_1^2 + x_2^2} + \sqrt{y_1^2 + y_2^2}$ falls $x$ und $y$ nicht auf einer Geraden durch $\pair{0,0}$ liegen und sonst $d\of{x,y} \coloneqq \sqrt{\pair{x_1 - y_1}^2 + \pair{x_2 - y_2}^2}$.
        \item Sei $\pair{X, d}$ ein metrischer Raum. Dann können wir eine Metrik $d_1$ definieren durch
        \begin{align*}
            d_1\of{x,y} &\coloneqq \frac{d\of{x,y}}{1+d\of{x,y}}
            \intertext{Dass $d_1$ die Eigenschaften (i) und (ii) im Sinne der Definition erfüllt, rechnet sich leicht nach. Für (iii) gilt für ein $z\in X$}
            d\of{x,y} &\leq d\of{x,z} + d\of{z,y}
            \intertext{Außerdem gilt für $t\in\R$}
            \frac{t}{1+t} &= 1- \frac{1}{t+1} \tag{monoton steigend}\\
            \impl d_1\of{x,y} = \frac{d\of{x,y}}{1+d\of{x,y}} &\leq \frac{d\of{x,z} + d\of{z,y}}{1+d\of{x,z} + d\of{z,y}}\\
            &= \frac{d\of{x,z}}{1+d\of{x,z} + d\of{z,y}} + \frac{d\of{z,y}}{1+d\of{x,z} + d\of{z,y}}\\
            &\leq \frac{d\of{x,z}}{1+d\of{x,z}} + \frac{d\of{z,y}}{1+d\of{z,y}} = d_1\of{x,z} + d_1\of{z,y}
        \end{align*}
    \end{enumerate}
\end{beispiel}

\subsection{Normierte Vektorräume}

\begin{definition}[Vektorraum]
    Ein Vektorraum $V$ über $\K\in\set{\R, \C}$ ist eine Menge auf der es eine Vektoraddition
    \begin{align*}
        +: V \times V&\fromto V\\
        \pair{x,y}&\mapsto x+y
        \intertext{und eine (skalare) Multiplikation}
        \cdot: \K \times V&\fromto V\\
        \pair{\alpha, x}&\mapsto \alpha \cdot x
    \end{align*}
    gibt, welche den üblichen Axiomen aus der Algebra\footnote{Zum Beispiel $x+y = y+x$, $\alpha\pair{x+y} = \alpha x + \alpha y$, etc. Für eine genaue Aufzählung siehe \textsc{Lineare Algebra I}} genügen.
\end{definition}

\begin{definition}[Norm]
    Eine Norm auf einem VR $V$ ist eine Abbildung $\norm{\cdot}: V \fromto \R,~x\mapsto\norm{x}$ mit den Eigenschaften
    \begin{enumerate}[label=(\roman*)]
        \item $\norm{x} \geq 0$ und $\norm{x} = 0 \equivalent x = 0~\forall x\in V$
        \item $\norm{\lambda x} = \abs{\lambda}\norm{x}~\forall \lambda\in\K, x\in V$
        \item $\norm{x+y} \leq \norm{x} + \norm{y}~\forall x,y\in V$
    \end{enumerate}
\end{definition}

\begin{definition}[Normierter Vektorraum]
    Ein normierter Vektorraum ist ein Paar $\pair{V, \norm{\cdot}}$ aus einem VR $V$ und einer Norm $\norm{\cdot}$ auf $V$.
\end{definition}

\begin{satz} % Satz 3
    Ist $\pair{V, \norm{\cdot}}$ ein normierter VR, so wird durch $d\of{x,y}\coloneqq \norm{x-y}$ eine Metrik auf $V$ definiert.
    \begin{proof}
        Die Axiome der Metrik folgen unmittelbar aus den Axiomen der Norm.
    \end{proof}
\end{satz}

\begin{bemerkung}[Halbnorm]
    Gelten für eine Abbildung die Norm-Eigenschaften (ii) und (ii), aber statt (i) nur $\norm{x} \geq 0~\forall x\in V$. Dann heißt $\norm{\cdot}$ eine Halbnorm. Ein Beispiel dafür im $\R^2$ wäre die Abbildung $\pair{x_1~x_2}\mapsto x_1$.
\end{bemerkung}

\begin{beispiel}
    \theoremescape
    \begin{enumerate}
        \item Sei $V$ ein euklidischer reeller VR mit symmetrischem, positiv-definitem Skalarprodukt $\sprod{\cdot,\cdot}$. Dann ist $\norm{x}\coloneqq \sqrt{\sprod{x,x}}$ eine Norm.
        \item Analog funktioniert der Fall, dass $V$ ein komplexer VR mit positiv-definiter sesquilinearer Bilinearformen $\sprod{\cdot,\cdot}$ ist. Das heißt $\sprod{\cdot, \cdot}: V\times V\fromto\C$ hat die Eigenschaften
        \begin{enumerate}[label=(\roman*)]
            \item $\sprod{x,y} = \conj{\sprod{y,x}}$ und $\sprod{x,x} \geq 0~\forall x,y\in V$ (\footnote{Die Forderung $\sprod{x,x} \geq 0$ ist wohldefiniert, weil durch die anderen Eigenschaften folgt, dass $\forall x\in V\colon \sprod{x,x}\in\R$})
            \item $\sprod{x, u+w} = \sprod{x,u} + \sprod{x,w}$ sowie $\sprod{u+w, y} = \sprod{u, y} + \sprod{w, y}$
            \item $\sprod{x, \alpha y} = \alpha\sprod{x,y}$ sowie $\sprod{\alpha x, y} = \conj{\alpha}\sprod{x,y}$ (\footnote{Physiker-Konvention, anders herum als in der Vorlesung \textsc{Lineare Algebra I/II}})
        \end{enumerate}
        Dann definieren wir eine Norm durch $\norm{x} \coloneqq \sqrt{\sprod{x,x}}$.
        \item Im $\R^d$ definieren wir für zwei Vektoren $x=\pair{x_1, \ldots, x_d}$ und $y=\pair{y_1, \ldots, y_d}$ ein Skalarprodukt durch
        \begin{align*}
            \sprod{x,y} &\coloneqq \sum_{j=1}^{d} x_j \cdot y_j
            \intertext{Und eine Norm durch}
            \norm{x} &\coloneqq \norm{x}_2 \coloneqq \sqrt{\sprod{x,x}} = \sqrt{\sum_{j=1}^{d} \abs{x_j}^2}
        \end{align*}
        \item Wir definieren die Maximums-Norm auf $\R^d$ durch $\norm{x}_{\infty} \coloneqq \max_{1\leq j\leq d} \abs{x_j}$. Daraus folgt auch $\norm{x}_{\infty} \leq \norm{x}_2 \leq \sqrt{d}\cdot\norm{x}_{\infty}$.
        \item Wir definieren die Manhattan-Norm auf $\R^d$ durch
        \begin{align*}
            \norm{x}_1 &\coloneqq \sum_{j=1}^{d} \abs{x_j}
        \end{align*}
        und es gilt $\norm{x}_{\infty} \leq \norm{x}_1 \leq d \cdot \norm{x}_{\infty}$.
        \item Sei $X$ eine beliebige Menge und $\mathcal{B}\of{X}$ der Vektorraum der reellwertigen beschränkten Funktionen $f: X\fromto \R$. Dann definieren wir
        \begin{align*}
            \norm{f}_{L^{\infty}\of{x}} &\coloneqq \sup\set{\abs{f\of{x}}: x\in X} = \sup_{x\in X} \abs{f\of{x}}
        \end{align*}
        als eine Norm auf $\mathcal{B}\of{X}$. Das ist eine Verallgemeinerung der Maximumsnorm.
    \end{enumerate}
\end{beispiel}

\subsection{[*] Umgebungen und offene Mengen}

\begin{definition}[Umgebung]
    Sei $\pair{X, d}$ ein metrischer Raum
    \begin{enumerate}[label=(\alph*)]
        \item Für $a\in X$, $r > 0$ heißt $B_r\of{a} \coloneqq \set{x\in X ~\middle\vert~ d\of{a, x} < r}$ die (offene) Kugel um $a$ mit dem Radius $r$ und dem Mittelpunkt $a$.
        \item Eine Teilmenge $U\subseteq X$ heißt \emph{Umgebung} von $x\in X$, falls $\exists\varepsilon > 0$, sodass $B_{\varepsilon}\of{x} \subseteq U$. Insbesondere ist $B_{\varepsilon}\of{x}$ eine Umgebung von $x$. Wir nennen $B_{\varepsilon}\of{x}$ die $\varepsilon$-Umgebung von $x$.
    \end{enumerate}
\end{definition}

\begin{satz}[Hausdorffsches Trennungsaxiom in metrischen Räumen]
    \label{satz:hausdorff-trennungsaxiom}
    Sei $\pair{X, d}$ ein metrischer Raum. Dann existieren zu $x,y\in X$ mit $x\neq y$ $\varepsilon$-Umgebungen mit $B_{\varepsilon}\of{x} \cap B_{\varepsilon}\of{y} = \emptyset$.
    \begin{proof}
        Wir wählen $\varepsilon = \frac{1}{2}\cdot d\of{x,y} > 0$. Angenommen $\exists z\in B_{\varepsilon}\of{x} \cap B_{\varepsilon}\of{y}$. Dann gilt
        \begin{align*}
            2\varepsilon &= d\of{x,y} \leq d\of{x,z} + d\of{z,y}\\
            &< \varepsilon + \varepsilon = 2\varepsilon\qedhere
        \end{align*}
    \end{proof}
\end{satz}

\begin{definition}[Offene Menge]
    \marginnote{[11. Jun]}
    Sei $\pair{X, d}$ ein metrischer Raum. Eine Teilmenge $U\subseteq X$ heißt offen, wenn $\forall x\in U\ex \varepsilon > 0\colon B_{\varepsilon}\of{x}\subseteq U$.
\end{definition}

\begin{beispiel}
    Seien $a, b\in\R$ und $a < b$
    \begin{enumerate}
        \item Dann sind $\pair{a, b}$, $\pair{a, \infty}$ und $\pair{-\infty, a}$ offen
        \item Die Mengen $\linterv{a,b}$, $\interv{a,b}$, $\linterv{a,\infty}$ sind nicht offen, weil für den ersten Fall $\pair{a-\varepsilon, a + \varepsilon}\nsubseteq \linterv{a,b}$.
        \item In jedem metrischen Raum $\pair{X, d}$ ist für beliebige $a\in X$, $r > 0$ die Menge $B_r\of{a}$ offen. Deshalb heißt $B_r\of{a}$ die offene Kugel um $a$ mit Radius $r$.
        \item In $\R^d$ gilt $U\subseteq\R^d$ ist offen bezüglich $\norm{\cdot}_{\infty} \equivalent U$ ist offen bezüglich $\norm{\cdot}_2$
    \end{enumerate}
    \begin{proof}[Beweis für 3.]
        Sei $x\in B_{r}\of{a}$ und $\varepsilon \coloneqq r - d\of{a, x} > 0$. Sei $z\in B_{\varepsilon}\of{x}$. Dann gilt
        \begin{align*}
            d\of{a,z} \leq d\of{a,x} + d\of{x,z} &< d\of{a, x} + r - d\of{a,x} = r\\
            \impl d\of{a,z} &< r\\
            \impl z\in &B_r\of{a}\qedhere
        \end{align*}
    \end{proof}
    \begin{proof}[Beweis für 4.]
        Sei $B_{\varepsilon}^{\infty}\of{x}\coloneqq\set{y\in \R^d: \norm{x-y}_{\infty} < \varepsilon}$ und $B_{\varepsilon}^{(2)}\of{x} \coloneqq \set{y\in\R^d: \norm{x-y}_2 < \varepsilon}$. Für ein $y\in B_{\frac{\varepsilon}{\sqrt{d}}}^{(2)}\of{x}$ gilt
        \begin{align*}
            \sqrt{ \sum_{i=1}^{d} \pair{y_i - x_i}^2} &< \frac{\varepsilon}{\sqrt{d}}\\
            \impl \sum_{i=1}^{d} \pair{y_i - x_i}^2 &< \frac{\varepsilon^2}{d}\\
            \impl \max_{1 \leq i \leq d} \abs{x_i - y_i} &< \varepsilon\\
            \impl y&\in B_{\varepsilon}^{\infty}\of{x}
            \intertext{Außerdem lässt sich zeigen, dass daraus auch folgt, dass $y\in B_{\varepsilon}^{(2)}\of{x}$. Das heißt}
            B_{\frac{\varepsilon}{\sqrt{d}}}^{(2)}\of{x} \subseteq B_{\varepsilon}^{\infty}\of{x} &\subseteq B_{\varepsilon}^{(2)}\of{x}\qedhere
        \end{align*}
    \end{proof}
\end{beispiel}

\begin{satz} % Satz 7
    \label{satz:offene-mengen-metr}
    Für die offenen Mengen eines metrischen Raums $\pair{X, d}$ gilt
    \begin{enumerate}[label=(\roman*)]
        \item $\emptyset$ und $X$ sind offen
        \item Sind $U$ und $V$ offen, so ist auch $U\cap V$ offen
        \item Ist $I$ eine beliebige Indexmenge und $(U_j)_{j\in I}$ eine Familie offener Teilmengen von $X$. So ist $\bigcup_{j\in I} U_j$ offen.
    \end{enumerate}

    \begin{proof}[Beweis von (ii)]
        Sei $x\in U\cap V$. Da $U$ und $V$ offen sind, gibt es $\varepsilon_1 > 0$ und $\varepsilon_2 > 0$ mit $B_{\varepsilon_1}\of{x}\subseteq U$, $B_{\varepsilon_2}\of{x}\subseteq V$. Sei $\varepsilon\coloneqq \min\set{\varepsilon_1, \varepsilon_2} \impl B_{\varepsilon}\of{x}\subseteq U\land B_{\varepsilon}\of{x}\subseteq V$. Damit gilt $B_{\varepsilon}\of{x}\subseteq U\cap V$.
    \end{proof}

    \begin{proof}[Beweis von (iii)]
        Sei $x\in \bigcup_{j\in I} U_j \impl \exists i\colon x\in U_{i}$. Außerdem ist $U_{i}$ offen. Das heißt es existiert ein $\varepsilon > 0$, sodass $B_{\varepsilon}\of{x}\subseteq U_{i} \impl B_{\varepsilon}\of{x}\subseteq \bigcup_{j\in I} U_{j}$.
    \end{proof}
\end{satz}

\begin{bemerkung}
    Seien $U_1, \ldots, U_k$ offen. Dann folgt aus Satz~\ref{satz:offene-mengen-metr}, dass $U_1\cap U_2\cap\dots\cap U_k$ offen ist. Das gilt allerdings nur für $k < \infty$.\\
    Wir betrachten für einen Schnitt über unendlich viele Mengen das folgende Beispiel:
\end{bemerkung}

\begin{beispiel}[Schnitt über unendlich viele offene Mengen]
    Sei $U_n = \pair{-\frac{1}{n}, 1+\frac{1}{n}}$. Dann ist $U_n$ offen $\forall n\in\N$. Allerdings ist $\bigcap_{n=1}^{\infty} U_n = \interv{0, 1}$ nicht offen.
\end{beispiel}

\begin{definition}[Abgeschlossene Menge]
    In einem metrischen Raum $\pair{X, d}$ nennen wir eine Menge $A\subseteq X$ abgeschlossen, wenn ihr Komplement $A^{\mathrm{C}}\coloneqq X \exclude A = \set{y\in X: y\not\in A}$ offen ist.
\end{definition}

\begin{bemerkung}
    Eine andere (aber äquivalente) Definition von Abgeschlossenheit wurde 1884 von Cantor gegeben. Diese Definition basiert auf Folgen.
\end{bemerkung}

\begin{definition}[Konvergenz in metrischen Räumen]
    Sei $\pair{X, d}$ ein metrischer Raum. Eine Folge $(x_n)_n\subseteq X$ konvergiert gegen den Punkt $a\in X$, wenn
    \begin{align*}
        &\forall\varepsilon>0\ex N\in\N\colon d\of{a, x_n} < \varepsilon\quad\forall n> N
        \intertext{das heißt}
        \equivalent &\forall\varepsilon > 0\ex N\in\N\colon x_n\in B_{\varepsilon}\of{a}\quad\forall n> N\\
        \equivalent &\forall\varepsilon > 0 \text{ ist } x_n\in B_{\varepsilon}\of{a} \text{ für fast alle $n$}
    \end{align*}
    Wir schreiben dann $\biglim{\ntoinf} x_n = a$.
\end{definition}

\begin{definition}[Folgenabgeschlossenheit]
    Sei $\pair{X, d}$ ein metrischer Raum. Eine Menge $A\subseteq X$ ist folgenabgeschlossen, falls für jede Folge $(x_n)_n \subseteq A$, die gegen einen Punkt $a\in X$ konvergiert, auch gilt $a\in A$.
\end{definition}

\begin{beispiel}
    Wir betrachten die Menge $A = \rinterv{0, 1}$ und die Folge $x_n = \frac{1}{n}$. Dann liegt $\lim_{\ntoinf} x_n$ nicht in $A$. Das heißt $A$ ist nicht folgenabgeschlossen. Die Menge $\interv{0, 1}$ hingegen schon.
\end{beispiel}

\begin{satz} % Satz 11
    \label{satz:vergleich-folgen-abgeschlossen}
    Sei $\pair{X, d}$ ein metrischer Raum. Für $A\subseteq X$ gilt, $A$ ist genau dann abgeschlossen, wenn $A$ folgenabgeschlossen ist.
\end{satz}

Um diesen Satz zu beweisen, benötigen wir zunächst noch folgende Lemmata

\begin{lemma} % Lemma 12
    \label{lemma:komplement-folgen-abgeschlossen}
    Sei $\pair{X, d}$ ein metrischer Raum. Ist $U\subseteq X$ offen, so ist $U^{\mathrm{C}}$ folgenabgeschlossen.

    \begin{proof}
        Angenommen es gibt eine Folge $(x_n)_n \subseteq U^{\mathrm{C}}$ mit $x_n\fromto a$, aber $a\not\in U^{\mathrm{C}}$. Das heißt $a\in U$. Da $U$ offen ist, existiert eine Kugel $B_{\varepsilon}\of{a} \subseteq U$ (für ein $\varepsilon > 0$). Da $x_n\fromto a$ für $\ntoinf$ gilt
        \begin{align*}
            \exists N\in\N\colon &d\of{a, x_n} < \varepsilon\quad\forall n> N \\
            \equivalent \exists N\in\N\colon &x_n \in B_{\varepsilon}\of{a}\quad\forall n > N\\
            \impl &x_n\not\in U^{\mathrm{C}}\quad\forall n > N\tag*{(Widerspruch)\qedhere}
        \end{align*}
    \end{proof}
\end{lemma}

\begin{lemma}
    \label{lemma:komplement-offen}
    Sei $\pair{X, d}$ ein metrischer Raum und $A\subseteq X$ folgenabgeschlossen. Dann ist $A^{\mathrm{C}}$ offen.

    \begin{proof}
        Angenommen $A^{\mathrm{C}}$ ist nicht offen. Das heißt $\exists x_0 \in A^{\mathrm{C}}$, sodass für jedes $\varepsilon > 0$ die Kugel $B_{\varepsilon}\of{x_0}$ nicht ganz in $A^{\mathrm{C}}$ enthalten ist. Das heißt
        \begin{align*}
            \forall\varepsilon > 0\colon B_{\varepsilon}\of{x_0}\cap A \neq \emptyset
        \end{align*}
        Wir wählen eine Folge $(x_n)_n\subseteq A$ mit $x_n\in B_{\frac{1}{n}}\of{x_0}$. Also gilt $x_n\fromto x_0\in A^{\mathrm{C}}$. Das heißt $A$ ist nicht folgenabgeschlossen. Widerspruch.
    \end{proof}
\end{lemma}

\begin{proof}[Beweis von Satz~\ref{satz:vergleich-folgen-abgeschlossen}]
    ~\\
    \anf{$\impl$} Wenn $A$ abgeschlossen ist, dann gilt nach Definition, dass $A^{\mathrm{C}}$ offen ist. Mit Lemma~\ref{lemma:komplement-folgen-abgeschlossen} folgt dann, dass $A$ folgenabgeschlossen ist.\\
    \anf{$\Leftarrow$} Sei $A$ folgenabgeschlossen. Dann folgt direkt aus Lemma~\ref{lemma:komplement-offen}, dass $A$ abgeschlossen.
\end{proof}

\newpage

\begin{beispiel}
    \marginnote{[14. Jun]}
    Seien $A_1\subseteq\R^k$, $A_2\subseteq\R^m$ abgeschlossen. Dann ist auch $A_1\times A_2 \subseteq \R^k \times \R^m$ abgeschlossen.
    \begin{proof}
        Sei $\pair{x,y}\in \pair{A_1 \times A_2}^{\mathrm{C}} \supseteq A_1^C \times \R^m \cup \R^k \times A_2^C$.\\[.5\baselineskip]
        \textsc{Fall 1}: $x \in A_1^C$. Dann gilt wegen der Abgeschlossenheit von $A_1$
        \begin{align*}
            \exists \varepsilon > 0\colon B_{\varepsilon}\of{x} &\subseteq A_1^C\\
            \impl B_{\varepsilon}\of{x, y} &\subseteq A_1^{\mathrm{C}} \times \R^m
            \intertext{Aber $B_{\varepsilon}\of{x,y}$ ist eine offene Menge mit}
            B_{\varepsilon}\of{x,y} \subseteq A^{\mathrm{C}}_1 \times \R^m &\subseteq \pair{A_1 \times A_2}^{\mathrm{C}}
            \intertext{\textsc{Fall 2}: $y\in A_2^{\mathrm{C}}$. Dann gilt analog zu \textsc{Fall 1}}
            \impl \exists \varepsilon > 0\colon B_{\varepsilon}\of{x,y} &\subseteq \pair{A_1 \times A_2}^{\mathrm{C}}
        \end{align*}
        Das heißt $\pair{A_1 \times A_2}^{\mathrm{C}}$ ist offen und damit ist $\pair{A_1 \times A_2}$ abgeschlossen.
    \end{proof}
\end{beispiel}

\begin{bemerkung}
    \theoremescape
    \begin{enumerate}[label=(\roman*)]
        \item In jedem metrischen Raum $\pair{X, d}$ sind die Mengen $\emptyset$ und $X$ sowohl offen als auch abgeschlossen
        \item Das Intervall $\linterv{a,b}\subseteq\R$ ist nicht offen und nicht abgeschlossen
    \end{enumerate}
\end{bemerkung}

\subsection{Grundzüge der Topologie}

\begin{definition}[Topologie]
    Sei $X$ eine Menge. Dann heißt ein Mengensystem $\Tau\subseteq\mathcal{P}\of{X}$ eine Topologie auf $X$, falls
    \begin{enumerate}[label=(\roman*)]
        \item $\emptyset, X\in\Tau$
        \item $U, V\in \Tau \impl U \cap V \in \Tau$ und
        \item Für eine beliebige Indexmenge $I$ mit $\forall j\in I\colon V_j \in \Tau$ folgt $\displaystyle\bigcup_{j\in I} V_j \in \Tau$
    \end{enumerate}
    Ein topologischer Raum ist ein Paar $\pair{X, \Tau}$. $V\subseteq X$ heißt offen, falls $V\in\Tau$ und $A\subseteq X$ heißt abgeschlossen, falls $A^{\mathrm{C}} \in\Tau$.
\end{definition}

\begin{beispiel}
    \theoremescape
    \begin{enumerate}
        \item Das System von offenen Mengen eines metrischen Raums $X$ ist eine Topologie
        \item $\R^d$ ist ein topologischer Raum (mit dem System der offenen Mengen als implizierte Topologie)
        \item Auf jeder Menge $X$ gibt es mindestens 2 Topologien: $\Tau_0 \coloneqq \set{\emptyset, X}$ und die feinste Topologie $\Tau_1 = \mathcal{P}\of{X}$. Ist $\abs{X} \geq 2$, so ist $\Tau_0 \neq \Tau_1$
        \item Sei $\pair{X, \Tau}$ ein topologischer Raum und $Y\subseteq X$ eine Teilmenge von $X$. Wir definieren ähnlich zur induzierten Metrik eine \emph{induzierte Topologie} (Relativ-Topologie)
        \begin{align*}
            \Tau_{Y} \coloneqq \Tau \cap Y \coloneqq \set{U \cap Y \colon U\in \Tau}
        \end{align*}
        Dann ist $\pair{Y, \Tau_Y}$ ein topologischer Raum.
    \end{enumerate}
\end{beispiel}

\begin{bemerkung}
    Ist $Y$ nicht offen in $X$, so ist $V\in \Tau_Y$ nicht notwendigerweise offen in $X$.
\end{bemerkung}

\begin{definition}[Umgebung]
    Sei $\pair{X, \Tau}$ ein topologischer Raum und $x\in X$ (Punkt in $X$). Eine Teilmenge $V\subseteq X$ heißt Umgebung von $x$, falls es eine Menge $U\in\Tau$ gibt mit $x\in U \subseteq V$.
\end{definition}

\begin{satz} % Satz 16
    Sei $\pair{X, \Tau}$ ein topologischer Raum. Eine Menge $V\subseteq X$ ist genau dann offen, wenn $V$ eine Umgebung für jeden Punkt $x\in V$ ist.
    \begin{proof}
        \anf{$\impl$} Sei $V$ offen. Dann existiert für jedes $x\in V$ eine offene Menge $U\in\Tau$ mit $x\in U\subseteq V$. Also ist $V$ eine Umgebung jedes Punktes $x\in V$.\\[.5\baselineskip]
        \anf{$\Leftarrow$} Sei $V$ eine Umgebung für alle Punkte $x\in V$. Dann gilt $\forall x\in V\ex U_x \in \Tau\colon x\in U_x \subseteq V$. Dann ist $V=\bigcup_{x\in V} U_x$. Das heißt $V$ ist als Vereinigung offener Mengen offen.
    \end{proof}
\end{satz}

\begin{definition}[Hausdorff-Raum]
    Ein topologischer Raum heißt Hausdorff-Raum, falls das Hausdorffsche Trennungsaxiom gilt. Das heißt zu zwei Punkten $x,y\in X$ mit $x\neq y$ existieren offene Mengen $U,V\in\Tau$ mit $x\in U$, $y\in V$ und $U\cap V = \emptyset$.
\end{definition}

\begin{beispiel}
    \theoremescape
    \begin{enumerate}
        \item Nach Satz~\ref{satz:hausdorff-trennungsaxiom} ist jeder metrische Raum ein Hausdorff-Raum.
        \item Sei $X=\set{0, 1}$ und $\Tau\coloneqq \set{\emptyset, \set{0}, \set{0, 1}}$ eine Topologie. Dann ist $\pair{X, \Tau}$ kein Hausdorff-Raum. Um das einzusehen, betrachten wir $x=0$ und $y=1$. Für diese zwei Elemente finden wir keine entsprechenden Mengen.
    \end{enumerate}
\end{beispiel}

\subsection{[*] Berührpunkt, Häufungspunkt und Randpunkt}

\begin{mdframed}
    \begin{center}
        Für die folgenden Definitionen sei $\pair{X, \Tau}$ ein topologischer Raum und $A\subseteq X$.
    \end{center}
\end{mdframed}

\begin{definition}[Berührpunkt]
    Ein Punkt $x\in X$ heißt Berührpunkt von $A$, wenn in jeder offenen Umgebung $U$ von $x$ mindestens ein Punkt aus $A$ liegt. Das heißt $U\cap A \neq \emptyset$ für alle offenen Mengen $U$ mit $x\in U$.
\end{definition}

\begin{definition}[Häufungspunkt]
    Ein Punkt $x\in X$ heißt Häufungspunkt von $A$, wenn für jede offene Umgebung $U$ von $x$ ein von $x$ verschiedener Punkt in $A$ liegt. Das heißt $A\cap \pair{U\exclude\set{x}} \neq \emptyset$ für alle offenen Mengen $U$ mit $x\in U$.
\end{definition}

\begin{definition}[Randpunkt]
    Ein Punkt $x\in X$ heißt Randpunkt von $A$, falls es in jeder offenen Umgebung $U$ von $x$ mindestens einen Punkt aus $A$ und einen Punkt aus $A^{\mathrm{C}}$ gibt. Das heißt für alle offenen Mengen $U$ mit $x\in U$ ist $U\cap A \neq \emptyset$ und $U\cap A^{\mathrm{C}} \neq \emptyset$. Wir schreiben $\partial A$ für die Menge aller Randpunkte von $A$.
\end{definition}

\begin{satz} % Satz 19
    \marginnote{[18. Jun]}
    \label{satz:top-rand}
    Ist $\pair{X, \Tau}$ ein topologischer Raum und $A\subseteq X$. Dann gilt
    \begin{enumerate}[label=(\roman*)]
        \item $A \exclude \partial A$ ist offen
        \item $A \cup \partial A$ ist abgeschlossen
        \item $\partial A$ ist abgeschlossen
    \end{enumerate}
    \begin{proof}
        \theoremescape
        \begin{enumerate}[label=(\roman*)]
            \item Sei $x\in A \exclude \partial A$ beliebig. Dann folgt es existiert eine offene Umgebung $V$ von $x$ mit
            \begin{align*}
                V \cap A^{\mathrm{C}} &= V\cap \pair{X\exclude A} = \emptyset\tag{(1)}
                \intertext{denn ansonsten wäre $x\in\partial A$}
                \impl V \cap \partial A &= \emptyset\tag{(2)}
                \intertext{Denn wäre $V\cap \partial A \neq \emptyset$. Dann würde folgen}
                \impl \exists y&\in V \cap \partial A\\
                \impl V \text{ offene Umgebung von  } y&\in \partial A\\
                \impl V\cap A^{\mathrm{C}} &\neq \emptyset\tag{Widerspruch zu (1)}
                \intertext{Mit (1) und (2) folgt}
                V \cap\pair{\partial A \cup A^{\mathrm{C}}} &= \emptyset\\
                \equivalent V \subseteq A \exclude \partial A\\
                \impl A\exclude \partial A \text{ ist offen }
            \end{align*}
            \item $A^{\mathrm{C}} = X \exclude A$. Aus der Definition des Randes folgt $\partial A = \partial A^{\mathrm{C}}$. Nach (i) gilt dann
            \begin{align*}
                A^{\mathrm{C}} \exclude \partial A = A^{\mathrm{C}} \exclude \partial A^{\mathrm{C}} \text{ ist offen }\\
                \equivalent \pair{A^{\mathrm{C}} \exclude \partial A}^{\mathrm{C}} \text{ ist abgeschlossen }\\
                \equivalent \pair{A^{\mathrm{C}} \exclude \partial A}^{\mathrm{C}} &= X\exclude\pair{A^{\mathrm{C}} \exclude \partial A}\\
                &= X\exclude A^{\mathrm{C}} \cup \partial A = A\cup \partial A \text{ ist abgeschlossen }
            \end{align*}
            \item
            \begin{align*}
                \partial A &= \pair{A\cup \partial A} \cap \pair{A^C \cup \partial A}\\
                &= \pair{A \cup \partial A} \cap \pair{A^C \cup \partial A^{\mathrm{C}}} \text{ ist abgeschlossen}
            \end{align*}
        \end{enumerate}
    \end{proof}
\end{satz}

\begin{definition}[Inneres und Abschluss, abgeschlossene Hülle]
    Sei $\pair{X, \Tau}$ ein topologischer Raum und $A\subseteq X$.
    \begin{enumerate}[label=(\roman*)]
        \item $U\of{A} \coloneqq \set{U\in\Tau: U\subseteq A}= $ alle offenen Teilmengen von $A$.\\
        Das Innere von $A = A^{\circ} \coloneqq \bigcup_{\sigma\in U\of{A}} \sigma= $Vereinigung aller offenen Teilmengen von $A$
        \item $B\of{A} \coloneqq \set{B\subseteq X \text{ abgeschlossen}: A \subseteq B} \neq \emptyset$ (da $X\in B\of{A}$)\\
        Der Abschluss von $B$ sei $\overline{B} =$ abgeschlossene Hülle von $B \coloneqq \bigcap_{B\in B\of{A}} B$ abgeschlossen
    \end{enumerate}
    Damit gilt $A \subseteq \overline{A}$.
\end{definition}

\begin{bemerkung}
    $A^{\circ} =$ größte offene Teilmenge von $A$ und $\overline{A} = $ kleinste abgeschlossene Menge, die $A$ enthält.
\end{bemerkung}

\begin{satz} % Satz 21
    $A^{\circ} = A \exclude \partial A$ und $\overline{A} = A \cup \partial A$.
    \begin{proof}
        \textsc{Teil 1}: Nach Satz~\ref{satz:top-rand} ist $A\exclude \partial A$ offen und $A \exclude \partial A \subseteq A$. Damit folgt $A \exclude \partial A \subseteq A^{\circ}$.\\
        Damit ist noch zu zeigen, dass $A^{\circ} \subseteq A \exclude \partial A$. Ist $U\subseteq A$ offen
        \begin{align*}
            \impl U\cap A^{\mathrm{C}} &= \emptyset \\
            \impl U \cap \partial A &= \emptyset
        \end{align*}
        Falls nicht $\exists y \in U \cap \partial A$, dann folgt $U\cap A^{\mathrm{C}} \neq \emptyset$. Das ergibt einen Widerspruch.\\
        Das heißt für jede offene Teilmenge $U\subseteq A$ gilt $U\subseteq A\exclude \partial A$.
        \begin{align*}
            A = \bigcup_{U\in U\of{A}} U \subseteq A \exclude \partial A\\
            \impl A^{\circ} \subseteq A \exclude \partial A\\
            \impl A \exclude \partial A = A^{\circ}
        \end{align*}
        \textsc{Teil 2}: Behauptung: Aus $B\subseteq X$ ist abgeschlossen und $A \subseteq B$ folgt $\partial A \subseteq B$. Angenommen die Behauptung ist falsch. Dann würde gelten $B^{\mathrm{C}} \cap \partial A \neq \emptyset$.\\
        $x\in \partial A \cap B^{\mathrm{C}}$. $B^{\mathrm{C}}$ ist offene Umgebung von $x$. Nach Definition von $\partial A$ ist $A\cap B^{\mathrm{C}}\neq \emptyset$. Das heißt $A \subseteq B$. (Widerspruch).\\
        Aus der Behauptung folgt jetzt also $B \supseteq A \cup \partial A~\forall B\supseteq A$ und $B$ ist abgeschlossen
        \begin{align*}
            \impl A \cup \partial A \subseteq \bigcap_{B\in B\of{A}} B = \overline{A}
        \end{align*}
        Andererseits ist nach Satz~\ref{satz:top-rand} $A \cup \partial A$ abgeschlossen und sicherlich $A \subseteq A \cup \partial A$. Daraus folgt $\overline{A} \subseteq A \cup\partial A$. Damit folgt $\overline{A} = A \cup \partial A$.
    \end{proof}
\end{satz}

\newpage


    \section{[*] Konvergenz und Stetigkeit in metrischen Räumen}
    \section{[*] Konvergenz und Stetigkeit in metrischen Räumen}
\imaginarysubsection{[*]Konvergenz und Stetigkeit in metrischen Räumen}
\thispagestyle{pagenumberonly}

\begin{definition}[Konvergenz in metrischen Räumen]
    Sei $\pair{M, d}$ ein metrischer Raum. Eine Folge $(x_n)_n \subseteq M$ konvergiert gegen $a \in M$, falls
    \begin{align*}
        &\forall\varepsilon > 0\ex N\in\N\colon d\of{x_n, a} < \varepsilon\quad\forall n\geq N\\
        (\equivalent &\forall \varepsilon > 0\colon x_n \in B_{\varepsilon}\of{a} \text{ für fast alle } n)
    \end{align*}
    Wir schreiben $a = \lim_{\ntoinf} x_n$ oder $x_n\fromto a$ für $\ntoinf$.\\
    $(x_n)_n$ heißt \emph{Cauchy-Folge}, falls
    \begin{align*}
        &\forall\varepsilon>0\ex N\in\N\colon d\of{x_n, x_m} < \varepsilon\quad\forall n,m\geq N
    \end{align*}
\end{definition}

\begin{definition}[Durchmesser]
    Sei $\pair{M, d}$ ein metrischer Raum. Wir definieren den Durchmesser von einer Menge $A$ mit
    \begin{align*}
        \diam\of{A} &\coloneqq \sup \set{d\of{x,y}: x,y\in A}
    \end{align*}
    $A$ ist beschränkt, falls $\diam\of{A} < \infty$.
\end{definition}

\begin{bemerkung}
    Eine Menge $A\subseteq M$ ist genau dann beschränkt, wenn
    \begin{align*}
        \exists x\in M, r> 0\colon A \subseteq B_r\of{x}
    \end{align*}
    In diesem Fall ist $\diam\of{A} \leq 2r$.
\end{bemerkung}

\begin{satz}
    \theoremescape
    \begin{enumerate}[label=(\roman*)]
        \item Konvergiert die Folge $(x_n)_n$, so ist sie eine Cauchy-Folge
        \item Jede Cauchy-Folge ist beschränkt.
    \end{enumerate}
    \begin{proof}
        ~\\
        \begin{enumerate}[label=(\roman*)]
            \item Sei $a = \lim_{\ntoinf} x_n$ und $\varepsilon > 0$
            \begin{align*}
                \impl \exists N\in\N\colon d\of{x_n , a} &< \frac{\varepsilon}{2}\\
                \impl d\of{x_n, x_m} &\leq d\of{x_n, a} + d\of{a, x_m} < \varepsilon\quad\forall n,m\geq N\\
                \impl (x_n)_n &\text{ ist eine Cauchy-Folge }
            \end{align*}
            \item Sei $(x_n)_n$ eine Cauchy-Folge. Wir wählen $\varepsilon = 1$
            \begin{align*}
                \impl \exists N\in\N\colon d\of{x_n, x_m} &< 1\quad\forall n,m\geq N
                \intertext{Wir definieren $a\coloneqq x_N$}
                \impl d\of{x_n, a} &< 1\quad\forall n\geq N
                \intertext{$r\coloneqq \max\of{d\of{x_1, a}, d\of{x_2, a}, \ldots, d\of{x_{N-1}, a}} + 1$}
                \impl \forall n\in\N\colon d\of{x_n, a} &< r\\
                \impl x_n&\in B_r\of{a}\quad\forall n\in \N
                \intertext{Das heißt}
                \bigcup_{n\in\N} \set{x_n} &\subseteq B_r\of{A} \text{ ist eine beschränkte Menge }
            \end{align*}
        \end{enumerate}
    \end{proof}
\end{satz}

\begin{definition}[Banachräume]
    Ein metrischer Raum $\pair{M, d}$ heißt vollständig, falls jede Cauchy-Folge in $M$ konvergiert. Ein vollständiger, normierter Vektorraum heißt Banachraum.
\end{definition}

\begin{beispiel}
    $\pair{\R^d, \norm{\cdot}_2}$, $\pair{\R^d, \norm{\cdot}_{\infty}}$ oder $\pair{\R^d, \norm{\cdot}_1}$ sind vollständige, normierte Vektorräume. Genauso $\pair{\C^d, \norm{\cdot}_2}$ etc.
\end{beispiel}

\begin{satz}[Schachtelungsprinzip] % Satz 4
    Sei $\pair{M, d}$ ein vollständiger metrischer Raum und $(A_n)_n$ eine absteigende Folge abgeschlossener Mengen ($A_0 \supseteq A_1 \supseteq A_2 \supseteq \dots$) mit $\diam\of{A_n}\fromto 0$ für $\ntoinf$ und $A_n\neq \emptyset~\forall n\in\N$. Dann existiert genau ein $x\in M$ sodass $x\in \bigcap_{n\in\N} A_n$. Das heißt $x\in A_n~\forall n\in\N$.
    \begin{proof}
        \textsc{Eindeutigkeit}: Angenommen $x, y\in A_n~\forall n\in \N$. Dann gilt $d\of{x,y} \leq \diam\of{A_n}\fromto 0$ für $\ntoinf$. Das heißt $d\of{x,y} = 0\equivalent x = y$.\\[.2\baselineskip]
        \textsc{Existenz}: $\forall n\in\N\ex x_n\in A_n$. Behauptung: $(x_n)_n$ ist eine Cauchy-Folge. Sei $n\geq m$
        \begin{align*}
            x_n\in A_n \subseteq A_{n-1} &\subseteq A_{n-2} \subseteq \dots \subseteq A_m
            \intertext{Das heißt}
            x_n &\in A_m\quad\forall n\geq m\\
            \impl x_n, x_m &\in A_m\quad\forall n\geq m\\
            d\of{x_n, x_m} &\leq \diam\of{A_m}\fromto 0 \text{ für } m\toinf\\
            \intertext{Das heißt $(x_n)_n$ ist eine Cauchy-Folge. Nach der Vollständigkeit von $M$ existiert ein $x\coloneqq \lim_{\ntoinf} x_n$. Behauptung: $x\in A_m~\forall m\in\N$}
            x_n &\in A_m\quad\forall n\geq m\\
            \impl \lim_{\ntoinf} x_n &\in A_m \text{ da $A_m$ abgeschlossen ist }\\
            \impl x&\in A_m\quad\forall m\in \N\qedhere
        \end{align*}
    \end{proof}
\end{satz}

\begin{definition}[Stetigkeit in metrischen Räumen]
    \marginnote{[21. Jun]}
    \label{definition:stetigkeit-metr}
    Seien $\pair{M, d_M}$, $\pair{N, d_N}$ metrische Räume.
    \begin{enumerate}[label=(\alph*)]
        \item $\varepsilon$-$\delta$-Definition von Stetigkeit: Eine Funktion $f: M\fromto N$ ist stetig im Punkt $a\in M$ falls
        \begin{align*}
            \forall\varepsilon > 0\ex\delta > 0\colon d_N\of{f\of{x}, f\of{a}} < \varepsilon\quad\forall x\in M \text{ mit } d_M\of{x,a} < \delta
        \end{align*}
        \item Folgenstetigkeit: Eine Funktion $f: M\fromto N$ ist folgenstetig in $a\in M$, falls für jede Folge $(x_n)_n \subseteq M$ mit $\biglim{\ntoinf} x_n = a$ auch $\biglim{\ntoinf} f\of{x_n} = f\of{a}$ folgt.
    \end{enumerate}
\end{definition}

\begin{bemerkung}
    Definition~\ref{definition:stetigkeit-metr} (a) ist dabei äquivalent zu
    \begin{align*}
        \forall \varepsilon > 0\ex \delta > 0\colon f\of{B_{\delta}^M\of{a}} \subseteq B_{\varepsilon}^{N}\of{f\of{a}}
    \end{align*}
\end{bemerkung}

\begin{satz} % Satz 6
    \label{satz:stetigkeit-def-equiv}
    Für eine Funktion $f: M\fromto N$ und $a\in M$ sind äquivalent
    \begin{enumerate}[label=(\roman*)]
        \item $f$ ist $\varepsilon$-$\delta$-stetig in $a$
        \item $f$ ist folgenstetig in $a$
        \item Für jede Umgebung $U$ von $f\of{a}$ ist $V=f^{-1}\of{U}$ eine Umgebung von $a$
    \end{enumerate}
    Dabei ist (iii) die topologische Definition von Stetigkeit.
    \begin{proof}
    (i)
        $\impl$ (ii): Sei $(x_n)_n \subseteq M$ mit $x_n\fromto a$. Für $\varepsilon > 0$ wähle $\delta > 0$, sodass
        \begin{align*}
            d_N\of{f\of{x}, f\of{a}} < \varepsilon\quad\forall x&\in M \text{ mit } d_M\of{x,a} < \delta
            \intertext{Da $a=\biglim{\ntoinf} x_n$ existiert ein $N\in\N$, sodass $\forall n > N\colon d_M\of{x_n, a} < \delta$}
            \impl \forall n > N\colon d_N\of{f\of{x_n}, f\of{a}} &< \varepsilon\\
            \impl f\of{x_n} \fromto f\of{a}&
        \end{align*}
        (ii) $\impl$ (i): Kontraposition. Angenommen (i) gilt nicht. Dann $\exists\varepsilon_0$, sodass
        \begin{align*}
            \forall\delta > 0\ex &x\in M\colon d_M\of{x, a} < \delta \text{ und } d_N\of{f\of{x}, f\of{a}} > \varepsilon_0
            \intertext{Wir wählen $\delta_n = \frac{1}{n}$. Dann gilt $x_n\fromto a$, da $d\of{x_n, a} < \frac{1}{n}$. Aber}
            \exists x_1\colon d_N&\of{f\of{x_1}, f\of{a}} > \varepsilon_0\\
            \exists x_2\colon d_N&\of{f\of{x_2}, f\of{a}} > \varepsilon_0\\
            &\vdots\\
            \exists x_n\colon d_N&\of{f\of{x_n}, f\of{a}} > \varepsilon_0\\
            \impl d_N&\of{f\of{x_n}, f\of{a}}\quad\forall n\in\N > \varepsilon_0\\
            \impl f&\of{x_n} \not\fromto f\of{a}
        \end{align*}
        (i) $\impl$ (iii): Sei $U\subseteq N$ eine Umgebung von $f\of{a}$. Dann gilt
        \begin{align*}
            \exists\varepsilon > 0\colon B_{\varepsilon}^{N}\of{f\of{a}} &\subseteq N\\
            \annot{\impl}{(i)} \exists \delta > 0\colon f\of{B_{\delta}\of{a}} &\subseteq B_{\varepsilon}\of{f\of{a}}\\
            \impl \exists \delta > 0\colon B_{\delta}\of{a} &\subseteq f^{-1}\of{U}\\
            \impl f^{-1}\of{U} \coloneqq V &\text{ ist eine Umgebung }
        \end{align*}
        (iii) $\impl$ (i): Sei $\varepsilon > 0$. Dann ist $B_{\varepsilon}^{N}\of{f\of{a}}$ eine Umgebung von $f\of{a}$. Nach (iii) ist $V\coloneqq f^{-1}\of{B_{\varepsilon}^{N}\of{f\of{a}}}$ eine Umgebung von $a$. Das heißt
        \begin{align*}
            \exists\delta > 0\colon B_{\delta}^{M}\of{a} &\subseteq V\\
            \equivalent d_N\of{f\of{x}, f\of{a}} &< \varepsilon\quad\forall x\in B_{\delta}\of{a}\\
            \equivalent d_N\of{f\of{x}, f\of{a}} &< \varepsilon\quad\forall x\in M \text{ mit } d_M\of{x,a} < \delta\qedhere
        \end{align*}
    \end{proof}
\end{satz}

\begin{definition} % Definition 7
    \label{definition:stetigkeit-vollst-metr}
    Seien $\pair{M, d_M}$ und $\pair{N, d_N}$ metrische Räume und $f: M\fromto N$ eine Funktion.
    \begin{enumerate}[label=(\alph*)]
        \item $\varepsilon$-$\delta$-Version: $f$ heißt stetig auf $M$, falls
        \begin{align*}
            \forall x_0\in M\fa \varepsilon > 0\ex \delta > 0\colon d_N\of{f\of{x}, f\of{x_0}} < \varepsilon\quad\forall x\in M \text{ mit } d_M\of{x, x_0} < \delta
        \end{align*}
        \item Folgenversion: $f$ heißt stetig auf $M$, falls $\forall x_0\in M$ und jede Folge $(x_n)_n$ mit $x_n\fromto x_0$ gilt $f\of{x_n}\fromto f\of{x_0}$.
        \item Topologische Version: $f$ heißt stetig auf $M$, falls für jede offene Menge $U\subseteq N$ das Urbild $f^{-1}\of{U}$ offen in $M$ ist.
    \end{enumerate}
\end{definition}

\begin{satz} % Satz 8
    Für metrische Räume $\pair{M, d_M}$ und $\pair{N, d_N}$ und eine Funktion $f: M\fromto N$ sind die Versionen (a), (b) und (c) von Definition~\ref{definition:stetigkeit-vollst-metr} äquivalent.
    \begin{proof}
        Der Beweis von (a) $\equivalent$ (b) folgt direkt aus Satz~\ref{satz:stetigkeit-def-equiv}.\\
        (c) $\impl$ (a): Sei $U$ offen in $N$. Dann ist $f^{-1}\of{U}$ offen in $M$. Sei $x_0\in M$ beliebig und $\varepsilon > 0$. Dann ist
        \begin{align*}
            B_{\varepsilon}^{N}&\coloneqq B_{\varepsilon}^{N}\of{f\of{x_0}} = \set{y\in N: d_N\of{y, f\of{x_0}} < \varepsilon}
            \intertext{offen in $N$}
            &\impl V=f^{-1}\of{B_{\varepsilon}^{N}} \text{ offen in }M\\
            \intertext{Außerdem ist $x_0\in V$}
            &\impl \exists\delta > 0\colon B_{\delta}^{M} \subseteq V\\
            \impl &d_N\of{f\of{x}, f\of{x_0}} < \varepsilon\quad\forall x\in B_{\delta}^{M}\of{x_0}\\
            \impl &\forall d_M\of{x, x_0} < \delta\colon d_N\of{f\of{x}, f\of{x_0}} < \varepsilon
        \end{align*}
        (a) $\impl$ (c): Kontraposition. Angenommen es existiert eine Menge $U\subseteq N$, sodass $V\coloneqq f^{-1}\of{U}$ nicht offen ist. Dann existiert ein $x_0\in V$, sodass
        \begin{align*}
            \forall \delta> 0\colon B_{\delta}\of{x_0} &\not\subseteq V
            \intertext{Da $x_0\in V$ ist $f\of{x_0} \in U$. Da $U$ offen ist, gilt}
            \exists \varepsilon_0\colon B_{\varepsilon_0}^{N}\of{f\of{x_0}} &\subseteq U
            \intertext{Sei $\delta_n = \frac{1}{n}$. Da $B_{\delta}\of{x_0}$ keine Teilmenge von $V$ ist, folgt}
            \exists x_n&\in B_{\frac{1}{n}}\of{x_0}\\
            \impl x_n &\not\in V\\
            \impl x_n&\not\in f^{-1}\of{B_{\varepsilon_0}^N\of{f\of{x_0}}}\\
            \impl d_N\of{f\of{x_n}, f\of{x_0}} &> \varepsilon_0\\
            \impl d_M\of{x_n, x_0} &< \frac{1}{n} \text{ aber } d_N\of{f\of{x_N}, f\of{x_0}} > \varepsilon_0
        \end{align*}
        Damit ergibt sich ein Widerspruch zu (a).
    \end{proof}
\end{satz}

\begin{bemerkung}
    Für $N=\R^d$ und $M\subseteq\R$ ist die Stetigkeit von $f: M\fromto \R^d,~x\mapsto\pair{f_1\of{x}, f_2\of{x}, \ldots, f_n\of{x}}$ äquivalent dazu, dass jedes $f_j$ stetig ist.
\end{bemerkung}

\begin{bemerkung}
    \marginnote{[25. Jun]}
    Sei $\pair{M, d}$ ein metrischer Raum. Wenn $g, f: M\fromto\R$ stetig sind, dann sind auch $f+g: M\fromto\R,x\mapsto f\of{x}+g\of{x}$ sowie $f\cdot g: M\fromto\R,~x\mapsto f\of{x}\cdot g\of{x}$ stetig.\\
    Außerdem ist die Menge $D_{g\neq 0} \coloneqq \set{x\in M: g\of{x}\neq 0}$ für ein stetiges $g$ offen. Das heißt wenn $g\of{x_0} \neq 0$, dann ist $D_{g\neq 0}$ eine Umgebung und wir können eine Funktion $\frac{f}{g}: D_{g\neq 0}\fromto\R$ definieren, die stetig in $x_0$ ist.
    \begin{proof}
    (Übung)
    \end{proof}
\end{bemerkung}

\begin{definition}[Homöomorphismus] % Def 9
    Seien $\pair{M, d_M}$ und $\pair{N, d_N}$ metrische Räume. Dann heißt eine Funktion $f: M\fromto N$ Homöomorphismus, wenn sie bijektiv und stetig ist und $f^{-1}: N\fromto M$ auch stetig ist.
\end{definition}

\begin{beispiel}
    \theoremescape
    \begin{enumerate}
        \item Die Funktion
        \begin{align*}
            f: \R&\fromto\pair{-1, 1}\\
            x &\mapsto \frac{x}{1+\abs{x}}
        \end{align*}
        ist ein Homöomorphismus.
        \item Die Funktion
        \begin{align*}
            f: \R^d &\fromto B_{1}\of{0} = \set{x\in\R^d: \norm{x} < 1}\\
            x &\mapsto f\of{x} = \frac{x}{1+\norm{x}}
            \intertext{ist ein Homöomorphismus mit der Umkehrabbildung}
            f^{-1}\of{y} &= \frac{y}{1-\norm{y}}
        \end{align*}
    \end{enumerate}
\end{beispiel}

\begin{definition} % Def 10
    Seien $\pair{M, d_M}$ und $\pair{N, d_N}$ metrische Räume. Eine Funktionenfolge $(f_n)_n$ mit $f_n: M\fromto N$ konvergiert gleichmäßig gegen $f: M\fromto N$, falls
    \begin{align*}
        \forall\varepsilon > 0\ex N\in\N\colon d_N\of{f\of{x}, f_n\of{x}} &< \varepsilon\quad\forall x\in M\fa n\geq N
        \intertext{Das heißt}
        \limsup_{\ntoinf} \sup_{x\in M} d_N\of{f\of{x}, f_n\of{x}} &= 0
    \end{align*}
\end{definition}

\begin{satz}
    Seien $f_n: M\fromto N$ stetig für alle $n\in\N$ und konvergiere $(f_n)_n$ gleichmäßig gegen $f: M\fromto N$. Dann ist $f$ stetig.
    \begin{proof}
        Wir verwenden den $\frac{\varepsilon}{3}$-Trick. Angenommen $\varepsilon > 0$ und $x_0\in M$ beliebig. Wegen der gleichmäßigen Konvergenz gilt
        \begin{align*}
            \exists N\in\N\colon d_N\of{f\of{x}, f_n\of{x}} &< \frac{\varepsilon}{3}\quad\forall n\geq N
            \intertext{$f_N$ ist stetig in $x_0$. Das heißt}
            \exists\delta > 0\colon d_N\of{f_N\of{x}, f_N\of{x_0}} &< \frac{\varepsilon}{3}\quad\forall x\in M, d_M\of{x, x_0} < \delta
        \end{align*}
        Sei nun $x\in B_{\delta}\of{x_0} = \set{y\in M: d_M\of{y, x_0} <\delta}$. Dann folgt durch das Verwenden der Dreiecksungleichung
        \begin{align*}
            d_N\of{f\of{x}, f\of{x_0}} &\leq d_N\of{f\of{x}, f_N\of{x}} + d_N\of{f_N\of{x}, f\of{x_0}}\\
            &\leq d_N\of{f\of{x}, f_N\of{x}} + d_N\of{f_N\of{x}, f_N\of{x_0}} + d_N\of{f_N\of{x_0}, f\of{x_0}}\\
            &<\frac{\varepsilon}{3} + \frac{\varepsilon}{3} + \frac{\varepsilon}{3} = \varepsilon\qedhere
        \end{align*}
    \end{proof}
\end{satz}

\newpage


    \section{[*] Kompakte Mengen und metrische Räume}
    \section{[*] Kompakte Mengen und metrische Räume}
\imaginarysubsection{Kompakte Mengen}
\thispagestyle{pagenumberonly}

\begin{definition}[Kompaktheit]
    Sei $\pair{M, d}$ ein metrischer Raum. Eine Teilmenge $A\subseteq M$ ist kompakt, falls jede Folge in $A$ eine konvergente Teilfolge besitzt, deren Grenzwert wieder in $A$ liegt. Das heißt
    \begin{align*}
        \forall \text{Folge } (x_n)_n\subseteq A\ex\text{Teilfolge } (x_{n_k})_k\colon x\coloneqq \lim_{k\toinf} x_{n_k}\text{ existiert und }x\in A
    \end{align*}
\end{definition}

\begin{bemerkung}[Topologische Kompaktheit]
    Es gibt auch eine topologische Definition, die nicht über Folgen argumentiert und im metrischen Fall äquivalent zur Folgenkompaktheit ist: $A\subseteq M$ ist topologisch kompakt, falls jede offene Überdeckung eine endliche Teilüberdeckung hat.\\
    Das heißt für eine beliebige Indexmenge $I$ und $U_j$ offen $\forall j\in I$ mit $A\subseteq \bigcup_{j\in I} U_j$ muss ein $N\in\N$ und $j_1, j_2, \ldots, j_N \in I$ existieren, sodass $A \subseteq U_{j_1} \cup U_{j_2} \cup \dots \cup U_{j_N}$.
\end{bemerkung}

\begin{definition}
    Ein metrischer Raum $\pair{M, d}$ ist kompakt, falls $M$ kompakt ist.
\end{definition}

\begin{bemerkung}
    Jeder kompakte metrische Raum ist vollständig.
    \begin{proof}
        Wir haben gezeigt, dass eine Cauchy-Folge genau dann konvergiert, wenn sie eine konvergente Teilfolge hat. Damit konvergiert jede Cauchy-Folge in einem kompakten Raum.
    \end{proof}
\end{bemerkung}

\begin{satz}
    Wir betrachten $\pair{\R^d, \norm{\cdot}_{\infty}}$ mit Maximumsnorm. Dann gilt $A\subseteq \R^d$ ist genau dann kompakt, wenn $A$ abgeschlossen und beschränkt ist.
    \begin{proof}
        \anf{$\impl$} Sei $A\subseteq\R^d$ kompakt. Sei $(x_n)_n \subseteq A$ eine Folge in $A$, welche gegen $x\in\R^d$ konvergiert. Dann existiert eine Teilfolge $(x_{n_k})_k$ mit $y\coloneqq \lim_{\ntoinf} x_{n_k}\in A$. Da $(x_n)_n$ konvergiert, ist $x=y\in A$.\\
        Angenommen $A$ ist nicht beschränkt. Das heißt
        \begin{align*}
            \forall n\in\N\ex x_n\in A\colon \norm{x_n}_{\infty} \geq n
        \end{align*}
        Dann gilt für jede Teilfolge $(x_{n_k})_k$ auch $\norm{x_{n_k}} \geq n_k \geq k$. Das heißt jede Teilfoge von $(x_n)_n$ ist unbeschränkt und kann somit nicht konvergieren.\\[.5\baselineskip]
        \anf{$\Leftarrow$} Sei $A$ abgeschlossen und beschränkt. Dann wählen wir eine Folge $(x_n)_n\subseteq A$. Dann besteht $x_n$ aus mehreren Koordinaten. Das heißt $x_n = \pair{x_n^1~x_n^2~\dots~x_n^d}$ Wir definieren $(x_n^j)_n$ als Folge in $\R$. Da $A$ beschränkt ist, muss auch $(x_n^j)_n$ beschränkt sein. Das heißt für $j=1$ gibt es eine Teilfolge $(x_{n_k}^1)_k$ von $(x_n^1)_n$, die konvergiert. Es gibt also eine Ausdünnung $\sigma_1: \N\fromto\N$ mit $\sigma_1\of{n+1} > \sigma_1\of{n}~\forall n\in\N$, sodass $x^1\coloneqq \lim_{k\toinf} x_{\sigma_{1}\of{k}}^1$ existiert. Dann konvergiert die erste Koordinate von $\pair{x_{\sigma_1\of{n}}^1~x_{\sigma_1\of{n}}^2~\ldots~x_{\sigma_1\of{n}}^d}$.\\
        Wir führen das Prinzip iterativ fort. Das heißt $(x_{\sigma\of{n}}^2)_n$ ist beschränkt in $\R$. Damit hat es eine konvergente Teilfolge. Wir wählen eine neue Ausdünnung $\sigma_2: \N\fromto\N$ analog zu $\sigma_1$ und setzen $\kappa_2 \coloneqq \sigma_2 \circ \sigma_1$. Dann hat $x^2_{\kappa_2\of{n}}$ einen Grenzwert $x^2\coloneqq \biglim{\ntoinf} x^2_{\kappa_2\of{n}}$. Dabei konvergiert $x_{\kappa_2\of{n}}^1$ immer noch.\\
        Dieses Prinzip wenden wir für jede Koordinate an. Das heißt wir definieren $\sigma_j: \N\fromto\N$ und $\kappa_{j+1} \coloneqq \sigma_{j+1} \circ \kappa_j$. Damit hören wir bei $j=d$ auf. Dann haben wir $x_{\kappa_d\of{n}}^j$ konvergiert gegen ein $x^j$ für alle $j\in\set{1, \ldots, d}$. Das heißt wir haben einen Grenzwert $x=\pair{x^1~x^2~\dots~x^d}$ von $(x_{\kappa_d\of{n}})_n$. Da $A$ abgeschlossen ist und $x_{\kappa\of{n}} \fromto x$ ist auch $x\in A$. Damit ist $A$ kompakt.
    \end{proof}
\end{satz}

\begin{satz} % Satz 4
    \label{satz:weierstrass-alg}
    Sei $\pair{M, d}$ ein metrischer Raum und $A\subseteq M$ kompakt. Sei außerdem $f: A\fromto\R$ stetig. Dann nimmt $f$ sein Maximum und Minimum an. Das heißt
    \begin{align*}
        \exists\un{x}\in A\colon f\of{\un{x}} &= \inf\set{f\of{x}: x\in A} \text{ und } \\
        \exists\ov{x}\in A\colon f\of{\ov{x}} &= \sup\set{f\of{x}: x\in A}
    \end{align*}
    \begin{proof}
        \textsc{Schritt 1}: Sei $a\coloneqq\inf\set{f\of{x}: x\in A}$. Angenommen $a=-\infty$. Dann würde gelten
        \begin{align*}
            &\forall n\in\N\ex x_0\in A\colon f\of{x_n} \leq -n\\
            \intertext{Das heißt es existiert eine konvergente Teilfolge $(x_{n_k})_k$, da $A$ kompakt ist. Wir definieren $x \coloneqq \biglim{k\toinf} x_{n_k}\in A$. Wegen der Stetigkeit von $f$ gilt dann}
            \R\ni f\of{x} &= f\of{\lim_{k\toinf} x_{n_k}} = \lim_{k\toinf} f\of{x_{n_k}} = -\infty
        \end{align*}
        \textsc{Schritt 2}: Sei $a\coloneqq\inf\set{f\of{x}: x\in A}$. Das heißt
        \begin{align*}
            \forall n\in\N\ex x_n\in A\colon &f\of{x_n} < a + \frac{1}{n}\\
            \impl a \leq &f\of{x_n} < a + \frac{1}{n}
            \intertext{Da $A$ kompakt ist, existiert eine konvergente Teilfolge $(x_{n_k})_k$ von $(x_n)_n$. Wir definieren}
            \un{x}\coloneqq&\lim_{k\toinf} x_{n_k} \leq a + \frac{1}{n}\\
            \impl a \leq &f\of{\un{x}} = f\of{\lim_{k\toinf} x_{n_k}} = \lim_{k\toinf} x_{n_k} = a\\
            \impl &f\of{\un{x}} = a
        \end{align*}
        \textsc{Schritt 3}: Für das Maximum betrachte $-f$ und wende \textsc{Schritt 1} und \textsc{Schritt 2} an.
    \end{proof}
\end{satz}

\begin{satz} % Satz 5
    \marginnote{[28. Jun]}
    Auf endliche-dimensionalen Vektorräumen sind alle Normen äquivalent. Das heißt für einen endlich-dimensionalen Vektorraum $X$ mit zwei Normen $\norm{\cdot}_a, \norm{\cdot}_b$ gibt es $0 <c_1 \leq c_2 < \infty$, sodass
    \begin{align*}
        c_1\cdot\norm{h}_a &\leq \norm{h}_b \leq c_2\cdot\norm{h}_a\quad\forall h\in X
    \end{align*}
    \begin{proof}
        \textsc{Schritt 1}: Sei $d\coloneqq\dim\of{X} < \infty$. Wir wählen eine geordnete Basis $\pair{e_1, e_2, \ldots, e_d}$ von $X$. Dann können wir einen beliebigen Vektor $h\in X$ schreiben als
        \begin{align*}
            h &= \sum_{j=1}^{d} t_j\cdot e_j
            \intertext{Für eindeutige $t_j\in\R$. Damit erhalten wir eine Bijektion}
            h: \R^d &\fromto X\\
            \R^d \ni t&\mapsto h\of{t} = \sum_{j=1}^{d} t_j\cdot e_j \in X
            \intertext{Also reicht es, zu zeigen, dass für beliebige Normen $\norm{\cdot}$ auf $X$ und gewählte $0< c_1 \leq c_2 < \infty$ gilt}
            c_1\cdot\norm{t}_{\infty} &\leq \norm{h\of{t}} \leq c_2\cdot\norm{t}_{\infty}\quad\forall t\in\R^d\tag{1}
            \intertext{Setzen wir nämlich voraus, dass (1) gilt, dann gilt für zwei beliebige Normen $\norm{\cdot}_a, \norm{\cdot}_b$ auf $X$}
            c_1\cdot\norm{t}_{\infty} &\leq \norm{h\of{t}}_a \leq c_2\cdot\norm{t}_{\infty}\\
            d_1\cdot\norm{t}_{\infty} &\leq \norm{h\of{t}}_b \leq d_2\cdot\norm{t}_{\infty}
            \intertext{Sei $h\coloneqq h\of{t}$}
            \impl \norm{h}_a &\leq c_2\cdot \norm{t}_{\infty} \leq \frac{c_2}{d_1}\cdot\norm{h}_b\\
            &\leq \frac{c_2}{d_1}\cdot d_2 \cdot \norm{t}_{\infty}\\
            &\leq \frac{c_2\cdot d_2}{c_1\cdot d_1} \cdot\norm{h}_a\\
            \impl \frac{d_1}{c_2} \cdot\norm{h}_a &\leq \norm{h}_b \leq \frac{d_2}{c_1}\cdot\norm{h}_a
            \intertext{Damit wäre die Behauptung gezeigt. Das heißt wir müssen nur noch (1) zeigen. \textsc{Schritt 2}: Sei $\norm{\cdot}$ eine beliebige Norm auf $X$. Wir definieren}
            f: \R^d &\fromto \R\\
            h &\mapsto f\of{h\of{t}} = \norm{h\of{t}}
            \intertext{Dann existiert ein $0 < C < \infty$, sodass}
            \abs{f\of{t}} = f\of{t} &\leq C\cdot\norm{t}_{\infty}\quad\forall t\in\R^d\tag{2}
            \intertext{und}
            \abs{f\of{t_1} - f\of{t}} &\leq C\cdot\norm{t_1 - t_2}_{\infty}\quad\forall t_1, t_2\in\R^d\tag{3}
            \intertext{Dabei folgt aus (3) gerade, dass $f$ Lipschitz-stetig ist. Um (2) zu beweisen, betrachten wir}
            \abs{f\of{t}} &= \norm{h\of{t}} = \norm{ \sum_{j=1}^{d} t_j\cdot e_j}\\
            &\leq \sum_{j=1}^{d} \norm{t_j\cdot e_j}\\
            &\leq \max_{1\leq j \leq d} \abs{t_j} \cdot \underbrace{\sum_{j=1}^{d} \norm{e_j}}_{\eqqcolon C} = C\cdot\norm{t}_{\infty}
            \intertext{Außerdem zeigen wir (3):}
            \abs{f\of{t_1} - f\of{t_2}} &= \abs{\norm{h\of{t_1}} - \norm{h\of{t_2}}}\\
            &= \abs{\norm{h_1} - \norm{h_2}} \leq \norm{h_1 - h_2}\\
            \impl \abs{f\of{t_1} - f\of{t_2}} &\leq \norm{h\of{t_1} - h\of{t_2}}\\
            &= \norm{h\of{t_1 - t_2}} \leq C\cdot\norm{t_1 - t_2}_{\infty}
            \intertext{Damit haben wir (3) gezeigt. \textsc{Schritt 3}: Nach (3) gilt $\norm{h\of{t}} \leq C\cdot\norm{t}_{\infty}$. Also reicht es zu zeigen, dass}
            \exists c_1 > 0\colon c_1\cdot\norm{t}_{\infty} &\leq \norm{h\of{t}}
            \intertext{Sei $t\neq 0$. Dann ist auch $\norm{t}_{\infty} > 0$}
            f\of{\frac{t}{\norm{t}}_{\infty}} &= \norm{h\of{\frac{t}{\norm{t}_{\infty}}}}\\
            h\of{\frac{t}{\norm{t}_{\infty}}} &= \sum_{j=1}^{d} \frac{t_j}{\norm{t}_{\infty}}\cdot e_j\\
            &= \frac{1}{\norm{t}_{\infty}} \sum_{j=1}^{d} t_j\cdot e_j\\
            &= \frac{1}{\norm{t}_{\infty}}\cdot f\of{t}\\
            \impl \norm{t}_{\infty}\cdot f\of{\frac{t}{\norm{t}_{\infty}}} &= f\of{t}\quad\forall t\in \R^d\exclude\set{\textbf{0}}\tag{4}
            \intertext{Wir definieren}
            S &\coloneqq \set{t\in\R^d : \norm{t}_{\infty} = 1}
            \intertext{Damit ist $S$ beschränkt und abgeschlossen. Wir wählen eine Folge $(t_n)_n\subseteq S$ mit $t_n\fromto t$}
            \impl \norm{t}_{\infty} &= \norm{\lim_{\ntoinf} t_n}_{\infty} = \lim_{\ntoinf}\norm{t_n}_{\infty} = 1\\
            \impl t&\in S
            \intertext{Wir definieren außerdem}
            c_1 &\coloneqq \inf\set{f\of{t}: t\in S}\\
            \annot{\impl}{(4)} f\of{t} &= \norm{t}_{\infty} \cdot f\of{\frac{t}{\norm{t}_{\infty}}} \geq c_1\cdot\norm{t}_{\infty}\quad\forall t\in\R^d
            \intertext{Frage: Was ist ein guter Grund, dass $c_1 > 0$? \textsc{Schritt 4}: $S$ ist kompakt, da $\R^d$ endlich-dimensional ist und $S$ beschränkt und abgeschlossen ist. Außerdem ist $f: S\fromto\R$ stetig und $f\of{t} > 0~\forall t\in S$. Nach Satz~\ref{satz:weierstrass-alg} nimmt $f$ auf $S$ sein Minimum ein. Das heißt es existiert ein $\un{t}\in S$, sodass}
            c_1 &= \inf\set{f\of{t}: t\in S} = f\of{\un{t}} > 0\qedhere
        \end{align*}
    \end{proof}
\end{satz}

\begin{bemerkung}
    Sei $d = \dim\of{X}$ und $\pair{e_1, e_2,\ldots, e_d}$ eine Basis von $X$. Dann ist
    \begin{align*}
        F: \R^d &\fromto X\\
        t &\mapsto F\of{t} = \sum_{j=1}^{d} t_j\cdot e_j
        \intertext{eine lineare Bijektion. Wir haben im vorherigen Beweis gesehen, dass}
        \norm{F\of{t}} & \leq C\cdot\norm{t}_{\infty}\text{ für } C \coloneqq \sum_{j=1}^{d} \norm{e_j}
        \intertext{Damit ist $F$ sogar stetig. Frage: Ist $F^{-1}: X\fromto\R^d$ auch stetig?}
        c_1 \cdot\norm{t}_{\infty} &\leq \norm{F\of{t}}\\
        t &= F^{-1}\of{h}\tag{$h\coloneqq F\of{t}$}\\
        \impl c_1\cdot\norm{F^{-1}\of{h}} &\leq \norm{h}\\
        \impl \norm{F^{-1}\of{h}} &\leq \frac{1}{c_1}\cdot\norm{h}
    \end{align*}
    Das heißt $F^{-1}: X\fromto \R^d$ ist auch stetig.
\end{bemerkung}

\begin{satz} % Satz 6
    Seien $\pair{M, d_M}$ und $\pair{N, d_N}$ metrische Räume, $M$ kompakt und $f: M\fromto N$. Dann gilt $f$ ist genau dann stetig, wenn $f$ gleichmäßig stetig ist. Das heißt
    \begin{align*}
        \forall \varepsilon > 0\ex\delta > 0\fa x,y\in M\colon d_M\of{x,y} < \delta \impl d_N\of{f\of{x}, f\of{y}} < \varepsilon
    \end{align*}
    \begin{proof}
        \anf{$\Leftarrow$} Ist klar.\\
        \anf{$\impl$} Angenommen $f$ ist stetig, aber nicht gleichmäßig stetig. Das heißt
        \begin{align*}
            \exists\varepsilon > 0\fa\delta > 0\ex x,y\in M\colon d_M\of{x,y} < \delta \text{ aber } d_N\of{f\of{x}, f\of{y}} \geq \varepsilon
            \intertext{Wähle $\delta = \frac{1}{n}$. Das heißt es existieren zwei Folgen $(x_n)_n, (y_n)_n \subseteq M$, sodass}
            d_N\of{x_n, y_n} < \frac{1}{n} \text{ aber } d_N\of{f\of{x_n}, f\of{y_n}} \geq \varepsilon\quad\forall n\in\N
            \intertext{Da $M$ kompakt ist, existiert eine Teilfolge $(y_{n_k})_k$ von $(y_n)_n$, die konvergiert. Sei}
            y \coloneqq \lim_{k\toinf} y_{n_k} \in M
            \intertext{Behauptung: $x_{n_K} \fromto y$, da}
            d_M\of{x_{n_k}, y} \leq d_M\of{x_{n_k}, y_{n_k}} + d_M\of{y_{n_k}, y} \fromto 0
            \intertext{Da $f$ stetig ist folgt}
            \impl d_N\of{f\of{x_{n_k}}, f\of{y}}\fromto 0 \text{ für } k\toinf\\
            0 < \varepsilon \leq d_N\of{f\of{x_{n_k}}, f\of{y_{n_k}}} \leq d_N\of{f\of{x_{n_k}}, f\of{y}} + d_N\of{f\of{y}, f\of{y_{n_k}}}\fromto 0
        \end{align*}
        Damit ergibt sich ein Widerspruch, da $\varepsilon$ fest gewählt war.
    \end{proof}
\end{satz}

\newpage


    \section{[*] Differentialrechnung im $\R^d$}

    \subsection{Die Ableitung}
    \thispagestyle{pagenumberonly}
    Bisher haben wir die Ableitung in $\R$ definiert. Die Idee dahinter war, eine Funktion $f\of{x}$ an einer Stelle durch eine affine Funktion zu approximieren. Das heißt
    \begin{align*}
        f\of{x+h} &= f\of{x} + f'\of{x}\cdot h + \fehler_x\of{h}\\
        &=b + a\cdot h + \fehler_x\of{h}
        \intertext{Dabei galt}
        &\frac{\abs{\fehler_x\of{h}}}{h}\fromto 0 \text{ für } h\toinf\\
        \equivalent &\lim_{h\toinf} \frac{f\of{x+h} - f\of{x}}{h} \text{ existiert }
        \intertext{Wir wollen dieses Prinzip auf den $\R^d$ übertragen und fragen uns: Gibt es eine affine Funktion}
        g: \R^n &\fromto \R^m\\
        h &\mapsto g\of{h} = b + A\cdot\interv{h} = b + A\cdot h
        \intertext{sodass}
        f\of{x + h} &= g\of{h} + \fehler_x\of{h} \in\R^m
        \intertext{mit}
        &\frac{\norm{\fehler_x\of{h}}_{\R^m}}{\norm{h}_{\R^n}}\fromto 0 \text{ für } h\toinf
        \intertext{Angenommen wir finden eine solche Funktion $f: D\fromto Y\coloneqq \R^m$ mit $D\subseteq X\coloneqq\R^u$, sodass}
        f\of{x+h} &= b + A\cdot\interv{h} + \fehler_x\of{h}\\
        \frac{\norm{\fehler_x\of{h}}_Y}{\norm{h}_X} &= \frac{\norm{\varepsilon_x\of{h}\cdot\norm{h}_X}}{\norm{h}_X} = \norm{\varepsilon_x\of{h}}_Y\fromto 0\\
        \varepsilon_x\of{h} &\coloneqq \frac{\fehler_x\of{h}}{\norm{h}_X}\\
        f\of{x+h} &= g\of{h} + \varepsilon_x\of{h}\cdot\norm{h}_X\tag{*}
    \end{align*}
    Behauptung: Angenommen (*) und $\varepsilon_x\of{h} \fromto 0$ gilt. Dann folgt, dass $g$ eindeutig bestimmt ist.
    \begin{proof}
        Angenommen es existieren $g_1, g_2$ affine Abbildungen, sodass
        \begin{align*}
            f\of{x+h} &= g_1\of{h} + \varepsilon_x^1\of{h}\cdot\norm{h}_X\\
            &= g_2\of{h} + \varepsilon_x^2\of{h}\cdot\norm{h}_X
            \intertext{Dann ist zu zeigen, dass $g_1 = g_2$. Wir definieren}
            g\of{h} &= g_1\of{h} - g_2\of{h} = -\varepsilon_x^1\of{h}\cdot\norm{h}_X + \varepsilon_x^2\of{h}\cdot\norm{h}_X\\
            &\eqqcolon \varepsilon_x\of{h}\cdot\norm{h}_X\fromto 0 \text{ für } h\fromto 0\\
            \impl g\of{0} &= 0 = b_1 - b_2\\
            \impl b1 &= b_2
            \intertext{Damit sind bereits die $b$ gleich. Also gilt}
            \impl g\of{h} &= A_2\interv{h} - A_1\interv{h} = \pair{A_2 - A_1}\interv{h}
            \intertext{Außerdem gilt $x+h\in U$. Wir ersetzen $h$ durch $th$. Dann ist $x+th\in U$}
            A\interv{h} = \impl g\of{th} &= \pair{\varepsilon_1\of{th} - \varepsilon_2\of{th}}\cdot\norm{th}_X
            \intertext{Das heißt $\abs{t} < \delta = \frac{r}{\norm{h}_X}$}
            \impl A &= A_2 - A_1\\
            \impl t\cdot A\interv{h} = A\interv{th} &= \pair{\varepsilon_1\of{th} - \varepsilon_2\of{th}}\cdot\norm{h}_X\cdot\abs{t}\\
            \impl \norm{tA\interv{h}}_Y &= \abs{t}\cdot\norm{h}_X \cdot\norm{\varepsilon_1\of{th} + \varepsilon_2\of{th}}_Y\\
            \impl \norm{A\interv{h}}_Y &= \norm{h}_X \cdot\norm{\varepsilon_1\of{th} - \varepsilon_2\of{th}}_Y \fromto 0\\
            \impl \norm{A\interv{h}}_Y &= 0\quad\forall h\in X\\
            \impl 0=A&= A_2 - A_1\\
            \impl A_1 &= A_1\qedhere
        \end{align*}
    \end{proof}

    \begin{definition}
        \marginnote{[02. Jul]}
        Sei $U\subseteq\R^n$ offen und $f: U\fromto\R^m$, $x\in U$. $f$ ist differenzierbar in $x$, falls eine lineare Abbildung $A: \R^n\fromto \R^m$ existiert mit
        \begin{align*}
            f\of{x+h} &= f\of{h} + A\interv{h} + \varepsilon_x\of{h}\cdot\norm{h}\\
            &= f\of{x} + A\interv{h} + \fehler_x\of{h}
        \end{align*}
        $\varepsilon_x\of{h} \fromto 0$ (in $\R^m$) für $h\fromto 0$ ($\in \R^n$). Wir schreiben $\D f\of{x} \coloneqq A \in\mL\of{\R^n, \R^m}$ (lineare Abbildungen von $\R^n$ nach $\R^m$).
    \end{definition}

    \begin{definition}
        Seien $\pair{X, \norm{\cdot}_X}$, $\pair{Y, \norm{\cdot}_Y}$ beliebige normierte Vektorräume und $U\subseteq X$ offen. Dann ist $f: U\fromto Y$ differenzierbar in $X$, falls $A\in\mL\of{X, Y}$ eine \underline{stetige} lineare Abbildung von $X$ nach $Y$ ist, sodass
        \begin{align*}
            f\of{x+h} &= f\of{x} + A\interv{h} + \fehler_x\of{h}\\
            &= f\of{x} + A\interv{h} + \varepsilon_x\of{h}\cdot\norm{h}_X,\\
            &\norm{\varepsilon_x\of{h}}_Y \fromto 0 \text{ für } h\fromto 0.
        \end{align*}
    \end{definition}

    \begin{bemerkung}
        Wir müssen im Allgemeinen verlangen, dass $A$ eine stetige lineare Abbildung $\equivalent A$ ist beschränkte lineare Abbildung.
        \begin{align*}
            &= \sup_{h\in X, \norm{h}_X \leq 1} \norm{A\interv{h}}_Y = \sup_{h\in X, \norm{h}_X = 1} \norm{A\interv{h}}_Y\\
            &= \sup_{h\in X\exclude\set{0}} \frac{\norm{A\interv{h}}_Y}{\norm{h}_X} = \norm{A\interv{\frac{h}{\norm{h}_X}}}_Y\\
            \norm{A}_{X\fromto Y} &\coloneqq \sup_{h\in X\exclude\set{0}} \frac{\norm{A\interv{h}}_Y}{\norm{h}_X}\tag{Operatornorm von $A: X\fromto Y$}
        \end{align*}
        $\mL\of{X,Y}$ mit $\norm{\cdot}_{X\fromto Y}$ ist ein normierter Vektorraum. Dieser ist vollständig, falls $Y$ vollständig ist.
    \end{bemerkung}

    \begin{definition}
        Eine Funktion $f: U\fromto Y$ mit $U\subseteq X$ offen ist differenzierbar, falls es in jedem Punkt $x\in U$ differenzierbar ist. Wir definieren also eine Abbildung
        \begin{align*}
            \D f: U&\fromto\mL\of{X, Y}\\
            x&\fromto\D f\of{x}
        \end{align*}
    \end{definition}

    \begin{satz}
        Für eine lineare Funktion $A: X\fromto Y$ sind äquivalent
        \begin{enumerate}[label=(\roman*)]
            \item $A$ ist Lipschitz-stetig
            \item $A$ ist gleichmäßig stetig
            \item $A$ ist stetig
            \item $A$ ist stetig in $0\in X$
            \item $A$ ist beschränkt, das heißt $\norm{A}_{X\fromto Y} < \infty$
        \end{enumerate}
    \end{satz}

    \begin{satz}
        Die Ableitung ist linear. Das heißt für $X,Y$ normierte Vektorräume und $U\subseteq X$ offen sowie $f,g: U\fromto Y$ differenzierbar in $x\in U$ gilt
        \begin{align*}
            \D \pair{\lambda f}\of{x} &= \lambda \D f\of{x}\\
            \D\pair{f+g} \of{x} &= \D f\of{x} + \D g\of{x}
        \end{align*}
        \begin{proof}
            Folgt direkt aus der Definition der Ableitung.
        \end{proof}
    \end{satz}

    \begin{beispiel}
        Sei $X$ ein normierter Vektorraum und $\sprod{\cdot, \cdot}$ ein Skalarprodukt auf $X$. Außerdem definieren wir $f\of{x} \coloneqq \sprod{x, x}$.
        \begin{align*}
            f\of{x+h} &= \sprod{x+h, x+h}\\
            &= \sprod{x, x+h} + \sprod{h, x+h}\\
            &= \sprod{x,x} + \sprod{x,h} + \sprod{h,x} + \sprod{h,h}\\
            &= f\of{x} + \underbrace{2\sprod{x,h}}_{\text{linear in $h$}} + \underbrace{\norm{h}^2}_{\text{Fehler}}\\
            \frac{\fehler\of{h}}{\norm{h}} &= \frac{\norm{h}^2}{\norm{h}} = \norm{h} \fromto 0
        \end{align*}
    \end{beispiel}

    \subsection{[*] Richtungsableitung und partielle Ableitung}
    \begin{definition}[Richtungsableitung]
        Sei $U\subseteq X$ offen und $f: U\fromto Y$ sowie $h\in X$ beliebig. Die Richtungsableitung von $f$ (in $x$) in Richtung $h$ ist gegeben durch
        \begin{align*}
            \lim_{t\fromto 0} \frac{f\of{x+th} - f\of{x}}{t} &\eqqcolon\text{D}_h\!f\of{x}
            \intertext{Also ist}
            \text{D}_h\!f\of{x} &= \frac{\dif}{\dif t}f\of{x+th}
        \end{align*}
    \end{definition}

    \begin{satz} % Satz 6
        Ist $f: U\fromto Y$ in $x$ differenzierbar, so folgt
        \begin{align*}
            \D f\of{x}\interv{h} = \text{D}_h\!f\of{x}\quad\forall h\in X
        \end{align*}
        \begin{proof}
            Sei $A = \D f\of{x}$
            \begin{align*}
                f\of{x+h} &= f\of{x} - A\interv{h} = \varepsilon\of{h}\norm{h}_X\\
                f\of{x+th} - f\of{x} - A\interv{th} &= \varepsilon\of{th}\cdot\norm{th}_X\\
                &= \varepsilon\of{th}\cdot\abs{t}\cdot\norm{h}_X\\
                \frac{f\of{x+th} - f\of{x}}{t} - A\of{h} &= \frac{\abs{t}}{t}\cdot\varepsilon\of{th} \cdot\norm{h}_X\fromto 0 \text{ für } t\fromto 0\\
                \impl \text{D}_h\!f\of{x} &= \lim_{t\fromto 0} \frac{f\of{x+th}- f\of{x}}{t} = A\interv{h} = \D f\of{x}\interv{h}\qedhere
            \end{align*}
        \end{proof}
    \end{satz}

    \begin{mdframed}
        \begin{center}
            Im Folgenden sei $X=\R^n$ und $\pair{e_1, e_2, \ldots, e_n}$ die Standardbasis.
        \end{center}
    \end{mdframed}

    \begin{definition}[Partielle Ableitung]
        Sei $f: U\fromto Y$ mit $U\subseteq \R^n$ und $f$ differenzierbar in $x\in U$. Außerdem sei $A \coloneqq \D f\of{x}$. Dann definieren wir die partielle Ableitung $\partial_{x_j} f\of{x} = \partial_j f\of{x}$ bezüglich der Standardbasen durch
        \begin{align*}
            \partial_{x_j} f\of{x} &= \partial_j f\of{x} = \text{D}_{e_j}\! f\of{x}\\
            &= \lim_{t\fromto 0} \frac{f\of{x+t\cdot e_j} - f\of{x}}{t}\\
            &= \frac{\dif}{\dif t}f\of{x+t\cdot e_j}\vert_{t = 0}
        \end{align*}
    \end{definition}

    \begin{bemerkung}
        \marginnote{[05. Jul]}
        Sei $f: U\fromto Y$  in $x\in U$ differenzierbar. $\D f\of{x}\in\mL\of{\R^n, Y}$. Dann gilt
        \begin{align*}
            \D f\of{x}\interv{h} &= \D f\of{x}\cdot h = \D f\begin{pmatrix}
                                                                h_1    \\
                                                                \vdots \\
                                                                h_n
            \end{pmatrix} = \begin{pmatrix}
                                \partial_1 f_1 & \cdots & \partial_n f_1 \\
                                \vdots         & \ddots & \vdots         \\
                                \partial_1 f_m & \cdots & \partial_n f_m
            \end{pmatrix}\begin{pmatrix}
                             h_1    \\
                             \vdots \\
                             h_n
            \end{pmatrix}
        \end{align*}
    \end{bemerkung}

    \noindent\textbf{Warnung:} Die Existenz der partiellen Ableitungen und sogar aller Richtungsableitungen impliziert nicht die Existenz der Ableitung!

    \begin{beispiel}
        Für die Funktion $f: \R^2 \fromto \R$ mit
        \begin{align*}
            f\of{x} &= \begin{cases}
                           \dfrac{x_1 x_2}{\sqrt{x_1^2 + x_2^2}} &x\neq \pair{0,0}\\
                           0 &x = \pair{0,0}
            \end{cases}
            \intertext{existiert $\frac{\partial f}{\partial x_2}$ für $x_1 \neq 0 \impl \sqrt{x_1^2 + x_2^2} \neq 0$. Außerdem gilt für $x_1 = 0$}
            \frac{\partial f}{\partial x_2} &= 0
            \intertext{Damit existieren die partiellen Ableitungen. Außerdem gilt}
            f\of{th} &= \frac{t^2 h_1 h_2}{\sqrt{t^2 h_1^2 + t^2 h_2^2}}\\
            &= \frac{t^2 h_1 h_2}{t\sqrt{h_1^2 + h_2^2}} = t\cdot f\of{h}\\
            \impl \text{D}_n\! f\of{0} &= \lim_{t\fromto 0} \frac{f\of{th}-f\of{0}}{t} = \lim_{t\fromto 0} \frac{tf\of{h} - 0}{t} = f\of{h} = \frac{h_1 h_2}{\sqrt{h_1^2 + h_2^2}}
        \end{align*}
        $f\of{x}$ ist nicht stetig in $\pair{0,0}$.
    \end{beispiel}

    \begin{satz} % Satz 8
        Sei $f: U\fromto Y$, $U$ offen, $x\in U$ und $f$ sei differenzierbar in $x$. Dann ist $f$ stetig in $x$.
        \begin{proof}
            Sei $h$ klein genug, dass $x+h \in U$. Dann gilt
            \begin{align*}
                f\of{x+h} &= f\of{x} + \underbrace{A\interv{h}}_{\fromto 0 \text{ für } \norm{h}\fromto 0} + \underbrace{\varepsilon\of{h}}_{\fromto 0 \text{ für } \norm{h}\fromto 0}\cdot\norm{h}_X\fromto f\of{x}
            \end{align*}
            Damit ist $f$ in $x$ stetig.
        \end{proof}
    \end{satz}

    \subsection{Kettenregel}

    \begin{satz}[Kettenregel] % Satz 9
        Seien $X, Y, Z$ normierte Vektorräume mit $V\subseteq X$, $U\subseteq Y$ offen und $f: V\fromto U$, $g: U\fromto Z$. Sei außerdem $f$ diffenzierbar in $x\in V$, $g$ differenzierbar in $y=f\of{x}$. Dann gilt $g\circ f: V\fromto Z$ ist differenzierbar in $x$ und $\text{D}_{g\circ f}\of{x} = \text{D}_g\of{f\of{x}}\circ \D f\of{x}$. Das heißt
        \begin{align*}
            \text{D}_{f\circ g}\of{x}\interv{h} &= \text{D}_g\of{f\of{x}}\interv{\D f\of{x}\interv{h}}
        \end{align*}
        \begin{proof}
            Sei $A = \D f\of{x}$, $B= \D g\of{y}$, $y=f\of{x}$. Nach Definition gilt
            \begin{align*}
                f\of{x+h} &= f\of{x} + A\interv{h} + \varepsilon_1\of{h}\cdot\norm{h}_X\quad\forall\pair{x+h}\in V\\
                g\of{y+k} &= g\of{y} + B\interv{k} + \varepsilon_2\of{k}\cdot\norm{k}_X \quad\forall\pair{y+k}\in U\\
                \impl \pair{g\circ f}\of{x+h} &= g\of{f\of{x+h}} = g\of{f\of{x} + \underbrace{A\interv{h} + \varepsilon_1\of{h}\cdot\norm{h}_X}_{\eqqcolon k}}\\
                &= g\of{y+k} = g\of{f\of{x}} + B\interv{k} + \varepsilon_2\of{k}\cdot\norm{k}_{Y}\\
                &= g\of{f\of{x}} + B\interv{A\interv{h}} + \norm{h}_X \cdot B\interv{\varepsilon_1\of{h}}\\
                &~~+ \varepsilon_2\of{A\interv{h} + \varepsilon_1\of{h}\cdot\norm{h}_X}\cdot\norm{A\interv{h} + \varepsilon_1\of{h}\cdot\norm{h}_X}
                \intertext{Durch Analyse der Grenzwerte der einzelnen Summanden, ergibt sich, dass}
                \pair{g\circ f}\of{x+h} - \pair{g\circ f} \of{x} &= B\interv{A\interv{h}} + \varepsilon\of{h}\cdot\norm{h}_X
                \intertext{Damit sit $g\circ f$ in $x$ differenzierbar und}
                \text{D}_{g\circ f}\interv{h} &= B\interv{A\interv{h}}\qedhere
            \end{align*}
        \end{proof}
    \end{satz}

    \begin{bemerkung}[Richtungsableitung aus Ableitung]
        \marginnote{[09. Jul]}
        Sei $I=\interv{a,b}, f: I \fromto Y$ und existiere
        \begin{align*}
            f'\of{x} &= \lim_{s\fromto 0} \frac{F\of{x+s} - F\of{x}}{s}
        \end{align*}
        Dann gilt
        \begin{align*}
            \text{D}_h\! f\of{x} = \frac{\dif f}{\dif t} f\of{x+th} &= \lim_{t\fromto 0} \frac{f\of{x+th} - f\of{x}}{t\cdot h} \cdot h = \lim_{s\fromto 0} \frac{f\of{x+s} - f\of{x}}{s} \cdot h = f'\of{x}\cdot h
        \end{align*}
    \end{bemerkung}

    \begin{bemerkung}
        Sei $f: \pair{a,b}\fromto Y$ differenzierbar. Dann gilt für alle $x\in\pair{a,b}$
        \begin{align*}
            \norm{\D f\of{x}}_{\R\fromto Y} \coloneqq \sup_{h\neq 0} \frac{\norm{\D f\of{x}\interv{h}}_Y}{\abs{h}} = \frac{\norm{f'\of{x}\cdot h}}{\abs{h}} = \frac{\abs{h}\cdot\norm{f'\of{x}}_Y}{\abs{h}} = \norm{f'\of{x}}_Y
        \end{align*}
    \end{bemerkung}

    \begin{satz}[Mittelwertsatz oder Schrankensatz] % Satz 10
        \label{satz:mittelwertsatz}
        Sei $\pair{Y, \norm{\cdot}_Y}$ ein vollständiger Vektorraum, $I=\interv{a,b}\subseteq\R$ und $f: I\fromto Y$ stetig auf $\interv{a,b}$ und differenzierbar auf $\pair{a,b}\eqqcolon I^{\circ}$. Sei außerdem $\norm{\D f\of{s}}_{\R\fromto Y} = \norm{f'\of{s}}_Y \leq M$ für alle $a < s < b$. Dann gilt
        \begin{align*}
            \norm{f\of{b}-f\of{a}}_Y &\leq M\cdot\pair{b-a}
        \end{align*}

        \begin{proof}
            Sei $\eta > 0$ fixiert. Wir definieren
            \begin{align*}
                A\coloneqq\set{\xi\in\interv{a,b}: \forall s, a\leq s < \xi\text{ gilt } \norm{f\of{s}-f\of{a}}_{Y} \leq \pair{M+\eta}\cdot\pair{s-a}}
            \end{align*}
            Dann ist $a\in A\impl A\neq \emptyset$. Sei $c\coloneqq \sup A \leq b$. Da $f$ stetig ist, gilt $c\in A \impl \interv{a,c}\subseteq A$. Zu zeigen ist, dass $c=b$. Angenommen $c<b$. Dann gilt
            \begin{align*}
                \exists\delta > 0, c+\delta \leq b, c\leq t < c+\delta\colon f\of{t} - f\of{c} - \D f\of{c}\interv{t-c} = f\of{t} - f\of{c} - f'\of{c}\cdot\pair{t-c} = \varepsilon\of{t\cdot c}
            \end{align*}
            \begin{align*}
                \norm{f\of{t} - f\of{c}}_Y &\leq \norm{f'\of{c}}_Y \cdot\pair{t-c} + \norm{\varepsilon\of{t-c}}_Y \leq M\cdot\pair{t-c} + \norm{\varepsilon\of{t-c}}_Y\cdot\pair{t-c}\\
                \norm{f\of{t} - f\of{a}}_Y &\leq \norm{f\of{c} - f\of{a}}_Y + \norm{f\of{t} - f\of{c}}_Y\\
                &\leq \pair{M + \eta}\cdot\pair{-a+c} + \pair{M+\frac{\eta}{2}}\pair{t-c} < \pair{M+\eta}\pair{t-a}\\
                \impl t&\in A\\
                \impl c&= b\qedhere
            \end{align*}
        \end{proof}
    \end{satz}

    \begin{bemerkung}
        Eine Menge $A\subseteq X$ heißt konvex, falls $\forall x,y\in A\colon \interv{x,y}\subseteq A$.
    \end{bemerkung}

    \begin{anwendung}
        Sei $U\subseteq X$ offen und $x,y\in U$ mit $\interv{x,y}\subseteq U$. Dann ist\\ $\interv{x,y} \coloneqq \set{x+s\cdot\pair{y-x}: 0 \leq s \leq 1}$. Das heißt wir definieren $X\of{s} = x + s\cdot\pair{y-x}$ und können für eine Funktion $f$, die auf $U$ definiert ist schreiben
        \begin{align*}
            F\of{s} \coloneqq f\circ X\of{s}
        \end{align*}
        Damit haben wir nach Kettenregel für $h\in \R$
        \begin{align*}
            \D F\of{s}\interv{h} &= \D f\of{X\of{s}}\interv{\D X\of{s}\interv{h}}\\
            \frac{\dif}{\dif s} &= \lim_{t\fromto 0} \frac{F\of{s+t} - F\of{s}}{t} = \text{D}_1\! F\of{s} = \D F\of{s}\interv{1} = \D f\of{X\of{s}}\interv{y-x}\\
            \norm{F'\of{s}}_Y &= \norm{\D f\of{X\of{s}}\interv{y-x}}_Y \leq \norm{\D f\of{X\of{s}}}_{X\fromto Y} \cdot\norm{y-x}_X
            \intertext{Dann definieren wir}
            M &\coloneqq \sup_{0\leq s \leq 1} \norm{F'\of{x}}_Y \leq \sup_{0\leq s \leq 1}\norm{\D f\of{X\of{s}}}_{X\fromto Y}\cdot\norm{y-x}_X = \norm{\D f\of{\xi}}_{X\fromto Y}\cdot\norm{Y-x}_X\tag{$\xi = \pair{1-s}\cdot x + sy$}\\
            \impl \norm{f\of{y} - f\of{x}}_Y &= \norm{F\of{1} - F\of{0}}_Y
            \intertext{Nach Satz~\ref{satz:mittelwertsatz} können wir abschätzen}
            &\leq \sup_{0\leq s \leq 1} \norm{F'\of{s}}_Y \cdot 1
        \end{align*}
    \end{anwendung}

    \begin{satz}[Schrankensatz II] % Satz 11
        \label{satz:schrankensatz-ii}
        Seien $\pair{X, \norm{\cdot}_X}, \pair{Y, \norm{\cdot}_Y}$ normierte Vektorräume und $U\subseteq X$ offen, $x,y\in U$, $\interv{x,y}\subseteq U$. Ferner sei $f: U\fromto Y$ differenzierbar auf ganz $U$. Dann gilt
        \begin{align*}
            \norm{f\of{y}  -f\of{x}}_Y \leq \sup_{\xi\in\interv{x,y}} \norm{\D f\of{\xi}}_{X\fromto Y} \cdot \norm{y-x}_X &= \sup_{0\leq s \leq 1}\norm{\D f\of{\pair{1-s}\cdot x + sy}}_{X\fromto Y}\cdot\norm{y-x}_X
            \intertext{Außerdem $\forall A\in\mL\of{X, Y}$}
            \norm{f\of{y} - f\of{x} - A\interv{y-x}}_Y &\leq \sup_{\xi\in\interv{x,y}} \norm{\D f\of{\xi} - A}_{X\fromto Y} \cdot\norm{y-x}_X\numberthis\label{eq:schrankensatz-ii}
        \end{align*}

        \begin{proof}[Beweis von (\ref{eq:schrankensatz-ii})]
            Seien $x,y\in U$, $\interv{x,y} \subseteq U$, $A\in \mL\of{X, Y}$. Sei $f: U\fromto Y,~x\mapsto F\of{x} \coloneqq f\of{x} - A\interv{X}$.
            \begin{align*}
                \D F\of{x}\interv{h} &= \D f\of{x}\interv{h} - A\interv{h}\\
                \impl \norm{f\of{x} - f\of{y} - A\interv{x-y}}_Y &= \norm{F\of{y} - F\of{x}}_Y\\
                &\leq \sup_{\xi\in\interv{x,y}} \norm{\D F\of{\xi}}_{X\fromto Y}\cdot\norm{y-x}_X\qedhere
            \end{align*}
        \end{proof}
    \end{satz}

    \begin{satz} % Satz 12
        Sei $U\subseteq X$, $x,y\in U$, $\interv{x,y}\subseteq U$ und $f: U\fromto \R$ differenzierbar. Dann gibt es ein $\xi\in\interv{x,y}$ mit
        \begin{align*}
            f\of{x} - f\of{y} &= \D f\of{\xi}\interv{y-x}
        \end{align*}

        \begin{proof}
            Setze $F: \interv{0,1}\fromto\R,~s\mapsto f\of{\pair{1-s}\cdot x + sy}$
            \begin{align*}
                \exists t\in\pair{0,1}\colon F\of{1} - F\of{0} = F'\of{t}\cdot\pair{1-0} = F'\of{t}\\
                F'\of{s} = \frac{\dif}{\dif s}f\of{\pair{1-s}\cdot x + sy} = \frac{\dif}{\dif s} f\of{x+s\cdot\pair{y-x}} = \D f\of{x+s\pair{y-x}}\interv{y-x}
            \end{align*}
            Dann setze $\xi\coloneqq \pair{1-t}\cdot x + ty$.
        \end{proof}
    \end{satz}

    \subsection{Existenz von Ableitungen}

    Das Ziel dieses Teilkapitels ist es, ein Differenzierbarkeitskriterium zu finden und in diesem Sinne den folgenden Satz zu beweisen.

    \begin{satz} % Satz 13
        \marginnote{[11. Jul]}
        \label{satz:existenz-ableitung}
        Eine Funktion $f: U\fromto Y$ für einen Banachraum $Y$ und eine offene Menge $U\subseteq \R^n$ ist genau dann stetig differenzierbar, wenn alle partiellen Ableitungen $\partial_j f: U\fromto Y$ ($j\in\set{1,\ldots, n}$) stetig sind.
    \end{satz}

    \begin{definition}[Verallgemeinerte partielle Ableitung]
        Wir können den $\R^n$ zerlegen, indem wir $\R^{n_1}, \R^{n_2}, \ldots, \R^{n_k}$ für $k\in\N$ und $n_j\in\N$ finden, sodass $\sum n_j = n$. Dann schreiben wir eine Zerlegung als $\R^n = \R^{n_1}\times \R^{n_2}\times\dots\times\R^{n_k}$. Außerdem können wir einen Vektor $x\in\R^n$ schreiben als $x=\pair{x_1, \ldots, x_k}$, wobei $x_j\in\R^{n_j}$.\\
        Wir betrachten eine Funktion $f: U\fromto Y$ mit $U$ offen und $a=\pair{a_1, \ldots, a_k}\in U$. Die Ableitung von $f\of{a_1, \ldots, a_{j} + x_j, a_{j+1},\ldots, a_k}$ nach $x_j$ an der Stelle $x_j = 0$ heißt verallgemeinerte partielle Ableitung von $f$ nach $j$ an der Stelle $a$.
    \end{definition}

    \begin{lemma} % Lemma 15
        \label{lemma:veralg-part-diff}
        Sei $f: U\fromto Y$ eine Abbildung mit $U\subseteq\R^n$ und haben wir eine Zerlegung von $\R^n$ mit $\sum n_j = n$. Dann gilt für die verallgemeinerte partielle Ableitung $\text{D}_j\! f\of{a}$ mit $h=\pair{h_1, \ldots, h_k}\in\R^{n_1}\times\dots \times\R^{n_k}$
        \begin{align*}
            \D f\of{a} &= \sum_{j=1}^{k} \text{D}_j f\of{a}\interv{h_j}
        \end{align*}
        \begin{proof}
            Sei $I_j: \R^{n_j}\fromto\R^{n}$, $I_j\of{x_j} = \pair{0,\ldots, 0, x_j, 0,\ldots, 0}$. Dann gilt
            \begin{align*}
                \D I_j\interv{h_j} &= I_j\of{h_j} = \pair{0,\ldots, 0, h_j, 0, \ldots, 0}\\
                \text{D}_j f\of{a} &= \D f\of{a}\circ I_j\\
                \impl \sum_{j=1}^{k} \text{D}_j f\of{a}\interv{h_j} &= \sum_{j=1}^{k} \D f\of{a} \cdot I\of{h_j} = \sum_{j=1}^{k} \D f\of{a}\interv{I_j\interv{h_j}}\\
                &= \D f\of{a} \cdot \underbrace{\sum_{j=1}^{k} I_j\interv{h_j}}_{h} = \D f\of{a}\interv{h}
            \end{align*}
        \end{proof}
    \end{lemma}

    \begin{satz} % Satz 16
        \label{satz:diff-krit-veralg-part}
        Eine Funktion $f: U\fromto Y$ für einen Banachraum $Y$ und eine offene Menge $U\subseteq \R^n$ ist genau dann stetig differenzierbar, wenn die verallgemeinerten partiellen Ableitungen $\text{D}_j f$ für alle Zerlegungen von $\R^n$ auf $U$ stetig differenzierbar sind.
        \begin{proof}
            \anf{$\impl$} Sei $f$ differenzierbar. Dann gilt nach Lemma~\ref{lemma:veralg-part-diff}, dass
            \begin{align*}
                \text{D}_j f\of{x}\interv{h_j} = \D f\of{x}\interv{I_j h}
            \end{align*}
            Da die rechte Seite in $x$ stetig differenzierbar ist, muss auch die linke Seite stetig differenzierbar sein.\\[.2\baselineskip]
            \anf{$\Leftarrow$} \textsc{Schritt 1}: Wir betrachten zunächst den Fall $\R^n = \R^{n_1} \times\R^{n_2}$ und definieren dafür $\norm{h}\coloneqq\max\of{\norm{h_1}_1, \norm{h_2}_2}$. Dann gilt
            \begin{align*}
                &f\of{x_1 + h_1, x_2 + h_2} - f\of{x_1, x_2} - \text{D}_1 f\of{x_1, x_2}\interv{h_1} - \text{D}_2\of{x_1, x_2}\interv{h_2}_Y\\
                = &f\of{x_1 + h_1, x_2 + h_2} - f\of{x_1 + h_1, x_2} - \text{D}_2 f\of{x_1+h_1, x_2}\interv{h_2} + \text{D}_2 f\of{x_1 + h_1, x_2}\interv{h_2}\\
                &- \text{D}_2 f\of{x_1, x_2}\interv{h_2} + f\of{x_1 + h_1, x_2} - f\of{x_1, x_2} - \text{D}_1 f\of{x_1, x_2}\interv{h_1} - \text{D}_2 f\of{x_1, x_2}\interv{h_2}
            \end{align*}
            Wir wollen die Normen der Terme nacheinander nach oben abschätzen
            \begin{alignat*}{2}
            (1)
                \quad&&\norm{f\of{x_1 + h_1, x_2} - f\of{x_1, x_2} - \text{D}_1 f\of{x_1, x_2}\interv{h_1}}_Y &\leq \frac{\varepsilon\of{h_1}}{4}\cdot\norm{h_1}_1\\
                (2)\quad&&\norm{\text{D}_2 f\of{x_1 + h_1, x_2} - \text{D}_2 f\of{x_1, x_2}}_{\R^{n_2}\fromto Y} &< \frac{\varepsilon}{4}\tag{Stetigkeit}\\
                &&\impl \norm{\text{D}_2 f\of{x_1 + h_1, x_2}\interv{h_2} - \text{D}_2 f\of{x_1, x_2}\interv{h_2}}_Y &< \frac{\varepsilon}{4}\cdot\norm{h_2}\\
                (3)\quad&&\norm{f\of{x_1 + h_1, x_2 + h_2} - f\of{x_1 + h_1, x_2} - \text{D}_2 f\of{x_1 + h_1, x_2}\interv{h_2}}_y \annot[{&}]{\leq}{\ref{satz:schrankensatz-ii}}\\
                &&\norm{h_2}\cdot\sup_{0\leq s \leq 1} \norm{\text{D}_2 f\of{x_1 + h_1, x_2 + sh_2} - \text{D}_2 f\of{x_1 + h_1, x_2}} &< \norm{h_2} \cdot \frac{\varepsilon}{4}
            \end{alignat*}
            Das heißt insgesamt folgt
            \begin{align*}
                \norm{f\of{x_1 + h_1, x_2 + h_2} - f\of{x_1, x_2} - \text{D}_1 f\of{x_1, x_2}\interv{h_1} - \text{D}_2\of{x_1, x_2}\interv{h_2}}_Y &\leq \varepsilon\of{\norm{h}}\cdot\norm{h}
            \end{align*}
            Damit folgt die Behauptung für diesen speziellen Fall.\\[5pt] \textsc{Schritt 2}: Im allgemeinen Fall haben wir $\R^n = \R^{n_1}\times\dots\times\R^{n_k}$. Dann gilt nach \textsc{Schritt 1}, dass wir mit $\text{D}_1 f$, $\text{D}_2 f$ auch $\text{D}_{12} f$ erhalten. Wir nehmen eine neue Zerlegung $\R^n = \pair{\R^{n_1}\times\R^{n_2}}\times \R^{n_3}\times\dots\times\R^{n_k}$. Dann erhalten wir aus $\text{D}_{12} f$ und $\text{D}_3 f$ auch $\text{D}_{123} f$. Dieses Vorgehen setzen wir induktiv fort. So ergibt sich nach $k$ Schritten unsere Behauptung.
        \end{proof}
    \end{satz}

    \begin{proof}[Beweis von Satz~\ref{satz:existenz-ableitung}]
        Der Satz folgt dann aus Satz~\ref{satz:diff-krit-veralg-part}, indem wir als initiale Zerlegung die Zerlegung $\R^n = \R^{1}\times \dots \times \R^{1}$ wählen.
    \end{proof}

    \subsection{[*] Symmetrie der zweiten partiellen Ableitung - Der Satz von Schwartz}


    \begin{mdframed}
        \begin{center}
            Für dieses Teilkapitel sei $Y$ ein Banach-Raum und $U\subseteq\R^n$ eine offene Menge.
        \end{center}
    \end{mdframed}

    \begin{bemerkung}
        \marginnote{[12. Jul]}
        Wir betrachten eine Abbildung $f: U\fromto Y$. Dann gilt $\D f: U\fromto\mL\of{\R^n, Y}$ und $\text{D}\of{\D f} = \text{D}^2 f\of{x}\in\mL\of{\R^n, \mL\of{\R^n, Y}}$. Außerdem haben wir
        \begin{align*}
            \D f\of{x}\interv{h} &= \sum_{j=1}^{n} \partial_j f\of{x}\interv{h}\\
            &= \pair{\partial_1 f\of{x}, \partial_2 f\of{x}, \ldots, \partial_n f\of{x}}\cdot\begin{pmatrix}
                                                                                                 h_1\\
                                                                                                 \vdots\\
                                                                                                 h_n
            \end{pmatrix}\\
            &= \sprod{\nabla f, h}
            \intertext{Dabei ist $\partial_i f\of{x}: U\fromto Y^n$}
            \text{D}_v \D f\of{x} &= \frac{\dif}{\dif t} \D f\of{x+tv}\vert_{t = 0}\\
            &= \pair{\frac{\dif}{\dif t} \partial_1 f\of{x + v}, \frac{\dif}{\dif t}\partial_2 f\of{x+tv}, \ldots, \frac{\dif}{\dif t} \partial_n f\of{x+v}}\vert_{t=0}\\
            &= \sum_{k=1}^{n} v_k\cdot\pair{\partial_k \partial_1 f\of{x}, \ldots, \partial_k \partial_n f\of{x}}
            \intertext{$v=\pair{v_1, \ldots, v_n}$}
            \text{D}_v \D f\of{x}\interv{h}  &= \frac{\dif}{\dif t}\pair{\partial_1 f\of{x+tv}, \ldots, \frac{\dif}{\dif t}\pair{\partial_n f\of{x+tv}}}\vert_{t=0}\\
            &= \sum_{k=1}^{n} v_k\pair{\sum_{j=1}^{n} \partial_k\partial_j f\of{x}\cdot h_j} = \sum_{k=1}^{n} \sum_{j=1}^{n} v_k \partial_k\partial_j f\of{x} h_j\\
            &= \sprod{v, H_{e}h}
            \intertext{wobei}
            H_e\of{x} &= \begin{pmatrix}
                             \partial_1\partial_1 f\of{x} & \partial_1\partial_2 f\of{x} & \dots & \partial_1\partial_n f\of{x}\\
                             \partial_2 \partial_1 f\of{x} & \partial_2\partial_2 f\of{x}& \dots & \partial_2 \partial_n f\of{x}\\
                             \vdots & \ddots &\ddots & \vdots\\
                             \partial_n\partial_1 f\of{x} & \dots & \dots&\partial_n \partial_n f\of{x}
            \end{pmatrix}\tag{Hesse-Matrix}
        \end{align*}
    \end{bemerkung}

    \begin{definition}
        Sei $f: U\fromto Y$ eine Abbildung. Dann gilt $\D f: U\fromto \underbrace{\mL\of{\R^n, Y}}_{\text{Banachraum}}$.\\
        Ist $\D f$ stetig differenzierbar, so heißt $f$ zweimal differenzierbar auf $U$. Wir nennen $\text{D}^2 f\of{x}$ dann die zweite Ableitung von $f$. Dabei ist
        \begin{enumerate}[label=(\roman*)]
            \item $\D f: U\fromto\mL\of{\R^n, Y}$
            \item $\text{D}^2 f: U\fromto \mL\of{\R^n, \mL\of{\R^n, Y}}$
        \end{enumerate}
        Sei $L\in\mL\of{\R^n, \mL\of{\R^n, Y}}$, $B: \R^n\times\R^n,~\pair{v,h}\mapsto \sprod{v, Bh} = \sum_{k=1}^n v_k\cdot \sum_{j=1}^n B_{k_j} h_j$. Definiere $B\of{v,h} \coloneqq \pair{L\interv{h}}\interv{h}$. Sei $\pair{e_1, \ldots, e_n}$ die Standardbasis in $\R^n$
        \begin{align*}
            v &= \sum_{}^{} v_k\cdot e_k\\
            h &= \sum h_k\cdot e_k\\
            \pair{L\interv{v}}\interv{h} &= \pair{L\interv{\sum_{k=1}^{n} v_k\cdot e_k}}\interv{\sum_{j=1}^{n} h_j e_j}\\
            &= \sum_{k=1}^{n} \sum_{j=1}^{n} \underbrace{\pair{L\interv{e_k}\interv{e_j}}}_{B_{k_j} = B\of{e_k, e_j}}\cdot v_k h_j\\
            &= \sum_{k=1}^{n} \sum_{j=1}^{n} B_{k_j} v_k h_j = \sprod{v, Bh}\\
            \pair{\text{D}^2 f\of{x}\interv{v}}\interv{h} &= \sum_{k=1}^{n} \sum_{j=1}^{n} v_k\pair{\text{D}^2 f\of{x}\interv{e_k}}\interv{e_j} h_j\\
            &= \sum_{k=1}^{n} \sum_{j=1}^{n} v_k \partial_k \partial_j f\of{x} h_j\\
            &= \sprod{v, H_e h}
        \end{align*}
    \end{definition}

    \begin{satz} % Satz 1
        Die folgenden Aussagen sind äquivalent
        \begin{enumerate}[label=(\roman*)]
            \item Die 2. Ableitung existiert und ist stetig
            \item Alle gemischten partiellen Ableitungen $\partial_k\partial_j f\of{x}$, $1\leq k,j \leq n$ existieren und sind stetig
            \item Die Hesse-Matrix $H_e \of{x}$ existiert und ist stetig
        \end{enumerate}

        \begin{proof}
            Folgt direkt aus Satz~\ref{satz:existenz-ableitung} und Satz~\ref{satz:diff-krit-veralg-part}.
        \end{proof}
    \end{satz}

    \begin{satz} % Satz 2
        Sei $f: U\fromto Y$ zwei mal differenzierbar und $x\in U$. Dann ist die bilineare Abbildung $\text{D}^2 f\of{x}: \R^n \times \R^n\fromto Y$ symmetrisch. Das heißt für alle $v,h\in\R^n$ gilt
        \begin{align*}
            \text{D}^2 f\of{x}\interv{v,h} &= \text{D}^2 f\of{x}\interv{h,v}\\
            \intertext{oder}
            \text{D}_v\text{D}_h f\of{x} &= \text{D}_h \text{D}_v f\of{x}
            \intertext{oder}
            \partial_k\partial_j f\of{x} &= \partial_j \partial_k f\of{x}
        \end{align*}
        \begin{proof}
            Wir wollen zeigen, dass $\text{D}_v \text{D}_h f\of{x} = \text{D}_h \text{D}_v f\of{x}$. Wir betrachten den Fall $Y=\R$, $U\subseteq\R^n$. Definieren den Differenz-Operator
            \begin{align*}
                \Delta_h f\of{x} &\coloneqq f\of{x+h} - f\of{x}
                \intertext{\textsc{Schritt 1}: Für $f,h\in\R^n$ mit $x+h, x+v\in U$ gilt}
                \Delta_v \Delta_h f\of{x} &= \Delta_v \pair{f\of{x+h} - f\of{x}} = \Delta_v f\of{x+h} - \Delta_v f\of{x}\\
                &= f\of{x+h+v} - f\of{x+h} - f\of{x+v} + f\of{x}\\
                &= f\of{x+h+v} - f\of{x+v} - f\of{x+h} + f\of{x} = \Delta_h \Delta_v f\of{x}
                \intertext{\textsc{Schritt 2}: Sei $h,v\in\R^n$ beliebig und $x\in U$, $U$ offen. Dann $\exists\delta > 0\colon B_{2\delta}\of{x} \subseteq U$}
                \abs{s} &< \frac{\delta}{\min\of{1,\abs{v}}}\\
                \abs{t} &< \frac{\delta}{\min\of{1, \abs{h}}}
                \intertext{Wir wollen zeigen, dass $\frac{1}{st} \Delta_{sv}\Delta_{th} = \text{D}_h \text{D}_v f\of{x+s, v+t_1 h}$ für $s_1\in\pair{0,s}$, $t_1\in\pair{0,t}$}
                \Delta_{sv} \Delta_{th} f\of{x} &= g\of{x+sv} - g\of{x} = k'\of{s_1}\cdot\pair{s-0} = k'\of{s_1}\cdot s = \frac{\dif}{\dif s} g\of{x+sv}\vert_{s=s_1}\cdot s\\
                &= \text{D}_v g\of{x+s_1 v}\cdot s = \text{D}_v\of{\Delta_{th} f}\pair{x+s_1v}\cdot s\\
                &= \text{D}_v\of{f\of{x+s_1 v + th} - f\of{x+s_1v}\cdot s}\\
                &= \text{D}_v f\of{x+s_1 v + th}\cdot s - \text{D}_v f\of{x+s_1 v}\cdot s\\
                \text{D}_h \text{D}_v f\of{x+s_1 v + t_1 h}\cdot s \cdot t\\
                \impl \frac{1}{st}\Delta_{sv} \Delta_{th} f\of{x} &= \text{D}_h \text{D}_v f\of{x+s_1 v + t_1 h}
                \intertext{\textsc{Schritt 3}: Nach \textsc{Schritt 1} haben wir $\Delta_{sv}\Delta_{th} f\of{x} = \Delta_{th} \Delta{sv} f\of{x}$}
                \impl \frac{1}{st}\Delta_{th} \Delta_{sv} f\of{x} &= \text{D}_v \text{D}_h f\of{x+s_1 v + t_1 h} = \text{D}_h \text{D}_v f\of{x+s_1 v + t_1 h}
            \end{align*}
            Für $s\fromto 0$ und $t\fromto 0$ haben wir so die Behauptung.
        \end{proof}
    \end{satz}



\end{document}
